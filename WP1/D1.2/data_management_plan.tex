\documentclass[12pt]{amsbook}


\usepackage[utf8]{inputenc}
\usepackage{ae,aecompl,aeguill}	% pour utiliser << et >>
\usepackage{times}
\usepackage[babel=true,kerning=true]{microtype}

\title{Data Management Plan for OpenDreamKit}
\author{Benoît Pilorget}

\begin{document}

\maketitle

\section{Bla}

sdffjsadfgadnads adsf asd adsff asd fasdfd asdd adsfd asdfd sdadf
sdffjsadfgadnads adsf asd adsff asd fasdfd asdd adsfd asdfd sdadf
sdffjsadfgadnads adsf asd adsff asd fasdfd asdd adsfd asdfd sdadf
sdffjsadfgadnads adsf asd adsff asd fasdfd asdd adsfd asdfd sdadf
sdffjsadfgadnads adsf asd adsff asd fasdfd asdd adsfd asdfd sdadf
sdffjsadfgadnads adsf asd adsff asd fasdfd asdd adsfd asdfd sdadf


sdffjsadfgadnads adsf asd adsff asd fasdfd asdd adsfd asdfd sdadf
sdffjsadfgadnads adsf asd adsff asd fasdfd asdd adsfd asdfd sdadf
sdffjsadfgadnads adsf asd adsff asd fasdfd asdd adsfd asdfd sdadf

\begin{itemize}
\item bla
\item ble
\item bli
\end{itemize}

\subsection{Bla blae}

Une liste à puces numérotée:
\begin{enumerate}
\item adsfasdf
\item sdaf
\item sadfd asd fasd fas dfa sdf asdf asdf asdf asdf asddf as
\item asd asd asd dfa sd asd fasd dfas df asdf asdf asddf adsf f
\end{enumerate}

Un tableau
\begin{tabular}{|l|c|r|r|}
\hline
1 & tea & asdfas & asdad sds\\
1 & tea & asdfas & asdad sds\\
1 & tea & asdfas & asdad sds\\\hline
1 & tea & asdfas & asdad sds\\\hline
\end{tabular}

\begin{description}
\item[Mean of production] asdl asdf asdf 
\item[Format] asdl asdf asdf szdasd fasd dfas asdl asdf asdf szdasd
  fasd dfas asdl asdf asdf szdasd fasd dfas asdl asdf asdf szdasd fasd
  dfas asdl asdf asdf szdasd fasd dfas asdl asdf asdf szdasd fasd dfas
  asdl asdf asdf szdasd fasd dfas asdl asdf asdf szdasd fasd dfasasdl
  asdf asdf szdasd fasd dfas asdl asdf asdf szdasd fasd dfas
\item[Usage for further experiments] 
\end{description}

\section{Truc}

asdf \textbf{adsf} asd \textit{dasdf} asdfasdf asdf

\Huge

$\mathfrak{X\pi}$

Déjà ? «une citation».

asdf
sda
sadf
asdf
asdf

\end{document}
