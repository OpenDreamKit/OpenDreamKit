\documentclass[12pt]{amsbook}

\usepackage{hyperref}
\usepackage[utf8]{inputenc}
\usepackage{ae,aecompl,aeguill}	% pour utiliser << et >>
\usepackage{times}
\usepackage[babel=true,kerning=true]{microtype}

\title{Draft Data Management Plan for OpenDreamKit}
\author{Benoît Pilorget}

\begin{document}

\maketitle

\section{Datasets}



\subsection{UPSud}

This subsection will contain all datasets the UPSud is currently able to describe.
\begin{description}
\item[Data storage and security] Quickly explain how data are stored and protected within yout institution
\item[Dissemination] How data can be disseminated -> openaccess etc
\item[Preservation and future access] How data can be preserved and available in the next years
\end{description}


\begin{enumerate}


\item{Example}


\begin{description}
\item[Name of data] Photos from scottish landscapes
\item[Nature of data] Image, photo
\item[Reuse of existing data] We used old photos from 50 decades ago as well as paintings
\item[Mean of production] Camera
\item[Data standard] Photos are in black and white and printed with particular paper and ink
\item [Usage for further experiments] Photos will be put on this website so that they can be viewed by the max people. For this they will need to use that software
\end{description}


\item {Dataset 1}


\begin{description}
\item[Name of data]
\item[Nature of data]
\item[Reuse of existing data]
\item[Mean of production]
\item[Data standard]
\item [Usage for further experiments]
\end{description}


\item{Dataset 2}


\begin{description}
\item[Name of data]
\item[Nature of data]
\item[Reuse of existing data]
\item[Mean of production]
\item[Data standard]
\item [Usage for further experiments]
\end{description}

\end{enumerate}

\subsection{CNRS}

This subsection will contain all datasets the CNRS is currently able to describe
\begin{description}
\item[Data storage and security] Quickly explain how data are stored and protected within yout institution
\item[Dissemination] How data can be disseminated -> openaccess etc
\item[Preservation and future access] How data can be preserved and available in the next years
\end{description}


\subsection{Jacobsuni}



This subsection will contain all datasets the Jacobsuni is currently able to describe
\begin{description}
\item[Data storage and security] Quickly explain how data are stored and protected within yout institution
\item[Dissemination] How data can be disseminated -> openaccess etc
\item[Preservation and future access] How data can be preserved and available in the next years
\end{description}


All partners must add themselves to the list!


\subsection{University of Southampton}

There are no significant data sets associated with the work at Southampton. The most important data is resulting code and associated documentation and tutorials. The details below refer to this data set, and we expect the data set to be fairly small (order of 1 GB).
\begin{description}
\item[Data storage and security] Data Storage: The code is stored in a distributed repository (git at the moment), and a central clone of this repository is stored with Github.com in the cloud. We may use multiple repositories, and store a central copy of each on Github.com.

  Security: All all the code is public, and there no concern about unauthorised access. Through cloud hosting and local clones of repositories, there are backups and redundancy.
\item[Dissemination] Data can be accessed through the public repositories, and the public website (probably this URL: \href{http://joommf.github.io}{http://joommf.github.io}, tbc), providing open access.
\item[Preservation and future access] We rely on provision of the data through \href{github.com}{github.com} but maintain local copies of the repository in case github.com ceases to exist or suffers from catastrophic technology failure. It is likely that other online repository hosting providers would be able to fill the gap (bitbucket.org is an existing alternative). The University of Southampton offers long term storage of small data sets for 10 years -- the repositories would fall into this categories. While the data wouldn't be conveniently accessible, this provides an extra layer of backups, from which accessible repositories and websites could be created easily.
\end{description}



\end{document}
