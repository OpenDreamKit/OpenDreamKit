
\section{KPIs}

\begin{introduction}
 After some internal discussion and project meetings, we revamp our KPIs according 
to the oral written recommendations of the reviewers. We also finalized the classification
of the KPIs and added more metrics. We report here on the current state of our KPIs.
\end{introduction}







\subsubsection{KPIs for Aim 1}

\begin{Aim 1}
  Improve the productivity of researchers in pure mathematics and
  applications by promoting collaborations based on mathematical
  software, data, and knowledge.
\end{recommendation}

Qualitative metrics: success stories, reported as blog posts.

         success stories reported as blogposts

The OpenDreamKit proposal was submitted on January 15th of 2015. It was written collaboratively by a diverse team across Europe and science.
Video, produced with Gource, of the activity on our collaborative work space on github. A  good illustration of how a dispersed but
dedicated team can collaborate efficiently with modern tools see Blogpost: Open Collaborative Grant Writing 
http://inverseprobability.com/2015/01/14/open-collaborative-grant-writing

             Collaborative work articles:
             
Blogpost on how to generate documentation for Cython projects by Jeroen Demeyer:
https://github.com/OpenDreamKit/OpenDreamKit.github.io/blob/333b21524f753dd49e441b4b1157ded68999eeff/meetings/2018-02-01-SteeringCommitteeMeeting/ProgressReports/UGent.md

As part of its OpenDreamKit deliverable, the Pythran team has written an in-depth article about identifier binding computation within the Pythran compiler. http://serge-sans-paille.github.io/pythran-stories/identifier-binding-computation.html

As part of its OpenDreamKit deliverable D5.4 (Make Pythran typing better to improve error information), the Pythran team has written an in-depth article about an unsound type checker. http://serge-sans-paille.github.io/pythran-stories/from-pythran-import-typing.html

             Story of cocalc adoption for teaching in UK after ODK's Kickoff
             
Supported a number of lecturers in Sheffield to migrate to Jupyter and CoCalc (formerly SageMathCloud). Also, provided continued support to those that had already been using CoCalc and Jupyter notebooks for their courses. These included lecturers from Computer Science, Physics, Biomedical Science, Bioinformatics, and Materials Science. (D2.17, T2.6)
The previously generated CoCalc tutorial was extended by adding tutorial sections for students having courses in CoCalc as well as with a hands-on tutorial for lecturers to get started. The material can be found as a website at https://tutorial.cocalc.com/ (The repository for this is located at:https://github.com/sagemathinc/cocalc_tutorial) (D2.17, T2.6).
https://github.com/OpenDreamKit/OpenDreamKit.github.io/blob/7cbee70c3bff6d0778733be931c927981df87e65/meetings/2018-10-28-Luxembourg/SteeringCommitteeMeeting/ProgressReports/Sheffield-Leeds.md

               Reproducible GAP experiments on Binder (AlexK)
Here https://github.com/alex-konovalov/gap-teaching, you can find a collection of GAP Jupyter notebooks. It uses the Docker container with the latest public release of GAP, which is maintained in a separate repository at https://github.com/gap-system/gap-docker.
AlexK used GAP Jupyter interface in teaching at PGTC 2018 and "Software tools for mathematics" workshop.

             Stories of Mike (Michael Croucher)
Along with the Software Sustainability Institute, the UK Research Software Engineering Association and the EU-funded OpenDreamKit project, Michael Croucher actively campaign to improve the career prospects of the talented people who underpin a huge amount of computational research…. https://github.com/mikecroucher/MLPM_talk

             Katja, Michael, Samuel involvement in OpenDreamKit: "Software tools for mathematics" workshops (Katja, Samuel)
http://stm.famnit.upr.si/
http://matmor.unam.mx/software-tools-math/
The workshop had two self-contained parts, both aimed at students and researchers who wanted to improve the computing skills they need for their studies and research.

                        
                       
                        

\subsubsection{KPIs for Aim 2 / adoption of \ODK's technologies}

\begin{Aim 2}
  Make it easy for teams of researchers of any size to set up custom,
  collaborative Virtual Research Environments tailored to their
  specific needs, resources and workflows. The VRE should support the entire life-cycle of computational work in mathematical research, from 
  initial exploration to publication teaching and outreach;
\end{recommendation}

Qualitative metrics:
\begin{itemize}
\item Success stories about \ODK based VRE deployments (university-wide, Mathrice, EGI, Micromagnetics VRE, ...), 
and generally speaking adoption of \ODK's components.
\item added metrics:
 
The Jupyter kernels for mathematical software 
developed as part of OpenDreamKit make computational mathematical components accessible in a Jupyter environment, enabling a Jupyter-based 
deployment of the relevant tools for the researchers.
 
       June 2018, Jupyter awarded prestigious 2017 ACM Software System Award
       Previous winners include: UNIX, TCP/IP, the Web, TeX, Java, GCC, LLVM
       https://blog.jupyter.org/jupyter-receives-the-acm-software-system-award-d433b0dfe3a2
       
       
       Partnership with EGI (main stakeholder of the EOSC): there is an ongoing collaboration between EGI and OpenDreamKit to deploy
       JupyterHub and BinderHub-based EGI services. Proofs of concepts for both have been deployed by Enol Fernandez from EGI. Both parties
       are very satisfied with the collaboration and want to strengthen it. A Technology Provider Agreement was signed between EGI and
       Simula, on behalfF of OpenDreamKit/Jupyter developers. Joint applications to upcoming EOSC calls, and in particular INFRAEOSC-02-2019
       https://github.com/OpenDreamKit/OpenDreamKit.github.io/blob/7cbee70c3bff6d0778733be931c927981df87e65/meetings/2018-10-28-Luxembourg
       /SteeringCommitteeMeeting/ProgressReports/ParisSud.md
       
      
      Milestone 5: “ODK’s computational components available on major platforms” (month 42).
      User story: users shall be able to easily install ODK’s computational components on the three major platforms (Windows, Mac, Linux)
      via their standard distribution channels.
      Packages available for all components on Debian, Ubuntu, Fedora, Arch, Gentoo, ...Experimental Conda Forge packages
      
      Arch Based upon voluntary reports from ~30k users
      
      Debian Based on ~200K voluntary submissions
      
      Ubuntu, based on ~2.8M voluntary submissions
      
      Docker hub https://hub.docker.com/r/sagemath/ and https://hub.docker.com/u/gapsystem/

          10k pulls for sagemath-jupyter
          5.7k pulls for sagemath
          3.8K pulls for gap-docker
Compare with https://hub.docker.com/r/jupyter/, which states 1M+ pulls for scipy-notebook.
 
   
  Quantitative metrics:
\begin{description}
\item List of known \ODK based VRE deployments, as tracked on
  \url{https://github.com/OpenDreamKit/OpenDreamKit/issues/174}
\item Number of installs of \ODK's components via platform-specific
  distribution channels: Debian popcon, Arch statistics, installer
  downloads, \dots
%\item[WP3] Decrease in set-up time for dissemination workshops based
%  on \ODK components. This is hard to measure objectively, better to
%  gather user stories.
\end{description}

       OpenDreamKit bet on Jupyter notebooks: It has paid off!
       Millions of notebooks online (over 3M on GitHub alone)

We tried to track collaborations with various institutions and projects to deploy instances of JupyterHub and CoCalc (formerly SageMathCloud), and collect estimates of the difficulties involved in such deployments in various use cases to:

    help plan future deployments
    seek what could be done to ease deployment

  Running:
  [x] Local CoCalc instance at Universität Zürich.
      Deployed September 2015 - February 2016.
      People involved: @pdehaye, @williamstein
      
  [x] [Instance of JupyterHub](https://jupyter.math.cnrs.fr/hub/) deployed by the [Mathrice group](http://mathrice.fr/)
      Host Infrastructure: France Grille's LAL cloud
      Users: members of math labs in France
      Main use case: casual use 
      People involved: the Mathrice group, @nthiery

  [x] Local [JupyterHub instance at Université Paris Sud / Paris Saclay](http://jupytercloud.lal.in2p3.fr/)
      Host Infrastructure: France Grille's LAL cloud
      Users: personnel and students of UPSud / Paris Saclay
      Main use case: use in classroom (Python, Sage, C++), casual use
      People involved: @VivianePons, @nthiery, @gouarin 
      
  [x] JupyterHub instance deployed on USheffield's HPC system
      http://docs.iceberg.shef.ac.uk/en/latest/using-iceberg/accessing/jupyterhub.html
      People involved: @mikecroucher
  
  [x] JupyterHub instance(s) deployed at UVSQ
      https://opendreamkit.org/2018/10/17/jupyterhub-docker/
      Main use case: use in classroom (Sage, Python, C, Apache Spark), casual use
     People involved: @defeo 
   
  [x] [Gallery of JupyterHub instances](https://jupyterhub.readthedocs.io/en/latest/gallery-jhub-deployments.html)
  [x] JupyterHub and Binder instances deployed on EGI infrastructure; see #205.
  [x] Easy deployment of live GAP/SageMath/... notebooks with [mybinder](mybinder.org), 
  thanks to the Docker containers (#58);potential alternatives: [Debian packaging](https://wiki.debian.org/DebianScience/Sage) 
  and [Conda packaging](https://wiki.sagemath.org/Conda).
  People involved: @nthiery, ...
  [x] Local instance of CoCalc (using the Docker container) at the University of Gent
    Main use case: teaching for mathematics students

  Investigated or under investigation:
  [ ] Interest in a JupyterHub deployment at the [Einstein Institute of Mathematics](http://math.huji.ac.il), part of the Hebrew University of Jerusalem
      Related: [Sage Days 79](https://wiki.sagemath.org/days79)
  [ ] Local Cocalc instance at UPSud
      Host infrastructure: UPSud's cloud.
      Users: personnel and students of UPSud
      Main use case: use in classroom (Python, Sage, C++), casual use
      People involved: @VivianePons, @nthiery, @gouarin 
      For now, priority has been given to the above JupyterHub instance
  [ ] Integration of Sage in the tmpnb.org's temporary notebook server
      People involved: @rgbkrk, @nthiery
      Status: some experiments run during a sprint at Pycon'15. Now that Sage's
      Jupyter kernel is well integrated in the stable version of Sage, it's just a question
      of installing Sage in tmpnb's docker container.
      This is superseded by [binder](http://mybinder.org); see above.
\end{itemize}




 

\subsubsection{KPIs for Aim 3}

\begin{recommendation}{Aim 3}
  Identify and promote best practices in computational mathematical
  research including: making results easily reproducible; producing
  reusable and easily accessible software; sharing data in a
  semantically sound way; exploiting and supporting the growing
  ecosystem of computational tools.
\end{recommendation}

Qualitative metrics: Success stories
\begin{itemize}
\item Mike Croucher's talk ``Is your research software correct''
Best practice and tools for correct and reproducible research]
         
         We refer here to the excellent talk
\href{https://mikecroucher.github.io/MLPM_talk/}{``Is your research software correct''} by Mike Croucher which highlights crucial best 
practice whenever software is used in research, including open code and data sharing, automation, use of high level languages, software 
training, version control, pair programming, literate computing, or testing. A lot of the work in ODK relates to disseminating this set of 
best practice (\longWPref{dissem}), and enabling it through appropriate technology (\longWPref{UI}).  Just to cite a few examples, 
\longdelivref{UI}{jupyter-collab}, and \longdelivref{UI}{jupyter-test} enable respectively version control and testing in the \Jupyter 
literate computing technology, while Mike's talk is and will be delivered in several of ODK's many training events (see 
\longdelivref{dissem}{workshops-1} for the list of training events in year 1).


 OpenDreamKit member, Tania Allard, ran a hands-on workshop on Jupyter notebooks for reproducible research. This workshop focused on the use of Jupyter notebooks as a means to disseminate reproducible analysis workflows and how this can be leveraged using tools such as nbdime and nbval. Both nbdime and nbval were developed by members of the OpenDreamKit project as a response to the growing popularity of the Jupyter notebooks and the lack of native integration between these technologies and existing version control and validation/testing tools.

An exceptional win was that this workshop was, in fact, one of the most popular events of the conference and we were asked to run it twice as it was massively oversubscribed. This reflects, on one hand, the popularity of Jupyter notebooks due to the boom of literate programming and its focus on human-readable code. Allowing researchers to share their findings and the code they used along the way in a compelling narrative. On the other hand, it demonstrates the importance of reproducible science and the need for tools that help RSE and researchers to achieve this goal, which aligns perfectly with the goals of OpenDreamKit.

The workshop revolved around 3 main topics:
    Version control of the Jupyter notebooks
    Notebooks validation
    The basics of reproducible software practices.
    
    https://github.com/OpenDreamKit/OpenDreamKit.github.io/blob/6faf6eb2f1532f342f86c8da633078067ca40c85/_posts/2018-03-07-opendreamkit-at-     
    the-rse-conference.md

 
\item Some research paper that showcases a range of best practices supported by ODK work (paper written collaboratively on e.g. github,
  software distributed as e.g. SageMath package, live demo and logbooks on binder, nbdime for collaboration, ...).
\item ...
\end{itemize}

      logbooks 
      https://github.com/OpenDreamKit/OpenDreamKit.github.io/blob/651460262b39df9b01f468c2e7561b42e83b5382/_posts/2017-11-02-use-case-
      publishing-reproducible-notebooks.md
      

\begin{event}{2018 Leeds: Introduction to reproducible workflows in Python}{leeds_repro_python}{University of Leeds, June 14th 2018}{LEEDS}{20}{1}{http://arc.leeds.ac.uk/training/spc-1-introduction-to-reproducible-workflows-in-python/}
This is an introductory course to reproducible analysis workflows in Python. It is aimed at people with some experience in Python for data analysis or computational research (e.g. people already developing scripts or using Jupyter notebooks). By the end of the course the attendees will have learnt about best practices for reproducible scientific code development and should be able to implement these techniques to their day to day workflows.
Materials at \url{https://github.com/trallard/ReproduciblePython}
\textbf{Results and impact.} Around 20 researchers received training.


Workshop June 4th - June 8th 2018 University of St Andrews:
Blog post about Jupyter in GAP and other CAS: https://www.gapdays.de/gap-jupyter-days2018/
Many new technologies for interactive documents, such as the GAP native Jupyter Kernel, MyBinder, ThebeLab have been developed recently. The aim of this workshop was to bring together people who are developing or using these technologies, and people who want to use them. While combining the use of GAP and Jupyter was the main focus of the workshop, people developing Jupyter or using Jupyter to access other software were welcomed to attend. The topics included:

      ThebeLab

- [x] Prepared and delivered a demo; now included in ThebeLab's [examples](https://github.com/minrk/thebelab/tree/master/examples) directory (Nicolas)
- [GAP demo](https://sebasguts.github.io/thebelab_test_gap/chap42)
- [SageMath demo](http://sage-package.readthedocs.io/en/latest/sage_package/sphinx-demo.html) ([about](http://sage-package.readthedocs.io/en/latest/sage_package/thebe.html))
- [x] Improved ThebeLab's documentation (Nicolas, Frank) 
- [x] Finalized [Sage's ticket](https://trac.sagemath.org/ticket/24593) enabling live documentation with ThebeLab (Nicolas)
  Sage package for ThebeLab, fetch ThebeLab locally if needed

      Around Binder

- [Main deployment](http://mybinder.org/)
- [Alternative deployment on EGI's cloud](https://binderhub-jupyter.fedcloud-tf.fedcloud.eu/)

- [x] Fix the alternative deployement and upgrade its docker (reported by Nicolas; done by Eñol Fernandez)
 
      Docker image for GAP

- [x] Update the official GAP Docker to 4.9.1 (Sebastian, Alex)
- [x] Update the GAP Docker Master
         * (this builds nightly, but we will have to update it following the docker base image update)
- [x] Testing GAP Docker containers after Sebastian's updates and cleaning up old issues in their repositories (Alex)
- [ ] Use GAP Docker Master to create a Binder-ready image (Alex, Steve, Sebastian)
    - [x] Add Jupyter to it
    - [ ] Add francy to it (Sebastian)
    - [x] Provide fixed tags
    - [x] Provide a boiler-plate repo that shows how to use gap-docker-master on binder [gap-system/gap-docker-binder](https://github.com/gap-system/gap-docker-binder) [![Binder](https://mybinder.org/badge.svg)](https://mybinder.org/v2/gh/gap-system/gap-docker-binder/master)
    - [x] Provide a Readme

- [x] Update gap-docker-base, gap-docker, and gap-docker-master to gap-4.9.1 (Sebastian)
- [x] Included Jupyter Kernel and Jupyter(lab) in gap-docker-master (Sebastian)

     Docker image for SageMath

- [ ] Report [regression when using the new Docker image for SageMath on binder](https://trac.sagemath.org/ticket/24655#comment:225) (Nicolas)

    Refactoring of GAP's REPL (and I/O more generally)
This is needed, or at least would be helpful to make a number of the other projects easier and/or more robust.
[Some notes](https://hackmd.io/3v48t_SmSpmOjhiSd8HOAw)

    Jupyter kernel for GAP (Markus, Pedro, ...)

- [x] Multiple Bugfixes on Jupyter Kernel (Markus)
- [x] Added visjs demo to JupyterKernel repo (Pedro). It as examples to work with graphs either by nodes and edges, or using dot language.
- [x] Created [FrancyMonoids package](https://gap-packages.github.io/FrancyMonoids) (Manuel, Andrés and Pedro)
- [x] Working on francy async branch errors (Manuel and Pedro) 
- [x] Added dot functions to the [NumericalSgps package](https://gap-packages.github.io/numericalsgps) to be used either with `JupyterSplashDot` or with the provided function `DotSplash` (Pedro, Andrés) 
- [ ] Make syntax errors and warnings show in red (Markus)
- [ ] Most recent protocol (Markus)
- [x] Checked that we support latest version of Jupyter protocol (5.3 at the time of writing); We do not implement all features, but we handle the stuff we do support in a 5.3-compliant way.
- [ ] Fix to GAP's Help to make Jupyter help display work again
- [ ]  Constructors and some documentation for JupyterRenderables
- [x] Fix indenting in the syntax highlighting: [gap-packages/JupyterKernel/pull/46](https://github.com/gap-packages/JupyterKernel/pull/46)
- [ ] Consider viz.js as an alternative to command line `graphviz`: see [`DotSplash`](https://github.com/pedritomelenas/dot-numericalsgps/blob/08edd05783440f7618da57096c622efc9dd4a33d/dot.gi#L23) in dot-numericalsgps repo (Pedro, Manuel) or provide a way to choose engine in `JupyterSplashDot`
- [ ] Look at [inorder.js](https://github.com/minrk/ipython_extensions/blob/master/nbextensions/inorder.js) and see if we can use it to adress Frank's wishes (Sebastian)


- [ ]  Deposit JupyterKernel and Dependencies (uuid, crypting) with GAP for 4.9.2
    - [x] allow uuid and crypting in package archives for stable-4.9 branch (Alex)


### libgap (Markus, Max, Sebastian, Steve Nicolas, ...)
Goal: push further the integration of GAP's own libgap implementation, and experiment with it from Sage, Julia, ...
See:
- [Markus plan](https://hackmd.io/mRAwlBttRqWf8owwHTBgfQ#)
- the [notes from a recent workshop](https://annuel2.framapad.org/p/opendreamkit-cernay-2019-projects) (section GAP as a library)

- [x] extracted and analyzed the [GAP symbols used by Sage](https://hackmd.io/emNi76svSWCh1fBeLKqPdA#) (Max + Nicolas)

- [ ] try the latest libgap in Sage (Nicolas)
      get the master branch from GAP, make libgap, make some experiments with Sage

### libsemigroups


- [x] Wrote and [published on binder](https://mybinder.org/v2/gh/james-d-mitchell/libsemigroups/binder?filepath=index.ipynb) a sample notebook illustrating the interactive use of libsemigroups with cling (Nicolas)
- [x] Lots of improvements to libsemigroups (James, ...)

#### Python/Sage bindings(James + Nicolas)

In the Cernay meeting we started exploring the use of cppyy to obtain bindings for libsemigroups. The technology looks very promissing but still on the bleeding edge: it tends to segfault due to bugs in cppyy.
Bugs analyzed, reduced, and reported upstream:
- [x] [#24, explicit constructors](https://bitbucket.org/wlav/cppyy/issues/24) fixed (James)
- [x] [#28, virtual destructor, array refcounting](https://bitbucket.org/wlav/cppyy/issues/28) fixed (Nicolas)
- [x] [#30, private numbers](https://bitbucket.org/wlav/cppyy/issues/30) fixed (Nicolas)

With the fixes, a non trivial part of libsemigroups works now smoothly:
- [ ] Basic elements: transformations, ...
- [ ] Semigroups of elements preexisting types
- [ ] Semigroups of Python objects

### Notebook collection website for ODK and GAP (Sebastian, Nicolas, ...)
  Example: [OSCARExamples](https://sebasguts.github.io/OSCARDemo/)

### [FSR package](https://github.com/nzidaric/gap-fsr) (Nusa, Alex)
- [x] Set up Travis CI and CodeCov integration (Alex)
- [ ] Make a website using [GitHubPagesForGAP](https://github.com/gap-system/GitHubPagesForGAP)
- [ ] Make a release using [ReleaseTools](https://github.com/gap-system/ReleaseTools)
- [ ] Make a beta-release


### Experimenting with Azure (Alex)
- [ ] Try to use GAP in Azure Notebooks (Alex)
- [ ] Try https://docs.microsoft.com/en-us/azure/container-instances/container-instances-quickstart-portal

### Convert GAPDoc manuals into Jupyter notebooks (Frank, ...)

* How to turn the GAPDoc Books into collection of Notebooks?

### Update GAP Homebrew installer for GAP 4.9.1 (Alex)

## For the GAP developers meeting specifically
### Discuss GAP release/development cycle 
- [ ] Should we try and concentrate major disruptive developments and/or minor but wideranging changes (whitespace etc.) in particular parts of the time between major releases? (Steve)

- one result of the discussion was that we want to set a release schedule for GAP 4.10. Here is a suggestion:
  - 2018-11-01 release of GAP 4.10.1 (first user release)
  - 2018-10-01 release of GAP 4.10.0
  - 2018-09-17 prepare changelog (during GAP days)
  - 2018-09-01 branching of stable-4.10

### OpenDreamKit & Funding

- [x] [Brainstorm discussion on ODK D6.9: persistent memoization library](/1M5clex-TYWCuxxgi05k5A) (Alex, Markus, Nicolas, Steve)
- [x] [MitM brainstorm](https://hackmd.io/13i3MB8IQse2uY2lcYuHOw#) (Markus, Nicolas)
- [ ] Brainstorms about future funding opportunities (Nicolas, Pedro, Markus, Max, Alex, ...)




Quantitative metrics:
\begin{itemize}
\item Number of PyPI hosted packages for \Sage, and similarly for
  other components.
  https://github.com/OpenDreamKit/OpenDreamKit/blob/88ee2544e31e661c4eac7f39fa7333f24b62ab2a/WP3/D3.3/report.tex
  
         The Math-in-the-Middle (MitM) ontology and the system API theories in the MitM paradigm are big theory graphs with thousands of 
         nodes and edges. Understanding and interacting with such large and complex objects is very difficult.The FAU group has conducted
       research into whether virtual reality technologies are helpful for this task. We have presented a first working prototype at the
       Conference on Intelligent Computer Mathematics CICM 2018 and the author: Richard Marcus - a master's student at FAU has received a
       prize for best presentation.https://github.com/OpenDreamKit/OpenDreamKit.github.io/blob/1f175efc40146570fafa28123bed6b1198c41b00
       /_posts/2018-08-20-tgview3d.md
  
    
  The future
         Packaging (D3.10)

         Consolidate available packages, work on inter-compatibility.
         Make packages for Conda.
         Move SageMath to Python 3.
         Improve workflows for user code sharing.

         Conda packages From Conda Forge:

         Jupyter 380K downloads: https://anaconda.org/conda-forge/jupyter
         Sage 6K downloads: https://anaconda.org/conda-forge/sage

         Sage on PyPI

         80 packages in PyPI, not all Sage-related.
         Make them easily discoverable, document workflows.
         
         GAP packages code coverage: 69% (4.9) → 75% (4.10).
Freshness of GAP packages: 50% released in the last year.
SageMath on Windows: 44% (happy) Windows users.
Packaging (not counting alt. methods, such as Conda):
Arch: 50% of Jupyter users are also ODK users;
Debian: 10% of Jupyter users are also ODK users.
Medium sized VRE deployments: 20h of work, 244 LOCs.
Docker Hub: 4K-10K pulls of ODK images.

https://github.com/OpenDreamKit/OpenDreamKit.github.io/blob/af29108dd216972485a17a3444caf6eee569033a/meetings/2018-10-28-Luxembourg/ProjectReview/WP3.ipynb

KPIs and Deliverables for WP6
MitM-connected Systems: four (GAP, Sage, LMFDB, Singular)(See D6.5)
Formal MitM Ontology: 55 files, 2600 LoF, 360 commits (See D6.8)
Informal MitM Ontology: 815 theories, 1700 concepts in English, German,(Romanian, Chinese)
MitM System API Theories (GAP, Sage, LMFDB, Singular): 1.000+ Theories, 22.000 Concepts.
Multi-Site involvement of Researchers (Mobility of Researchers)
PD. Dr. Florian Rabe (Joint appointment UPSud/FAU)
Felix Schmoll Summer Internship (From JacU to St.Andrews)
Prof. Nathan Carter (Bentley Univ.) in St. Andrews (Sabbatical)
Heavy interest by the theorem proving community about MitM Ontology
Logipedia (http://logipedia.science) adopts the MitM principle of integrating (logical) systems by aligning concepts.
First ODK-external MitM “user” for the next months: Andrea Thevis, Saarbrücken 





In the OpenMath terminology, the MitM ontology acts as a set of content dictionaries (CDs) that anchor the OpenMath objects semantically. 
For communication between the systems we only need to equip them with OpenMath phrasebooks (I/O libraries for OpenMath objects). The MitM 
ontology will be semi-automatically curated and connected to "system interface ontologies" (CDs for the system objects).

The MitM Ontology can enable multiple added value services including:

    remote evaluation of mathematical expressions via SCSCP
    MONET-style service discovery via the MitM CDs.

The current state of play is that we have initial exports of system interface ontologies for three systems (the exporters are under development still, so your mileage may vary).

    SageMath (265 CDs); see them on MathHub.info
    GAP (210 CDs with 2996 symbols) see them on MathHub.info, and
    (partially) and the LMFDB.

The MitM ontology is still very much experimental.

All CDs are encoded in OMDoc/MMT, which is legal by the OpenMath2 standard. Lossful coversions to standalone OMCDs are possible, but have not been pursued at the moment.

https://github.com/OpenDreamKit/OpenDreamKit.github.io/blob/9be9fdac647d0c2c4ccb068cb75f31184e970308/_posts/2016-06-30-OM_in_ODK.md


“Computational Foundation for Python/Sage”. In the course of the deliberations in the WP6 workshops we saw a shift from the development of computational foundations and verification towards API/Interface function specifications to enable semantic system interoperability via the Math-in-the-Middle Ontology. Consequently, emphasis has changed to the generation of API Content Dictionaries (API CDs) for GAP, LMFDB and SAGE. We have a prototypical set of GAP and SAGE Content Dictionaries in OMDoc/MMT form (GAP: 218 CDs, 2996 entries; SAGE: 512 CDs, 2800 entries overall). The computational foundations exist but are rather more simple than originally anticipated. Much of the functionality has been offloaded to the SCSCP standard – remote procedure call with OpenMath representations of the mathematical objects – developed in the SCIENCE Project. As a direct consequence of the work in OpenDreamKit the OpenMath Society has promoted the SCSCP protocol into as an OpenMath Standard. 676541 OpenDreamKit 5

Conversely, the GAP and SAGE CDs are rather more elaborated than anticipated in the proposal, and thus form a viable basis for alignment with the MitM Ontology.

  https://github.com/OpenDreamKit/OpenDreamKit.github.io/blob/333b21524f753dd49e441b4b1157ded68999eeff/meetings/2018-02-01-SteeringCommitteeMeeting/ProgressReports/Zurich.md
  
  
\item Number of additional systems made interoperable with the
  Math-in-the-Middle architecture, on top of the three for the Month
  36 prototype
  % Michael: I am expecting about 2 (OEIS and Findstat) any more than that would be a success.conv: {dissem: PU, due: '24', issue: '138', label: D6.4, lead: ZH, nature: DEC,
page: '54', short: 'Conversion of existing and new Databases (  LMFDB, OEIS,   FindStat) to unified interoperable System', status: canceled, title: 'Conversion of existing and new Databases (  LMFDB, OEIS,   FindStat ) to unified interoperable System'}




\subsubsection{KPIs for Aim 4 / Dissemination / Impact}

\begin{recommendation}{Aim 4}
  Maximise sustainability and impact in mathematics, neighbouring fields, and scientific computing.
\end{recommendation}

Qualitative metrics: success stories. For example:
\begin{itemize}
\item Women in \Sage workshops;
\item Use of \Jupyter, \cocalc, ... for teaching, at Sheffield, at the
  African Institute for Mathematical Sciences, ...;
\item Stories about the impact of the Micromagnetics VRE;
\item Impact of nbdime, 3D widgets, Thebelab, ...
% D4.13 (Sphinx) might have impact beyond ODK. For example, Simon King
%is interested in things done for D4.13 for his
%\software{p\_group\_cohomology} package. It might also lead to a PEP,
%which by itself counts as impact.
\end{itemize}

 
 
 The under-representation of women in the scientific world is even
  more visible if we intersect science with software
  development. As we know, we have many talented women in our
  community, and we have organised some events targeted at women in the
  spirit of the "Women in Sage" days that happened many times in the
  US already. We are planning to have one more on 2019......

Women in Sage

    Organised by OpenDreamKit in Paris January 2017
    Another event (independent of ODK) in Montreal in June 2018
    Next ODK one: Spring/Summer 2019 in Crete

-- Women in computing

    Developed training materials and provided training for over 130 women in the last 12 months at Sheffield and Manchester in partnership with CodeFirstGirls.
    Tania Allard participated in the Diversity and Inclusion in Scientific Computing unconference by direct invite of NumFOCUS
    Tania Allard was diversity chair for the 2017 International Research Software Engineering conference.

 
 Women in sage workshops
 Last January, Viviane Pons, Jessica Striker and Jennifer Balakrishnan organized the first WomenInSage event in Europe with OpenDreamKit. 20 women spent a week together coding and learning in a rented house in the Paris area.
We took advantage of the diverse knowledge background of our group to work together and learn from each other. It was an occasion for many "first times" among participants who had very little experience with Sage:

    5 participants installed a source version of Sage for the first time (so that they could edit the source).
    3 used git for the first time.
    5 used git within Sage for the first time.
    11 got their first Trac account .
    5 got their first contribution to a Sage ticket.
    8 are in the process of getting their first code integrated to Sage.

We worked on 14 tickets during the week, 6 of those which have been merged since the conference. All participants said they had learned new things and it would impact their careers.

https://github.com/OpenDreamKit/OpenDreamKit.github.io/blob/6faf6eb2f1532f342f86c8da633078067ca40c85/_posts/2017-04-06-WomenInSage.md
Impact of nbdime; thebelab...

KPI: Usage/impact statistics (since last reporting period)

Nbdime: 855 stars on github, 64 contributors (36 in 12 months prior), 611 
comments, 239 new issues (241 closed).
Thebelab: 44 stars, 15 contributors, 151 new issues (118 closed), 323 
comments.
K3D-Jupyter: 48 stars, 14 contributors, 140 new issues (129 closed), 303 
comments.

htps://github.com/OpenDreamKit/OpenDreamKit.github.io/blob/master/meetings/2018-10-28-Luxembourg/ProjectReview/WP2.md

Highlight: OpenDreamKit kernels
Now 117 Jupyter kernels (49 when ODK started), 6 contributed to by ODK.
Further improved kernels from first reporting period (GAP, PARI, Singular). 
Delivered as D4.7, and MMT as part of WP6
Also contributed to kernels: cling (C++), SageMath
KPI: ODK Kernels on GitHub
Notebooks found on GitHub using each kernel (code):
SageMath: 6199
Xeus-cling: 684
GAP: 63
Singular: 8
PARI/GP: 3
MMT: 1
Highlight: nbdime
Further development on the nbdime project, delivered in the first period 
(D4.6). Has been met with enthusiasm, adoption in the community.
Jupyter Notebook and Jupyter Lab extensions, git integration.
Highlight: real-time collaboration
D4.15 prototype and plan for live collaboration in JupyterLab.
Optimistic about good integration during the final year of ODK.
Highlight: 3D visualization in Jupyter notebooks
D4.12: Jupyter extension for 3D vis, demonstrated with fluid-dynamics
Packages:
k3d-jupyter
ipydatawidgets
ipyscales○unray
Improved distribution
Highlight: Dynamic documentation and exploration system
Delivered D4.16 as two new packages, 
built on Jupyter widgets for interactively 
exploring objects in Sage
○Sage Combinat Widgets
○Sage Explorer
Interactive Documents
Key areas:
Active Documents
Interactive Documentation
Highlight: Active Documents Workshop
Workshop on live documents hosted in Oslo. Resulted in new package: 
thebelab, for embedding execution on any page, integrating tools from 
JupyterLab and MyBinder.org, demonstrating value of coordination.
Highlight: MathHub notebook integration
MathHub.info is a portal for active mathematical documents. As part of D4.11, a 
notebook integration with MathHub was added. This allows:
Authoring MathHub documents as a Notebook
Interactively exploring existing MathHub documents as a Notebook.
Highlight: Sage documentation
Refactorisation of SageMath’s Sphinx documentation system as part of D4.13
Improve Sphinx support for Cython projects.
○Enabled building proper documentation for fpylll, CyPari2, CySignals.
To completely enable Cython documentation out of the box, Python needs to 
be fixed. For this, we submitted PEP (Python Enhancement Proposal) 580.
Highlight: PARI bindings
Improved PARI/GP bindings delivered as D4.10
CyPari2 used to be part of SageMath, but it was made a separate package in 
D4.10 (see also D4.1). This ties into WP3.


Quantitative metrics:
\begin{itemize}
\item Statistics on workshops organized and conference presentations
  delivered as part of our dissemination activities, including
  estimates of number of attendees and what (if anything) happened as
  a follow-up.
\item Number of courses and departments we worked with directly and an
  estimate of how many students this subsequently affected.

  For example:
  \begin{itemize}
  \item Department of physics, University of Sheffield. Estimated
    number of new users: 500 undergraduate students
  \item Department of Biomedical Sciences, University of Sheffield.
    BMS353. 2015: N students. 2016: Y students
  \end{itemize}
\end{itemize}

We have been systematically collecting these metrics since the
beginning of \ODK, as part of our dissemination reports
\delivref{dissem}{workshops-1}, \delivref{dissem}{workshops-2},
\delivref{dissem}{workshops-3}, \delivref{dissem}{workshops-4}.

Over the last four years OpenDreamKit has been involved in 66 events, 31 organized/co-organized by UPSud.

                        - 21 development workshops and project meetings, among these: 14 organized or co-organized by UPSud
                        - 26 training workshops or sessions,  7 org./co-org. by UPSud
                        - 13 community building workshops, 8 org./co-org. by UPSud
                        - 6 research workshops, 2 org./co-org. by UPSud

And 19 external events ( 9 organized/co-organized by UPSud) 
















KPI: JupyterHub deployments
Local CoCalc instance at Universität Zürich.
Deployed September 2015 - February 2016.
People involved: @pdehaye, @williamstein
Instance of JupyterHub
 deployed by the 
Mathrice group
Host Infrastructure: France Grille's LAL cloud
Users: members of math labs in France
Main use case: casual use
 Local 
JupyterHub instance at Université Paris Sud / Paris Saclay
Host Infrastructure: France Grille's LAL cloud
Users: personnel and students of UPSud / Paris Saclay
Main use case: use in classroom (Python, Sage, C++), casual use
People involved: @VivianePons, @nthiery, @gouarin
JupyterHub instance deployed on USheffield's HPC system
People involved: @mikecroucher
JupyterHub instance(s) deployed at UVSQ
Main use case: use in classroom (Sage, Python, C, Apache Spark), 
casual use
People involved: @defeo
JupyterHub and Binder instances deployed on EGI 
infrastructure
Easy deployment of live GAP/SageMath/... notebooks with 
mybinder, thanks to the Docker containers (#58); 
potential alternatives: Debian packaging and Conda packaging.
People involved: @nthiery, ...
Local instance of CoCalc (using the Docker container) at the University of Gent
Main use case: teaching for mathematics students
Deployed at jupyter.mathhub.info
With MMT kernel
People involved: @tkw1536
Highlight: Simulagora
Logilab VRE deployment for application development and deployment.
Can use JupyterLab for application development, which can then be deployed 
with a simplified parameters form input.



Use of ODK technologies for teaching

\textbf{Event summary.} A group of researchers based within the Sheffield Institute for Translational Neuroscience (SITRAN, \url{http://sitran.org/}) were taught about Bioinformatics workflows using Jupyter notebooks with computation provided by the free Micosost Azure Notebook service.

\textbf{Results and impact.} The event demonstrated that \ODK supported technologies could be applied to the field of Bioinformatics and led to a new collaboration between Dr Cutillo and \ODK member Mike Croucher.

Following the success of this workshop, Dr Cutillo independently taught an introductory workshop on statistics using Jupyter notebooks on Azure at Parthenope University of Naples (Materials at \url{https://github.com/luisacutillo78/RbasicStats})

Dr Cutillo has since moved to University of Leeds where she will be teaching statistics to 200+ undergraduates. She plans to use \ODK developed technologies in collaboration with the Research Software Engineering group at Leeds.


The event required the development of a website that was linked to the Jupyter notebooks (\url{https://bitsandchips.me/BAD_days/}). The website caught the attention of Eleni Vasilaki, Head of Machine Learning at University of Sheffield who wanted to do something similar for her course on Adaptive Intelligence. We supported her in this endeavour and the result is at (\url{http://bitsandchips.me/COM3240_Adaptive_Intelligence/}).

In order to better support this, \ODK member Tania Allard, developed a Jekyll template for use by academics and researchers using Jupyter notebooks for course materials and dissemination. Such a template allows the creation of Jupyter notebooks based websites using Jekyll, which is the default static website framework supported by GitHub. It also allows for easy display of notebooks connected to cloud computing resources such as Microsoft Azure Notebooks (D2.17, T2.6).

This led to the development of a Python package: nbjekyll (\url{https://github.com/trallard/nbjekyll}) that complements the Jekyll template. This package converts Jupyter notebooks into .md files that can be readily usable by Jekyll (this uses nbformat for the conversion). It also uses \ODK-developed nbval to perform notebook validation and add custom headers indicating the last update date, version and test status of the notebook.

As well as being used internally at Sheffield, The nbjekyll package received some attention on twitter \url{https://twitter.com/jdblischak/status/1009800776305332224} and \url{https://twitter.com/walkingrandomly/status/1009414151716909057} receiving a total of 42 retweets and 80 'likes'

\end{event}

https://github.com/OpenDreamKit/OpenDreamKit/blob/88ee2544e31e661c4eac7f39fa7333f24b62ab2a/WP4/D4.8/report.tex

