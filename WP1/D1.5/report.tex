\documentclass{deliverablereport}

\deliverable{management}{ipr2}
\duedate{31/08/2018 (M36)}
\deliverydate{Unknown}

\usepackage[style=alphabetic,backend=bibtex]{biblatex}
\addbibresource{../../lib/kbibs/kwarcpubs.bib}
\addbibresource{../../lib/kbibs/extpubs.bib}
\addbibresource{../../lib/kbibs/kwarccrossrefs.bib}
\addbibresource{../../lib/kbibs/extcrossrefs.bib}
\addbibresource{../../lib/deliverables.bib}
\addbibresource{../../lib/publications.bib}
% temporary fix due to http://tex.stackexchange.com/questions/311426/bibliography-error-use-of-blxbblverbaddi-doesnt-match-its-definition-ve
\makeatletter\def\blx@maxline{77}\makeatother

\usepackage[show]{ed}
\makeatletter
%%%%%%%%%%%%%%%%%%%%%%%%%%%%%%%%%%%%%%%%%%%%%%%%%%%%%%%%%%%%%%%%%%%%%%%%%%%%%%
% Styling: adapt amsart's subsubsection macro to put a newline after the title
%%%%%%%%%%%%%%%%%%%%%%%%%%%%%%%%%%%%%%%%%%%%%%%%%%%%%%%%%%%%%%%%%%%%%%%%%%%%%%
\renewcommand\subsubsection{\@startsection{subsubsection}{2}%
  \z@{.5\linespacing\@plus.7\linespacing}{.1\linespacing}%
  {\normalfont\bfseries}}

% Variant of taskref that links to the section on the task in this file
\newcommand\localtaskref[2]{\hyperref[#1@#2]{\csname task@#1@#2@label\endcsname}}
\newcommand\longlocaltaskref[2]{\hyperref[#1@#2]{\csname task@#1@#2@label\endcsname: ``\csname task@#1@#2@title\endcsname''}}
\newcommand\longmilestoneref[1]{\textbf{\csname mile@#1@label\endcsname}:
  ``\csname mile@#1@title\endcsname''
  (month \csname mile@#1@month\endcsname)}
\makeatother

%% another level of numbered sectioning
%\usepackage{titlesec}
\setcounter{secnumdepth}{4}

% \titleformat{\paragraph}
% {\normalfont\normalsize\bfseries}{\theparagraph}{1em}{}
% \titlespacing*{\paragraph}
% {0pt}{3.25ex plus 1ex minus .2ex}{1.5ex plus .2ex}

%\usepackage{todonotes}
\author{Nicolas M. Thiéry, Benoît Pilorget, et al.}

\begin{document}
\enlargethispage{4ex}
\maketitle
\githubissuedescription
\tableofcontents\newpage

\section{Explanation of the work carried out by the beneficiaries and Overview of the progress}

In this section, we give a general overview of the progress of the
project during the third reporting period, ranging from September 2018
to August 2019. For new readers, we start by recalling some context of
\ODK's approach that is important to understand and evaluate the
progress; except for the last paragraphs, this piece of text is
unchanged from the preview reporting periods. Then we provide a brief
overview of the work carried out by objective of the project; finally
we detail the progress work package by work package.

\subsubsection*{Some context: \ODK's approach}
\ODK's approach to delivering a Virtual Research Environment (VRE) for
mathematics is not to build a monolithic one-size-fits-all VRE, but
rather a toolkit from which it is easy to set up VRE's that are
customised to specific needs by combining the appropriate components
(collaborative workspaces, user interfaces, computational software,
databases, \dots) on top of available physical resources (from
personal laptops to cloud infrastructure). This approach --- chosen by
design --- allows users to flexibly put together lean computational
environments and tools for particular research challenges. These tools
provide the required functionality but due to the component based
approach carry no unnecessary bloat that would reduce effectiveness in
terms of installation process, size, computation time, and
reproducibility.

Most of the components preexist as an ecosystem of open source
software, developed by well established communities of developers. For
example, for interactive computing and data analysis, OpenDreamKit
promotes Jupyter, a web-based general purpose flexible notebook
interface\footnote{a notebook is a document that contains live code,
  equations, visualizations and explanatory text} that targets all
areas of science. A number of Virtual Research Environment already
exist, e.g.\ powered by \cocalc (formerly \SMC) or \JupyterHub.

Hence most of the work in \ODK\ is to foster this ecosystem, improving
the components themselves and their composability. The technical work is
distributed over the work packages:
\begin{itemize}
\item \emph{Component Architecture} (\textbf{WP3}): ease of
  deployment: modularity, packaging, portability, distribution, for
  individual components and combinations thereof. sustainability of
  the ecosystem: improving the development workflows.
\item \emph{User Interfaces} (\textbf{WP4}): enable Jupyter as uniform notebook
  interface, and further improve it; foster the collaboration between
  \cocalc and JupyterHub; generally speaking investigate
  collaborative, reproducible, and active documents.
\item \emph{Performance} (\textbf{WP5}): make the most of available hardware
  (multi-core, HPC, cloud), for individual computational components and
  combinations thereof.
\item \emph{Data/Knowledge/Software} (\textbf{WP6}): enable rich and robust
  interaction between computational components, data bases, knowledge
  bases, and users through explicit common semantic spaces, a language to
  express them, and tools to leverage them.
\end{itemize}
These technical work packages are supported by%
\footnote{The project originally had another work package on
  \emph{Studies of Social Aspects} (\textbf{WP7}); following the
  formal review for reporting period 1, it was decided in agreement
  with the reviewers and advisory board to shut down the work package
  after reporting period 1, moving some of its tasks to other work
  packages, and redirecting the man power for the others to more
  central tasks.}
\begin{itemize}
\item \emph{Community Building and Dissemination} (\textbf{WP2}): developer and
  training workshops, conferences, teaching material with focus on
  making the created value accessible to a wide, varied and growing user community.
\end{itemize}

As a result of \ODK's approach, the work programme for \ODK\ consists
of a large array of loosely coupled tasks, each being useful in its
own right, and none being absolutely critical.

All three reporting periods confirmed that this is a strong
feature of \ODK's approach. Indeed, as analysed in the proposal, this
kind of project is subject to the following risks:
\begin{enumerate}
\item Recruitment of qualified personnel;
\item Different groups not forming an effective team;
\item Implementing infrastructure that does not match the needs of end-users;
\item Lack of predictability for tasks that are pursued jointly with
  the community;
\item Reliance on external software components.
\end{enumerate}
Together with ambitious software challenges, this made the accurate
prediction of workload and precise timeline of work packages
difficult, especially over a period of four years in a field of
rapidly evolving technologies.

And indeed the project actually faced each of the risks above,
especially 1, 4, and 5.
The toughest situation we encountered was the volatility of the personnel at
Sheffield, where the PIs and hired personnel all left for industry at
various stage of the project, some after an intermediate move to
Leeds. However, thanks to the flexibility enabled by
the loose coupling, those risks could be mitigated by adapting the
tasks schedule and human resources allocation, with little influence
on the general aims and objectives.

% The toughest situation we faced was the volatility of the personnel at
% Sheffield, where the PIs and hired personnel all left for industry at
% various stage of the project, some after an intermediate move to
% Leeds. Luckily, most of the work planned for this site was on
% dissemination tasks

OpenDreamKit's approach has one downside: it impedes formal
evaluation, as much for us to assess the adequacy and impact of our
tools, as for our reviewers to assess the depth and value of our
contribution. Indeed, we do not have a main well-defined product and
we capture a very diverse range of end-users and use-cases; hence
quantitative methods of evaluation of the adequacy for end users, like
satisfaction surveys, are rather elusive. In addition, when tools are
jointly developed with the community, how should one attribute the
merit of their success to the project or to the community?

In practice, this certainly added complexity to our reporting efforts
and to our reviewers efforts. It is our \emph{belief} however that
this did not impact the work itself. Indeed, co-design is intrinsic to
the by-users for-users development model of the ecosystem, most of the
participants were also end-users themselves, and we maintained deep
contact with the user community notably through our continuous
dissemination actions. Therefore informal evaluation through first
hand experience or witnessing was largely sufficient to inform the
design and execution of the project.

%  LocalWords:  Jupyter visualizations cocalc composability emph textbf taskref dissem
%  LocalWords:  dissemination-of-oommf-nb-virtual-environment oommf-python-interface hpc
%  LocalWords:  dissemination-of-oommf-nb-workshops oommf-py-ipython-attributes delivref
%  LocalWords:  oommf-tutorial-and-documentation oommf-nb-ve oommf-nb-evaluation ednote
%  LocalWords:  pythran-typing sage-paral-tree oldpart subsubsection WPtref dksbases

\subsection{Explanation of work carried out per Objective}
%List the specific  objectives  for  the  project  as  described  in  section  1.1  of  Part B   and describe  the  work  carried  out  during  the  reporting  period  towards  the  achievement  of  each listed objective. Provide clear and measurable details.
For reference, let us recall the aims of \ODK.
\begin{compactenum}[\bf {A}1\rm:]
\item \label{aim:collaboration} Improve the productivity of
  researchers in pure mathematics and applications by promoting
  collaborations based on mathematical \textbf{software},
  \textbf{data}, and \textbf{knowledge}.
\item \label{aim:vre} Make it easy for teams of researchers of any
  size to set up custom, collaborative \emph{Virtual Research
    Environments} tailored to their specific needs, resources and
  workflows. The \VREs should support the entire life-cycle of
  computational work in mathematical research, from initial
  exploration to publication, teaching and outreach.
  % and bridge the gaps between
  % code, published results, and educational material.
\item \label{aim:sharing} Identify and promote best practices in
  computational mathematical research including: making results easily
  reproducible; producing reusable and easily accessible
  software; sharing data in a semantically sound way; exploiting and
  supporting the growing ecosystem of computational tools.
\item \label{aim:impact} Maximise sustainability and impact in
  mathematics, neighbouring fields, and scientific computing.
\end{compactenum}

Those aims are backed up in our proposal by nine objectives; we now
highlight our main contributions during this reporting period toward
achieving each of them.

\begin{compactenum}[\bf O1\rm:]
\item\label{objective:framework} ``\emph{To develop and standardise an architecture
    allowing combination of mathematical, data and software components with off-the-shelf
    computing infrastructure to produce specialised \VREs for different communities.}''

  This objective is by nature multilevel; achievements include:
  \begin{itemize}
  \item Collaborative workspaces: major \JupyterHub developments,
    see~\longlocaltaskref{UI}{notebook-collab}; study and documentation of the \SMC
    architecture, see \longlocaltaskref{component-architecture}{extract-smc};
  \item User interface level: enabling \Jupyter as uniform interface for all computational
    components; see \longlocaltaskref{UI}{ipython-kernels}.
  \item Interfaces between computational or database components: short term: refactoring
    of existing ad-hoc interfaces, see \longlocaltaskref{UI}{pari-python}; long term:
    investigation of patterns to share data, ontologies, and semantics uniformly across
    components, see \longlocaltaskref{component-architecture}{interface-systems}, and
    Section~\ref{dksbases} about \WPref{dksbases}, where we report on the
    ``Math-in-the-Middle'' (MitM) paradigm for semantic system integration and non-trivial
    mathematical use cases. 
  \end{itemize}

\item\label{objectives:core} ``\emph{To develop open source core components
  for \VREs where existing software is not suitable. These components
  will support a variety of platforms, including standard cloud
  computing and clusters. This primarily addresses Aim~\ref{aim:vre},
  thereby contributing to Aim \ref{aim:collaboration}
  and~\ref{aim:sharing}.}''

At this stage, it has been possible to implement most of the required developments within
existing components or extensions thereof. New software components includes the tools
nbmerge, nbdiff and nbval (see \delivref{UI}{jupyter-test} and
\delivref{UI}{jupyter-collab}), and planetaryum (see \delivref{dissem}{ils-tool}). For the
Math-in-the-Middle paradigm for semantic system interoperability we have developed
knowledge-based Mediator based on the MMT system. 

\item \label{objective:community} ``\emph{To bring together research
  communities (e.g. users of \Jupyter, \Sage, \Singular, and \GAP) to
  symbiotically exploit overlaps in tool creation building efforts,
  avoid duplication of effort in different disciplines, and share best
  practice. This supports Aims~\ref{aim:collaboration},
  \ref{aim:sharing} and~\ref{aim:impact}.}''

  We have organized or co-organized a dozen users or developers
  workshops (see~\longlocaltaskref{dissem}{devel-workshops}) which brought
  together several communities. Some key outcomes include:
  \begin{itemize}
  \item Enabling \Jupyter as uniform interface for all computational
    components; see \longlocaltaskref{UI}{ipython-kernels}.
  \item Sharing best practices for development, packaging, building
    containers
    (see~\longlocaltaskref{component-architecture}{mod-packaging}),
    and continuous integration
    (see~\longlocaltaskref{component-architecture}{portability});
  \item A smooth collaboration between \JupyterHub, \SMC, and \Simulagora;
    see~\longlocaltaskref{component-architecture}{extract-smc},
    \longlocaltaskref{component-architecture}{simulagora-dev} and
    Section~\ref{infrastructures};
  \item Work on interfaces between systems; see
    \longlocaltaskref{component-architecture}{interface-systems}, \longlocaltaskref{UI}{mathhub},
    and \longlocaltaskref{UI}{pari-python};
    % \item Steps toward \longlocaltaskref{UI}{sage-sphinx}
  \item Sharing of best practices and tools for authoring live structured
    documents (see~\longlocaltaskref{UI}{structdocs});
  \item Sharing of best practices when using VRE's like \cocalc or \Jupyter for research and
    education;
  \item Collaboration on interactive visualization
    \longlocaltaskref{UI}{vis3d}, \longlocaltaskref{UI}{cfd-vis},
    \longlocaltaskref{UI}{dynamic-inspect}.
  \end{itemize}

\item \label{objective:updates} ``\emph{Update a range of existing open source
  mathematical software systems for seamless deployment and efficient
  execution within the VRE architecture of objective~\ref{objective:framework}.
  This fulfils part of Aim~\ref{aim:vre}.}''

Achievements include:
\begin{itemize}
  \item Continuous efforts of development, release and integration within \Sage
    have been put for
    \begin{itemize}
    \item  the linear algebra computational kernels of LinBox,
    fflas-ffpack and Givaro (Deliverable~\longdelivref{hpc}{LinBox-algo})
    \item the PARI library for computational number theory
    (Deliverable~\longdelivref{hpc}{pari-hpc2} still ongoing)
    \item the GAP software for computational group theory
    (Deliverable~\longdelivref{hpc}{GAP-HPC-report} still ongoing)
  \end{itemize}
\item Packaging efforts: docker containers (delivered and regularly
  updated), Debian and Conda packages (beta); see
  \longlocaltaskref{component-architecture}{mod-packaging}.
\item Continued efforts on portability of \Sage and its dependencies
  (see \longlocaltaskref{component-architecture}{portability}, in
  particular \delivref{component-architecture}{portability-cygwin}).
\item Improved continuous integration and development workflow;
  (see~\longlocaltaskref{component-architecture}{workflow}), and
  \longdelivref{component-architecture}{multiplatform-buildbot}.
\item Integration of all the relevant mathematical software in the
  uniform \Jupyter user interface, in particular for integration in
  the VRE framework (delivered, ongoing); see
  \longlocaltaskref{UI}{ipython-kernels}.
\item Ongoing work in \WPref{hpc} to better support HPC in the
  individual mathematical software system and combinations thereof;
  see Section~\ref{hpc}.
\end{itemize}

\item \label{objective:sustainable} ``\emph{Ensure that our ecosystem of
  interoperable open source components is \emph{sustainable} by
  promoting collaborative software development and outsourcing
  development to larger communities whenever suitable. This fulfils
  part of Aims~\ref{aim:sharing} and~\ref{aim:impact}.}''

Achievements include:
\begin{itemize}
\item Continued work on outsourcing the computational system user
  interfaces by migrating to \Jupyter; see \longlocaltaskref{UI}{ipython-kernels};
\item Refactoring \Sage's documentation build system to contribute many local developments
  upstream (\Sphinx) \longlocaltaskref{UI}{sage-sphinx};
\item Outsourcing and contributing upstream as \Python bindings the existing \Sage
  bindings for many computational systems; see \longlocaltaskref{UI}{pari-python}.
\end{itemize}

\item \label{objective:social} ``\emph{Promote collaborative mathematics and
  science by exploring the social phenomena that underpin these
  endeavours: how do researchers collaborate in Mathematics and
  Computational Sciences?  What can be the role of \VREs?  How can
  collaborators within a VRE be credited and incentivised? This
  addresses parts of Aims~\ref{aim:sharing}, \ref{aim:collaboration},
  and~\ref{aim:vre}.}''

This objective was the social science research side of
\WPref{social-aspects}. Following the work plan revisions after
Reporting Period 1, the manpower originally allocated to this
objective was reallocated to other objectives. There thus was no new
achievements in Reporting Period 2.

\item \label{objective:data} ``\emph{Identify and extend ontologies and
  standards to facilitate safe and efficient storage, reuse,
  interoperation and sharing of rich mathematical data whilst taking
  account of provenance and citability. This fulfills parts of
  Aims~\ref{aim:vre} and~\ref{aim:sharing}.}''
 
This objective is at the core of \WPref{dksbases}; see Section~\ref{dksbases} for details.
In the first two reporting periods \WPref{dksbases} has developed the Math-in-the-Middle ontology that acts as the pivot point mediating between system languages in the MitM interoperability framework.
This work has been reported in deliverables \delivref{dksbases}{psfoundation} and \delivref{dksbases}{lfmverif}.

In the third reporting period the focus of \WPref{dksbases} has been on instantiating the FAIR principles for mathematics (we call the result \textbf{deep FAIR}) and turning mathematical datasets into deep FAIR VRE components.
This work has been reported in \delivref{dksbases}{nbad-search}.
The \dmh system, which implements deep FAIR datasets from scratch meets exactly the objectives stated above -- but the system is still very young and needs to attract a critical mass of datasets and community.
The LMFDB system which has both has been retrofitted with aspects of deep FAIR in \pn, and is much more interoperable than at the start of \pn. 

\item \label{objective:demo} ``\emph{Demonstrate the effectiveness of Virtual
  Research Environments built on top of \ODK components for a
  number of real-world use cases that traverse domains. This addresses
  part of Aim~\ref{aim:vre} and through documenting best practices in
  reproducible demonstrator documents Aim~\ref{aim:sharing}.}''

Most of the work toward this objective is by nature planned for the last period of the \pn
project. Nevertheless, work has started e.g.  toward the OOMMF demonstrator; see
\longlocaltaskref{dissem}{dissemination-of-oommf-nb-virtual-environment}
\longlocaltaskref{dissem}{dissemination-of-oommf-nb-workshops},
\longlocaltaskref{component-architecture}{oommf-python-interface}.

%Long term sustainability
\item \label{objective:disseminate} ``\emph{Promote and disseminate
  \ODK to the scientific community by active communication,
  workshop organisation, and training in the spirit of open-source
  software. This addresses Aim~\ref{aim:impact}.}''

This objective is at the core of \WPref{dissem}, with in particular
more than 30 meetings, developer, training, and community building
workshops organized during the second reporting period. See
Section~\ref{dissem} and \longdelivref{dissem}{workshops-3} for
details.
\end{compactenum}

%%% Local Variables:
%%% mode: latex
%%% mode: visual-line
%%% fill-column: 5000
%%% TeX-master: "report"
%%% End:

%  LocalWords:  compactenum textbf aim:vre emph JupyterHub longtaskref notebook-collab Simulagora mathhub visualization cfd-vis fflas-ffpack Givaro LinBox-algo multiplatform-buildbot nbad-search dmh
%  LocalWords:  extract-smc Jupyter localtaskref ipython-kernels dksbases WPref dksbases
%  LocalWords:  nbmerge nbdiff nbval delivref delivref jupyter-collab dissem organized
%  LocalWords:  cocalc portability-cygwin hpc taskref citability fulfills
%  LocalWords:  dissemination-of-oommf-nb-virtual-environment oommf-python-interface
%  LocalWords:  dissemination-of-oommf-nb-workshops

\subsection{Explanation of the work carried per work package}\label{sec:achievements}
We will now tabulate and explain the achievements of the OpenDreamKit project by the work packages.

\addtocounter{wpno}{1}

\begin{Workpackage}{\thewpno}
\WPTitle{\wpname{\thewpno}}
\WPStart{Month 1}
\WPParticipant{SA}{60}

\begin{WPObjectives}
The objectives of \theWP{} are to undertake all project management activities, including ...

\end{WPObjectives}

\begin{WPDescription}
This workpackage will perform ...
\end{WPDescription}

\begin{WPDeliverables}
\begin{itemize}
\item
\ref{mgt:mailinglists}
(Month 1): 
Internal and external mailing lists.
\item
\ref{mgt:swrepository}
(Month 1): 
Internal software repository.
\item
\ref{mgt:periodic-rep-1}
(Month 12): 
Project Periodic Report (first year).
\item
\ref{mgt:periodic-rep-2}
 (Month 24): 
Project Periodic Report (second year).
\item
\ref{mgt:periodic-rep-3}
(Month 36): 
Project Periodic Report (third year).
\item
\ref{mgt:periodic-rep-4}
(Month 48): 
Project Periodic Report (fourth year).
\item
\ref{mgt:final-mgt-rep}
(Month 36): 
Project Final Report
\end{itemize}
\end{WPDeliverables}
\end{Workpackage}
\newpage
\subsubsection{WorkPackage 2:  Community Building, Training, Dissemination, Exploitation, and Outreach}
\label{dissem}
%Explain, task per task, the work carried out in WP during the reporting period giving details of the work carried out by each beneficiary involved.

%%%%%%%%%%%%%%%%%%%%%%%%%%%%%%%%%%%%%%%%%%%%%%%%%%%%%%%%%%%%%%%%%%%%%%%%%%%%%%
\paragraph{Overview}

We continued the line of work from the previous reporting periods,
especially on \longlocaltaskref{dissem}{dissemination-communication}
and \longlocaltaskref{dissem}{dissemination}: we organized or
participated in about 30 events throughout the last year, including
the large dissemination conference \emph{Free Computational
  Mathematics} at CIRM in February 2019. Over the whole project, this
accumulates to 110 events.

We have also pursued our effort towards greater diversity in the
open-source community, organizing the first ever \Sage workshop in
Nigeria as well as our second Women in Sage event in Archanes, Crete.

About our communication, we continued our work to reach the community
through multimedia. We produced comics with Juliette Belin from
Logilab and short motion graphics videos with the Pix Videos company
to explain the project and common use cases. This self-explanatory
visual material reinforces our website to keep informing the community
about the project long after it is officially over.

%%%%%%%%%%%%%%%%%%%%%%%%%%%%%%%%%%%%%%%%%%%%%%%%%%%%%%%%%%%%%%%%%%%%%%%%%%%%%%
\paragraph{Tasks}

\subparagraph{\longtaskref{dissem}{dissemination-communication}}
\label{dissem@dissemination-communication}

 Press Releases were considered an important dissemination and communication tool at the start of the project and will also be at the end. During the  first year, the project has covered six press releases describing the general goals of the project. To promote \ODK innovative method and highlight its results to the general public, we plan to submit press releases at the beginning of November, after the Final review meeting. The procedure for the press release production and distribution is still under revision. The text proposal was made available to all the partners  inviting them to finalize its publication through their press offices. The final press releases will be published in French, the Coordinator’s and 3 partners main language, but also translated in English for the others beneficiaries to enable its publication in local media. We also plan to send this article proposal to our European communication officer to publish it in the EC newsletter and submit it for publication in the Horizon magazine.  These Press releases will  be  addressed  to   the general press in the high education, research area but also in local press, to audiences that do not require a detailed knowledge of the work carried out.

 Beside that, to raise interest of the scientific community on the project topic and its impact, our communication strategy was accompanied by audio-visually enhanced materials targeted at non-specialist. With the help of Pix Videos, we created several explainer comics and life motion-design videos based on the sketches by Juliette Belin from Logilab. These multimedia creations describe different common use-cases of tools either directly developed by \ODK or that can be used in conjunction with our software. As an example, it describes how to use Binder (an external tool) and \Jupyter (developed by \ODK) in the context of scientific collaboration. We plan to use those videos and comics to promote open source software now that the project itself is ending.



\subparagraph{\longtaskref{dissem}{training-portal}}


The website for the project has been continuously updated with new content, and virtually all work in progress was openly accessible. It  became  a  repository  for  a  wide  type  of information and communication material. It was used to update on new technical results, and events that might be of interest for our targeted communities, and also to help to share the demos experience, facilitate adoption of project results by the users, to support best practices, \emph{etc.}.
Some of our ``Use Cases`` were illustrated by comics designed by
Juliette Belin, giving a quick overview on how to use some of \ODK
tools.

\subparagraph{\longtaskref{dissem}{devel-workshops}}
\label{dissem@devel-workshops}

Development workshops are a key aspect of OpenDreamKit development model. The aim of these workshops is to bring together developers from the different communities to design and implement some
of the wanted features such as user interface, and documentation and to ensure cross compatibility.
As reported in \longdelivref{dissem}{workshops-4}, we have organized
or co-organized 7  of these workshops throughout year 4 of the project. The thematic varies
for each event: \PariGP, \Linbox, Data, and cross-thematic events such as \GAP-\Sage and \GAP-\Singular days. They were aimed at a specific software components to improve joint developments. It fostered collaboration between scientists and developers from different backgrounds to build tools that are needed by all. These workshops were essential in order to disseminate our work while improving it.

\subparagraph{\longtaskref{dissem}{tech-review}}

By nature, most of the work on this task occurred in the earlier
reporting periods, especially through \longdelivref{dissem}{techno}.
Of course, we continued keeping track of new technologies and writing
about them on our website.

\subparagraph{\longtaskref{dissem}{dissemination}}
\label{dissem@dissemination}

During Reporting Period 3, we organized 10 more training workshops on
various components of \ODK such as \Sage, \Jupyter, \GAP, Ubermag and
more, to disseminate them to the scientific community. Other the four
years of OpenDreamKit, this accumulate to 45 training events and about
1800 attendees.


One of our main dissemination event was the CIRM conference \emph{Free Computational Mathematics}, in Marseille which was aimed at the general scientific community.  It was an occasion to showcase  many of the tools developed and supported by \ODK and to promote the spirit of collaboration, free software, and best practices. We had 58 participants mixing different level of expertise, from newcomers to advanced developers, and different software communities (\GAP, \Jupyter, \Linbox,  \MPIR, \PariGP, \Sage, \Singular). Another important dissemination event was the Sage Days 105, organized as a satellite event to the main yearly international conference  on algebraic combinatorics involving 50 participants. It featured several tutorial demos and presentations by \ODK participants including best practices.

We pursued our effort towards better diversity in the open source software community. We widely advertised our
  \emph{Free Computational Mathematics} conference to reach a large audience and provided funding to many attendees.
  This allowed in particular the attendance of three researchers from University of Ibadan, Nigeria. They were very enthusiastic
  about the conference and it was then decided to organize a \Sage workshop directly in Ibadan for the benefit of Nigerian and
  West-African mathematical community. This event happened in July 2019 and welcomed 80 participants, mostly from Nigeria and
  neighboring countries.

  We also had another \emph{Women in Sage} event, following the one we organized in Paris in 2017. The event was co-organized
  by Viviane Pons from \ODK and Eleni Tzanaki, who attended the 2017 Women in Sage workshop. It was held in the village of Archanes
  in Crete. We welcomed 22 women from 8 different countries, many of them \Sage beginners. The organization of the event was also
  an occasion for a series of \Sage lectures at the mathematics department of University of Crete which initiated the inclusion of \Sage
  in the students curriculum.

All the material we developed for presentation at all events organized throughout the project were made publicly available. The impact of development and training workshops was the awareness rising of project results and of the possibilities to strengthen our collaborative open source development model.
\smallskip
\subparagraph{\longtaskref{dissem}{project-intro}}

Training and disseminating to Researchers and Teachers is at the heart
of OpenDreamKit and the participants doubled up their efforts during
the last reporting period. This included the organization of training
events (see \longtaskref{dissem}{dissemination} above), but also many
more evaluation and dissemination activities: teaching with
OpenDreamKit technology (thereby training students and other
instructors alike), local consulting, contributing course material,
templates and utilities. This is reported on in
\longdelivref{dissem}{IntroODK}, together with some reflection on the
lessons learned at the occasion of these activities: adoption,
adequateness for the needs, best practice.

It should be noted that Sheffield (now Leeds) has been the lead on
this task until its participants got compelling opportunities in the
industry in Fall 2018. This did not reduce the overall dissemination
activities of the project: indeed, the freed resources were
redistributed to other participants that were eager to organize more
activities than originally planned. There was some impact however:
with continued leadership some more of the lessons learned at the
occasion of those activities could have been formally collated, when
currently many are in the state of shared folklore. Luckily this
information is still spreading in the community through many channels:
informal discussions, blog posts, mailing lists, etc.

\smallskip
\subparagraph{\longtaskref{dissem}{dissemination-of-oommf-nb-virtual-environment}}
\label{dissem@dissemination-of-oommf-nb-virtual-environment}

This task was mostly carried out during the first reporting period. The Ubermag
(previously called JOOMMF) project is working and available on GitHub
(\href{https://github.com/ubermag}{Ubermag repo}). For each Ubermag
package we use continuous integration on both Travis CI and AppVeyor,
where we perform tests and monitor the test coverage, which we then
make available on \href{https://codecov.io/}{Codecov}. Documentation
for each package consists of APIs (automatically generated from the
code) and different tutorials created in Jupyter notebooks. Both of
them are tested on Travis CI. Documentation is built and made publicly
available on \href{http://discretisedfield.readthedocs.io}{Read the
  Docs}. After every major milestone, we upload each package to the
Python Package Index repository and build a Conda package, which can
later be easily installed on different operating systems. We encourage
the early use of our software and invite for feedback for which we
provide several different communication channels. Ubermag can also be
used in the cloud as a Virtual Research Environment, by using Binder
services.

\smallskip
\subparagraph{\longtaskref{dissem}{dissemination-of-oommf-nb-workshops}}
\label{dissem@dissemination-of-oommf-nb-workshops}

We had several workshops and tutorials during major events where we demonstrated the use of our Micromagnetic VRE, received feedback and feature requests from the community:

\begin{compactitem}
\item IOP Magnetism in April 2017, univ. of York.
\item Intermag in April 2017, Dublin.
\item MMM in November 2017, Pittsburgh.
\item Advances in Magnetism in February 2018, Italy.
\end{compactitem}

\smallskip
\subparagraph{\longtaskref{dissem}{ibook}}
\label{dissem@ibook}

In \longdelivref{dissem}{ibook3c} we report on the delivery of two new
open interactive textbooks. Together with the two books delivered
during RP2 (\longdelivref{dissem}{ibook1}), this was the occasions to
explore various approaches to exploit OpenDreamKit technology for
authoring textbooks. In the deliverable report, we reflect on their
respective merits and suggest some best practice.

\smallskip
\subparagraph{\longtaskref{dissem}{index-librorum-salvificorum}} Not
applicable for this period.  The web toolkit \textit{planetaryum}
(\delivref{dissem}{ils-tool}) has been delivered in the 2nd reporting
period, closing the task.


%%% Local Variables:
%%% mode: latex
%%% TeX-master: "report"
%%% End:

%  LocalWords:  subsubsection dissem longtaskref organized co-organized longdelivref emph
%  LocalWords:  Jupyter compactitem dissemination-of-oommf-nb-virtual-environment Piwik
%  LocalWords:  delivref centralized cocalc Cython Pythran textbf organizing EuroScyPy
%  LocalWords:  nbgrader nbgrader specialized Codecov joommf-news Micromagnetic Intermag
%  LocalWords:  dissemination-of-oommf-nb-workshops Fruehjahrstagung Sagecell taskref
%  LocalWords:  adcomp index-librorum-salvificorum
\newpage
\subsection{WorkPackage 3:  Component Architecture}
%Explain, task per task, the work carried out in WP during the reporting period giving details of the work carried out by each beneficiary involved.

%%%%%%%%%%%%%%%%%%%%%%%%%%%%%%%%%%%%%%%%%%%%%%%%%%%%%%%%%%%%%%%%%%%%%%%%%%%%%%
\subsubsection{Overview}

This Work Package focuses on the structure of the components that make
up a mathematical software and their interactions. Such components can
be separate modules inside a unique software, or separate softwares
interacting through library calls and/or through APIs.

The latest reporting period has focused mainly on improving
development workflows and user experience, in particular targeting
notoriously ``difficult'' platforms such as Windows.

%%%%%%%%%%%%%%%%%%%%%%%%%%%%%%%%%%%%%%%%%%%%%%%%%%%%%%%%%%%%%%%%%%%%%%%%%%%%%%
\paragraph{Milestones} Helping end users perform computations on
whatever hardware they possess is one of the major goals of
OpenDreamKit, and of WP3 in particular. The only milestone involving
WP3 is

\subparagraph{\longmilestoneref{component-architecture-distribution}}

\emph{“User story: users shall be able to easily install ODK's
    computational components on the three major platforms (Windows,
    Mac, Linux) via their standard distribution channels.”}

  With the completion of
  \longdelivref{component-architecture}{portability-cygwin}, all
  OpenDreamKit components now run on Windows. Packages for the major
  Linux distributions (Debian, Ubuntu, Fedora, Arch, ...) have also
  been available for at least a year, thanks to the efforts of the
  community\footnote{Note that the role of OpenDreamKit is to
    facilitate packaging for Linux distributions, by simplifying
    dependency management and build chains, and keeping up to date
    with dependencies. It is not OpenDreamKit's goal to directly take
    the lead on packaging for the dozens of available distributions,
    as this would not be sustainable. We will keep monitoring the
    status of Linux packages, prioritizing more popular distributions
    such as Ubuntu, and continue our efforts to make our components
    easy to package.}. MacOS binaries are regularly released, albeit
  with the usual hiccups typical of the Apple ecosystem.

  The bulk of the milestone is thus completed, ahead of schedule,
  although there is still work to do, the most important items on the
  agenda being better continuous integration, and support for Python 3
  in SageMath\footnote{Python 3 support is vital for SageMath going
    into the year 2020, when Python 2 will be officially
    deprecated.}. We will focus on these items for the remaining year.
  
%%%%%%%%%%%%%%%%%%%%%%%%%%%%%%%%%%%%%%%%%%%%%%%%%%%%%%%%%%%%%%%%%%%%%%%%%%%%%%
\subsubsection{Tasks}

  \paragraph{\longtaskref{component-architecture}{portability}}
  \label{component-architecture@portability}
  The first task of this workpackage is to improve the portability of
  computational components.

  The most challenging target is the Windows platform, and indeed
  SageMath has not had native Window support for years. With the
  completion of
  \longdelivref{component-architecture}{portability-cygwin}, we are
  happy to announce Windows support for SageMath: since version 8.0,
  released in July 2017, a one-click installer based on Cygwin is the
  recommended way to install SageMath on Windows. With this
  deliverable, we achieved Windows support for 100\% of OpenDreamKit's
  components.

  In support of developing and maintaining OpenDreamKit's software on
  all platforms, we have also worked on infrastructures for continuous
  integration. No unique solution was found that could accommodate the
  needs of every project, however, through sharing information and
  experience returns, each of the software projects inside
  OpenDreamKit has managed to put in place the infrastructure better
  suited for its needs, leveraging various popular technologies such
  as Docker, Jenkins, etc. These efforts have been reported in
  \longdelivref{component-architecture}{multiplatform-buildbot}.

  \paragraph{\longtaskref{component-architecture}{interface-systems}}
  \label{component-architecture@interface-systems}
  In this task we investigate patterns to share data, ontologies,
  and semantics across computational systems, possibly connected
  remotely.

  The work concerning this work package was essentially completed in
  Year 1, through
  \href{http://www.symbolic-computing.org/science/index.php/SCSCP}{Symbolic
    Computation Software Composability Protocol (SCSCP)}. All
  subsequent planned work has been moved to WP6.

  
  \paragraph{\longtaskref{component-architecture}{mod-packaging}}
  \label{component-architecture@mod-packaging}
  In this task we investigate best practices for composing, sharing
  and interfacing computational components and data for connected
  mathematical systems.

  The main deliverable in this task is
  \delivref{component-architecture}{sage-distribution}, due in month
  48. Thanks to the joint efforts of OpenDreamKit and of the
  community, SageMath is now available as a Debian package, and
  recently also as a Conda package.

  This task is progressing as planned, and we expect to successfully
  complete it next year.

  \paragraph{\longtaskref{component-architecture}{simulagora-dev}}
  The goal of this task is to deliver every six months a new Simulagora
  VM image containing all the software components released over the
  period.

  To this date, five OpenDreamKit VMs have been released in
  Simulagora. The latest version, released in March 2018,
  showcases virtual desktops available from a web browser and
  collaboration workflows based on ``tools'' that can be described as
  micro web applications that require very little development skills
  to set up, but make it easy to make available complex simulations to
  users.
  
  \paragraph{\longtaskref{component-architecture}{component-for-HPC}}
  Not applicable for this period.

  \paragraph{\longtaskref{component-architecture}{extract-smc}}
  \label{component-architecture@extract-smc}
  Recall \cocalc used to be called \SMC at the beginning of this project.
  This task has been terminated early due to the cancellation of
  \longdelivref{component-architecture}{personal-smc}, achieved by the
  \cocalc developers \emph{before the start of \ODK}.

  The resources planned for this task were diverted to other work
  packages.
  
  \paragraph{\longtaskref{component-architecture}{workflow}}
  This task seeks new ways of accepting contributions to mathematical
  software in a scalable way.

  Deliverable \longdelivref{component-architecture}{smc-trac} was
  considerably reshaped to take into account the recent developments
  in the ecosystem. This caused a one year delay in the delivery.

  Thanks to the work done, the entry barrier for developing SageMath
  has been considerably lowered. It is now possible for users with a
  GitHub or GitLab account to contribute to SageMath without having to
  go through a manual (and slow) registration process, and editing
  documentation is now easier then ever, even for the inexperienced
  user.

  With the delivery of
  \delivref{component-architecture}{smc-trac}, this task is now
  complete.

  \paragraph{\longtaskref{component-architecture}{oommf-python-interface}}
  \label{component-architecture@oommf-python-interface}
  Not applicable for this period.

  
%%% Local Variables:
%%% mode: latex
%%% TeX-master: "report"
%%% End:

%  LocalWords:  subsubsection longmilestoneref emph longdelivref portability-cygwin
%  LocalWords:  prioritizing longtaskref multiplatform-buildbot Composability delivref
%  LocalWords:  simulagora-dev Simulagora extract-smc cocalc personal-smc smc-trac
%  LocalWords:  oommf-python-interface
\newpage
\subsubsection{WorkPackage 4: User Interfaces}
%Explain, task per task, the work carried out in WP during the reporting period giving details of the work carried out by each beneficiary involved.

%%%%%%%%%%%%%%%%%%%%%%%%%%%%%%%%%%%%%%%%%%%%%%%%%%%%%%%%%%%%%%%%%%%%%%%%%%%%%%
\paragraph{Overview}

The objective of WorkPackage 4 is to provide modern, robust, and flexible user interfaces for
computation, supporting real-time sharing, integration with collaborative problem-solving,
multilingual documents, paper writing and publication, links to databases, etc. This work is focused primarily around the \Jupyter project, in the form of:

\begin{itemize}
    \item Enhancing existing \Jupyter tools (\localtaskref{UI}{notebook-collab})
    \item Building new tools in the \Jupyter ecosystem (\localtaskref{UI}{notebook-verification}, \localtaskref{UI}{notebook-collab}, \localtaskref{UI}{vis3d})
    \item Improving the use of \ODK components in \Jupyter and \Sage environments (\localtaskref{UI}{ipython-kernels}, \localtaskref{UI}{sage-sphinx}, \localtaskref{UI}{dynamic-inspect}, \localtaskref{UI}{pari-python})
    \item Demonstrating effectiveness of WorkPackage 4 results in specific scientific applications (\localtaskref{UI}{cfd-vis}, \localtaskref{UI}{oommf-py-ipython-attributes}, \localtaskref{UI}{oommf-nb-ve}, \localtaskref{UI}{oommf-tutorial-and-documentation})
    \item Work on Active Documents, which have some goals in common with \Jupyter notebooks (\localtaskref{UI}{structdocs}, \localtaskref{UI}{mathhub})
\end{itemize}

All deliverables for WorkPackage 4 have been delivered and highly successful in previous reporting periods.
There are no new deliverables in Reporting Period 3.
However, the work of software is never really complete.
Work has continued on some tasks to further improve,
mature, and maintain the results of WorkPackage 4
toward sustainability and to best serve \ODK objectives
based on feedback from \ODK and the wider user community.

%%%%%%%%%%%%%%%%%%%%%%%%%%%%%%%%%%%%%%%%%%%%%%%%%%%%%%%%%%%%%%%%%%%%%%%%%%%%%%
\subparagraph{Milestones}

\subparagraph{\longmilestoneref{UI-vre}}

\emph{“The prototype VRE shall be extended with improved ease of deployment, new
  functionality such as interactive 3D visualization and real-time
  collaboration, enabling researchers to collaborate productively in a shared
  computational environment. Finally, integrating notebooks and semantic
  knowledge into a publication / knowledge system enable a continuous process
  of leveraging \ODK components from research to publication.”}


The \Jupyter-based prototype for this has been previously delivered in \longmilestoneref{UI-vre-prototype},
and is extended in \longtaskref{UI}{notebook-collab} to more mature functionality.

WorkPackage 4 has resulted in a number of useful pieces of software
for mathematical researchers,
sometimes creating new software,
improving existing software,
or establishing new or improved connections between two existing systems.

Combining the above, Milestone~\longmilestoneref{UI-vre} has
been reached:
from the obtained toolkit, we can produce a \Jupyter-based VRE,
integrating \ODK components.
The Jupyter kernels delivered in \localtaskref{UI}{ipython-kernels}
enable access to a broader collection of mathematical software.
The interactive utility of software such as \Pari is improved in \localtaskref{UI}{pari-python},
and general interactivity and exploration of mathematical objects in \Sage is improved in \localtaskref{UI}{dynamic-inspect}.
The scope of what classes of work can be made interactive is increased
by the development of interactive three-dimensional visualization tools in \localtaskref{UI}{vis3d}.
Further, the process of collaboration on notebook documents is improved by \localtaskref{UI}{notebook-collab}
and prototype support for live collaboration with \localtaskref{UI}{notebook-collab}.
By focusing on \Jupyter as our User Interface of choice,
all of these tools can be combined in a single VRE,
hosted in the cloud or and made accessible to any researcher,
building on the Docker images created in \longdelivref{component-architecture}{virtual-machines}.

The work in this final reporting period has focused on stabilising and maturing the software delivered in previous periods.

%%%%%%%%%%%%%%%%%%%%%%%%%%%%%%%%%%%%%%%%%%%%%%%%%%%%%%%%%%%%%%%%%%%%%%%%%%%%%%
\paragraph{Tasks}

\subparagraph{\longtaskref{UI}{ipython-kernels}}
\label{UI@ipython-kernels}

All deliverables for this task have been delivered in previous reporting periods.

Kernels for \ODK components \GAP, \Pari, \Sage, and \Singular,
had been delivered in the form of \delivref{UI}{ipython-kernels-basic}
in RP1 and \longdelivref{UI}{ipython-kernels} in RP2.
Work has continued to develop these kernels in this reporting period
to bring them to further maturity and sustainability.

\smallskip
\subparagraph{\longtaskref{UI}{notebook-collab}}
\label{UI@notebook-collab}

All deliverables for this task have been delivered in previous reporting periods.

Prototype components and plan for \delivref{UI}{jupyter-live-collab} had been delivered in RP2.
This has been developed to further complete prototypes of real-time collaboration in JupyterLab in collaboration with the \Jupyter community.
We are optimistic about its completion and adoption in JupyterLab in the near future.
Real-time collaboration has proven to be the largest and most challenging
effort in WP4,
both in terms of technical effort and in community engagement.
The reason being that real-time collaboration needs extensive work
in development in the core of JupyterLab itself,
which required collaboration and coordination with the JupyterLab community for assembling plans and implementation,
aligning with other goals of the JupyterLab project,
including development of new features in the phosphorjs framework on with JupyterLab is based,
and a complete refactor of the JupyterLab data model.
This work has involved participation in workshops and meetings,
as well as addition of \ODK team members to the core JupyterLab team.
As of August 2019, real-time collaboration has been implemented in JupyterLab in a \texttt{datastore} branch on the official jupyterlab repository on GitHub,
and is expected to arrive in a public release of JupyterLab soon.

In addition, further releases of \texttt{nbdime} from \delivref{UI}{jupyter-collab} have been made.

This work furthers \ODK objective 5 of promoting sustainable software in math and science.


\smallskip
\subparagraph{\longtaskref{UI}{notebook-verification}}
\label{UI@notebook-verification}

All deliverables for this task have been delivered in previous reporting periods.

\longdelivref{UI}{jupyter-test} was delivered in the form of a new Python package, \texttt{nbval},
which enables testing and verification of existing notebooks via a plugin to the Python testing
framework \textbf{pytest}.
In this reporting period, nbval has received further activity and contributions and new releases.
nbval integrates with nbdime from \delivref{UI}{jupyter-collab} to deliver
testable, reproducible notebooks via traditional software development testing practices.
This work furthers \ODK objective 5 of promoting sustainable software in math and science.

\smallskip
\subparagraph{\longtaskref{UI}{sage-sphinx}}
\label{UI@sage-sphinx}

%%% Updated for RP3 by Jeroen Demeyer %%%
Even though this reporting period contains no explicit deliverables
for this task, significant foundation work was carried out which we
now describe. Documentation tools such as Sphinx rely on introspection
to harvest the documentation out the sources. For performance, a large
fraction of the SageMath sources is however written in Cython
(compiled Python) which, until recently, had an incompatible and
limited introspection API. This forced SageMath and other projects to
maintain bespoke and fragile Sphinx extensions to harvest their
documentation.

Tackling this required to dig deep into the system and design,
implement, and get accepted a change to Python itself: PEP (Python
Enhancement Proposal) 590. PEP 590 makes available Python's fast
calling protocol to custom code, thereby enabling full support for
introspection and documentation to Python functions implemented in C
-- e.g. Cython functions --, with no performance loss. This has been
implemented in the upcoming Python~3.8 and Cython~3.0 releases. We
expect not only Cython and therefore SageMath to benefit from this,
but also other similar projects such as Pythran or Numba.

\smallskip
\subparagraph{\longtaskref{UI}{dynamic-inspect}} Due M36 (\delivref{UI}{ipython-advanced-interacts})
\label{UI@dynamic-inspect}

All deliverables for this task have been delivered in previous reporting periods.

As planned in \delivref{UI}{ipython-advanced-interacts}, \ODK
packages \emph{Sage-Combinat-Widgets} and \emph{Sage-Explorer} were
further developed during RP3.
%
%In versions 0.5.0 to 0.7.6,
\emph{Sage-Combinat-Widgets} has gained in
flexibility and has been applied to a range of new mathematical
objects. User interfaces features like feedback have been enhanced,
and documentation has been augmented and gained a tutorial.
%
%With version 0.5.0,
\emph{Sage-Explorer} has gone through a complete new design and reengineering process,
at the same time for better modularity in the code and for better ergonomics.
%
Finally, the \emph{Francy} Jupyter-based graph visualisation library
was generalized to support \Python -- and therefore \SageMath -- in
addition to \GAP.
%
All three benefited from feedback, if not contributions, from end-users.

% Both build on the robust
% foundation of Jupyter Widgets, and explore what it can bring to
% interactive mathematics. The former focuses on interactive
% visualization and edition of mathematical objects, taking
% combinatorics and discrete math as use case. The latter, which uses
% the former as building block, provides rich, detailed, and efficient
% interactive exploration of objects, their properties and
% interrelations. Both are
% \href{https://github.com/sagemath/sage-explorer}{demonstrated online}
% via the Binder service.


\smallskip
\subparagraph{\longtaskref{UI}{structdocs}}
\label{UI@structdocs}

All deliverables for this task have been delivered in previous reporting periods.

Active structured documents are a common need with many use cases, and has many potential
solutions.  Requirements and venues for collaborations were explored through discussions
between participants, in particular at the occasion of
\href{https://wiki.sagemath.org/days77/}{Sage Days 77} workshop (see the
\href{https://wiki.sagemath.org/days77/live-structured-documents}{notes}), and the ODK
meeting in Bremen. The findings were reported in \longdelivref{UI}{adstex}.

In \longdelivref{UI}{adcomp}, We have presented a general framework for in-situ computation in active documents. This is
a contribution towards using mathematical documents -- the traditional form mathematicians
interact with mathematical knowledge and computations -- as a user interface for a
mathematical virtual research environments. This is also a step towards integrating the
two main UI frameworks under investigation in the \ODK project: \Jupyter notebooks and
active documents -- see~\longdelivref{UI}{adstex} -- at a conceptual level. The system is
prototypical at the moment, but can already be embedded into active documents via a
Javascript framework and is ready for use in the \ODK project. The user interface and \SCSCP
connections are quite fresh and need substantial testing and optimizations.

\ODK hosted a workshop on live structured documents in October 2017,
which resulted in the development of \href{https://github.com/minrk/thebelab}{thebelab} software for interactive computing on any website,
enabling interactivity in traditional web-based documentation,
and further development of the \MathHub facilities for evaluation in structured documents.

We developed a JupyterLab extension dedicated to teaching computerscience languages,
such as Python or Sage. JupyterLabTraining (\href{https://gitlab.com/logilab/jupyterhub-training})
is an extension that provides an environment where learners can autonomously do a
series of exercises in order to learn a new programming language. Each exercise is
an independant Jupyter notebook containing the questions, a cell where the learner will
write her code, a hidden cell containing automated tests, and a button to run these tests
and check the code that has been written answers the questions. The left panel shows
the list of all the exercises; they can be sorted by topic (keyword), complexity or
learning track. Thanks to this environment, each learner can do the exercises at his
own pace and choose the exercises that focus on his own points of interest. The
learning process is thus much more efficient for each person.


\subparagraph{\longtaskref{UI}{mathhub}}
\label{UI@mathhub}

All deliverables for this task have been delivered in previous reporting periods.

One of the most prominent features of a virtual research environment (VRE) is a unified user interface. The \ODK approach is to create a mathematical VRE by integrating various pre-existing mathematical software systems. There are two approaches that can serve as a basis for the \ODK UI: computational notebooks and active documents. The former allows for mathematical text around the computation cells of a read-eval-print loop of a mathematical software system and the latter makes semantically annotated documents active.

\MathHub is a portal for active mathematical documents ranging from formal libraries of theorem provers to informal – but rigorous – mathematical documents lightly marked up by preserving LaTeX markup.

As the authoring, maintenance, and curation of theory-structured mathematical ontologies and the transfer of mathematical knowledge via active documents are an important part of the \ODK VRE toolkit, the editing facilities in \MathHub play a great role for the project,
as delivered in \longdelivref{UI}{mathhub-editing}.

\subparagraph{\longtaskref{UI}{vis3d}}
\label{UI@vis3d}

All deliverables for this task have been delivered in previous reporting periods.

The software developed for this task has been delivered in earlier reporting periods.
Packages such as ipyvolume and k3d-jupyter have received further development,
improved compatibility with JupyterLab,
and developed toward maturity and stability,
with growing community adoption.
Several contributions have been made to JupyterLab and
the \Jupyter ecosystem to further support similar work,
benefiting a wide user community.

\subparagraph{\longtaskref{UI}{cfd-vis}} % M12-36
\label{UI@cfd-vis}

No work to report in this period.


\subparagraph{\longtaskref{UI}{Sage-display}} % M24, no deliverables

No work to report in this period.

\subparagraph{\longtaskref{UI}{oommf-py-ipython-attributes}} % M13-19
\label{UI@oommf-py-ipython-attributes}

\ednote{@fangohr: proofread/update report on T4.11: micromagnetics VRE case study}

The micromagnetic virtual research environment is hosted in the
\Jupyter Notebook. The computational backend is the existing \OOMMF
(Object Oriented MicroMagnetic Framework) simulation tool, which is
accessible through the new Python interface that has been created as
part of \ODK
(\localtaskref{component-architecture}{oommf-python-interface}). The
\Jupyter Notebook allows us to integrate the micromagnetic model
specification, the execution of the simulation, and the postprocessing
and data representation within a single executable document; providing
a new computational research environment for micromagnetic simulation
that uses the most widely used simulation code. We have enhanced this
environment further by exploiting that the notebook allows objects to
represent themselves in different ways within the notebook. For
example, Python objects that represent mathematical equations in the
micromagnetic VRE appear rendered as \LaTeX{} in the notebook. It
allows users to interactively compose and explore computational
models, and to be able to inspect what they have put together in the
language of the scientist (i.e. through equations) rather than through
the language of the computer (i.e. code). The addition of this
representation options does not stop the code from being valid \Python
that can be run outside the notebook. We have also provided a
graphical representation of the mesh and discretisation cell as the
appropriate representation of a finite difference mesh to further
assist the effective communication between code and science user and
graphical representation of vector field objects.  We have used
dissemination workshops to seek feedback from users and to refine
interface.

\subparagraph{\longtaskref{UI}{pari-python}}
\label{UI@pari-python}

\ednote{@jdemeyer, @videlec: proofread/update report on T4.12: Pari bindings}

There has been a great deal of progress delivering improved \Pari.
This work has resulted in benefits to the wider Python and \Sage communities
via substantial contributions to the \Sage codebase,
the benefits of which go well beyond this deliverable,
being used by projects outside \ODK.

The end results of this first state of the work are the packages
\href{https://github.com/sagemath/cysignals}{cysignals} and
\href{https://github.com/defeo/cypari2}{CyPari2}, both installable
in a pure \Python environment via the standard tool
\texttt{pip}. Starting from version 8.0, installation via \texttt{pip}
is \Sage's default way of providing the \Pari interface.

\longdelivref{UI}{pari-python-lib2} has been delivered, further improving the \Pari packages
by adding new features, in particular to the Python interface to \Pari.
\emph{cypari2} has gained the ability produce high-resolution SVG plots.
It now also supports the dynamic array type from PARI/GP, \verb/t_LIST/.
The source code of cypari2 is automatically generated.
This automatic generation has been greatly improved
and can be re-used outside cypari2 for any Python package that wants to interface efficiently with PARI.
The cypari2 documentation is also greatly improved,
as a direct result of improvements to the Sphinx documentation system
in \localtaskref{UI}{sage-sphinx}.

\subparagraph{\longtaskref{UI}{oommf-tutorial-and-documentation}
  has been merged into
  \longlocaltaskref{dissem}{dissemination-of-oommf-nb-virtual-environment}
}
\label{UI@oommf-tutorial-and-documentation}

\subparagraph{\longtaskref{UI}{oommf-nb-ve}
  has been merged into
  \longlocaltaskref{dissem}{dissemination-of-oommf-nb-virtual-environment}
}
\label{UI@oommf-nb-ve}

%%% Local Variables:
%%% mode: latex
%%% TeX-master: "report"
%%% End:

%  LocalWords:  subsubsection Jupyter taskref notebook-collab ipython-kernels cfd-vis
%  LocalWords:  oommf-py-ipython-attributes oommf-nb-ve oommf-tutorial-and-documentation
%  LocalWords:  mathhub longmilestoneref emph visualization longdelivref UI-vre delivref
%  LocalWords:  jupyter-live-collab ipython-kernel-sage jupyter-collab texttt nbdime
%  LocalWords:  nbval textbf pytest Cython-generated ipython-advanced-interacts adstex
%  LocalWords:  adcomp optimizations thebelab ipyvolume pythreejs threejs ipyscales unray
%  LocalWords:  ipydatawidgets micromagnetic oommf-python-interface cysignals cypari2
%  LocalWords:  dissem dissemination-of-oommf-nb-virtual-environment
\newpage
\subsubsection{WorkPackage 5: High Performance Mathematical Computing}
  \label{hpc}
%Explain, task per task, the work carried out in WP during the reporting period giving details of the work carried out by each beneficiary involved.


  %%%%%%%%%%%%%%%%%%%%%%%%%%%%%%%%%%%%%%%%%%%%%%%%%%%%%%%%%%%%%%%%%%%%%%%%%%%%%% 
  \paragraph{Overview}

  This work package is about better exploiting modern parallel
  computer architectures in computational mathematics software,
  notably when deployed within a Virtual Research Environment. It is
  addressed at the level of individual computational components
  (\Pari, \GAP, \Linbox, \MPIR, \Sage, \Singular, ...), and also at
  the level of interfacing and exposing core parallel features to
  higher level programming interfaces.

  Key results obtained over the reporting period are the following:
  %% Only list deliverables produced in the reporting period
  \begin{compactitem}
  \item \ednote{@ClementPernet: nothing to highlight about LinBox?}
  %% \item A closer integration of \Linbox in \Sage with improved reliability and
  %%   computing efficiency.
  \item A full-featured parallelisation engine, supporting POSIX threads and
    MPI, for \PariGP in production release of the software.
  \item Release of GAP-4.9 allowing compilation  in HPC-GAP compatibility mode.
  %    \item A new super-optimizer for vectorized assembly code and its
  %  exploitation to improve the performances of the MPIR code.
  \item A new symmetric matrix factorization algorithm over finite fields, and
    its high-performance implementation in the \texttt{fflas-ffpack} library.
  \item Major redesign of the polynomial arithmetic used in Singular,
    delivering state of the art efficiency.
  \end{compactitem}
  In addition, we investigated how to exploit parallelism when
  combining computational software; see
  \longlocaltaskref{component-architecture}{component-for-HPC}, and
  the following milestone.

%%%%%%%%%%%%%%%%%%%%%%%%%%%%%%%%%%%%%%%%%%%%%%%%%%%%%%%%%%%%%%%%%%%%%%%%%%%%%%
\paragraph{Milestones}

\ednote{@ClementPernet: this is a milestone for the previous reporting
period; do we want to keep it around? mention further work?}

\subparagraph{\longmilestoneref{hpc-prototype}}

\emph{“User story: Astrid wants to run compute intensive routines
    involving both dense linear algebra and combinatorics. She has
    access through a JupyterHub-based VRE to a high end multi-core
    machine which includes a vanilla \Sage installation. She
    automatically benefits from the HPC features of the underlying
    specialized libraries (\Linbox, ...). This is a proof of concept
    of the overall framework to integrate the HPC advances of
    specialized libraries into a general purpose VRE.
    %
    It will prepare the final integration of a broader set of such
    parallel features for the end of the project.”}

With Deliverable~\delivref{hpc}{LinBox-algo}, we developed, released and integrated in the
\Sage the LinBox library and its core dependencies: fflas-ffpack and givaro.
When installing the latest \Sage release on a multithreaded multicore server, it
only takes one configure option to let fflas-ffpack use a multi-threaded BLAS
and therefore expose its parallel speed-up to the end-user of Sage. This feature
is compliant with the use of a higher level of parallelism, through process
workstealing queues that \textit{Astrid} may be using in her combinatorics code, as those
exposed in \delivref{hpc}{sage-HPCcombi}. Now that this first proof of concept has been
successfully achieved, we are working in exposing the more advanced parallel
routines of fflas-ffpack into \Sage, following~\delivref{hpc}{LinBox-DSL}. It
should in particular make Gaussian elimination and related routines enjoy a
better scaling with respect to available CPU cores.

%%%%%%%%%%%%%%%%%%%%%%%%%%%%%%%%%%%%%%%%%%%%%%%%%%%%%%%%%%%%%%%%%%%%%%%%%%%%%%
\paragraph{Tasks}

\medskip
\subparagraph{\longtaskref{hpc}{hpc-pari}}
Deliverable~\longdelivref{hpc}{pari-hpc1} was merged with
\longdelivref{hpc}{pari-hpc2} in the revised workplan. The deliverable
\delivref{hpc}{pari-hpc2} is released on time.

The release 2.12 of the \PariGP suite features a MultiThread engine,
used transparently in all tools from the suite: the \Pari library, the
command line interface \texttt{gp} and the GP2C compiler. Written in 2015 and
2016, the engine supports sequential evaluation (no parallelism), POSIX
threads and MPI within the same code base. It is now being progressively
used wherever it makes sense in the code base and this is by nature work
in progress. In \Pari-2.12, the MT engine is a central component of
\begin{itemize}
\item fast (near linear time) Chinese remaindering;
\item fast linear algebra over $\mathbb{Q}$ and cyclotomic fields,
  a critical component of the new "Modular Forms" package;
\item polynomial resultants in
  $\mathbb{Z}[X] \times \mathbb{Z}[X,Y]$ (via fast Chinese remainders and
    evaluation / interpolation), a basic tool for algebraic number theory;
\item computation of classical modular polynomials for about 20 classical
invariants (j, Weber functions, small eta quotients\dots);
\item discrete logarithm over finite fields (prime fields and
$\mathbb{F}_{p^e}$ for word-sized prime $p$) ;
\item Adleman-Pomerance-Rumely-Cohen-Lenstra primality proof;
\item Fourier coefficients of $L$-functions (Hasse-Weil and Artin
  $L$-functions);
\item hi-resolution plot of mathematical functions using parallel evaluation.
\end{itemize}

The \texttt{master} branch on the public development server includes further
\begin{itemize}
  \item values of complex $L$-functions (via parallel computation of
    Meijer $G$-functions);
  \item a new thread-safe version of the Multiple Polynomial Quadratic Sieve
    (MPQS) integer factoring algorithm, ready to be parallelized;
\end{itemize}
The parallel-enabled components of the \PariGP suite have been advertised 
(including tutorial sessions) and tested by participants during PARI/GP
workshops in Grenoble (2016), Lyon (2017), Besançon (2018) and Bordeaux
(2019).

\medskip
\subparagraph{\longtaskref{hpc}{hpc-gap}}
\label{hpc@hpc-gap}

Deliverable~\longdelivref{hpc}{GAP-HPC-report} was completed at the end of this period,
reporting all of the developments in the \GAP system during the project which are relevant
to, or provide essential context for, this workpackage.

During reporting period 3 the main areas of effort were:
\begin{itemize}
\item Follow-up work to the integration of HPC-GAP into the main codebase reported in
  the previous period. This work improves the robustness of the system and dramatically reduced the
  differences between the two versions of the source code.
\item
  \ednote{@stevelinton, @alex-konovalov: briefly explain why meataxe is fundamental?
    Something like: this is the workhorse for higher level
    representation theoretical computations (character tables, etc).}
  Development of the \GAP interface to the meataxe64 high performance linear algebra library (to which
  we also contributed significant development effort). This system targets large calculations over
  small finite fields on multi-core shared memory computers.
  The interface makes almost all of the
  capabilities of meataxe64 callable from \GAP, something which is not only a quantum leap in
  performance for \GAP in this critical area, but also allows easy prototyping in \GAP of new
  algorithms for meataxe64.  This library can make use of multiple cores whether or not it is being
  called from \HPCGAP. A full set of benchmarks are included in D5.15, but as a highlight, two
  dense random $320\,000\times 320\,000$ matrices over $GF(2)$ can be multiplied in just over 1000
  seconds on a 64 core AMD ``bulldozer'' system.
  \item Release of ``libGAP'' a general-purpose C API for \GAP. This allows any program, including in particular HPC code, to call on the
    functionality of \GAP efficiently and without the need to run a separate \GAP process. This makes
    fine grained interaction possible.

    \ednote{@defeo, @embray: the following statement could be here or in WP3}
    To interface with \GAP, SageMath formerly used a bespoke
    implementation of ``libGAP``, requiring heavy patching of \GAP.
    Having the functionality available upstream reduced considerably
    the maintenance burden for SageMath developers and packagers
    alike.

  \item Development and release of a new linguistic reflection API in \GAP, allowing \GAP programs to
    access and modify the executable representation of their own functions at run-time. This will be
    the basis of future automatic parallelisation and optimisation tools.
  \item Very much improved profiling tools
  \item Release of the new package ``ferret'' which achieves world-leading performance in partition
    backtrack, a critical, and notoriusly challenging computational kernel
  \item Extensive developments in our testing and release infrastructure with the overall goal of
    ensuring that \GAP users have easy access to a reliable, up-to-date and mutually compatible set of
    versions of the large suite of packages redistributed with \GAP.
\end{itemize}

\medskip
\subparagraph{\longtaskref{hpc}{hpc-linbox}}
  \label{hpc@hpc-linbox}

During this reporting period, we delivered~\longdelivref{hpc}{LinBox-distributed}.

A first focus was made on distributed computing, with an MPI parallelization of a Chinese remainder based
algorithm. The first proof-of-concept implementation was then cleanly integrated in the mainstream code of the
library. Its performance shows a very nice scaling with the number of compute nodes on a 256 cores cluster.

\ednote{@ClementPernet: T5.3: proofread my changes; I was confused by the original text}

Although this approach is best suited for parallelization on a large
number of nodes, its total computational complexity ($O(n^?)$) becomes
a major concern on large instances.
%
The usual alternative approach based on $p$-adic lifting has a better total computational complexity ($O(n^?)$), but is intrinsically more
sequential, and therefore less suited for large scale parallelization.
A major contribution in this task is a new algorithm combining $p$-adic lifting and Chinese remaindering in order to
expose more parallelism without sacrifying the gain in complexity.
We also provide a full-featured  implementation of this new algorithm in \Linbox, which delivers high
sequential efficiency and nevertheless scales well up to 16 cores.
This compromise is a good fit for personal computers or the typical
lab-wide computational server researchers have access to.

Lastly, we introduced support for GPUs in the \texttt{fflasffpack} library and showed how matrix product over a finite
field benefit from these accelerators.

All these software improvements are closely integrated in the mainstream code of the \texttt{fflasffpack} and \Linbox libraries.

\medskip
\subparagraph{\longtaskref{hpc}{hpc-singular}}
  \label{hpc@hpc-singular}

  The only deliverable under consideration for this reporting period
  is~\longdelivref{hpc}{singular-polyarith}

Multivariate polynomials are represented in Singular using the sdmp format. While this data structure is generally amenable to parallelization, the implementation and some of the algorithms in Singular were not. Much work has been invested in updating the algorithms and data structures and making Singular polynomial arithmetic competitive with other systems. This work has been done in the Singular submodule Flint, whose code is available at \url{https://github.com/wbhart/flint2}.

We now support polynomial exponents of unlimited size with the three basic monomial orderings of lex, deglex, and degrevlex over the integers mod p and rationals. Continued engagement by colleagues in the HPC community including Bernard Parisse, Michael Monagan, Roman Pearce, and Micka\"el Gastineau has been invaluable.

Parallel and serial implementations of the operations of multiplication, division and GCD are complete and perform well in both the dense and sparse cases. The performance is more than competitive with all other systems we are aware of, both on a single core and on multiple cores.

Basic arithmetic in Singular now benefits directly from the implementation and researchers are already working on leveraging the new implementation in other areas, e.g. Gr\"{o}bner bases over rational functions, Gr\"{o}bner bases with a bottleneck on multivariate arithmetic and polynomial factorisation. Early indications are that all of these are going to experience a huge improvement for many real-world research applications.

Other systems such as the new Oscar computer algebra system already benefit directly from the new ODK implementation. 

Sage will automatically benefit directly at the next update of the Singular version in Sage.

\medskip
  \subparagraph{\longtaskref{hpc}{hpc-mpir}}
  \label{hpc@hpc-mpir}

  Not applicable for this period.
  
  \subparagraph{\longtaskref{hpc}{hpc-combi}}
  \label{hpc@hpc-combi}

  Not applicable for this period.


  \subparagraph{\longtaskref{hpc}{pythran}}
  \label{hpc@hpc-pythran}
  Not applicable for this period.

%%   The goal of this task is to make Pythran easily integratable in large-scale
%%   project, taking into account native dependencies, compilation time, memory
%%   footprint, speed and size of compiled binaries as well as multi-platform
%%   support. Integration with \software{cython} is a possible mean to achieve
%%   this goal.

%%   Two projects have been selected for this task: \software{scipy} and
%%   \software{scikit-image}. These projects are relevant for \software{pythran}
%%   because they have many small to medium kernels that can benefit from
%%   compilation. Even though \software{pythran} has not been selected as a scipy
%%   backend, the exchanges with the community have led to a great deal of
%%   improvements of which all \software{Pythran} users take advantage.  The
%%   \software{scikit-image} community is still examining the possibility of using
%%   \software{pythran} as an acceleration mean.

%%   Integration of  \software{pythran} as a \software{cython} backend for
%%   \software{numpy} has improved in various aspects: better error detection,
%%   more supported expression patterns and improved performance for the compiled
%%   expressions.

%% Deliverable~\longdelivref{hpc}{sage-HPCcombi} is shared with
%% Task~\longlocaltaskref{hpc}{hpc-combi}, the status of which we reported on above.

  \subparagraph{\longtaskref{hpc}{hpc-jupyter}}
  \label{hpc@hpc-jupyter}
  
%% It is common for academic High Performance Computing (HPC) clusters to make
%% use of schedulers based on Sun Grid Engine with Son of Grid Engine as one of
%% the most popular. It is used, for example, on the institutional HPC systems
%% in the Universities of Sheffield and Manchester in the United Kingdom. It is also used
%% on the regional N8 HPC facility, a system shared by the eight most research
%% intensive universities in the North of England.
  Not applicable for this period.

%%% Local Variables:
%%% mode: latex
%%% mode: visual-line
%%% TeX-master: "report"
%%% End:


%  LocalWords:  subsubsection hpc compactitem super-optimizer vectorized factorization
%  LocalWords:  texttt fflas-ffpack longmilestoneref emph JupyterHub-based specialized
%  LocalWords:  longtaskref longdelivref delivref finalized mathbb embarassingly taskref
%  LocalWords:  scienceproject refactorization organized LinBox-algo DumPerSul:fcrpmgbd16
%  LocalWords:  Pernet:cqm16,PerSto:tsegqm17 DumKalTho:lticmpdsm16,DumLucPer:cftearp17
%  LocalWords:  Cython Hongguang singular-polyarith sdmp parallelization deglex degrevlex
%  LocalWords:  Monagan Micka parallelized parallelize Imbach hpc-mpir sage-paral-tree
%  LocalWords:  Cilk libsemigroup optimized pythran integratable scipy scikit-image numpy
%  LocalWords:  sage-HPCcombi hpc-jupyter
\newpage
\subsubsection{WorkPackage 6:  Data/Knowledge/Software-Bases}\label{dksbases}
%Explain, task per task, the work carried out in WP during the reporting period giving details of the work carried out by each beneficiary involved.

%%%%%%%%%%%%%%%%%%%%%%%%%%%%%%%%%%%%%%%%%%%%%%%%%%%%%%%%%%%%%%%%%%%%%%%%%%%%%%
\paragraph{Overview}

In a series of workshops (September 2015 in Paris, January 2016 in St. Andrews, June 2016 in Bremen, and July 2016 in Bia{\l}ystok, 2017 in Orsay, 2018 in Cernay, 2019 in Cernay), the participants working on \WPref{dksbases} met and discussed the topic of integrating the \pn systems into a mathematical VRE toolkit.
Additionally, Florian Rabe was employed at both FAU and UPSud throughout 2018 and 2019 to deepen the integration.

Key results of the first two reporting periods were
\begin{compactitem}[\bf R1.]
\item the observation that \emph{knowledge-aware interoperability of software and database-systems is the most critical objective} for \WPref{dksbases} in the \pn project.
\item the consensus that this can be achieved by \emph{aligning the mathematical knowledge underlying the various systems},
\item the existing integration of mathematical computation systems in the Sage and Jupyter systems must be complemented with a similar integration of mathematical databases.
\end{compactitem}
This requires explicitly representing the three aspects of math VREs -- Data (D), Knowledge (K), and Software (S) -- and basing computational services and inter-system communication on a joint \DKS-base.
These results are engrained in the ``Math-in-the-Middle'' (MitM) paradigm~\cite{DehKohKon:iop16}, which gives a representational basis for specification-based interoperability of mathematical software systems -- so that they can be integrated in a VRE toolkit.
In the MitM paradigm, the mathematical knowledge underlying the VREs (K) and the interface for each system (S) are represented as modular theory graphs in the OMDoc/MMT format.
For the data aspect (D) we have extended the concept of OMDoc/MMT theories to ``virtual theories'' that allow the practical management of possibly infinite theories, see~\cite{ODK-D6.5} for details.

Through the concerted effort of the WP6 participants, we have been able to implement this design and instantiate itwith generate theory graphs for the \GAP and \Sage systems and integrating the \LMFDB (see~\cite{ODK-D6.5}.
Based on this, we were able to generically integrate \GAP, \Sage, and \LMFDB via the standardised SCSCP protocol~\cite{HorRoz:ossp09}. This case study shows the feasibility of the design. 

\begin{wrapfigure}r{6cm}\vspace*{-1em}
\documentclass{standalone}
\usepackage{tikzinput}
\begin{document}
\providecommand\myscale{4.5}
\begin{tikzpicture}[scale=\myscale]
  \node (center) at (0,.15) {Organization};
  \node (left) at (.2,-.3) {Computation};
  \node (right) at (.4,0) {Tabulation};
  \node (back) at (-.5,0) {Inference};
  \node (up) at (0,.5) {Narration};

  \draw[very thick] (center) -- (left);
  \draw[very thick] (center) -- (right);
  \draw[very thick] (center) -- (back);
  \draw[very thick] (center) -- (up);
  \draw[dotted] (left) -- (right) -- (back) -- (left);
  \draw[dotted] (up) -- (left);
  \draw[dotted] (up) -- (right);
  \draw[dotted] (up) -- (back);
\end{tikzpicture}
\end{document}
%%% Local Variables: 
%%% mode: latex
%%% TeX-master: t
%%% End: 
\vspace*{.5em}
\caption{Five Aspects of Math VREs, a Tetrapod Structure}\label{fig:tetrapod}\vspace*{-1.5em}
\end{wrapfigure}
In the \textbf{third reporting period}, the focus was on the \textbf{representation and curation of mathematical data}, building on the earlier work. We have refined the original notion of \DKS-bases from the grant proposal into a tetrapodal structure which joins four primary aspects of ``doing Maths'' which have to be supported in a VRE toolkit via a modular organization aspect -- see the introduction  of \cite{ODK-D6.10} and \cite{CarFarKohRab:bmobb19} for a discussion.

We have taken up the general discussion of research data, the FAIR principles and have adapted them to the case of mathematical resarch data. The outcome of this was the concept that -- as mathematics deals with ideal and abstract objects -- it is possible to fully describe the objects in question symbolically and often losslessly represent them as database structures while at the same time binding them to the symbolic description. The codecs from the ``virtual theories'' approach developed in \WPref{dksbases} do just this: they link the database level of mathematical data sets with the MitM ontology -- and from there via the interface theory graphs interface it to the mathematical software systems in \pn.

We undertook three larger case studies to bring this about:
\begin{itemize}
\item Developing the system data.mathhub.info for managing mathematical data sets MitM-style and equipping it with a search UI; see the report on \taskref{dksbases}{data-LMFDB} below
\item Exporting a the Isabelle knowledge base (via a subcontract), and equipping it with a semantic search facility;  see the report on \taskref{dksbases}{isabelle} below 
\item Extending the formula search capabilities developed in the first reporting period to Jupyter notebooks; see the repoet on
 \taskref{dksbases}{mws} below.
\end{itemize}
This wraps up and integrates the work in \WPref{dksbases} combining aspects of Data (D), Knowledge (K), and Software (S). The joint system addreasses central aspects of all four FAIR concerns for research data semanticaly. 
For a joint and integrated final report on this, see ~\cite{ODK-D6.10}.


%%%%%%%%%%%%%%%%%%%%%%%%%%%%%%%%%%%%%%%%%%%%%%%%%%%%%%%%%%%%%%%%%%%%%%%%%%%%%% 
\paragraph{Milestones}

% month 36
\subparagraph{\longmilestoneref{dksbases-interop1}}
This milestone was addressed in the second reporting period.
\medskip
%\emph{“User story: thanks to a fully functional prototype integrating of at least the systems \GAP, \Sage, \Singular, and \LMFDB via the \SCSCP Protocol, end users shall be able to run calculations involving any combination of those systems from any of them.
%  This prototype will be the basis for integration work for additional systems and the user interface from WP4.”}
%\medskip
%
%Workpackage \textbf{WP6} is fully on track with this milestone.
%After first integration and DKS prototypes (the MitM VRE middleware  framework) became available in late fall 2017 (see~\cite{KohMuePfe:kbimss17,WieKohRab:vtuimkb17}) we were able to develop more sophisticated -- and mathematically more realistic/relevant -- use cases~\cite{CreLow:mdcmds18} and generalize those parts of the framework that had been overly specific to the first use cases.
%This involved non-trivial investments in all parts of the framework, as well as the system API theory generation systems and (in particular) the MitM ontology. 

% month 42
\subparagraph{\longmilestoneref{dksbases-interop2}}
\begin{oldpart}{copied from TR 2, needs update by FAU}
\emph{“The goal of this milestone is to take into account all the operational experiences with the first prototype and add more systems and integrate some of the UI components from WP4.
  The experiences with the preparation of this prototype will allow us to estimate the joining costs of adding a system to the OpenDreamKit VRE toolkit, which is an important measure of the flexibility of the Math-In-the-Middle approach.”}
The state of the MitM VRE middleware is sufficiently mature that most of the functionality can be configured by writing domain and system knowledge in form of OMDoc/MMT theories, but not extending the system (programming the VRE systems or the MMT mediator).
This means that additional systems can be added at the cost of generating system API theories, extending the  MitM ontology and supplying alignments.
We are targeting the knowledge bases OEIS, and FindStat (see \localtaskref{dksbases}{data-findstat}) as well as PARI/GP.
We plan to extend the worked use cases substantially.
To this end we already have statements of interest from external researchers, who want to use the flexible integration in the MitM framework and do not mind the communication overheads involved. 
First work on UI integration work has already begun; see  \longdelivref{UI}{jupyter-import}, which presents a Jupyter kernel for MMT and prototypical MitM-based integration of Jupyter widgets.
\end{oldpart}

%%%%%%%%%%%%%%%%%%%%%%%%%%%%%%%%%%%%%%%%%%%%%%%%%%%%%%%%%%%%%%%%%%%%%%%%%%%%%% 
\paragraph{Tasks}
\medskip

\subparagraph{\longtaskref{dksbases}{data-assessment}}
\label{dksbases@data-assessment}
This task was addressed in the first reporting period.
\medskip

\subparagraph{\longtaskref{dksbases}{data-triform}}
\label{dksbases@data-triform}
This task was addressed in the first reporting period.
%For this task we have specified and implemented the concept of virtual theories that can contain large -- theoretically even infinite -- numbers of declarations and objects (e.g. 3.5M declarations in the LMFDB data base for elliptic functions) in OMDoc/MMT.
%Virtual theories are characterized by the fact that they are too large to keep in main memory of the MMT System and have to be partially and lazily imported from an external data store.
%We have reported on the design in \longdelivref{dksbases}{design}, on a first implementation on the international conference (MACIS 2017)~\cite{WieKohRab:vtuimkb17}, and finally on an extended use-case in \LMFDB in \longdelivref{dksbases}{psfoundation}. 
\medskip

\subparagraph{\longtaskref{dksbases}{data-design}}
\label{dksbases@data-design}
This task was addressed in the first reporting period.
%This task was directly addressed in the \WPref{dksbases} workshops in the first year and has led to the design and implementation in \delivref{dksbases}{design}. A first implementation has been presented on the international conference (MACIS 2017)~\cite{WieKohRab:vtuimkb17}, and finally on an extended use-case in \LMFDB in \longdelivref{dksbases}{psfoundation}.
% \medskip

\subparagraph{\longtaskref{dksbases}{data-foundationCAS}}
\label{dksbases@data-foundationCAS}
This task was addressed in the first reporting period.
%In the course of the deliberations in the \WPref{dksbases} workshops we saw a shift from the development of computational foundations and verification towards API/Interface function specifications to enable semantic system interoperability via the Math-in-the-Middle (MitM) Ontology.
%Consequently, emphasis has changed to the generation of system API theories for \GAP, \Sage, \Singular, and \LMFDB, which act as OpenMath content dictionaries.
%The computational foundations exist but are rather more simple than originally anticipated.
%Much of the functionality has been offloaded to the SCSCP standard -- remote procedure call with OpenMath representations of the mathematical objects -- developed in the SCIENCE Project.
%As a direct consequence of the work in \pn the OpenMath Society has promoted the \SCSCP protocol into as an OpenMath Standard.
%
%Conversely, the \GAP and \Sage CDs are rather more elaborated than anticipated in the proposal, and thus form a viable basis for alignment with the MitM Ontology.
%
%The MitM integration paradigm is the result of our research and development on the computer algebra foundations in this task has been presented on the international conference MACIS 2017~\cite{KohMuePfe:kbimss17} and is described in deliverable \longdelivref{dksbases}{psfoundation}, which presents an advanced CAS integration use case. 
%The MitM ontology and the system API theories have been developed to the point, where the data model is fully developed and the contents cover the use cases corresponding to this task and \longlocaltaskref{dksbases}{data-design} are surveyed in \longdelivref{dksbases}{lfmverif}.
\medskip

\subparagraph{\longtaskref{dksbases}{research-categories}}
\label{dksbases@data-research-categories}

\begin{oldpart}{this is the text from TR 2, update by FAU needed}
The MitM architecture developed in \WPref{dksbases} has given important impulses to make the code infrastructure of \Sage and \GAP more declarative (knowledge-based).
In \Sage, the category infrastructure was validated (it seems to be the right level of abstraction to generate API theories) and extended; we explore ways to enrich it with additional semantic
trough the use of annotations, to maximize the chance of
them being accepted and adopted by the Sage community.

In \GAP, the facilities for ``constructors'' was reformed, extended by an infrastructure for documentation and static typing/type analysis, and the code base refactored for over 2000 constructors.
Similarly, the online documentation subsystem for \GAP has been regularized and synchronized with the constructor level.
Already at this early stage of the task the new ``knowledge-based perspective'' has revealed a plethora of errors and inefficiencies and has contributed to the code quality in both systems.
\end{oldpart}
\medskip

\subparagraph{\longtaskref{dksbases}{data-OEIS}}
\label{dksbases@data-OEIS}
This task was addressed in the first reporting period.
%For the OEIS case study we have parsed the OEIS data and converted it into OMDoc/MMT theories (ca. 260,000).
%The main problem solved here was to parse the formula section (generating functions, relations between sequences, \ldots): they are represented in a human-oriented ASCII syntax, which is highly irregular, ill-separated from surrounding text, and interpunctuation.
%Nonetheless we managed to recover ca. 90\% of the formulae and
%\begin{compactenum}[\em i\rm)]
%\item generate ca. 100,000 new relations between sequences and
%\item provide a package of ca. 50,000 generating functions to Sage (which can be used
%  e.g. in the FindStat database).
%\end{compactenum}
%We use this theory set to test the functionalities of ``virtual theory graphs'' (one step up from the ``virtual theories'' developed in \localtaskref{dksbases}{data-design}).
\medskip

\subparagraph{\longtaskref{dksbases}{data-findstat}}
\label{dksbases@data-findstat}
This task was addressed in the second reporting period.
%We have seen that the \LMFDB already shows all the complexities needed to develop full-coverage DKS functionality for the \pn VRE toolkit.
%On the other hand our survey shows that our DKS design (OMDoc/MMT virtual theories) is sufficient for covering the FindStat use case as well.
%Therefore we have delayed this taks to the last year of the \pn project, when the system API theories for \Sage and OEIS have matured. With the declarative design of the virtual theories, task \localtaskref{dksbases}{data-findstat} becomes a matter of writing down the schema theories system API theories for FindStat and defining the requisite codecs. We expect this to be a matter of one of two weeks of joint development of the FAU team together with UPSud. 
\medskip

\subparagraph{\longtaskref{dksbases}{data-LMFDB}}
\label{dksbases@data-LMFDB}

\begin{oldpart}{this is the text from TR 2, update by FAU needed}
Work on this task has started. Given the concept of virtual theories developed in \localtaskref{dksbases}{data-triform} the task is to build a database connector that converts the MongoDB tables in LMFDB into ``mathematical objects''.
We have identified the problems -- e.g. that objects are reduced to ad-hoc database records: for instance elliptic curves are represented as a quadruple of integers, where the last is represented as a string of digits as the range of MongoDB integers is too small.
We have developed an architecture of language-specific Codecs which mitigate these problems in a knowledge-centered way (Codecs are OMDoc/MMT objects) that interpret database records as OMDoc/MMT objects and can thus be used populate virtual theories.
The next step is to extend the existing MMT query language by a query compiler into the underlying data store system; concretely to MongoDB underlying LMFDB for \localtaskref{dksbases}{data-LMFDB}.
\end{oldpart}
\medskip

\subparagraph{\longtaskref{dksbases}{data-memo}}
\label{dksbases@data-memo}
We have developed persistent memoization modules for Sage and Gap that can use both local and remote data stores.
Both use the same format so they can share the same data stores.\ednote{to be finished by sites US,PS,UW}

We report on this task in detail in \delivref{dksbases}{persistent-memoization}.
\medskip

\subparagraph{\longtaskref{dksbases}{mws}}
\begin{oldpart}{this is the text from TR 2, update by FAU needed}
Work on the first work phase has proceeded as planned and has culminated in \longdelivref{dksbases}{mws}.
The second work phase on this task presupposes the Math-in-the-Middle ontology (as we call it now.)
Where we already have that, e.g. for the OEIS (see \localtaskref{dksbases}{data-OEIS}) we already have a running search engine.
The main problem here is to devise intuitive query interfaces and integrate them into the \pn VRE framework.
\end{oldpart}

We report on this task in detail in \delivref{dksbases}{nbad-search}.
\medskip

\subparagraph{\longtaskref{dksbases}{isabelle}}
\begin{newpart}{FR: adapted from the deliverable}
For many decades, the development of a universal database of all mathematical knowledge, as envisioned, e.g., in the QED manifesto \cite{qed}, has been a major driving force of computer mathematics.
Today a variety of such libraries are available.
These are most prominently developed in proof assistants such as Coq \cite{coq} or Isabelle \cite{isabelle} and are treasure troves of detailed mathematical knowledge.
Within \pn, we have developed interface standards, specifically OMDoc for symbolic and ULO for relational knowledge, that allow maintainers of formal libraries to make their content available to outside systems.

In this task (which has been added in the last amendment of the grant agreement), we have exported the large Isabelle knowledge bases as both OMDoc/MMT and ULO format
Concretely, we have built an exporter from the Isabelle Theorem prover library (Archive of Formal Proof) to both MMT and RDF data.
This exporter is now part of the latest releases of both Isabelle and MMT, and the exported data is available online.

We report on this task in detail in \delivref{dksbases}{nbad-search}.
\end{newpart}


%%% Local Variables:
%%% mode: latex 
%%% mode: visual-line
%%% fill-column: 5000
%%% TeX-master: "report"
%%% End:

%  LocalWords:  subsubsection dksbases ystok WPref compactitem emph DehKohKon:iop16 textbf taskref longdelivref lfmverif triformal formalized biformal HorRoz:ossp09 medskip longmilestoneref dksbases-interop1 dksbases-interop2 characterized WieKohRab:vtuimkb17 psfoundation delivref KohMuePfe:kbimss17 regularized synchronized ldots interpunctuation compactenum mws KohMuePfe:kbimss17,WieKohRab:vtuimkb17 CreLow:mdcmds18 jupyter-import Jupyter MitM-based Jupyter


%%% Local Variables:
%%% mode: latex
%%% TeX-master: "report"
%%% End:

%  LocalWords:  newpage

  \subsection{Impact}
  % Include in this section whether the information on section 2.1 of the DoA (how your
  % project will contribute to the expected impacts) is still relevant or needs to be
  % updated. Include further details in the latter case

  All the information of section 2.1 of the DoA is still relevant. The KPIs 

  There is for now no
  change to bring to Key Performance Indicators. The evolution of the measures between
  Month 18 and Month 36 will allow the Coordinator to evaluate if the selected KPI are
  appropriate.

\subsection{Infrastructures}
% If access to research infrastructures has been provided under the grant please include
% access provision activities

\label{infrastructures}

Per design, \ODK focuses on delivering ``a flexible toolkit enabling
research groups to set up Virtual Research Environments''. As such,
there is no e-infrastructure deployed and managed by \ODK. Instead,
there are many e-infrastructures that use the software developed or
contributed to by \ODK, and we regularly help with new or updated
deployments.

Some of the typical content of this section (e.g. Selection Panel,
...) is therefore irrelevant for \ODK, and we simply provide some
informal information and figures on the main existing deployments and
their typical public, together with some assessment of the impact we
had on them.

\begin{itemize}
\item{cloud.sagemath.org} With 500k accounts worldwide and 30k active
  projects both for research and education, \SMC is the largest
  Virtual Research Environment based on the ecosystem \ODK contributes
  to. Predating \ODK, it benefits back from most of our actions. \ODK
  has been contributing to a healthy collaboration/competition
  relation between \JupyterHub and \SMC, with the competition
  occurring only at the level of specific individual components and
  both teams learning from each other.

\item{jupyter.math.cnrs.fr} We have helped setup this \JupyterHub
  service, deployed by the French CNRS for the benefit of the
  personnel of all math labs in France. This service includes all the
  \ODK computational components.

\item{mybinder.org} Binder is a web service that makes it easy for any
  user to publish live notebooks based on an arbitrary reproducible
  executable environments. It thus fosters dissemination and
  reproducible research. The current main instance
  (\url{http://mybinder.org/}) is overloaded by the demand, proving
  that it has identified just the right service for a critical need.

  Our work on packaging
  \localtaskref{component-architecture}{mod-packaging} and \Jupyter
  integration \localtaskref{UI}{ipython-kernels} is about to enable
  the easy definition of executable environments including \ODK's
  computational math software.

  We are further reaching toward EGI/EUDAT to use their
  e-infrastructure to contribute additional computing resources to the
  main mybinder instance or setup a new one for the EC community.
  \url{https://github.com/OpenDreamKit/OpenDreamKit/issues/205}

\item{JupyterHub at USFD}
  \href{http://docs.iceberg.shef.ac.uk/en/latest/using-iceberg/accessing/jupyterhub.html}{\JupyterHub}
  instance deployed on USheffield's HPC system.
\end{itemize}

Many other instances are being deployed by universities (e.g.
university Paris Sud) for their personnel. We are keeping track of
those we are aware of at
\url{https://github.com/OpenDreamKit/OpenDreamKit/issues/174}.

%%% Local Variables:
%%% mode: latex
%%% TeX-master: "report"
%%% End:


\section{Update of the plan for exploitation and dissemination of result (if
    applicable)}
 Not applicable
 % Include in this section whether the plan for exploitation and dissemination of results
  % as described in the DoA needs to be updated and give details.

\section{Update of the Data Management Plan}

% Include in this section whether the data management plan as
% described in the DoA needs to be updated and give details.

A second version of the Data Management Plan was released in
\longdelivref{management}{data-plan2}. Up to minor updates to the list
of data sets, there was no change since
\longdelivref{management}{data-plan1}.


\section{Follow-up of recommendations and Quality Management}

In this section, we will detail our actions in response to the recommendations and
comments of the reviewers and review the risk management and quality assurance procedures
adopted in in \pn project.

\section{Follow-up of recommendations and Quality Management}

In this section, we will detail our actions in response to the recommendations and
comments of the reviewers and review the risk management and quality assurance procedures
adopted in in \pn project.

\subsection{Follow-up of recommendations}

We are extremely grateful for the very constructive comments and
recommendations that were provided during the review itself and in the
formal report. Most recommendations were implemented right away at the
occasion of the Work Plan Revision process that followed the second
reporting period and involved the reviewers and advisory board. In the
sequel we explain how we took the recommendations into account.

\newenvironment{recommendation}[1]
{\noindent{\textbf{#1:}} \begingroup\it}
{\endgroup}

\begin{recommendation}{Recommendation 1} The deliverable D4.7 needs to
  be presented or reasons for not presenting it needs to be detailed
  in the progress report.
\end{recommendation}

Taken from our Work Plan Revision proposal:
This was in fact a communication glitch, which we clarified since with
the project officer. The deadline of M14 for
\delivref{UI}{ipython-kernels} was a typo in the original proposal: it
should have been M24, 12 months after the related deliverable
\longdelivref{UI}{ipython-kernels-basic}. After consulting our former
project officer back in Fall 2016, this typo was fixed in the second
amendment to the grant agreement. We oversaw that at the time of the
review this agreement was not yet effective and available to the
reviewers. We will make sure to mention any such change in the report
next time.

\begin{recommendation}{Recommendation 2} The final Progress Report
  must be presented and it should report on all deviations including
  cancelled or amalgamated deliverables.
\end{recommendation}

We followed the advice from our Project Officer: \emph{``The review
report says that you should amend the current progress report on all
deviations etc but as we have already started to process the report
and cost, you don’t need to do that anymore for the past period.
However, in the future please report on all changes/deviations in the
progress report to be clear.''}

In addition we made sure that changes and deviations for Reporting
Period 2 were reported in this document.

\begin{recommendation}{Recommendation 3}
  The workplan of WP7 should be assessed critically and the draft
  proposal of the revised plan submitted to the Commission as soon as
  possible and latest by 1 July 2017.
\end{recommendation}

This was implemented in the Work Plan Revisions.

\begin{recommendation}{Recommendation 4}
  The deliverables presentation should be clearer concerning their key
  content. They should include a clear ``executive'' summary, state
  the purpose and target audience of the deliverable and include clear
  conclusions. The full title of the deliverable should be used
  instead of the often used form ``Report on Dn.m''. Consolidation of
  the deliverables should be considered to reduce their number.
\end{recommendation}

Deliverables were consolidated at the occasion of the Work Plan
Revision, without affecting the content and work time line, beside
enabling a bit of flexibility to adapt to a quickly evolving landscape
(where e.g. some actions become more pressing and others less).
Together with the refactoring of \WPref{social-aspects} this
consolidation reduced the number of remaining deliverables from 55 to
below 40.

We tried hard to follow the presentation recommendations for all newly
submitted deliverables.

\begin{recommendation}{Recommendation 5}
  Provide clearer and more relevant work-package specific milestones
  to allow for effective monitoring of the project's progress.
\end{recommendation}

This was implemented in the Work Plan Revisions.

\begin{recommendation}{Recommendation 6}
  The web presence should be made more attractive for broader
  dissemination and it is recommended to create separate externally
  facing websites. The internal project website could be separated from
  the external one, as their purpose and target audiences are not the
  same. The social media presence and blogging should be made more
  vivid and attractive.
\end{recommendation}

At the occasions of the Work Plan Revisions, we proposed a plan to
improve our web presence, which we implemented during Reporting Period
2, with continued efforts on the content since then, including
interview videos and an upcoming motion design explainer video. To
this end, we used help from master students in communications, and two
specialized companies. For details, see
\longlocaltaskref{dissem}{dissemination-communication}.

\begin{recommendation}{Recommendation 7}
  The KPIs should be refined and alternatives suggested and the
  currently very generic milestones should be revised to be more
  specific and appropriate for project monitoring purposes.
\end{recommendation}

Alternative KPI's were chosen at the occasion of the Work Plan
Revisions.

\begin{recommendation}{Recommendation 8}
  Regarding the WP2, it is recommended to deploy some additional
  resources to improve the externally facing website as well as
  improving the internal site.
\end{recommendation}

See Recommendation 6 above.

\begin{recommendation}{Recommendation 9}
  WP3 work to be integrated into the proposed revisions of the
  website. Efforts need to be re-allocated as SMC developers have
  already done the work that was described to be in D3.4.
\end{recommendation}

This was implemented in the Work Plan Revisions.

\begin{recommendation}{Recommendation 10}
  Regarding WP5, make contacts with HPC community in order to
  ascertain current state- of-the-art. The work in this WP needs to be
  nearer the leading edge.
\end{recommendation}

The following text is updated from our Work Revision Proposal.

We would like to clarify the context of high performance \emph{mathematical} computing,
which is subject of \WPref{hpc}, and its relation to high
performance \emph{scientific} computing.

High performance computing is mostly driven by scientific computing and its
applications. As a consequence, it focuses on numerical computations using
approximate floating point arithmetic and parallel numerical linear algebra. Decades of
efforts in research and development in this field have produced a mature  set of
software and algorithmic practices and even impacted the design of most of nowadays computers.

On the other hand, computational mathematics, at the core of \ODK activity, differs from scientific
computing primarily on the type of arithmetic being used. There is in fact a
large variety of arithmetics to be supported depending on the application: finite
  fields of small, medium and large cardinalities, multiprecision integer and
  rationals, polynomials over these domains, etc. Obviously all these
  arithmetics need to be exact which defeats a direct use of floating point
  arithmetic. Another challenge is the deep interplay between these arithmetics
  which are often composed in high stacks of algebraic structures: e.g. tensor
  algebras built on top of algebras built on top of combinatorics and
  fractions, the latter being built on polynomials built on arithmetic.

Consequently, the experience of the numerical scientific computing community can not
be blindly exploited in the development of high performance mathematical
computing.
Driven by emerging applications, such as experimental mathematics, cryptanalysis,
discrete optimization, and bioinformatics, high performance mathematical computing has
become an active field in computer algebra over the last two decades, but is
comparatively at an earlier stage of development, given the smaller community
and the broader scope to be adressed.
However interactions with the numerical scientific computing community
have always been intense and several crucial innovations
resulted from a convergence between the two fields: floating point arithmetic
can be used under control for finite field linear algebra, sparse solvers used
in cryptography are adapted from iterative numerical methods, etc.

In the first evaluation period of the project, most efforts have been put
towards tasks specific to mathematical computing: improving some core computational kernels, for polynomial and finite
field arithmetic (\delivref{hpc}{MPIRsuperoptimiser}  and \delivref{hpc}{FFT}),
parallelization of recursive tree explorations~\delivref{hpc}{sage-HPCcombi}, etc. This could explain a
feeling that these contributions are not in line with mainstream numerical high
performance computing research.
Achievements during second reporting period, which will be presented
at the formal review, include significant contributions to parallel
exact linear algebra, which is much closer in nature to numerical HPC.

Participants of the \WPref{hpc} are (and have been for a long time) designing
leading edge software for high performance mathematical computing and are in
continuous interaction with the mainstream numerical HPC community.
More precisely, we identify that the major forthcoming interactions will be in
the following aspects:
\begin{description}
\item[Parallel runtime for task based parallelization] Parallel runtime systems
  are becoming a key component to properly harness the always growing number
  parallel cores. Shifting from low level thread management to higher level
  parallel programming languages and runtimes has become a hot topic in numerical
  HPC. Convergence in this area is already happening, with for instance our
  participation to a french national workshop ``Journée Runtime'' where we could exchange with the
  numerical HPC community on our experience using XKaapi, Cilk and OpenMP for the task
  based parallelization of exact linear algebra. During the 11th e-Concertation
  Meeting, we also started an interaction with the research group at the BSC
  (Barcelona Supercomputing Center) regarding their runtime OmpSS.
\item[Automated SIMD optimization.] SIMD is the lowest level of parallelization
  available inside each processor. Automating the optimization of such low level code is a common target
  between communities of exact and numerical computation. 
  Our project's contribution in  \delivref{hpc}{MPIRsuperoptimiser} is a major
  step in this direction.
  On the topic of SIMD vectorization, we have started several interactions with the developpers of the BLIS project
  (a framework instantiating blas-like libraries for numerical linear algebra),
  regarding implementations of AVX512 kernels for matrix multiplication\footnote{\url{https://github.com/flame/blis/issues/182}}. We also
  have interacted on the topic of implementations of Strassen's algorithms
  which is a core feature of \texttt{LinBox} (\localtaskref{hpc}{hpc-linbox}) and for which authors of BLIS have shown recent interest.
  
\end{description}



%% We therefore acknowledge the importance of keeping and increasing our interactions with the
%% mainstream HPC community and of keeping us up to date with the state of the art
%% practices in the domain.


%% One message we take back home here is that we really need
%% to better communicate about this work package and its rationale. Indeed,
%% The \WPref{hpc} participants are (and have been for a long time) in
%% continuous contact with the mainstream numerical  HPC community, trying hard to
%% reuse or adapt its know-how and technologies. However, in particular
%% due to the wide variety of data structures and algorithms, the
%% challenges arising in HPC for pure mathematics are often of rather
%% different nature, calling for custom solutions.

%% Here is some evidence:
%% \begin{itemize}
%% \item There is a dedicated yearly conference on the topic, PASCO.
%%   Incidentally, it turns out that PASCO is hosted by one of our site
%%   (Kaiserslautern); Florent Hivert from Orsay is invited speaker and
%%   will present, among other things, his work around
%%   \delivref{hpc}{sage-paral-tree}. Other \ODK
%%   participants from \ODK will be presenting work or attending too.
%% \item Several participants of this work package have taken part in the
%%   past to joint research projects with established members of the HPC
%%   community (Grenoble, Kaiserslautern).
%% \item Enabling HPC in GAP by itself was a major undertaking of the
%%   previous SCIEnce project (Symbolic Computation Infrastructure for
%%   Europe FP6 eRII3-CT-026133, 2006–2011).
%% \end{itemize}

% We are planning to write a blog post about the
% interplay and differences between high performance mathematical software and
% mainstream HPC.

We nevertheless doubled our efforts to deepen our contacts with the
mainstream HPC community, and very much appreciate the contacts that
were and will be supplied by the reviewers.


\begin{recommendation}{Recommendation 11}
  The activities of the WP7 should be assessed critically and a
  revised work plan for this WP should be presented that is of greater
  relevance to the aims and goals of the project. If a satisfactory
  resolution of the issues is not reached then this WP should be
  dropped and the effort re-assigned elsewhere within the project.
\end{recommendation}

See Recommendation 3 above.

%%% Local Variables:
%%% mode: latex
%%% TeX-master: "report"
%%% End:

%  LocalWords:  newenvironment noindent textbf begingroup endgroup delivref emph WPref
%  LocalWords:  ipython-kernels specialized longlocaltaskref dissem hpc optimization
%  LocalWords:  MPIRsuperoptimiser parallelization sage-HPCcombi XKaapi OmpSS
%  LocalWords:  vectorization localtaskref sage-paral-tree

\subsection{Risk management}
\ednote{This whole section seems to be about RP2}
\subsubsection{Recruitment of highly qualified staff}
Recruitment of highly qualified staff was planned to be a high risk
when the Proposal was written. And unfortunately it turned out we were
right. In such a field as computer science and software development,
potential candidates who are likely to be fairly young considering
only temporary positions are offered, are very scarce. Furthermore
they need to make a choice between public and private bodies which are
very attractive, and the choice between pure development and research.
Because of this difficulty to recruit in the past year, there have
been slight changes in the workplan, which do
not put the project results at risk.

The following people were hired in the past year or are in the hiring pipeline for next year\\
\begin{tabular}{|l|c|r|r|r|}\hline
  NAME&GENDER&PARTNER&POSITION&HIRING DATE\\\hline
  Theresa Pollinger & F & \site{FAU} & Junior Researcher & October 2017\\
  Tom Wiesing & M & \site{FAU}  & Junior Researcher & September 2017\\
  PD. Dr. Florian Rabe & M & \site{FAU}/\site{PS} & PostDoc & December 2017\\
  Dr. Katja Ber\v{c}i\v{c} & F & \site{FAU} & PostDoc &  November 2018\\
\hline
\end{tabular}\\
~\\
Dr. Florian Rabe is a joint appointment and splits his time and research between \site{FAU} and \site{PS}, which reflects
the close cooperation and cross-fertilization of methods in \WPref{dksbases}. 

\ODK partners had to face some Human Resources issues; this mostly
concerned Reporting Period 1 where most of the hiring occurred, with
reduced effects on Reporting Period 2:
\begin{itemize}
\item{\site{PS}:}
  Thanks to an early start in the recruitment process, and despite
  some difficulties in attracting experienced candidates for a part
  time position, the project manager position (24PM) was filled by
  Benoît Pilorget shortly after the start of the project. Unfortunately at month 36 the project
finds itself without a Project manager since the departure of B. Pilorget.

  The recruitment of \site{PS}'s first Research Engineer (48PM) was
  delayed by four months because the top ranked candidate for this
  position, Erik Bray, was originating from the US and needed time to
  arrange for his moving; there were also some administrative delays
  (visa, ...).

  The second Research Engineer position (36PM) was more problematic
  for internal administrative reasons. The top ranked candidate,
  Jeroen Demeyer, had the perfect profile; however for family reasons,
  he wished to work most of the time from Ghent in Belgium. After eight
  months investigating an administrative solution to hire him at
  \site{PS}, and a temporary four month solution, it was decided with
  OpenDreamKit's Steering Committee and Project Officer to instead add
  Ghent's university as new partner, hire Jeroen Demeyer there, with an
  adequate budget transfer and amendment to the Grant Agreement.

  Those delays have induced late start on several tasks, and costed
  much management time. However the excellence of the recruitment,
  well confirmed by the results obtained so far, was worth it and soon
  compensated for the late start.

  In addition to this, a three year PhD position was open to work on
  WP6, starting from Month 12. By lack of suitable candidate, this
  position was converted into a PostDoc position; this position was
  filled in half by Florian Rabe in Spring 2018. A research software
  engineer, Odile Benassy was hired in June 2018 until the end of the
  project using the remaining PMs.\\

\item{CNRS:} Because the research engineer offer (48PM) was still not
  filled in the Summer 2016, the CNRS decided to divide the position
  in two full positions of 24 PM each. As a result, a candidate was
  selected for one of the two positions and began his work in Fall
  2016. Thanks to the PM division, there should be no delay in any
  task or
  deliverable. \\

\item{JacobsUni:} Michael Kohlase, lead PI for Jacobs University, has
  moved on 01/09/2016 to Friedrich-Alexander-Universität
  Erlangen-Nürnberg, and most of his team will follow him. The necessary changes have been
  implemented in a grant agreement amendment in 2017.\\

\item{UJF:} The original tentative candidate for UJF's Research
  Engineer position (12PM, planned to start on Month 1), Pierrick
  Brunet, finally declined the position to accept an alternative
  permanent offer. The position was filled by another candidate in
  Autumn 2016. This induced a delay of Deliverable \textbf{D5.2} from Month 12
  to Month 18, without impact on other tasks.\\

\item{UNIKL:} UNIKL had to split the 12 PM planned for a software developer into 2 shorter positions (Anders Jensen and Alexander Kruppa) in order to deliver the planned work on time. Indeed the few qualified persons for this job were not able to accept this 12 months position during the timelapse planned within the project.\\

\item{USFD:} The University of Sheffield has also been struggling in
  the hiring process of a postdoc (36PM), and a move of the PI to the
  University of Leeds.\\

\item{Southampton:} Southampton faced administrative difficulties in
  the recruitment of Marijan Beg (38PM) as a post-doc, due to
  Marijan's Croatian nationality and recent changes in the relevant
  legislation. His recruitment was delayed by four months, and
  therefore some tasks and deliverables were postponed. We will
  compensate for the delay by putting more staff effort in at later
  stages in the project. We don’t expect any delay nor implication on
  the main tasks of OpenDreamKit.

\item{UVSQ:} Nicolas Gama was on a long-term leave until September
  2017. This did not affect the project in any way.\\

\item{UZH:} The University of Zürich partner is only composed of one
  person, Paul-Olivier Dehaye, who does not enjoy a permanent position
  there. There have been worries that Mr Dehaye's contract with his
  university might end earlier than planned within OpenDreamKit. But
  thanks to the action of the OpenDreamKit steering committee, Mr
  Dehaye has been technically rehired by UZH as a scientific
  consultant until the end of the planned implication.\\

\item{Simula:} Everything is fine concerning temporary staff
  recruitment on the Simula side, however we have had to endure the
  hazards of human ressources with Hans-Peter Langtanger (the PI when
  the Grant was signed) being on a long-term sick leave, and with
  Martin Alnaes replacing him as PI having a paternity leave. However
  Benjamin Ragan-Kelley has stepped in to lead the
  Simula contribution in the meantime and all planned tasks are on time.\\
\end{itemize}


Altogether, this confirmed that the recruitment of highly
qualified staff is indeed a risky endeavour, which induced delays on
several deliverables. However the planned mitigation measures --
taking into account the pool of potential candidates in the design of
the positions, aggressive advertisement, weak coupling between tasks
-- worked adequately: with appropriate reshuffling of the work plan,
this did not impact the overall progress of the project.

\subsubsection{Different groups not forming effective team}

As expected, this risk was tamed by the existence of many preexisting
collaborations between the partners and of ``joint itches to scratch
together'' (to use a common open source software metaphor). The
organization of many joint workshops (for example the Sage-GAP
workshop, the Atelier Pari attended by SageMath developers, the WP6
workshops) helped bootstrap joint activities through brainstorms and
coding sprints. Upcoming workshops are planned on Year~2 to strengthen
collaborations with the social aspects team in Oxford and the Singular
team in Kaiserslautern.


\subsubsection{Implementing infrastructure that does not match the needs of end-users}

The consortium is keeping in their minds the end-user needs. Since
OpenDreamKit is improving already existent software which have their
own users, their needs are naturally met. However Key performance
Indicators will evaluate the effects of OpenDreamKit on these
software. KPIs, indicated in the Proposal, will be launched this
Autumn with the help of the end-user group which was merged with the
Advisory Board. Constant links between the accomplished work and the
end-user needs should be made in WP2 deliverables and also in WP7
deliverables when relevant.  Open tracking of KPIs evolution can be
found on
\href{https://github.com/OpenDreamKit/OpenDreamKit/labels/KPI}{GitHub}.

\subsubsection{Lack of predictability for tasks that are pursued jointly
  with the community}

As planned, we are regularly shifting manpower around to adapt for the
variability of the involvement of the community in the different
tasks. For example, the SageMath Jupyter kernel of
\longdelivref{UI}{ipython-kernels-basic} was mostly implemented by the
community which allowed to focus on other tasks such as the long term
task~\longdelivref{component-architecture}{portability-cygwin}.  On
the other hand many other deliverables were implemented with very
little help from the community.

\subsubsection{Reliance on external software components}

There is not much to report on this front yet: none of the external
software component we rely on have failed us. Quite on the contrary,
critical software like \Jupyter have continued to blossom. Besides the
high modularity of the design means few components are critical to the
overall success of the project.

%%% Local Variables:
%%% mode: latex
%%% TeX-master: "report"
%%% End:

%  LocalWords:  hline cross-fertilization WPref dksbases Pilorget Pierrick textbf Kruppa
%  LocalWords:  Dehaye Dehaye's Dehaye Alnaes organization Jupyter longdelivref Benassy
%  LocalWords:  ipython-kernels-basic portability-cygwin subsubsection

\subsection{Quality assurance plan}
\label{section.QAP}

\subsubsection{Deliverables quality: Quality Review Board}

The Quality Review Board is the Consortium Body that fosters best
possible quality in the delivered work of the project.
All four members of the board
have a research interest in the quality of software in computational
science, and use and share their experience to benefit the quality of
the work.

The board was chaired by Hans Fangohr, from the University of
Southampton and European XFEL GmbH (Germany). He is supported in this task by
Mike Croucher from the University of Leeds and now
Numerical Algorithm Group (UK), Alexander Konovalov from
the University of St Andrews (UK), and by Konrad Hinsen from the Centre de
Biophysique Moléculaire (France) with whom a Non-Disclosure Agreement was
signed.

These board members engage with European initiatives working towards
improvement of the software quality in research, in particular in
computational and data science; both as voluntary activities and key
of their professional roles. Mike Croucher was the head of research
computing at Leeds, and is well known through his outreach blog; Alexander Konovalov
is a fellow of the Software Sustainability Institute and an active
member of the Software Carpentry community; Konrad Hinsen has
founded and is editing the ReScience Journal for reproducible Science,
and Hans Fangohr is the founder of the UK's only centre for
doctoral training in computational modelling with focus on software
engineering training for scientists, a fellow of the Software
Sustainability Institute, was chairing the EPSRC's national scientific
advisory committee on high performance computing, is heading
data analysis infrastructure development at the European XFEL research
facility, and leading the data analysis work package in the Photon and
Neutron Open Science Cloud H2020 project that works towards
implementation of the European Open Science Cloud.

The quality review board has reviewed deliverables after the reporting
period 1 and 2, identified good practice - both in terms of software
engineering content but also presentation of the work -, produced
reports, and shared the findings with all members in the project to
improve the quality of the remaining deliverables. An improvement of
the quality of deliverable reports at the end of reporting period 2 in
comparison to reporting period 1 was noted. The reports are available
on request, and a summary is included in
Sec.~\ref{sec:summ-recomm-deliv}. The board has stuck to its no-blame
culture in its reporting, but has pointed out deliverable reports of
very high quality.


\subsubsection{Good practice}

The quality review board notes that there is no firmly established
view on what best practice establishes, and that (i) we expect our
best practice checklist to grow and change, and (ii) that due to the
variety of possible outputs not all categories will be appropriate for
every item under review. Here is a summary of good practice, with
particular focus on software and computational projects.

\paragraph{Software engineering}
\label{sec:org2e9824e}

Use of
\begin{itemize}
\item[{$\square$}] version control
\item[{$\square$}] tests
\item[{$\square$}] automated tests
\item[{$\square$}] continuous integration
\item[{$\square$}] automatic building of releases
\end{itemize}

\paragraph{Dissemmination}
\label{sec:org1f65c9b}
\begin{itemize}
\item[{$\square$}] Host code publicly (Github, \ldots{})
\item[{$\square$}] Reference Manual (APIs)
\item[{$\square$}] Tutorial (for beginning users)
\item[{$\square$}] Examples
\item[{$\square$}] Offer live interactive online demos (for example
  through Binder)
\item[{$\square$}] Support mechanisms (email/forum/gitter/github issues/\ldots{})
\item[{$\square$}] How to cite the output?
\item[{$\square$}] Installation mechanism
\item[{$\square$}] High level description of tool/activity accessible to non-experts
\item[{$\square$}] URLs/Blog/etc to and from  OpenDreamKit project
\item[{$\square$}] Grant acknowledgements
\item[{$\square$}] Open Source license
\item[{$\square$}] Workshop
\item[{$\square$}] Engaging users
\end{itemize}

\paragraph{Pathways to impact}
\label{sec:orgc218a3a}
\begin{itemize}
\item[{$\square$}] Does the software address the needs of the users?
\item[{$\square$}] Workshops to gather feedback
\end{itemize}

\subsubsection{Summary of recommendations for deliverable reports}
\label{sec:summ-recomm-deliv}

For reports that are well written, the quality review board found good
software engineering practices. However, for some deliverables the
reports were more difficult to assess. To address this, the following
guidance has been developed:

\begin{itemize}
\item Context setting:
\begin{itemize}
\item what is the problem (we are addressing)? Why should we care about
the deliverable?
\item what is the thing that has been created?
\end{itemize}

\item State the obvious: for example if people (outside the project) are
excited about it.

\item If there are immediately explorable exhibits on the Internet, put a
link prominently on the first page or early on in the report.

\item Make it pleasant to read (easy flowing, images, \ldots{}).

\item Provide introduction to topics, in particular for people not too familiar with
the field.

\item Comment on other good practices we use without thinking about it
(version control, testing, continuous integration, distribution).

\item Comment on the testing that has been done, even if not automatic.

\item If there is a prototype / demonstrator:
\begin{itemize}
\item don't just mention it, but explain it in more detail - it makes a
great story,
\item provide a URL to it (in particular if it is an interactive
exhibit: Binder, SageMathCell server etc).
\end{itemize}

\item If you have done something in terms of sustainability, do mention
it.

\item Anything with impact should be mentioned (contribution / uptake to
Software Carpentry, other computational science and software
projects, GitHub, users, \ldots{}).

\item Have an executive summary on the first page (just half a page):
\begin{itemize}
\item why do we bother (context)
\item have we achieved everything we set out to
\item what would be / are the next steps
\end{itemize}
This could be just the summary from the Github issue.

\item Where we have been in contact with users of the tools we have
developed or improved, and we have feedback from them available,
then it is great to include that in the report - even if this might
be mostly anecdotal evidence: what difference does it make to have
the interactive widgets, be able to install the software via Docker,
have an interactive textbook over the previous version, 3D
visualisation in the notebook, etc.

\item Be careful with figures and captions: it is important that captions
describe what is shown in the figure, and that the content of figure
and caption can be understood without reading the main text. (People
often look at figures and captions before deciding if they want to
read the rest of the report. If the caption is very short, the
figure may be misunderstood.)

\item Be clear about the source of the displayed image in figures: is it
to motivate the work and from external sources, or are the figure(s)
created as part of OpenDreamKit work? This is often not obvious from
the outside.
\end{itemize}



\subsubsection{Infrastructure quality: End-user group}


It was decided by the Steering Committee during the
\href{http://opendreamkit.org/meetings/2015-09-02-Kickoff/management_structure/}{kick-off
  meeting} to slightly modify the management structure by having only
one gender-friendly Advisory Board composed of 6 people (as agreed a
few months later at the
\href{http://opendreamkit.org/meetings/2016-06-27-Bremen/minutes/}{Bremen
  meeting}), some of which to be end-users.

Members of the board are: Jacques Carette from the McMaster University, Istvan Csabai from the Eötvös University Budapest,
Françoise Genova from the Observatoire de Strasbourg, Konrad Hinsen from the Centre de Biophysique Moléculaire,
William Stein who is CEO of SageMath, Inc. (SME), and Paul Zimmermann from INRIA.

%%% Local Variables:
%%% mode: latex
%%% TeX-master: "report"
%%% End:

%  LocalWords:  subsubsection sec:summ-recomm-deliv Dissemmination ldots sec:orgc218a3a
%  LocalWords:  github Csabai


\clearpage
\section{Deviations from Annex 1 (if applicable)}
  % Explain the reasons for deviations from the DoA, the consequences and the proposed
  % corrective actions

There was no major deviation from Annex 1. All deliverables due for M18 were delivered
  within the timeframe of the 1st Reporting Period, and all milestones in this period were
  reached.  Slight modifications were brought to \WPref{hpc} and \WPref{dksbases} and were
  included in the AMD-676541-13.

\subsection{Tasks}
% Include explanations for tasks not fully implemented, critical objectives not fully
% achieved and/or not being on schedule.  Explain also the impact on other tasks on the
% available resources and the planning

No deviation from the tasks. All workplan is on time at the end of the Reporting
Period.

However, there were four deliverables that
were handed in late. We will explain the situation and implications in
each case. Therefore, administratively speaking, Milestone 2
(Implementations), originally due on Month 24, was only reached in
Month 36, though with little, if any, consequences on the project as a
whole.

\subsubsection{\protect\delivtref{dissem}{ils-tool}}
This deliverable, a tool for publishing and organizing curated
collections of Jupyter notebooks, was delivered in month 36, a delay
of 12 months after the initially planned schedule.

The initial plan had been to build upon an already existing prototype,
whose development had started at a \Sage meeting back in 2015. The
original tool was specific to \Sage, but broader in scope and not tied
to Jupyter. Given the constantly evolving context, it quickly became
apparent that this tool did not properly address the community needs.
We thus took on evaluating other available open source solutions,
which considerably delayed the deliverable.

As we were not able to find an appropriate solution to build upon, we
finally decided to bootstrap a new project, called
\emph{planetaryum}. Once the development of \emph{planetaryum}
started, we were able to complete the version 0.1 in the planned
time frame.

Since other solutions were available before \emph{planetaryum}, albeit
less powerful, the delay in the deliverable did not impact other tasks
in the project.


\subsubsection{\protect\delivtref{UI}{ipython-kernels}}

This deliverable of full-featured kernels for GAP, Pari, Singular, etc.  has been
delivered in month 36 after a delay of 12 months.  The initial plan was for delivery in
Month 24, but was delayed to ensure a high quality of the delivered software, as more
time-sensitive resources were directed away from this task during months 12-24.  The delay
had no impact beyond the deliverable itself, as no other tasks relied strongly on this
deliverable being ready, and the result is a much stronger collection of Jupyter kernels
for mathematical software.

\subsubsection{\protect\delivtref{UI}{sage-sphinx}}

This deliverable is delivered in month 36, a delay of 12 months after the initial schedule
of month 24.  Progress was slower than planned, due to the nature of coordinating with
large software collaborations.  Additionally, work was shifted to other tasks during early
stages, resulting in the delay of this deliverable.  There have been no negative
consequences of the delay, as its delivery was not a prerequisite for other tasks.  As a
result of the delay, we have delivered much greater work than initially planned, including
significant improvements to the Sphinx documentation system itself used by projects all
over the world, and an Enhancement Proposal to improve the Python language itself,
ensuring wide impact for this work.


\subsubsection{\protect\delivtref{dksbases}{psfoundation}}
This deliverable report was delayed, as we found it useful to extend the scope of the work
reported from just the format of the interface theories and aligments -- these are
extensively discussed in the report as well -- to a full account of the Math-in-the-Middle
interoperability paradigm for \pn and discuss two full-scale use cases. It just made more
sense to deliver this report together with D6.8 (the resources for the use cases) for
Milestone M9 \emph{First Math-In-The-Middle-based interoperability prototype} in month 36,
in particular, since the delay of D6.5 did not delay the research and development in WP6
(after all, an earlier version of much of the content of D6.5 has been pre-published
as~\cite{WieKohRab:vtuimkb17,KohMuePfe:kbimss17} near the original deadline of D6.5 and
was therefore available to the \pn partners). 

This way D6.5 can serve as as reference for opening the MitM paradigm to outside users. We
are currently working on a high-visibility Journal publication based on D6.5 and D6.8
(presumably Journal of Symbolic Computation). 

\subsection{Use of resources}
% Include explanations on deviations of the use of resources between actual and planned
% use of resources in Annex 1, especially related to person-months per work package.

All changes of use of resources were included in the two amendments previously cited and were
due to modifications in the personnel. Those adjustments were due to the change of positions
of some key \ODK participants and expected difficulties in hiring planned
staff. The work plan has been updated accordingly, with no foreseeable
impact on the achievement of tasks, deliverables, and milestones.

Another minor deviation in the proposed use of resource was that FAU hired students to do
some routine jobs (simple formalizations, and the creation of alignments in WP6 and the
creation of example documents in WP4) that did not require the attention of a mature
researchers. As the pay grade of student assistant is roughly 1/4 of that of full
researchers, this action was cost-effective. An unplanned effect was that the reported
person months went up considerably, exceeding the planned amount, without incurring
additional cost. 

\subsubsection{Unforeseen subcontracting (if applicable)}

Not applicable.
% Specify in this section: a) the work (the tasks) performed by a subcontractor which
  % may cover only a limited part of the project; b) explanation of the circumstances
  % which caused the need for a subcontract, taking into account the specific
  % characteristics of the project; c) the confirmation that the subcontractor has been
  % selected ensuring the best value for money or, if appropriate, the lowest price and
  % avoiding any conflict of interests

\subsubsection{Unforeseen use of in kind contribution from third party against payment or
  free of charges (if applicable)}

 Not applicable. 
 % Specify in this section: d) the identity of the third party; e) the resources made
  % available by the third party respectively against payment or free of charges f)
  % explanation of the circumstances which caused the need for using these resources for
  % carrying out the work.

\clearpage
\printbibliography

\end{document}

%%% Local Variables:
%%% mode: latex
%%% TeX-master: t
%%% End:

%  LocalWords:  maketitle githubissuedescription newpage newcommand xspace Jupyter dissem
%  LocalWords:  tableofcontents visualizations composability itemize analyzed taskref hpc
%  LocalWords:  dissemination-of-oommf-nb-virtual-environment taskref dissem taskref pn
%  LocalWords:  dissemination-of-oommf-nb-workshops dissem ibook taskref taskref taskref
%  LocalWords:  oommf-python-interface oommf-py-ipython-attributes taskref oommf-nb-ve
%  LocalWords:  oommf-tutorial-and-documentation taskref oommf-nb-evaluation taskrefs
%  LocalWords:  delivref pythran-typing sage-paral-tree subsubsection organized Dagstuhl
%  LocalWords:  co-organized organization modularization ipython-kernels nbdime Pythran
%  LocalWords:  jupyter-collab ystok WPref dksbases compactitem emph WPtref DehKohKon
%  LocalWords:  iop16 textbf tasktref lfmverif triformal formalized biformal ossp09 Dima
%  LocalWords:  hline Marijan Pilorget Pierrick Kruppa Dehaye Dehaye's Dehaye's Alnaes
%  LocalWords:  Konovalov Hinsen github printbibliography enlargethispage
