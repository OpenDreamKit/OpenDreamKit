  \subsubsection{WorkPackage 3:  Component Architecture}
%Explain, task per task, the work carried out in WP during the reporting period giving details of the work carried out by each beneficiary involved.

\TODO{Update for Reporting Period 2}

%%%%%%%%%%%%%%%%%%%%%%%%%%%%%%%%%%%%%%%%%%%%%%%%%%%%%%%%%%%%%%%%%%%%%%%%%%%%%%
\paragraph{Overview}

%%%%%%%%%%%%%%%%%%%%%%%%%%%%%%%%%%%%%%%%%%%%%%%%%%%%%%%%%%%%%%%%%%%%%%%%%%%%%%
\paragraph{Milestones}

\subparagraph{\longmilestoneref{component-architecture-distribution}}

\emph{“User story: users shall be able to easily install ODK's
    computational components on the three major platforms (Windows,
    Mac, Linux) via their standard distribution channels.”}

%%%%%%%%%%%%%%%%%%%%%%%%%%%%%%%%%%%%%%%%%%%%%%%%%%%%%%%%%%%%%%%%%%%%%%%%%%%%%%
\paragraph{Tasks}

  \paragraph{\longtaskref{component-architecture}{portability}}
  \label{component-architecture@portability}
  The first task of this workpackage is to improve the portability of
  computational components. A particular challenge is the portability
  of \Sage\ (and therefore all its dependencies) on Windows, which has
  remained elusive for a decade, despite many efforts of the
  community.

  No deliverable is due for the evaluation period, but we are happy to
  report that Erik Bray of \site{PS} has made considerable progress on
  \longdelivref{component-architecture}{portability-cygwin} by producing a
  one-click Windows installer based on Docker. Although the Docker
  based installer has proven itself a viable solution, our plan is
  still to deliver a Cygwin-based installer on month 24. This project
  is at a very advanced stage: we have a patched version of \Sage
  compiling and runnning in Cygwin, and a beta version of the
  installer was made available in December. We expect to deliver
  \delivref{component-architecture}{portability-cygwin} on time.

  \paragraph{\longtaskref{component-architecture}{interface-systems}}
  \label{component-architecture@interface-systems}
  In this task we investigate patterns to share data, ontologies,
  and semantics across computational systems, possibly connected
  remotely.

  The
  \href{http://www.symbolic-computing.org/science/index.php/SCSCP}{Symbolic
    Computation Software Composability Protocol (SCSCP)} is a remote
  procedure call protocol by which a computer algebra system (CAS) may
  offer services to a variety of possible clients, including e.g.\
  another CAS running on the same computer system or remotely. The
  goal of \delivref{component-architecture}{scscp-sage} was to bring
  SCSCP support to all relevant components of \ODK. Thanks to the
  joint efforts of \site{SA}, \site{UV}, \site{JU}, \site{UG} and
  \site{UB}, SCSCP is now supported in \GAP, \Sage and \MathHub, and
  plans have been made to extend support to \Singular and \PariGP at
  an appropriate time.

  This task benefits from the work done in
  Task~\taskref{UI}{pari-python} on low level interfaces to software
  components. In particular, the results of
  deliverables~\longdelivref{UI}{pari-python-lib1}
  and~\longdelivref{UI}{pari-python-lib2} will be crucial to supporting
  SCSCP in \PariGP.

  Furthermore, experimental work on a semantic interface between \GAP
  and \Sage
  (\delivref{component-architecture}{semantic-interface-sage-gap}, due
  on month 36) has started during the joint GAP-Sage days, and a
  working prototype is already available. The current prototype uses
  \emph{ad hoc} language mechanisms to transfer the semantics from one
  system to the other; these mechanisms will be replaced with a
  generic API (Application programming Interface) once the MitM
  (Math-in-the-Middle) approach developed in WP6 will be mature
  enough.

  \paragraph{\longtaskref{component-architecture}{mod-packaging}}
  \label{component-architecture@mod-packaging}
  In this task we investigate best practices for composing, sharing
  and interfacing computational components and data for connected
  mathematical systems.

  The first deliverable,
  \longdelivref{component-architecture}{virtual-machines} was
  delivered on time early in the project, producing the expected
  results.

  A focused workshop in March 2016 (Sage Days 77, followed by Sage
  Days 85 in March 2017) also triggered much work and progress on the
  packaging side, both by \ODK participants and the community. Partly
  thanks to this, \Sage is now on its way to be available as a
  standard package in major distributions such as Debian and
  Anaconda. We stress the fact that this task is very much ahead of
  schedule: having \Sage in Debian was the goal set for
  \delivref{component-architecture}{sage-distribution}, which is only
  due in month 48.

  The workshops were also the occasion to clarify the modularization,
  packaging, and distribution needs and challenges. This has paved the
  way for \longdelivref{component-architecture}{sage-repository}, due
  in month 24.

  \paragraph{\longtaskref{component-architecture}{simulagora-dev}}
  The goal of this task is to deliver every six months a new Simulagora
  VM image containing all the software components released over the
  period.

  To this date, three OpenDreamKit VMs have been released in
  Simulagora.

  \paragraph{\longtaskref{component-architecture}{component-for-HPC}}
Not applicable for this period.

  \paragraph{\longtaskref{component-architecture}{extract-smc}}
  \label{component-architecture@extract-smc}
  From its inception in 2013, \SMC\ has quickly developed into a full
  featured VRE.  Because of the tight
  development cycles, it is quite difficult to keep track of \SMC's
  structure and goals.  The goal of this task is to participate in the
  evolution of \SMC, by helping with documentation and
  interoperability.

  The first deliverable of the task,
  \longdelivref{component-architecture}{smc-documentation} was delayed
  because of the slow recruitment process in \site{PS}. We were nevertheless able
  to catch up on time for the first reporting period, the result being
  a remarkable
  contribution\footnote{\url{https://github.com/sagemathinc/smc/blob/master/src/doc/design_overview/overview.rst}}
  by Erik Bray to the internal documentation of \SMC.

  On the other hand,
  \longdelivref{component-architecture}{personal-smc}, due in month
  24, was achieved by the \SMC developers \emph{before the start of
    \ODK}. We are currently evaluating the most useful way to
  re-allocate the planned effort.

  \paragraph{\longtaskref{component-architecture}{workflow}}
  This task seeks new ways of accepting contributions to mathematical
  software in a scalable way.  No deliverable is due for the
  evaluation period.

  \paragraph{\longtaskref{component-architecture}{oommf-python-interface}}
  \label{component-architecture@oommf-python-interface}
  This task provides a Python interface to the Object Oriented
  Micromagnetic Framework (OOMMF). This allows to access the
  capabilities of this package as one component of a virtual research
  environment, together with the existing ecosystem of scientific
  python libraries and tools. The task has been completed \cite{Beg2017a}, and the
  resulting software is available online on
  GitHub\footnote{\url{https://github.com/joommf/oommfc}} and through
  the Python packaging index under the name \texttt{oommfc}. The
  interface has been presented to users in the micromagnetic community
  through our dissemination workshops
  (\taskref{dissem}{dissemination-of-oommf-nb-virtual-environment}). This
  task is part of a number of steps towards a virtual research
  environment for micromagnetic simulations
  (\delivref{dissem}{oommfnb-vre-deliver}).

%%% Local Variables:
%%% mode: latex
%%% TeX-master: "report"
%%% End:
