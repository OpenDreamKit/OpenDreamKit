\subsubsection{Work Package 1: Project Management}

%Explain, task per task, the work carried out in WP during the reporting period giving details of the work carried out by each beneficiary involved.

\TODO{Merge in the latest updates by Benoît from the commented out text below
  (commits around 13265449a1f84bdd6d56ce152d2b5712c724e2da)}

% \subsubsection{WP1: Project management}

% As planned in WP1, \site{PS} has been coordinating \ODK.  Most of the
% management effort for years 2 and 3 has been made in
% \longtaskref{management}{project-finance-management} (see
% \delivref{management}{data-plan2}).

% The coordinator organised the Project review with the Funding Agency. It made sure that all deliverables were delivered and milestones reached on time so that the results could be presented to the Project officer and the reviewers.
% Some objectives as well as participants evolved during years 2 and 3, which forced the coordinator to negotiate 3 amendments to the contract:
% - to terminate JacobsUni and Southampton as beneficiaries and add FAU and XFEL to the consortium: this is due to the move of key participants to the project
% - to cancel Work Package 7: Social aspects
% - to terminate the University of Sheffield and add the University of Leeds to accomodate the move of key participants to the project.


% We also organised to project meetings: one that was remote in February 2018 and a another one in June 2018 which took place at the XFEL premises.
% Concerning the future of OpenDreamKit and of its software that the project is developing, the coordinator took action to access information and get involved in the development of the European Open Science Cloud that is currently promoted by the European Commission. We took advantage of the EOSC stakeholder forum on 28-29 November and the 2017 edition of the DI4R (Digital Infrastructures for Research) in Brussels. During these events we gathered information on the potential of EOSC and how \ODK could fit in there. Furthermore we began a partnership with EGI which is a key participant to EOSC.

% More information on
% \longtaskref{management}{project-quality-management} can be found in
% Section 4 of this document: Quality assurance plan.
% Task \longtaskref{management}{project-innovation-management} will provide
% further detail and will be available at month 18.

%%%%%%%%%%%%%%%%%%%%%%%%%%%%%%%%%%%%%%%%%%%%%%%%%%%%%%%%%%%%%%%%%%%%%%%%%%%%%%
\paragraph{Overview}

The general objectives of Work Package 1 are:

\begin{enumerate}
\item{Meeting the objectives of the project within the agreed budget and timeframe and carrying out control of the milestones and deliverables}
\item{Ensure all the risks jeopardising the success of the projects are managed and that the final results are of good quality}
\item{Ensuring the innovation process within the project is fully aligned with the objectives set up in the Grant agreement}
\end{enumerate}

\WPref{management} has been divided into three tasks. In the following, progress is reported with respect to these individual tasks.
Key results of WP1 since the beginning of the project include:
\begin{enumerate}
\item The management of Consortium Agreement, including three
  amendments, and a work plan revision process.
\item The organization of project meetings and bi-yearly steering
  committee meetings.
\item The organization of an interim review at month 9 and a formal
  review at month 18.
\item The setting up of a new version of the \ODK website, with a more end-user friendly interface.
\item The setting up and interaction with an Advisory Board and a
  Quality Review Board to control the quality and the relevance of the
  software development relative to the end-user needs.
\item All milestones have been reached and deliverables achieved within the second Reporting Period timeframe.
\item Continued success in the recruitment of highly qualified staff.
\end{enumerate}

Concerning the recruitment: the strategies we used (tailoring of the
positions according to the known pool of potential candidates, in
particular among previous related projects, strong advertisement, ...)
seem to have paid off, and we are really happy with the top notch
quality of our recruits. However, despite many steps to foster women
applications to apply (e.g. through reaching personally toward
potential candidates or including women in the committees), we had
almost no female candidate, and none made it to the short list. This
is alas unsurprising in the very tight segment of experienced research
software engineers for mathematics on temporary positions which is
highly gender imbalanced; this is nevertheless a failure.

%%%%%%%%%%%%%%%%%%%%%%%%%%%%%%%%%%%%%%%%%%%%%%%%%%%%%%%%%%%%%%%%%%%%%%%%%%%%%%
\paragraph{Tasks}

\subparagraph{\longtaskref{management}{project-finance-management}}

As planned in WP1, \site{PS} has been coordinating \ODK and took care
of the budget management together with the administration body, the
D.A.R.I. (Direction des Activités de Recherche et de l'Innovation) and
its finance service.

During the second reporting period, \site{PS} lead the 2nd, 3rd and
4th amendment to the consortium, in particular to cater for the moving
of permanent and/or non-permanent researchers who are key personnel
for the success of \ODK. A collateral effect is the termination of the
sites \site{USH}, \site{USO}, \site{JU}, \site{ZH} since no relevant
staff for \ODK remained at these institution. Following the
recommendations from the reviewers at the end of the first reporting
period, \site{PS} also lead the work plan revision process that were
implemented in the 3rd amendment.

\site{PS} also coordinated regular project meetings, including two
steering committee meetings organized in Brussels and online.

Concerning the communication, \site{PS} as been maintaining the intern
communication tools were described in
\longdelivref{management}{infrastructure}. As for external
communication the website for the project has been continuously
updated with new content, and virtually all work in progress is openly
accessible on the Internet to external experts and contributors (for
example through open source software on Github). A new version of the
website were released. Its end-user friendly interface and content
makes it a tool not only for internal communication but very much for
dissemination and progress tracking by the reviewers and the
community.

Finally \site{PS} produced a second version of the Data Management
Plan (\delivref{management}{data-plan2}).


\begin{oldpart}{update to reporting period 2; all of this mainly concentrates on RP1 achievements}
\subparagraph{\longtaskref{management}{project-quality-management}}


The Quality Assurance Plan is described in detail in
\longdelivref{management}{ipr}. We will describe the main points
below.  \site{PS} launched a Quality Review Board which is chaired by
Hans Fangohr. The four members of the board have a track record of
caring about the quality of software in computational science. This
board is responsible for ensuring key deliverables do reach their
original goal and that best practice is followed in the writing
process as well as in the innovation production process.  The board
will meet after the end of each Reporting Period (RP), and before the
Review following that RP.

The other structure supporting \ODK to ensure the quality of the
infrastructure is the End-user group that is composed of some members
of the Advisory Board. It is composed of seven members:

\begin{itemize}
\item{Lorena Barba from the George Washington University}
\item{Jacques Carette from the McMaster University}
\item{Istvan Csabai from the Eötvös University Budapest}
\item{Françoise Genova from the Observatoire de Strasbourg}
\item{Konrad Hinsen from the Centre de Biophysique Moléculaire}
\item{William Stein, CEO of SageMath Inc.}
\item{Paul Zimmermann from the INRIA}
\end{itemize}

This Advisory Board being composed of Academics and/or software
developers from different backgrounds, countries and communities, it
will be a strong asset to understand the needs of a variety of
end-user profiles. This Technical Report for the first Reporting
Period will be the first occasion to ask for their feedback on the
potential of the VRE and our strategy to promote its use around the
world. According to the Consortium Agreement, all Advisory Board
members have signed a lightweight Non-Disclosure Agreement with the
consortium.

\site{PS} has also been managing risks. In \delivref{management}{ipr}
all potential risks were assessed by the Coordinator at Month 12. Here
is a brief update on Risk 1 concerning the recruitment of highly
qualified staff. This risk has been globally well managed thanks to a
flexible workplan enabling adjustments in the timing of some tasks or
deliverables, and thanks to legal actions taken by the Coordinator to
allow key personnel, permanent or not, to remain in the Consortium
even though their positions changed. The addition of the three
partners is representative of these actions. The assessment for the
other risks remain valid at Month 18, and we refer to
\delivref{management}{ipr} for details.

\subparagraph{\longtaskref{management}{project-innovation-management}}

\longdelivref{management}{imp1} was produced at month 18 and is mainly focused on:

\begin{itemize}
\item{The open source aspect of the innovation produced within \ODK}
\item{The various implementation processes the project is dealing with}
\item{The strategy to match end-users needs with the promoted VRE}.
\end{itemize}

  The second version of the Innovation Management Plan will add content to explain all the
  innovations that the VRE is bringing to end-users. However the open source approach and
  the "by users for users" development process will not change.  One
  of the assessed risks for \ODK is to have different groups not forming effective teams. Put
  in other words, having developers of the different pieces of software working solely for
  the benefit of the programme they were initially working on and for. This risk is
  tackled by the Coordinator in order to reach the final goals of the VRE which are the
  unification of open source tools with overlapping functionality, the simplification of
  the tools for end-users without coding expertise, and the development of user-friendly
  interfaces. For this, the Scientific Coordinator is for example wilfully pushing
  for joint actions and
  workshops. Even if it takes time to bend some of the old implementation processes and coding habits,
  more and actions are taken by \ODK participants from different communities to work
  together. More information on joint workshops can be found in the section below.
\end{oldpart}

%%% Local Variables:
%%% mode: latex
%%% TeX-master: "report"
%%% End:

%  LocalWords:  subsubsection longtaskref delivref organized Bougeret ipr Csabai
