\documentclass{deliverablereport}

\deliverable{management}{tickets}
\duedate{28/02/2017 (M18)}
\deliverydate{??/??/2017}

\usepackage[style=alphabetic,backend=bibtex]{biblatex}
\addbibresource{../../lib/kbibs/kwarcpubs.bib}
\addbibresource{../../lib/kbibs/extpubs.bib}
\addbibresource{../../lib/kbibs/kwarccrossrefs.bib}
\addbibresource{../../lib/kbibs/extcrossrefs.bib}
\addbibresource{../../lib/deliverables.bib}
% temporary fix due to http://tex.stackexchange.com/questions/311426/bibliography-error-use-of-blxbblverbaddi-doesnt-match-its-definition-ve
\makeatletter\def\blx@maxline{77}\makeatother

\author{Nicolas M. Thiéry, Benoît Pilorget}

\begin{document}
\enlargethispage{4ex}
\maketitle
\githubissuedescription
\tableofcontents\newpage


\section{Introduction}

OpenDreamKit (Open Digital Research Environment Toolkit) is a four-year Horizon 2020 European Research Infrastructure project (\#676541) that will run for four years. It provides substantial funding to the open source computational mathematics ecosystem, and in particular popular tools such as LinBox, MPIR, SageMath, GAP, PARI/GP, LMFDB, Singular, MathHub, and the IPython/Jupyter interactive computing environment.

From this ecosystem, OpenDreamKit will eventually deliver a flexible toolkit enabling research groups to set up Virtual Research Environments (VRE), customised to meet the varied needs of research projects in pure mathematics and applications, and supporting the full research life-cycle from exploration, through proof and publication, to archival and sharing of data and code. 
This means the potential end-user, meaning the person that the software and/or tool is designed for, is a person in need of a toolkit to optimise their work process. This person can be a researcher (whether academic or not), a professor from any field related to mathematics or using mathematics, etc. As OpenDreamKit is promoting an opensource VRE there is no discrimination as to whom will be allowed to use the latter. It is therefore possible that Small and Medium Enterprises also become end-users.

However potential end-users are more likely to be people from the scientific community. The main reason for that being the developers of the VRE come themselves from the scientific community. Most of the OpenDreamKit participants are indeed academics who often have several hats (developer, mathematician, computing scientist etc.) and who are willing to develop the tools they need for themselves to simplify their work. For that reason OpenDreamKit is unique in the fact the software developers are end-users as well, with a "by users for users model". The software developers/end-users are also connected to the opensource software communities that they are improving and developing. Therefore some innovations brought to the toolkit are partially or entirely accomplished outside of the OpenDreamKit project. This is not a problem in itself as the communities and the project share the same values and objectives.
That being said, targeted end-users comprise the software developer communities but not only, the VRE will also benefit to end-users who are not experts in the field and who simply need tools to enhance their results and improve their processes.

Once the OpenDreamKit project ends, end-users will benefit from innovations in the multiple pieces of software forming the opensource VRE. In the case of OpenDreamKit the innovation will be the unification of opensource tools with overlapping functionality, the simplification of the tools for end-users without coding expertise, and the development of user-friendly interfaces.

The following document will first explain the innovations that OpenDreamKit will bring to end-users, then explain the processes enabling the implementation of the said innovations, and finally will explain how OpenDreaMkit will target its end-users.


\section{Innovations OpenDreamKit is bringing to the research community}

\begin{itemize}
\item{blabla}
\item{blablabla}
\end{itemize}


\section{Implementation processes of the innovations}

Description of the collaboration and communication between developers when improvements are brought.



\section{The end-users targeted at by innovations}

As it was previously said, the originality of OpenDreamKit is that the VRE is its "by users for users" model. Indeed, OpenDreamKit's participants have many hats: they are mathematicians computer scientists, developers, researchers, professors, end-users and sometimes all of this at the same time. 

But since OpenDreamKit is promoting opensource, it is aiming at reaching out as many end-users as possible. In order to do that OpenDreamKit comprises two boards whi can help reach out end-users outside the developer communities.

\subsection{Boards within OpenDreamKit}

\subsubsection{Quality-Review board}

The Quality Review board is composed of four members: Hans Fanghor (chair), Alexander Konovalov, Konrad Hinsen and Mike Croucher. They are all turned towards the end-user needs and towards the development of opensource research software.
Some of their objectives are to:

\begin{itemize}
\item{emphasis the quality of the software engineering aspects}
\item{identify best practice}
\item{improve future work within OpenDreamKit and to disseminate knowledge to a wider audience, i.e. potential end-users.}
\end{itemize}
  


\subsubsection{The Advisory board and its end-user group}

\subsection{End-users targeted by OpenDreamKit innovations}



In order to attract new users, OpenDreamKit has been organising workshops and schools in Europe, South-America and in Africa. These activities are organised within the Work Package 2 about dissemination. (cite deliverable about these workshops). All the pieces of software improved and developed by the project such as /Sage, /Singular, /ETC are presented during these schools and workshops targeted at students and potential new users. Of course this software being originally developed for end-users who are potentially very skilled in computing and software development, one of the goals is to make it easy  for unskilled end-users to start using our opensource toolkit. 
For example, one of the main objectives in the development of /Sage in the attraction of new end-users is to develop a !!!!Windows docker!!!!!. With such a tool, the toolkit will enlarge its target to all Windows users. 

Conferences for end users
Tutorials


End-users are developing their own tools. But other users outside of the community could be targeted
\printbibliography

\end{document}

%%% Local Variables:
%%% mode: latex
%%% TeX-master: t
%%% End:

%  LocalWords:  maketitle githubissuedescription newpage newcommand xspace Jupyter dissem
%  LocalWords:  tableofcontents visualizations composability itemize analyzed taskref hpc
%  LocalWords:  dissemination-of-oommf-nb-virtual-environment taskref dissem taskref pn
%  LocalWords:  dissemination-of-oommf-nb-workshops dissem ibook taskref taskref taskref
%  LocalWords:  oommf-python-interface oommf-py-ipython-attributes taskref oommf-nb-ve
%  LocalWords:  oommf-tutorial-and-documentation taskref oommf-nb-evaluation taskrefs
%  LocalWords:  delivref pythran-typing sage-paral-tree subsubsection organized Dagstuhl
%  LocalWords:  co-organized organization modularization ipython-kernels nbdime Pythran
%  LocalWords:  jupyter-collab ystok WPref dksbases compactitem emph WPtref DehKohKon
%  LocalWords:  iop16 textbf tasktref lfmverif triformal formalized biformal ossp09 Dima
%  LocalWords:  hline Marijan Pilorget Pierrick Kruppa Dehaye Dehaye's Dehaye's Alnaes
%  LocalWords:  Konovalov Hinsen github printbibliography
