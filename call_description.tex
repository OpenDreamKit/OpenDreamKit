This is a synthesis of Antonios Barbas' slides describing the Call 3
Topic 9-2015 EINFRA-9: e-Infrastructure for Virtual Research Environment

See file:../Documentation/VirtualEnvironmentsWorkProgramme2014-2015.ppt

** Suggested EU contribution per proposal: 2 to 8 M€ ; Total budget: 42 M€
** Dates: 14/01/2015
- H2020-EINFRA-2014-1 15/04/2014
- H2020-EINFRA-2014-2 02/09/2014
- H2020-EINFRA-2015-1 14/01/2015(tbc)
** European contacts: Antonios Barbas
   See file:Documentation/Contacts.docx
** Definition:
- Groups of researchers, typically widely dispersed, who are working
  together
- through ubiquitous, trusted and easy access to services for
  scientific data, computing and networking
- in a collaborative, virtual environment:
  > the e-Infrastructures
** Characteristics:
- Address the needs of specific scientific communities – in support of
  e-Science;
- Have users from both academia and industry;
- Involve bottom-up research and develop user-oriented services;
- Are based on e-infrastructures
** Specific challenge:
- Capacity building in interdisciplinary research
- through community-led development and deployment of service-driven
  digital environments
- for large-scale cross-disciplinary research collaboration and data
  interoperability
** Expected impact:
- More effective collaboration between researchers and increased
  take-up of collaborative research by new disciplines;
- Easier discovery, access and re-use of data, resulting in higher
  productivity of researchers;
- Accelerate innovation via access to integrated digital research
  resources across disciplines;
*** Scope: Proposals are expected to
Notations: [X]: easy to argue; [?]: we have some lead, but that will take some arguing
- [?] Integrate resources across all layers of the e-infrastructure
  (networking, computing, data, software, user interfaces) to foster
  cross-disciplinary data interoperability
- [?] Build on requirements from real use cases, i.e. integrate
  heterogeneous data from multiple sources and re-use tools and
  services from existing infrastructures
- [X] Target any area of Science and Technology, especially
  interdisciplinary ones, including ICT, mathematics, web science and
  social sciences and humanities
- [X] Use standardised building blocks and workflows, well-documented
  interfaces and interoperable components;
- [?] Define semantics, ontologies and metadata to enable data citation
  and promote data sharing, as to ensure interoperability;
- [X] Target easy-to-use functionalities; and indicate the number of
  researchers they target as potential users;
** Specific conditions for the Call on e-Infrastructures:
- [X?] Proposals should be structured around Networking, Service
  and Joint Research Activities
- [X] The Software to be developed needs to be open source
- [ ] A Data Management Plan to be developed enabling data preservation,
  on-line discoverability, authorisation and re-use of data
- [X] Clear Metrics (KPIs) to be proposed and used;
- [?] Open Access to Publications resulting from the project;
- [X] Usefulness of services to the end user community and
      financial sustainability to be ensured;
** Where should the emphasis be?
- [?] Services
- [X] Thinking innovation
      With both suppliers or users
- [X] Mainstreaming skills development
- [ ] Integration between data and computing
- [X] Business plans for financial sustainability
  ...and partnerships with the private sector
- [ ] Supporting policies
- [X] open data and software
- [X] Sharing basic operations services and building blocks
- [X] Monitoring performance (KPIs)
** Simplified funding model
   - Up to 100% for Research and Innovation
     - Flat 25% rate for indirect costs (overhead?)
