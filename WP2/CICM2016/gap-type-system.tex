\subsection{Exploring GAP types}

\subsubsection{Brief introduction to the GAP type system.}\label{gap-types-intro}

\ednote{TODO: This is just the citation from \url{http://www.gap-system.org/Manuals/doc/ref/chap13.html}}

``Every GAP object has a type. The type of an object is the information 
which is used to decide whether an operation is admissible or possible 
with that object as an argument, and if so, how it is to be performed.

For example, the types determine whether two objects can be multiplied 
and what function is called to compute the product. Analogously, the 
type of an object determines whether and how the size of the object 
can be computed. It is sometimes useful in discussing the type system, 
to identify types with the set of objects that have this type. Partial 
types can then also be regarded as sets, such that any type is the 
intersection of its parts.

The type of an object consists of two main parts, which describe 
different aspects of the object.

The family determines the relation of the object to other objects. 
For example, all permutations form a family. Another family consists 
of all collections of permutations, this family contains the set of 
permutation groups as a subset. A third family consists of all 
rational functions with coefficients in a certain family.

The other part of a type is a collection of filters (actually stored 
as a bit-list indicating, from the complete set of possible filters, 
which are included in this particular type). These filters are all 
treated equally by the method selection, but, from the viewpoint of 
their creation and use, they can be divided (with a small number of 
unimportant exceptions) into categories, representations, attribute 
testers and properties. Each of these is described in more detail below.

A discussion of the type system can be found in \cite{breuer-linton}
(\url{http://dl.acm.org/citation.cfm?id=281540})''

For example, in GAP \texttt{IsGroup} is a filter defined as 
\texttt{IsMagmaWithInverses and IsAssociative}. Out of these
two, \texttt{IsMagmaWithInverses} is a category defined as an intersection
of categories \texttt{IsMagmaWithInversesIfNonzero} and
\texttt{IsMultiplicativeElementWithInverseCollection}, 
and \texttt{IsAssociative} is a property defined for objects
in the category \texttt{IsMagma}. Furthermore, \texttt{IsMagma} is
an intersection of categories \texttt{IsDomain} and 
\texttt{IsMultiplicativeElementCollection}.

As another example, \texttt{IsSet} is a synonym for a property \texttt{IsSSortedList} 
defined for objects in the category \texttt{IsList}, and \texttt{IsFinite} 
is a property defined for objects in the category \texttt{IsCollection} 

%GAP example:
%gap> IsGroup;
%<Filter "(IsMagmaWithInverses and IsAssociative)">
%gap> IsMagmaWithInverses;
%<Category "IsMagmaWithInverses">
%gap> IsAssociative;
%<Property "IsAssociative">
%gap> IsSet;
%<Property "IsSSortedList">
%gap> IsFinite;
%<Property "IsFinite">
%gap> IsSet=IsSSortedList;
%true

\ednote{NT: It could be informative to add an example of category and 
property. For example Magma and Associative. Or Set and Finite. Ideally, 
we would use the same example in the GAP and Sage section.}

\ednote{TODO (???) Compare and contrast it with the Sage type system}

\subsubsection{Tentative approaches to exporting GAP types.}\label{gap-types-export}

GAP allows to accessed programmatically information about types of all 
objects that are defined in the GAP workspace after the system is loaded.
Having a clear picture of the relations between different objects is 
very helpful to GAP developers, package authors and users, who may be 
interested in questions like, for example, 
``Which methods are installed for this particular operation'' or 
``Which attributes a group may have?''. 

We are developing tools to access and export this information in several possible ways:
\begin{itemize}
\item presented immediately in the GAP session
\item exported to a JSON file to be used by other software systems 
such as, for example, MMT or SageMath
\item visualised for better understanding and exploration
\end{itemize}
\ednote{picture based on \url{https://github.com/OpenDreamKit/OpenDreamKit/issues/165}?}

At this stage, during GAP-SageMath Days and OpenDreamKit workshop in St Andrews 
in January 2016 we have produced prototype implementations. We plan to make them
more user-friendly to be ready for the inclusion in the next major release of GAP. 
We hope that they will be also enhanced by the GAP Jupyter interface developed in 
\url{https://github.com/gap-packages/jupyter-gap}.

\ednote{NT: Do you have anything to say about the GAP-MMT formalization? 
E.g. hints on potential ways this formalization may be written?}

\subsubsection{An application: consistency checker for the GAP documentation.}\label{gap-types}

One of the immediate outcomes of the development of the tools described in the
previous section is the consistency checker for the GAP documentation. 

GAP uses special format for its main manuals. It is called GAPDoc and is 
provided by the GAP package with the same name \cite{gapdoc}. Besides main 
manuals, it is adopted by 97 out of 130 packages currently redistributed 
with GAP. Using GAPDoc, one builds text, PDF and HTML versions of the manual
from a common source given in XML.

GAPDoc defines XML constructions to specify the type of the documented object 
(function, operation, attribute, property, etc.). However, due to the 
limitations of the semi-automated conversion of GAP manuals from the \TeX-based
manuals used in GAP 4.4.12 and earlier, a number of objects had their types
stated incorrectly. 

We developed the consistency checker for the GAP documentation, which extracts
type annotations from the documented GAP objects and compares them with their
actual types. It immediately reported almost 400 inconsistencies out of 3674 
manual entries. In the subsequent cleanup, we by now have eliminated about 
75\% of them. The  consistency checker will appear in the next release of
GAP 4.8.3, and will be available via \texttt{make check-manuals}.
It also performs other useful checks: for example, it produces a list of
manual sections having no examples. Thus, the new tool helps to improve
the quality of GAP documentation, and may be useful for the similar checks
of those GAP packages which use GAPDoc-based manuals.

% \url{https://github.com/gap-system/gap/pull/675}
% \url{https://github.com/gap-system/gap/pull/538}

%%% Local Variables:
%%% mode: latex
%%% TeX-master: "paper"
%%% End:
