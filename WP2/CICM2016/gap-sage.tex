\section{Distributed Collaboration with GAP/Sage}\label{sec:gapsage}
\begin{newpart}{MK@MK: completely rewrite from a high-level perspective the material that
    is now in \texttt{deleted-scenes.tex} and should be extended to a technical paper}
Another aspect of interoperability in a mathematical VRE is the possibility of distributed
computation, where one the systems delegates a sub-computation to a specialist system or
multiple system share components of a larger computation or reasoning task.

There are already a variety of peer-to-peer integrations (see Figure~\ref{fig:interop})
based on the ``handle paradigm'' between systems in the \ODK project, e.g. between \Sage a
the master and \GAP, \Singular, and \Pari. 

In the ``handle paradigm'', when a system $A$ delegates a calculation to a system $B$, the
result $r$ of the calculation is not converted to a native $A$ object; instead $B$ just
returns a handle (or reference) to the object $r$. Later $A$ can run further calculations
with $r$ by passing it as argument to $B$ functions or methods. The advantages of this
approach include that we can avoid the overhead of back and forth conversions between $A$
and $B$ and that we can manipulate objects of $B$ from $A$ even if they have no native
representation in $A$.

Given a mapping of corresponding methods in the systems, we can use the adaptor pattern to
implement this. For example, calling the method \texttt{h.cardinality()} on a \Sage handle
\texttt{h} to a \GAP object \texttt{G}, triggers in \GAP a call to \texttt{Size(G)} if
\texttt{cardinality} and \texttt{Size} are marked as corresponding. But this dispatch
depends on an alignment of the type systems in \Sage and \GAP. For example, if \texttt{h}
is a handle to a set \texttt{S}, \Sage only knows that \texttt{h.cardinality()} can be
computed by \texttt{Size(S)} in \GAP if \texttt{S} is a group; in fact if \texttt{h} has
been constructed through the \texttt{PermutationGroup} or \texttt{MatrixGroup}
constructors. Whereas we would want this method to be available as soon as \texttt{S} is a
set.

To get around this problem we have worked on a more semantic integration, where adaptor
methods are made aware of the type hierarchies of the respective other system, see
Listing~\ref{lst:adaptor} below. But for scaling this up to a mathematical VRE with a
douzen systems, we want to use the MitM paradigm.
 
\begin{lstlisting}[language=Python,label=lst:adaptor,
  caption=A Semantic Adaptor Mehthod in \Sage]
class Sets: # Everything about sets in Sage
    class GAP: # The adapter methods relevant to Sets in the Sage-Gap interface
         class ParentMethods: # Adapter methods for sets
             def cardinality(self): # The adapter for the cardinality method
                 return self.gap().Size().sage()
         class ElementMethods: # Adapter methods for set elements
             ...
         class MorphismMethods: # Adapter methods for set morphisms
             ...
\end{lstlisting}
The main problem now becomes in generating interface theories and interviews into the core
MitM ontology so that the adaptor pattern above still applies, but can be made generic in
terms of the MitM ontology structure instead of the concrete structure of the respective
type systems. In our example, the correspondence between \texttt{cardinality} and
\texttt{Size} still holds, if the MitM interviews link the \texttt{cardinality} function
in the \Sage interface theory on sets with the \texttt{Size} function in the corresponding
interface theory for \GAP.

We will now show first results of our experiments of generating interface theories that
support distributed computation for \Sage and \GAP. 
\end{newpart}

\subsection{Semantics in the \Sage Category System}

The \Sage library includes 40k functions and allows for manipulating thousands of
different kinds of objects. As usual in such large systems, it’s critical for taming code
bloat to
\begin{inparaenum}[\em i\rm)]
\item identify the core concepts describing common behavior among the objects; 
\item exploit this to implement generic operations that apply on all object having a given
  behavior, with appropriate specializations when performance calls for it.  
\item design or choose a process for selecting the best implementation available when
  calling an operation on one or several objects.
\end{inparaenum}
Following mathematical tradition, \Sage has developed a category-theory-inspired
``category system'' and found a way to implement it in terms of the underlying \Python
object system\ednote{MK@NT; cite your paper on this here}. This cagegory system moddels
taxonomic knowledge from mathematics explicity and uses it to support genericity, control
the method selection process, structure the code and documentation, enforce consistency,
and provide generic tests.

\begin{wrapfigure}r{7cm}\vspace*{-2.5em}
\begin{lstlisting}[language=Python]
@semantic(mmt="sets")
class Sets:
    class ParentMethods:
         @semantic(mmt="o", gap="Size")
         @abstractmethod
         def cardinality(self):
             r"""
             Return the cardinality of ``self``.
             """
\end{lstlisting}
\vspace*{-.5em}
\caption{An anotated Category in \Sage}\label{fig:anncat}\vspace*{-1.5em}
\end{wrapfigure}
To generate interface theories from the \Sage category system, we are experimenting with a
system of annotations in the \Sage source files. Consider for instance the situtation in
Figure~\ref{fig:anncat} where we have annotated the \texttt{Magmas()} category in \Sage
with \texttt{@semantic} lines that state correspondences to other interface theories. From
these the \Sage-to-MMT exporter can generate the respective interface theories and views.

Several variants of the annotations exist to allow for adding annotations on existing
categories without touching their file, and also for specifying directly the corresponding
method names in other systems when this has not yet been formalized elsewhere. Similarly,
one could provide directly the signature information in case that is not yet modelled in
MMT.

\subsection{Exporting the \GAP Type System}
\ednote{summarize the \GAP stuff here, also add the picture}

%%% Local Variables:
%%% mode: latex
%%% TeX-master: "paper"
%%% End:
