\documentclass{llncs}
\pagestyle{plain}
\usepackage[show]{ed}
\usepackage{listings}
% \usepackage{url}
% \usepackage{wrapfig}
% \usepackage{xspace}

\usepackage[style=alphabetic,backend=bibtex]{biblatex}
\addbibresource{kwarc.bib}
\def\pn{OpenDreamKit}
\title{The OpenDreamKit Project:\\
  Towards Enhanced Interoperability\\
  via a Math-in-the-Middle Approach}
\author{Nicolas M. Thi\'ery\inst{1} Michael Kohlhase\inst{2}}
\institute{Universit\'e Paris-Sud, Paris, France\and
Jacobs University, Bremen, Germany}

\begin{document}
\maketitle
\begin{abstract}
  OpenDreamKit --- ``Open Digital Research Environment Toolkit for the
  Advancement of Mathematics'' --- is an H2020 EU Research
  Infrastructure project that aims at supporting, over the period
  2015--2019, the ecosystem of open-source mathematical software
  systems, and in particular popular tools such as LinBox, MPIR,
  SageMath, GAP, Pari/GP, LMFDB, Singular, MathHub, and the
  IPython/Jupyter interactive computing environment. From that
  ecosystem, OpenDreamKit will deliver a flexible toolkit enabling
  research groups to set up Virtual Research Environments, customised
  to meet the varied needs of research projects in pure mathematics
  and applications.

  An important step in the OpenDreamKit endeavor is to foster the
  interoperability between a variety of systems, ranging from computer
  algebra systems over mathematical databases to front-ends.

  In this paper, we describe the OpenDreamKit project and report on experiments and future
  plans with the \emph{math-in-the-middle} approach.  This information architecture
  consists in a central mathematical ontology that documents the domain and fixes a joint
  vocabulary, combined with specifications of the functionalities of the various
  systems. Interaction between systems can then be enriched by pivoting off this
  information architecture.
\end{abstract}

\section{Towards a Mathematical Virtual Research Environment}
The OpenDreamKit project aims to build a nimble and modular toolkit for computational
mathematics, built around open source software. This would enable easy set up of custom
collaborative environments tailored for specific needs, resources and workflows. These
environments should support the entire life cycle of computational work in mathematical
research, from initial exploration to publication, teaching and outreach.

%good sounding introduction themes from proposal:
data archiving and sharing in a semantically sound way component architecture, user
interfaces, deployment frameworks and standardisation and interoperability of software
systems.  collaborations based on software, data and knowledge

update range of existing open source software systems for seamless deployment identify and
extend ontologies and standards to facilitate safe and efficient storage, reuse,
interoperation and sharing of rich mathematical data whilst taking into account of
provenance and citability

core is sage, GAP, jupyter, singular developers

\ednote{MK@NT: we need a general description here, cite \cite{OpenDreamKit:on,ODKproposal:on}}


% somewhere we should briefly present how GAP, sage, findstat and LMFDB store data

\section{Integrating Mathematical Software Systems via the Math-in-the-Middle Approach}

% Mathematics has a rich notion of data: it can be either numeric or symbolic data;
% knowledge about mathematical objects given as statements (definitions, theorems or
% proofs); or software that computes with these mathematical objects. All this data is
% really a common resource, and should be maintained as such


To achieve the goal of assembling the ecosystem of mathematical software systems in the
\pn project into a coherent mathematical VRE, we have to make the systems interoperable at
a mathematical level. In particular, we have to establish a common meaning space that
allows to share computation, visualization of the mathematical concepts, objects, and
models (COMs) between the respective systems. Building on this we can build a VRE with
classical IDE techniques. 

Concretely, the problem is that the software systems in \pn have their own representations
of and functionalities for the COMs involved. This starts with simple naming issues:
elliptic curves are represented by \lstinline|ec| in the LMFDB, and as
\lstinline|EllipticCurve| in SageMath, persists through the underlying data structures,
and becomes virulent at the level of algorithms, their parameters, and domains of
appliccability.


% *** maybe use this above *** 
    % \begin{omtext}[title=Advantages] well-known Open Source Software
    %   \begin{enumerate}
    %   \item Let the specialists do that they do best and like\lec{and avoid what the don't}
    %   \item collaboration exponentiates results
    %   \item competition fosters innovation \lec{+ no vendor lock-in}
    %   \end{enumerate}
    % \end{omtext}


%   \item 
%     \begin{omtext}[title=Problem]
%       does an elliptic curve mean the same in GAP, SAGE, LMFDB?
%       \begin{itemize}
%       \item otherwise delegating computation becomes unsound
%       \item storing data in a central KB becomes unsafe 
%       \item the user cannot interpret the results in an UI
%       \end{itemize}
%     \end{omtext}
% \item
%     \begin{omtext}[title=Idea]
%       Need a common meaning space for safe distributed computation in a VRE!
%     \end{omtext}

\ednote{MK: essentially the St. Andrews stuff}
% outline:
% - MMT: distributed nature, flexibility, 
% - 
\section{Conclusion}
\ednote{MK: I am not sure we need this, but we will see}
\subsubsection*{Acknowledgements}
The authors gratefully acknowledge discussions and experimentation at the St.\ Andrews
workshop, which clarified the ideas behind the math-in-the-middle approach;
in particular John Cremona, Paul-Olivier Dehaye, Luca de Feo, Mihnea Iancu, Alexander
Konovalov, Samuel Leli\`evre, Steve Linton, Dennis M\"uller, Markus Pfeiffer, Viviane Pons,
Florian Rabe, and Tom Wiesing.

We acknowledge financial support from the OpenDreamKit Horizon 2020 European Research
Infrastructures project (\#676541).

\printbibliography
\end{document}
%%% Local Variables:
%%% mode: latex
%%% TeX-master: t
%%% End:

%  LocalWords:  maketitle endeavor ednote ODKproposal printbibliography subsubsection
