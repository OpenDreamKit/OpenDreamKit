\subsection{Presenting LMFDB data}
The \emph{$L$-functions and modular forms database} is a project involving dozens of 
mathematicians, who assemble computational data about $L$-functions, modular 
forms and related number theoretic objects. The main output of the project is a 
website, hosted at \url{http://www.lmfdb.org}, that presents this data in a way that could serve as a reference for research efforts and should be accessible at the graduate student level.  The mathematical concepts underlying the \LMFDB are extremely complex and varied, so part of the effort has been focused on how to relay mathematical definitions. For this purpose, the \LMFDB has developed so-called \emph{knowls}, which are a technical solution to present \LaTeX-encoded information interactively, heavily exploiting the concept of transclusion. The end result there is a very modular and highly interlinked set of definitions in mathematical natural language. 
 
The \LMFDB backend is written in \textsf{python}, with a heavy reliance on \Sage and the document database system \Mongo \cite{lmfdb-repo}. Again, due to the complexity of the objects considered, many idiosyncratic encodings are used for the data. This makes the whole data management lifecycle particularly tricky, and dependent on different select groups of individuals for each component. There are ongoing efforts to document the formats used \cite{lmfdb-formats}.

This whole situation therefore constitutes an interesting case study in mathematical knowledge management. In St Andrews we decided to experiment and see what could be achieved if we made the semantic layers more explicit: the same concepts are simultaneously defined and 
