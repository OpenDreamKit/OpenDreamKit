\hypertarget{deliverable-description-as-taken-from-github-issue-39-on-2019-08-31}{%
\section*{\texorpdfstring{Deliverable description, as taken from Github
issue
\href{https://github.com/OpenDreamKit/OpenDreamKit/issues/39}{\#39} on
2019-08-31}{Deliverable description, as taken from Github issue \#39 on 2019-08-31}}\label{deliverable-description-as-taken-from-github-issue-39-on-2019-08-31}}

\begin{itemize}
\tightlist
\item
  \textbf{WP2:}
  \href{https://github.com/OpenDreamKit/OpenDreamKit/tree/master/WP2}{Community
  Building, Training, Dissemination, Exploitation, and Outreach}
\item
  \textbf{Lead Institution:} University of Silesia
\item
  \textbf{Due:} 2019-07-31 (month 47)
\item
  \textbf{Nature:} Demonstrator
\item
  \textbf{Task:} T2.9
  (\href{https://github.com/OpenDreamKit/OpenDreamKit/issues/32}{\#32})
  \emph{Demonstrators:interactive text books}
\item
  \textbf{Proposal:}
  \href{https://github.com/OpenDreamKit/OpenDreamKit/raw/master/Proposal/proposal-www.pdf}{p.
  38}
\item
  \textbf{\href{https://github.com/OpenDreamKit/OpenDreamKit/raw/master/WP2/D2.14/report-final.pdf}{Final
  report}}
  (\href{https://github.com/OpenDreamKit/OpenDreamKit/raw/master/WP2/D2.14/}{sources})
\end{itemize}

Interactive tools have always been an attractive tool in education,
engaging the student to learn both by theory and by practice, to
immediately test their understanding, and to explore around the
material, all at their own pace.

In Task\textasciitilde T2.9, we explored different approaches to
authoring and distributing interactive textbooks using the Jupyter
toolkit. In D2.9
(\href{https://github.com/OpenDreamKit/OpenDreamKit/issues/49}{\#49}) we
reported on the writing of two interactive books. There, the books were
authored as structured text files in the ReST document format, and
exported as interactive html pages or pdf, using the \texttt{Sphinx}
documentation system and the \texttt{sage-cell} interactive html page
technology.

For this deliverable, we proceeded with two additional interactive
textbooks:

\begin{itemize}
\tightlist
\item
  Problems in Physics
  \url{https://github.com/marcinofulus/Mechanics_with_SageMath}
  \url{https://github.com/marcinofulus/Dynamical_Systems}
  \url{https://github.com/marcinofulus/Transport_Processes}
\item
  \emph{Introduction to Python Computational Science and Engineering}
  \url{https://github.com/fangohr/introduction-to-python-for-computational-science-and-engineering}
\end{itemize}

There we explored an alternative approach: the books were authored as
collections of Jupyter notebooks, and exported as notebooks, html or
pdf.

In this report, we set the stage by describing the benefits of
(Jupyter-based) interactive textbooks from the learners and authors
perspective, and review our two new interactive text books; we then
discuss the workflows we explored, their relative merits, and some best
practices to enhance quality and maintenability. We present a template
abstracted away from our books that enables new authors to kick-start
the writing of their own book. We conclude by highlighting the ease of
distribution of interactive textbooks thanks to the Binder Virtual
Environment. The table of contents of the two books is provided in the
appendix.

Altogether, this demonstrates that the OpenDreamKit efforts, notably
T4.1
(\href{https://github.com/OpenDreamKit/OpenDreamKit/issues/69}{\#69}),
T4.3
(\href{https://github.com/OpenDreamKit/OpenDreamKit/issues/71}{\#71}),
T4.6
(\href{https://github.com/OpenDreamKit/OpenDreamKit/issues/74}{\#74}),
and T4.8
(\href{https://github.com/OpenDreamKit/OpenDreamKit/issues/76}{\#76})
contributed to lower barriers for including computations in science
education while significantly improving the maintainability of such
interactive materials by proper use of automated validation.
