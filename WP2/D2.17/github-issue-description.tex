\hypertarget{deliverable-description-as-taken-from-github-issue-252-on-2019-09-05}{%
\section*{\texorpdfstring{Deliverable description, as taken from Github
issue
\href{https://github.com/OpenDreamKit/OpenDreamKit/issues/252}{\#252} on
2019-09-05}{Deliverable description, as taken from Github issue \#252 on 2019-09-05}}\label{deliverable-description-as-taken-from-github-issue-252-on-2019-09-05}}

\begin{itemize}
\tightlist
\item
  \textbf{WP2:}
  \href{https://github.com/OpenDreamKit/OpenDreamKit/tree/master/WP2}{Community
  Building, Training, Dissemination, Exploitation, and Outreach}
\item
  \textbf{Lead Institution:} University of Sheffield
\item
  \textbf{Due:} 2019-04-30 (month 48)
\item
  \textbf{Nature:} DEC: Websites, patent fillings, videos etc.
\item
  \textbf{Task:} T2.6
  (\href{https://github.com/OpenDreamKit/OpenDreamKit/issues/29}{\#29})
  Introduce OpenDreamKit to Reseachers and Teachers
\item
  \textbf{Proposal:}
  \href{https://github.com/OpenDreamKit/OpenDreamKit/raw/master/Proposal/proposal-www.pdf}{p.
  38}
\item
  \textbf{\href{https://github.com/OpenDreamKit/OpenDreamKit/raw/master/WP2/D2.17/report-final.pdf}{Final
  report}}
  (\href{https://github.com/OpenDreamKit/OpenDreamKit/raw/master/WP2/D2.17/}{sources})
\end{itemize}

Notebooks -\/- interactive documents mixing prose, code, outputs, and
visualization -\/- have existed for decades; we may cite the Maple,
Mathematica, or SageMath notebook. There is a long time experience in
the community of using such notebooks for a variety of purposes in
teaching and research, ranging from scratchpad for interactive
computation to narratives such as interactive teaching documents or
logbooks of computational research.

At the time of writing the OpenDreamKit Proposal in 2014, the Jupyter
notebook was a an emerging technology which had recently evolved from a
generalization of IPython notebook that supported a variety of
programming languages or interactive systems. Based on modern web
technologies, with a modular design informed by many former
implementations, and a growing ecosystem of related tools, it showed
strong potential as core user interface component for building Virtual
Research Environments. Of particular interest are the Jupyter widgets
that bridge the gap between interactive computation and interactive
visualization and applications, in effect making for a smooth continuous
learning curve from user to power user to developer, and enlarging the
range of use-cases.

In Task 2.6
(\href{https://github.com/OpenDreamKit/OpenDreamKit/issues/29}{\#29}),
we have disseminated Jupyter based Virtual Research Environment to
students, researchers, and teachers to act as multipliers of knowledge
and dissemination. This covered the technology but also best practices.
The notebook, despite all its benefits, also has pitfals, with which the
participants have extensive experience.

As reported on in our periodic deliverables on Community building:
Impact of development workshops, dissemination and training activities
D2.6
(\href{https://github.com/OpenDreamKit/OpenDreamKit/issues/46}{\#46}) ,
D2.11
(\href{https://github.com/OpenDreamKit/OpenDreamKit/issues/36}{\#36}) ,
D2.15
(\href{https://github.com/OpenDreamKit/OpenDreamKit/issues/40}{\#40}),
OpenDreamKit participants actively organized or participated in dozens
of events where they advertised and delivered training on OpenDreamKit
technology. In this report we start by reviewing other evaluation and
dissemination activities that were carried out during the project. In a
second section, we briefly reflect on the lessons learned. Indeed, these
activities were also first and foremost the occasion to gather first
hand testimony of the technology on the battle field, this to
continuously inform and motivate the project.
