\documentclass{deliverablereport}

\deliverable{dissem}{ibook1}
\deliverydate{31/08/2018}
\duedate{31/08/2018 (M36)}
\author{Marcin Kostur, Jerzy Łuczka, Jan Aksamit, Jolanta Marzec}

\newcommand{\screenshot}[2]{
\begin{figure}[ht]
  \includegraphics[width=\textwidth,trim={0 0 0 1px},clip]{#1}
  \caption{#2}
  \label{#1}
\end{figure}}

\newcommand{\screenshotsmall}[2]{
\begin{figure}[ht]
  \includegraphics[width=0.8\textwidth,trim={0 0 0 1px},clip]{#1}
  \caption{#2}
  \label{#1}
\end{figure}}



\begin{document}
\maketitle

\tableofcontents
% \githubissuedescription


\section{Introduction}

Interactive web pages have always been an attractive tool in
education. It has started in the age of Adobe-Flash technology, when a
lot of educational material was created. The advent of JavaScript has
led to significant improvements in that field: in the first place it
created a possibility to author the content using Open Source tools
and secondly it increased the portability of the created documents to
practically all devices equipped with a modern web browser.

The problem which was persistent in those solutions was the big gap
between the authoring process of interactive documents and their usage.
Usually the code behind an interactive examples was not very
educational and was not supposed to be edited by the student.

Then emerged the concept of interactive widgets in the notebook,
appearing first in Mathematica, then being popularized by the \Sage
notebook, and recently getting huge traction with the \Jupyter
notebook thanks to its wide audience. Interactive widgets, and in
particular the very simple to use \texttt{@interact} decorator,
delivered a versatile solution to this problem: they made it possible
to turn with little efforts a mere scientific calculation into a
visually appealing interactive application, all within the original
environment where the mathematical explorations took place.

The need of including computer technology for teaching at University
of Silesia was sparked by interdisciplinary courses. In those fields it
is common to master numerical tools for modeling and at the same time
to gain an intuition. Integration of modern computer technology allows
students to more efficiently analyse a system without tedious symbolic
calculations. This has led to systematic integration of \Sage system
with science education at the Institute of Physics since 2011. In this
deliverable we have created texbooks which consists of c.a.  500 pages
with over hundred interactive code cells and code examples and tens of
figures. The audience for those books are students of physics,
biophysics, econophysics and medical physics. This work took 25
PM of engagement of four people.

The source code of the interactive books is hosted on github within a
dedicated repository of the OpenDreamKit organization:
\url{https://github.com/OpenDreamKit/iODKbook2}.


\section{Interactive book technology}

We have used \Sphinx as main authoring tool: \Sphinx is originally a
documentation generator which is popular in the Python world where
it's used for the documentation of \Python itself, and many other
projects. More generally it can be used to author large structured
documents which are then exported to various formats, including HTML,
PDF, EPub, and now even Jupyter notebooks. \Sphinx contains a plugin
system enabling to taylor it for particular applications. We used
plugins for
\begin{itemize}
\item embedding \Sage code as live cells which can be edited, executed
  and interacted with (see Figure~\ref{fig:interact_sphinx}); under
  the hood the computations are run on a remote service thanks to the
  SageMathCell technology (\url{https://github.com/sagemath/sphinx-sagecell-ext});
  an alternative would be to use Thebelab (see~\longdelivref{UI}{ipython-kernels});
\item using conditional compilation of parts depending for example on
  whether the target medium is interactive of not, or whether
  solutions should be included,
\item automatically including \Jupyter notebooks with embedded output;
\item producing high quality pdf via \LaTeX system.
\end{itemize}

In the authoring process one can join interactive material in the form
of notebooks (based on sagenb or \Jupyter) that can be embedded in
full or in part into the \Sphinx book. In the HTML export, the
material is then rendered as live SageCellServer code that the user
can edit, execute, or interact with. It has to be stressed that not
only interactive widgets written with \texttt{@interact} are useful.
Practically all of exercises inside the book are illustrated by such
interactive material. It's our experience however that such
interactive material should not require or encourage the reader to do
anything but small edits to the provided code; indeed such edits are
volatile by essence; they take place anonymously in the browser with
no storage behind the scene.

\begin{figure}
\centerline{\includegraphics[width=.7\textwidth]{interact_in_sphinx.png}}
\caption{\label{fig:interact_sphinx} An example of \texttt{@interact} in
  \Sphinx generated web version of interactive book. The upper box
  shows the code and the lower box shows the output.}
\end{figure}


The authoring process is very similar to writing a \LaTeX paper. It
is based on edit-compile-view cycle which is known to most of
physicists and mathematicians. In some cases we have used a shared
account on a \Jupyter system for this process, where compilation commands
where pre-coded in a \Jupyter notebook, which also included a link to
web, and pdf versions. In such case is was very easy to help
academics less familar to computer science to just edit and modify the
content.


\section{ Nonlinear Processes in Biology }


The book is in large extend based on coursework taught at University
of Silesia. It was primarily based on sagenb system and gradually was
converted into the interactive book with live examples.


The book covers the classical topics in modeling in
biology. Methodologically it can be split into the following topics:

\begin{enumerate}
\item Methods of mathematical modelling of  processes based on ODE and PDE. 
\item Qualitative methods in ODE and PDE with application of Sage. It
  includes:
  \begin{itemize}
  \item stationary states of the system,
  \item the linear stability of fix points,
  \item phase curves,
  \item bifurcation diagrams,
  \item time-dependent solutions,
  \item a graphical  presentation of all above  elements. 
  \end{itemize}
  
  For each subject there is a list of related problems and
  supplementary tasks as homework. All requires symbolic manipulations
  as well as certain techniques from numerical analysis like root
  finding.
  
\item Numerical solving of ODEs. It is complementary to analytical and
  qualitative methods for ODE. 
\item Numerical solving of PDE-s - diffusion equation,
  reaction-diffusion systems and similar. It requires good tools in data
  processing, usually uses numpy for calculations.

\end{enumerate}


It turned out that conceptually simple methods - like plotting a
function of one variable, which depends on a parameter (see
Figure.~\ref{fig:interact2}), can be very useful for the analysis of
models containing analytically untractable expressions. In this
particular example, students construct just a simple plot of a
function; then, with the help of \texttt{@interact} construction, the
plot is easily turned intro an applet for the analysis of the system.

The book covers the following  models: 
 \begin{itemize}
  \item one-dimensional systems (Malthus, Verhulst, Ludwig, Alee models);
  \item two-dimensional systems (Lotka-Volterra, May models);
  \item kinetics of chemical reactions;
  \item Belousov-Zhabotinsky reactions;
  \item models of epidemics.
 \end{itemize}

\begin{figure}
\centerline{\includegraphics[width=.7\textwidth]{interact_npb.png}}
\caption{\label{fig:interact2} An example of \texttt{@interact}
  applied for graphical examination of roots of a function. It is a
  first step in qualitative analysis of model of tumor growth in a
  book.  }
\end{figure}



\section{Lectures on Linear Algebra}



{\it The Linear Algebra book is dedicated to Dr Jan Aksamit for his
devotion, and endless patience in work with students. He has lectured
linear algebra for physics students for almost 40 years of his
life. With restless passion he spent the last years of his life
preparing an electronic version of his lectures notes.}


This book is based on Linear Algebra courses for physics students
taught in recent years at the University of Silesia. It has been
gradually equipped with interactive materials and extended by sections
illustrating application of linear algebra in economy, computer
science, and other areas. The computations and the interactive part are
performed by \Sage, which is embedded in the book.

Along the theory the user becomes familiar with the code used in
\Sage. Their learning process is enhanced by various factors:
\begin{itemize}
\item numerous visualisations;
\item interactive images which allows for the construction of a
  graphical solution to a problem; the problem may be changed in real
  time by the user changing the code;
\item a place for experiments: the user may fill in the gaps in the code
  to produce graphical interpretation of a chosen operation;
\item possibility to perform the calculations with \Sage within the
  book; in particular, computations with real data and, thus, real
  life conclusions are possible;
\item some longer coding exercises are equipped with check points
  (without revealing the answer) to ensure that the discovered
  solution is correct.
\end{itemize}

Moreover, not only users benefit from interactive resources, but also
learn how to produce them themselves.

\section{Future work}


The two open textbooks are put to continuous use for the eponymous
courses at university of Silesia; they are frequently updated with new
material (new models, exercises, ...), and improved according to
feedback from teachers and students. Also there are effort being done
to encourage other lecturers to use them and contribute back.

An important improvement will be inclusion of automatic testing in
the compilation procedure.

Currently the material exist in Polish and English versions, which
are, however unsynchronized. In future we plan to update the English
version only, mostly because of growing internationallity of
studies. In some cases, mostly for theoretical parts of linear algebra
book, the polish version will be kept in synchronisation.



\clearpage
\appendix
\section{Screenshots}\label{screenshots}

\screenshot{toc1.png}{Table of contents of Linear Algebra book.}
\screenshot{toc2.png}{Table of contents of Linear Algebra book.}
\screenshot{toc3.png}{Table of contents of Linear Algebra book.}

\screenshot{la_book_ex.png}{An example of interactive element in web
  version of Linear Algebra book.}  \screenshotsmall{rst.png}{The
  source code of an example of interactive element in web version of
  Linear Algebra book.}


\end{document}

%%% Local Variables:
%%% mode: latex
%%% TeX-master: t
%%% End:

