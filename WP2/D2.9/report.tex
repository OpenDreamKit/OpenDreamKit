\documentclass{deliverablereport}

\deliverable{dissem}{ibook1}
\deliverydate{31/08/2018}
\duedate{31/08/2018 (M36)}
\author{Marcin Kostur, Jerzy Łuczka, Jan Aksamit, Jolanta Marzec}

\begin{document}
\maketitle

Linear Algebra and Nonlinear Processes in Biology are examples of
interactive capabilities

% \githubissuedescription

\section{Introduction}

Interactive webpages have always been an attractive tool in
education. It has started in the age of Adobe-Flash technology, when a
lot of educational materials have been created. The advent of
javascript has led to significant improvements in that field: in the
first place it created a possibility to author the materials with Open
Source tools and secondly it increased the portability to practically
all devices equipped with modern web browser.

The problems which was persistent in those solutions was the big gap
between authoring of interactive tools and usage. Usually the code
behind of interactive examples was not very educational and was not
supposed to be edited by the target user. On the other hand Open
Source tools such as \Sage, brought a solution to this problem
providing the \texttt{@interact} decorator. In classical \Sage
notebook it was possible to create with single slines of code even
quite complex web-based interactive applications. The biggest
advantage is that one can create a visually appealing scintific
interaction using the same environment in which the mathematical
exlorations take place.




\section{Interactive book technology}

We have used well estabilished sin Python community sphinx
documentation generator for the main authoring tool. Sphinx contains
many plugins  which in our particular application especially are able to:

\begin{itemize}
\item embedd \Sage code and allow for interactive execution on a
  remote server using SageMathCell  technology
\item use conditional compilation of parts depending e.g. if target medium is interactive of not.
\item include automatically \Jupyter notebooks with embedded output
\item  can produce high quality  pdf via \LaTeX system.
\end{itemize}  


In authoring process one can produce notebook (sagenb or \Jupyter
based) and decide if it should be included into the sphinx book or
choose selected examples and influde them as live SageCellServer code
which can be used interactively. It has to be stressed that not only
elemnents having \texttt{@interact} are usefull. Practically of
excercises can be done inside the book in the interactive code
boxes. It is reasonable, however, to propose reader to write only
short pieces of code in whose elements, since the code written in
browser without login is volatile and can dissappear if one reloads
the page.

\begin{figure}
\centerline{\includegraphics[width=1.0\textwidth]{interact_in_sphinx.png}}
\caption{\label{fig:jupyterdemo} An example of \texttt{@interact} in
  sphincs generated web version of interactive book. The upper box
  shows the code and the lower box shows the output.}
\end{figure}


\subsection{ Nonlinear Processes in Biology }

The book is in large extend based on coursework taught at University
of Silesia. It was primarily based on sagenb system and gradually was
converted into the interactive book with live examples. 

\subsection{Lectures on Linear Algebra}

This book is based on courses of linear algebra for physics students
taught in recent years at the University of Silesia. It has been
gradually equipped with interactive materials. \Sage was used as a
tool for performing computations. 


\end{document}

%%% Local Variables:
%%% mode: latex
%%% TeX-master: t
%%% End:

