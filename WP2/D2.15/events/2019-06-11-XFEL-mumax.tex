\begin{event}{Micromagnetism Workshop: Ubermag and mumax$^{3}$}{XFEL-ODK}{European XFEL, Schenefeld, Germany, 11-14 June 2019}{XFEL}{}{}{}

\textbf{Main goals.} The main goal of this workshop was to develop a
micromagnetic calculator for driving mumax$^{3}$ micromagnetic
simulation tool.

\textbf{\ODK implication.} The main implication of this workshop was
that another computational backend can be used (apart from OOMMF) to
run micromagnetic simulations. mumax$^{3}$ has certain advantages over
OOMMF, mostly in terms of its capabilities of running micromagnetic
simulations on GPU. This would enable us to reach a much larger target
audience.

\textbf{Event summary.} The first day of the workshop consisted of
presentations from all its participants in order to familiarise all
participants about the basics of Ubermag and mumax$^{3}$. Another 3
days were focused on coding and the implementation of the mumax$^{3}$
calculator, setting up the repository's infrastructure, and
discussions of future developments.

\textbf{Results and impact.} A micromagnetic calculator, based on
mumax$^{3}$, was developed and covers the most important capabilities
of mumax$^{3}$:

\centerline{\url{https://github.com/ubermag/mumax3c}}

\end{event}
