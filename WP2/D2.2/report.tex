\documentclass{deliverablereport}

\usepackage{xparse}
\usepackage{etoolbox}
\usepackage{caption}
 \ExplSyntaxOn

\newcounter{eventcounter}

\newenvironment{event}[6]{
\vspace{0.5cm}
\refstepcounter{eventcounter}
\label{event-#2}


\noindent\textbf{Event~\theeventcounter -~ #1}\newline % title

\noindent #3 \newline % location and date

\noindent ODK~partners~involved:~ \clist_map_inline:nn{#4}{\site{##1}~}\newline %partners

\ifx&#5&%
      % no participant #
\else
\noindent #5~participants\newline
\fi

\ifx&#6&%
      % no website
\else
\noindent \url{#6}\newline
\fi



}{\begin{center}\noindent\rule{4cm}{0.4pt}\end{center}}

 \ExplSyntaxOff


\deliverable{dissem}{workshops-1}
\duedate{31/08/2016 (M12)}
\deliverydate{31/08/2016}
\author{Viviane Pons}

\begin{document}
\maketitle
\githubissuedescription
\newpage
\tableofcontents
\newpage

% Other tasks and deliv impacted: T3.6, T4.1, T6.1, T6.3, T6.10, D6.2 WP5 D4.4 D4.7 T3.1 T4.4 T4.6

\section{Project meetings and development workshops}

We call a development workshop an event with a restricted number of participants
who meet to work on a specific task. These workshops are an inherent part
of \ODK development process as described in \taskref{dissem}{devel-workshops}:
 they bring together
developers from within and outside of \ODK and allow effective work
and discussions on many technical aspects. They also participate in building
and maintaining a community of developers inside \ODK and within the
open-source communities we belong to.

We list here all workshops which have been organized or co-organized by \ODK
as well as external workshops which have been attended by \ODK participants
with a significant impact on the project. We also include project meetings as they
participate to the same goal of bringing together participants of the project and
always include some development time.

Throughout the first year of the project, we have had
\begin{itemize}
\item 3 project meetings,
\item 2 workshops fully funded and organized by \ODK,
\item 2 workshops partially funded and co-organized by \ODK,
\item 1 external workshop.
\end{itemize}
This includes in particular one development edition of Sage Days and one Atelier
Pari that we had announced on \longtaskref{dissem}{devel-workshops}.
 These events had a major impact on other \ODK workpackages, tasks, and deliverables: WP3:
\taskref{component-architecture}{portability},
\taskref{component-architecture}{extract-smc}
WP4:
\taskref{UI}{ipython-kernels},
\taskref{UI}{sage-sphinx},
\taskref{UI}{structdocs},
\delivref{UI}{ipython-kernels-basic},
\delivref{UI}{ipython-kernels},
WP6:
\taskref{dksbases}{data-design},
\delivref{dksbases}{design}


\begin{event}{\ODK Kickoff meeting}{kickoff}{Orsay (France) 2015-09-02 to 2015-09-05}{PS,UK,LL,SA,JU,UW,USH,UV,ZH,USO,US,UJF,SR}{34}{http://opendreamkit.org/2015/09/02/KickoffMeeting/}

  \textbf{Main goals.} Build a joint vision by giving each participant
  an overview of the consortium and its wide variety of expertise, and
  of the project’s aims and specific tasks, as well as to expose them
  with key software components, web platforms and technologies in the
  ecosystem.

  \textbf{ODK implication.} As a lead partner, Paris-Sud organized and
  hosted this event which was fully funded by ODK.

  \textbf{Event summary.} The meeting started with presentations of
  the partners and work packages, as well as software component and
  technology previews. We also discussed the management structure,
  technical infrastructure, and planning. And we got to work –
  finally!  – on some collaborative tasks, through brainstorms and
  coding sprints.

  \textbf{Demographic.} 30 ODK participants were present, representing
  most of the partners. We had also invited a few external
  participants.

  \textbf{Results and impact.} This event set the tone for upcoming
  development workshops by creating a friendly and instructive working
  atmosphere. It also set the different goals of project, allowing
  everyone to share their views and understanding of ODK tasks.

  \begin{figure}[ht]
    \caption*{ODK Kickoff meeting}
    \includegraphics[scale=0.3]{pictures/kickoff1.jpg}
    \includegraphics[scale=0.5]{pictures/kickoff2.jpg}
  \end{figure}
\end{event}

\newpage
\begin{event}{Sage Days 70}{sd70}{Berkeley (US California), 2015-11-08 to 2015-11-14}{PS}{16}{https://wiki.sagemath.org/days70}

\textbf{Main goals.} Gather developers from Sage, SageMathCloud and Jupyter together to learn the
inner machineries of the different projects and code together towards common goals.

\textbf{ODK implication.} ODK was not the main organizer of this event, it was used to fund
European participants which are either part of ODK or very closely related.

\textbf{Event summary.} The event featured many interesting talks on the inner mechanics of
both SageMathCloud and Jupyter, in particular:
\begin{itemize}
\item \href{https://youtu.be/GOuy07Kift4}{How to contribute to SageMathCloud} by William Stein
\item \emph{The PARI Jupyter kernel} by Jeroen Demeyer
\item \emph{Jupyter Notebook development} by Jason Grout.
\end{itemize}
Lots of time was devoted to projects and code such as: installing a development version of SageMathCloud,
following tutorials on SageMathCloud development, working toward the integration of the Jupyter notebook
in Sage.

Furthermore a Jupyter interface for HPC-GAP was developed, and the Jupyter
interface for GAP was improved. A talk \emph{The current status of (HPC-)GAP}
was contributed.

\textbf{Results and impact.} 
This workshop was essential to some ODK planned tasks. This was especially related to WP3 and WP4. Here are some tasks that
were started during the Sage Days:
\begin{itemize}
\item T3.6 Document and modularize SageMathCloud's codebase. This task was started during the workshop using the 
knowledge of the main developer of SageMathClod, William Stein.

\item T4.1 Uniform notebook interface for all interactive components. This is a major task of WP4. This workshop
was an occasion to share first hands information between Sage, GAP, and Jupyter developers.

\end{itemize}
The knowledge we gathered during presentations was relevant to all tasks including notebook interfaces and cloud
systems.

\end{event}

\begin{event}{Atelier PARI/GP 2016}{AtelierPARI2016}{Grenoble (FR),
2016-01-11 to 2016-01-25}{PS,UB,UV,UW}{36}{http://pari.math.u-bordeaux.fr/Events/PARI2016/}

\textbf{Main goals.}

The PARI/GP Ateliers were established in 2012 as a yearly meeting
between developers and users of the PARI/GP system.

The main goals are advertising new features and improvements,
discussing further developments, sharing best practices, and collaborative
code writing (hacking sessions, doc reviews, bug-squashing parties).

You can find the list of previous PARI Ateliers at
\url{http://pari.math.u-bordeaux.fr/ateliers.html}

\textbf{ODK implication.} 
%Describe how ODK was involved and give a rough estimation of cost for ODK

\ODK participants: B. Allombert, K. Belabas, J. Demeyer, J.-P. Flori,
L. de Feo, as well as Aurel Page from the Warwick group (LMFDB).

\ODK provided the main funding source for the workshop (accommodation,
subsistence and travel expenses), for about 15k\euro. ERC Starting Grant
ANTICS, and the LabEx PERSYVAL-Lab co-funded the event.

\textbf{Event summary.} 
%Give a summary of your event

The 6th Atelier PARI/GP took place in Grenoble (France) from january
11th to 15th.

There were 36 registered participants from 16 different institutions
(no registration fees).

A typical day of the workshop had introductory talks and tutorials
in the morning; afternoons allowed ample time for hacking sessions,
discussions and training.

The Atelier featured 10 morning talks on

\begin{itemize}
\item mathematical topics and implementation projects : modular forms,
    L-functions, polylogs \& multizeta values,

\item packages and interfaces : PARI Jupyter notebook, a number field database,
    an elliptic curve library for cryptography, CADO-NFS, GIAC/XCAS,
    parallel programming with GP2C.
\end{itemize}

Slides and videos for all talks are available at
\url{https://www.youtube.com/playlist?list=PL0E0n75oNCDnWuydCHepxxSRc4UbtQQ}

\textbf{Results and impact.} 
% What did you achieve with this event? (If ever it impacted 
% other ODK tasks and deliverables, mention it here)

The workshop was very productive and was particularly beneficial to WP4 (user
interfaces) and WP5 (high-performance computing):
\begin{itemize}
\item it was a major boost to PARI/GP development; feebackd received allowed
the release of PARI/GP-2.8 in august 2016, a major release after two years of
development; (T2.3, development workshops)

\item issues related to the new PARI Jupyter notebook (D4.4, D4.7) and
Sage/PARI interaction were ironed out during the meeting; discussions related
to PARI parallelisation engine (D5.10)

\item the PARI developers learnt about technologies and created resources for
online GP deployment during the meeting using the \emph{emscripten} compiler,
see e.g.~\url{http://pari.math.u-bordeaux.fr/gp.html}.
\end{itemize}

\end{event}


\begin{event}{First Joint GAP-SageMath Days}{GAPSage2016}{St Andrews (UK),
2016-01-18 to 2016-01-22}{SA,PS,UV,UK,UO}{19}{http://gapdays.de/gap-sage-days2016/}

\textbf{Main goals.}

Both GAP and SageMath systems have traditions of regular developer meetings, where those
interested in these systems, from newcomers to contributors, are gathering together for
collaborative code writing, sharing best practices, advertising recent new features and
improvements, and discussing further developments. You can find the list of previous GAP 
days at \url{http://gapdays.de/} and of SageMath days at \url{https://wiki.sagemath.org/Workshops}.

Following these traditions, it was decided to organise the 1st Joint GAP-SageMath Days, with the
focus on improving GAP-SageMath integration and interaction between these systems and between
their developers.

\textbf{ODK implication.} 
%Describe how ODK was involved and give a rough estimation of cost for ODK

The 1st Joint GAP-SageMath Days were mainly supported by CoDiMa -- Collaborative Computational
Project (CCP) in the area of Computational Discrete Mathematics (EPSRC grant EP/M022641/1,
\url{http://www.codima.ac.uk/}). It was immediately followed by the WP7 Workshop ``Knowledge
representation in mathematical software and databases'' on January 25th-27th, 2016, therefore
\ODK participants involved in GAP and/or SageMath development could conveniently attend
both events. Accommodation, subsistence and travel expenses of partners from UPSud and UVSQ
were paid by the \ODK project, and those of partners from UNIKL,UOXF were reimbursed
by the CoDiMa project.

\textbf{Event summary.} 
%Give a summary of your event

A typical day of the workshop had one or two introductory talks to facilitate subsequent
discussions and coding sprints, in particular:
\begin{itemize}
\item \emph{Contributing to Sage} by Nicolas M. Thiéry and Volker Braun
\item \emph{Contributing to GAP } by Max Horn, Alex Konovalov, and Markus Pfeiffer
\item \emph{libGAP} by Volker Braun
\item \emph{GAP in the cloud} by Markus Pfeiffer
\end{itemize}
Other topics included, among others,
further integration of HPC-GAP into GAP;
working on the semantic-aware SageMath interface to GAP;
improving the installation of GAP in SageMath and in SageMath cloud;
creating and working with Docker containers;
development of GAP and SageMath teaching materials for Software Carpentry,
etc.

\textbf{Results and impact.} 
% What did you achieve with this event? (If ever it impacted 
% other ODK tasks and deliverables, mention it here)

The workshop was very productive. Only to the main GAP repository 
(\url{https://github.com/gap-system/gap/}) there were 51 new pull 
requests submitted, just 8 of which are still open; in addition,
28 new issues were created (9 of them are closed by now), and 
there was also progress achieved with GAP packages developed 
elsewhere; the work on converging GAP and HPC-GAP; discussing 
development workflows, etc. It helped to both GAP and SageMath 
teams to get further insights into each other's systems and was 
particularly beneficial to WP3 (component architecture), WP4 
(user interfaces) and WP5 (high-performance computing). 

\end{event}


%\begin{event}{WP6 Workshop: Knowledge representation in mathematical software and databases}{wp6-0}{St Andrews (UK), 2016-01-25 to 2016-06-27}{JU,SA,PS,UV,UW,ZH}{12}{}

  \textbf{Main goals.} Semiannual \ODK project meeting joined with a
  WP6 ``Data/Knowledge/Software bases'' kickoff workshop dedicated to
  the exploration of how mathematical knowledge could be better
  represented and exploited within systems and for communicating
  between systems.


\textbf{ODK implication.} St Andrews hosted this event which was
organized by Paris Sud (project meeting) and Jacobs Uni (WP6 workshop).

\textbf{Event summary.} Tuesday morning was dedicated to \ODK's
semiannual
\href{http://opendreamkit.org/meetings/2016-01-25-DKS/SteeringCommittee/minutes/}
{steering committee meeting}, including progress reports from all
sites. A couple additional \ODK participants joined physically or
remotely for this meeting.

The rest of the week was dedicated to Work Package 6 activities. To
bootstrap the discussions, we started with a presentation by Michael
Kohlhase of his ideas for a Knowledge First strategy, followed by
presentations of how knowledge is represented in various components of
\ODK (FindStat: Viviane Pons, LMFDB: Paul-Olivier Dehaye, GAP: Markus
Pfeiffer, SageMath: Nicolas M. Thiéry), in preparation for
\longdelivref{dksbases}{design}. There was also a brief presentation
of a proof-of-concept Knowledge-aware Sage-GAP interface developed the
week before at the GAP/Sage Days. Current practices where discussed as
well: commonalities in ``dumping math data on the web'' process
(FindStat, LMFDB,…). Then came a tutorial presentation of MMT to
explore how knowledge representation in MMT could serve as a generic
knowledge backbone for integrating the various systems.  After these
warm up activities, we moved on to brainstorms and joint code sprints,
developing proof-of-concepts formalizations of the knowledge in the
various components and exploring applications: detecting bugs and
inconsistencies in code and data, generating more complete
documentation, supporting generic handle interfaces, \ldots

A fine grained report of the activities is available
\href{http://opendreamkit.org/meetings/2016-01-25-DKS/report/}{here}.

\textbf{Results and impact.} The Math-in-the-Middle approach was born
at this workshop. This led to a joint paper at the CICM
conference~\cite{DehKohKon:iop16} and kicked off and fueled much of
the activity on WP6 since then, in particular around
\longtaskref{dksbases}{data-design}, \delivref{dksbases}{design},
\delivref{dksbases}{dkstheories}, and
\delivref{component-architecture}{semantic-interface-sage-gap}.

\begin{figure}[ht]
  \caption*{WP6 Workshop: Knowledge representation in mathematical software and databases}
  \includegraphics[scale=0.5]{pictures/2016-01-25-DKS-group-picture.jpg}
  \end{figure}
\end{event}


\begin{event}{Sage Days 77: packaging, portability, documentation tools}{sd77}{Cernay (France) 2016-04-04 to 2016-04-04}{PS,UV,LL}{15}{https://wiki.sagemath.org/days77}

  \textbf{Main goals.} This developer meeting was focused on
  initiating long term work on ODK tasks related to packaging,
  portability and documentation tools for SageMath.

  \textbf{ODK implication.} This event was organized and funded by
  Paris Sud.

  \textbf{Event summary.} An intensive week with some short informal
  presentations, and many brainstorms and coding sprints.

  \textbf{Demographic.} 9 ODK participants from three sites together
  with half a dozen other Sage, Sphinx, Guix, Gentoo, and Debian
  experts.

  \textbf{Results and impact.} Proper packaging and distribution has
  been a recurrent issue for SageMath, and is a major task for ODK
  (\href{https://github.com/OpenDreamKit/OpenDreamKit/issues/52}{T3.3:
    Modularisation and packaging}). Major brainstorms occurred during
  the week to clarify the needs, isolate the core difficulties, and
  explore potential approaches to tackle them. The outcome was posted
  on the \href{https://wiki.sagemath.org/days77/packaging}{Sage Wiki},
  to be shared and further edited by the community. This fostered
  tighter collaboration between the packaging efforts for various
  Linux distribution, and triggered major progress on the Debian
  packaging side.

  Similar brainstorms and coding sprints occurred around tasks
  \href{https://github.com/OpenDreamKit/OpenDreamKit/issues/50}{T3.1 portability}
  \href{https://github.com/OpenDreamKit/OpenDreamKit/issues/72}{T4.4: Refactor Sage 's Sphinx documentation system}
  \href{https://github.com/OpenDreamKit/OpenDreamKit/issues/74}{T4.6: Structured documents}

  Altogether
  \href{https://trac.sagemath.org/query?keywords=~days77&col=id&col=summary&col=status&col=type&col=priority&col=milestone&col=component&order=priority}{20
    Sage tickets} were actively worked on during the week.
  % \begin{figure}[ht]
  %   \caption*{ODK Kickoff meeting}
  %   \includegraphics[scale=0.3]{pictures/kickoff1.jpg}
  % \end{figure}
\end{event}


\begin{event}{WP6 Workshop (Bremen)}{wp6-1}{Bremen, Germany, 2016-05-30 to 2016-06-03}{JU, SA, PS}{7}{}

\textbf{Main goals.} Work meeting to understand the type systems of \GAP and
SAGE, and to develop a first interface between MMT, \GAP, and \Sage.

\textbf{ODK implication.} \ODK through Jacobs Uni Bremen was the main
organizer of this event to work on WP6. \PS participated remotely.

\textbf{Event summary.} The event featured a talk about the GAP type
system, and many discussions between the researchers in Bremen and
Markus Pfeiffer. We developed a substantial piece of software to
enable GAP to interface with MMT. Meanwhile \site{PS}, with the
support of \site{SA} and \site{JU}, made further progress on
\delivref{component-architecture}{semantic-interface-sage-gap} and on
the export of the math knowledge embedded in \Sage.

\textbf{Results and impact.}  This workshop was essential to ODK WP6,
in particular for \longtaskref{dksbases}{data-design} and
\delivref{dksbases}{design}.

\end{event}


\begin{event}{\ODK annual meeting}{2016-06-27-bremen}{Bremen (Germany) 2016-06-27 to 2016-07-01}{PS,LL,SA,JU,UW,USH,UV,ZH,USO,US,SR}{24}{http://opendreamkit.org/meetings/2016-06-27-Bremen/}

  \textbf{Main goals.} Annual project meeting, interim review and workshops

  \textbf{ODK implication.} JacobsUni (Bremen) hosted this event which
  was coorganized by Paris Sud and fully funded by ODK.

  \textbf{Event summary.} The beginning of the week was dedicated to
  ODK’s open and internal meetings, including an interim review with
  our Project Officer and three EU Commission reviewers. The rest of
  the week was dedicated to joint work sessions on WP4 (User
  Interfaces) and WP6 (Data/Knowledge/Software bases, aka
  Math-in-the-middle) activities.

  \textbf{Demographic.} 21 ODK participants together with ODK's
  project officer and three reviewers from the EU Commission.

  \textbf{Results and impact.} This meeting was the occasion to build
  a common overview of what was achieved during the first ten months,
  and plan together work on the upcoming tasks and deliverables. The
  project review was enormously helpful to get early feedback
  and start preparing for the upcoming review at Month 18
  (March 2017).


  % \begin{figure}[ht]
  %   \caption*{ODK Kickoff meeting}
  %   \includegraphics[scale=0.3]{pictures/kickoff1.jpg}
  % \end{figure}
\end{event}


\clearpage

\section{Dissemination and outreaching activities}

We describe here all activities related to \taskref{dissem}{dissemination}:
these are all events oriented towards dissemination, training, and outreach. This
includes events organized or co-organized by \ODK and also
participating in external events and many communication activities.

\subsection{Organization of Sage Days in established mathematical communities}

One goal of \ODK is to support local communities of researchers
and developers who contribute to the open-source softwares related to
the project. For Sage, this means supporting the organization of Sage-Days
workshops that arise from within all the different mathematical communities. The main 
goal of these workshops is mostly to improve the Sage coverage of some mathematical
area. They also play a major role in training and communication. The
impact for \ODK can be summarized this way:

\begin{itemize}
\item \textbf{Making ODK known to the end users}: by supporting Sage Days,
\ODK makes itself known to the Sage community and can
thus share the many developments of the project.

\item \textbf{Improving the overall quality of Sage}: by fostering researchers
in specific areas, Sage Days help bring interesting mathematics into
the software, which is beneficial for Sage and so \ODK.

\item \textbf{Training, bringing more user}: Sage Days are the perfect place
for new comers, especially students, to get their first experience with the software.

\item \textbf{Fostering a community}: Sage Days are helping making Sage a vibrant
community, which is vital for the success of \ODK.
\end{itemize}

Throughout the first year of the project, \ODK has been organizing or co-organizing
four Sage Days: one about geometry and dynamics of surfaces, one in differential geometry and topology,
 one in combinatorics and one in coding theory.

\begin{event}{Sage Days 73 May}{SD73}{Oaxaca, Mexico, 04 - 07 May 2016}{CNRS}{9}{https://wiki.sagemath.org/days73}

\textbf{Main goals.}
This Sage workshop was a satellite of the conference Flat Surfaces and
Dynamics of Moduli Space that happened in Oaxaca May 08-13. The aim was to
introduce participants to SageMath and share code and knowledge.

On the first day, we also had two participants from the University of 
Oaxaca.

\textbf{ODK implication.} ODK, via its Bordeaux node, supported the expenses
of participants. Vincent Delecroix made introductory and advanced talks about
SageMath and Python.
  
\textbf{Event summary.} The first day was dedicated to a SageMath introduction.
Each day in the afternoon, we had a demonstration from a participant. The rest
of the time was dedicated to programming.

\textbf{Demographic.}
11 persons took part in these Sage Days: 2 females and 9 males, originating
from the following countries: Mexico (3), Canada (3), France (3), USA (2)

\textbf{Results and impact.}
\begin{itemize}
\item A step toward the convergence of the IPython and SageMath notebooks with \url{https://trac.sagemath.org/ticket/20562} (REFERENCE TO WORKPACKAGE)
\item A fix in SageMath for a problem discovered Maxime Fortier-Bourque during the workshop (\url{https://trac.sagemath.org/ticket/20566})
\item Charles Fougeron's code about Lyapunov exponents gets integrated in \url{https://github.com/videlec/sage-flatsurf}.
\item New visualization tools for geometry of translation surfaces at \url{https://github.com/videlec/flatsurf-package}.
\item And several experimentations by the other participants.
\end{event}


\begin{event}{Sage Days 74: Differential geometry and topology}{SD74}{Observatoire de Paris, Meudon, France, 30 May - 2 June 2016}{PS, UB}{26}{https://wiki.sagemath.org/days74}

\textbf{Main goals.} This workshop was dedicated to the implementation of some
topology and differential geometry in SageMath, partly in connection with the
SageManifolds project \url{http://sagemanifolds.obspm.fr/}. 3D visualisation
in the Jupyter notebook was also discussed.

\textbf{ODK implication.} ODK, via its Orsay and Bordeaux nodes, supported the travel and living expenses of 7 speakers:
\begin{itemize}
\item Marck Bell (U. Illinois, Urbana-Champaign)
\item Marck Culler (U. Illinois, Chicago)
\item Nathan Dunfield (U. Illinois, Urbana-Champaign)
\item Patrick Hooper (City College of New York)
\item Vincent Delecroix (U. Bordeaux)
\item Jeremy L. Martin (U. Kansas, Lawrence)
\item John Palmieri (U. Washington, Seattle)
\end{itemize}

\textbf{Event summary.} Morning sessions were devoted to talks on various
topics relevant to the workshop theme, some of them involving codes that are
not part of SageMath (SnapPy, Flipper, Gyoto).
Afternoon sessions were devoted to working groups and coding sprints.

\textbf{Demographic.}
26 persons took part in these Sage Days: 5 females and 21 males, originating
from the following countries: France (11), USA (8), Poland (3), Germany (2), Russia (1) and UK (1).


\textbf{Results and impact.}
41 SageMath tickets have been written or reviewed during the workshop; the list
of them is available at \url{https://trac.sagemath.org/query?keywords=~sd74&or&keywords=~days74}
Progresses on the K3D-jupyter visualisation are reported at
\url{https://wiki.sagemath.org/K3D-tools}.

\end{event}


\begin{event}{Sage Days 78}{sd78}{Vancouver (Canada), 2016-06-29 to 2016-07-01}{PS, UB}{30}{https://wiki.sagemath.org/days78}

\textbf{Main goals.} The event was organized as a satellite event of the yearly international conference
in algebraic combinatorics \href{https://sites.google.com/site/fpsac2016/}{FPSAC}. The objective was to gather
the combinatorics community around Sage development, to introduce Sage to newcomers (especially graduate students) and
to bring new Sage contributions.

\textbf{ODK implication.} the event was co-organized by ODK (through Viviane Pons) and the 
\href{https://www.pims.math.ca/}{Pacific Institute for the Mathematical Science} where
it was hosted. The event costed around 4000 CAD (2000 CAD from ODK).
 A short presentation about ODK was made during the conference to present 
the project to the participants.

\textbf{Event summary.} We started the event by some introduction presentations and tutorials so that
the participants would familiarize themselves with Sage. Then the time was shared between lectures
and coding sprints. Here are some highlights:
\begin{itemize}
\item Our invited speaker \textbf{Mike Zabrocki} (York Univ.) gave a lecture on \emph{Open Problems in Combinatorial Representation Theory}.

\item \textbf{Emily Gunawan} (Univ. of Minnesota) and \textbf{Jessica Striker} (North Dakota State Univ.) gave respectively
a tutorial and a lecture on \emph{Research-based coding} for Sage.

\item An undergrad student \textbf{Amit Jamadagni} gave a presentation of the extensive package on \emph{Knot Theory} that
he developed during a Google Summer of code project.
\end{itemize}

The full program can be found on the website \url{https://wiki.sagemath.org/days78}. We planned lots of time for 
participants to work on development projects such as: Plane partitions, plotting functions for combinatorics
objects, Lie algebras, Rook replacement...

\textbf{Demographic.} The participants were required to fill out a demographic survey. We had 29 participants (24 males and
5 females), 27 identified as academics: 7 professors, 6 postdoc, 11 graduate students, and 3 undergrads. 19 participants
were from North America (10 from Canada and 9 from the US), 8 were from Europe (France, Austria, and Switzerland), and 3 from 
Asia (South Korea and India).

\textbf{Results and impact.} 
\begin{itemize}
\item \textbf{Newcomers got to use Sage for the first time:} around one
third of the participants had zero or very little experience with Sage before the meeting. By the
end of the three days, everyone had a way to use Sage (either online or on their machines)
and had written a bit of code.
\item \textbf{Newcomers got to contribute to Sage:} a lecture was given on how to contribute to Sage
and groups were formed on different projects mixing more experienced people with newcomers so
that the code that was written could end up being merged to the software. In particular, a implementation
of Plane Partitions was put together by a participant who had never used Sage before.
\item \textbf{New contributions were made in the combinatorics component of Sage:} we used the keyword
\textbf{days78} on the trac server of Sage to track the contributions that were submitted during the workshop.
 Altogether the participants
worked on 17 different tickets either reviewing
existing ticket, implementing, or creating new tickets. 6 of them already got positive reviews and are
on the process of being merged to the software.
\end{itemize}

\begin{figure}[ht]
\caption*{Participants of Sage Days 78 making Sage demo}
\begin{tabular}{cc}
\includegraphics[scale=0.3]{pictures/sd78-1.jpg}
&
\includegraphics[scale=0.3]{pictures/sd78-2.jpg}
\end{tabular}
\end{figure}



\end{event}

\begin{event}{Sage Days 75}{sd75}{Cernay la Ville (France), 2016-08-22 to  2016-08-26}{PS, UV, UJF}{30}{https://wiki.sagemath.org/days75}

\textbf{Main goals.} The event was organized primarily by the Inria project
Actis, to celebrate the termination of its two year lifetime period. The purpose
of the project was a major redesign and implementation of the coding theory
features of SageMath. Hence the workshop gathered researchers from coding
theory, and related topics, including cryptography, group theory, combinatorics,
and linear algebra.
The goal was to expose the results of the project to the community ensure its
proper integration into the main frame of the software, and initiate new
projects, so that its development would carry over with, after the end of the
Actis engineer position

\textbf{ODK implication.} the event was co-organized by ODK (through Clément
Pernet, UJF) and Inria's Actis project. The event costed around 5000eur (including 300eur for ODK).
A short presentation about ODK was made during the conference to present the project to the participants.

\textbf{Event summary.} We started the event by some introduction presentations and tutorials so that
the participants would familiarize themselves with Sage. Then the time was shared between lectures
and coding sprints.

The full program can be found on the website \url{https://wiki.sagemath.org/days75}. 

\textbf{Results and impact.} 
\begin{itemize}
\item \textbf{New comers got to use Sage for the first time} around one
third of the participants had zero or very little experience with Sage before the meeting. By the
end of the three days, everyone had a way to use Sage on their machines and had written a bit of code.
\item \textbf{New comers got to contribute to Sage}: a lecture was given on how to contribute to Sage
and groups were formed on different projects mixing more experienced people with new comers so
that the code that was written could end up being merged to the software. 
\item Formal thematic talks were given on the new coding theory component of Sage (David
  Lucas), exact linear algebra (Clément Pernet), Algebraic Coding Theory (Johan Rosenkilde) and Rank metric codes (Arpit
  Merchant), and other less formal, or lightning talks were organized upon
  request by participants.
\item \textbf{New contributions were made in the coding and linear algebra components of Sage}: we used the keyword
\textbf{days75} on the trac server of Sage to track the contributions that were submitted during the workshop.
 Altogether the participants
worked on 35 different tickets either reviewing existing ones, implementing, or
creating new tickets. 12 of them already got positive reviews and are on the process of being merged to the software.
\end{itemize}




\end{event}

\newpage
\subsection{Training activities for Sage in developing countries}

As open-source software developers, we wish our products
to be accessible to as many people as possible. Even though we offer
 a free access, there is still a technical gap in many 
developing countries that 
often prevents schools and researchers to benefit from our softwares.
This is why we believe the role of \ODK is to foster 
a wider community that does not leave a part of the world behind. In 
this section, we describe training activities that have been conducted 
through \ODK in this regard.



The events \ref{event-2015DIMACOS}, \ref{event-2016HADAT}, and
\ref{event-2016CATN} described in this section are part of a long term
plan from Adrien Boussicault (Bordeaux University) to create an active
and autonomous Sage community in the Mediterranean area by repeated
events and the creation of a local team. We also include event
\ref{event-2016ECCO}, where Viviane Pons (Université Paris-Sud)
represented \ODK (ODK), which took place in Colombia and was
\href{http://blogs.ams.org/matheducation/2016/08/22/an-inclusive-maths-conference-ecco-2016/}
{a positive example of the inclusive and open spirit} we want to
achieve.

\begin{event}{Conference-school on Discrete Mathematics and Computer Science 2015}{2015DIMACOS}{University of Sidi Bel Abbès, Algeria -- 15-19 November 2015}{UB}{30}{https://www.univ-sba.dz/ldm/dimacos/}

\textbf{Main goals.} DIMACOS 2015 is a mathematical conference that took place in 
Algeria. The first goals of ODK was to make an initiation on SageMath. 
The second goal was to create a team of sage developer in Algeria.

\textbf{ODK implication.} A. Boussicault was sent by ODK to give a lesson on
Sage Math. ODK paid the travel and accommodation of A. Boussicault.

\textbf{Event summary.}
Each intervention was 2 hours long. There were one intervention by day during 
a week.
The lessons were conducted on computers and the computer were prepared by 
Algerian researchers (Professor A. Belahcene).
All the evening, we made Linux installfests on laptops. 
The evening sessions were beginning at 20h30 and ended at midnight.

The lessons were presented by A. Boussicault (University of Bordeaux) and 
Z. Chemli (University of Paris-Est). The purpose was to present and make 
mathematical calculus with Sage. We made also some python lessons and 
Combinatorics lessons.

During that event, we discussed with professor H. Belbachir 
(University USTHB in Alger) and professor I. Boudabbous 
(University of SFAX in Tunisia) to plan another Sage Math event in the 
conference : the conference "Combinatoire, Algèbre et Théorie des Nombres" 
in Monastir - Tunisia. 


\textbf{Demographic.} 30 particpants were present for the lessons. 
DIMACOS is an international conference. 
The people present came from Lebanon, Algeria, Tunisia, Morocco, etc.

\textbf{Results and impact.} 
This event allowed us to work with Imad Eddine Bousbaa, a PhD student. 
He helped us during the sage lessons.
It was the starting point of a collaboration that allowed us to recruit him in
the ODK project.

His recruitment is part of the will to mount team of Sage developer in Algeria.

We could use this event to prepare the next conference : "Combinatoire, Algèbre et Théorie des Nombres"
in Tunisia (event \ref{event-2016CATN}).


\end{event}


\begin{event}{School in Lebanon University}{2016HADAT}{Lebanese University, HADAT -- 05-11 Mars 2016}{UB}{7}{}

\textbf{Main goals.} The goal was to make a lesson on Polya theory and then
to make an initiation on Sage Math to implement Polya Theory.

\textbf{ODK implication.} ODK paid the travel of A. Boussicault.

\textbf{Event summary.}
Each intervention was 4 hours long. There was one intervention per day during 
a week.

\textbf{Demographic.} 7 Students of Master 2 in mathematics in Lebanese University.
Their professor was Amine Sahili of Lebanese university.

\textbf{Results and impact.}
This was the occasion of a first contact with Amine Sahili.


\end{event}


\begin{event}{Conference Combinatoire, Algèbre et Théorie des Nombres}{2016CATN}{Monastir, Tunisia -- 24-28 Mars 2016}{UB}{30}{http://www.edsf.fss.rnu.tn/Colloque1/colloque3.html}

\textbf{Main goals.}
CTAN 2016 was a mathematical conference that took place in Tunisia.
The first goal of ODK was to make an initiation on SageMath. 
The second goal was to create a sage developer team in Tunisia.

\textbf{ODK implication.} A. Boussicault was sent by ODK to give a lesson on
SageMath. ODK paid the travel and accommodation of A. Boussicault.

\textbf{Event summary.}
Each intervention was 2 hours long. There was one intervention per day during 
a week. 
The lessons were conducted on participants' computers.
We worked with Sage by using a Debian live USB key.

All the evening, we made Linux installfests on partipant's laptops. 
The evening sessions were beginning at 20:30 and ended at midnight.

The lesson was presented by A. Boussicault (University of Bordeaux) and 
Imad Eddine Bousbaa (University of USTHB of Algeria). We met Imad during a 
previous conference in Algeria.
The purpose was to present and make mathematical calculus with Sage.
We made also some python lessons and Combinatorics lessons.

During that event, we discussed with CTAN organizers and some researchers 
to organize some important mathematical events in Tunisia, Algeria, Morocco and
Lebanon.
Many researchers were present in the discussion. For example, there were 
Professor H. Belbachir (University USTHB in Alger), 
Professor I. Boudabbous (University of Sfax in Tunisia)
Professor O. Khadir (University Hassan II of Casablanca),
Professor M. Pouzet (University of Lyon1),
professor H. Kheddouci, (University of Lyon1).
We decided to organize, in the next year, an event with the mathematical 
conference CTAN followed by mathematical schools and some sage Days.
In the school, we could make some Sage lessons and we could implement ideas 
and mathematical tools during the sage days.

\textbf{Demographic.} 30 participants. CTAN is an international conference. 
The people came from Lebanon, Algeria, Tunisia, Morocco, etc. 

\textbf{Results and impact.}
The will to organize an event with a mathematical conference 
followed by mathematical schools, and sage Days in Morocco or in Algeria.

We made some Tunisian contacts to create a developer team in University of 
Sfax.

\end{event}


\begin{event}{5th Encuentro Colombiano de Combinatoria}{ecco}{Medellin (Colombia), 2016-06-13 to 2016-06-24}{PS}{130}{http://ecco2016.combinatoria.co/}

\textbf{Main goals.} ECCO is a combinatorics summer school organized every other years in Colombia. It welcomes 
students from all over the world of all levels: from undergrad to postdocs. It is known to be a very
interesting event and has a great impact for combinatorics in Colombia and South America in general.
The Sage community in combinatorics being very active, it was a great occasion to introduce Sage
to a new generation of researchers.

\textbf{ODK implication.} Viviane Pons was sent by ODK to give two Sage interventions
during the school.

\textbf{Event summary.} Each intervention was 2 hours long with around 50 participants each time.
Most participants were using the university computers. 50 usb sticks were bought previous
to the conference and set up with bootable Linux and sage, allowing for a very quick setup
during the two sessions without relying on on-line options. The students worked on some
introduction tutorials and also specifically made tutorials in relation with the on-going
classes.  

\textbf{Demographic.} 66 participants came from South and Central America, 34 from North America
and 30 from Europe. 

\textbf{Results and impact.} 
\begin{itemize}
\item The organizers were very happy that ODK would propose to send someone at the school. 
They did not have anyone who could carry such an intervention which requires both Sage and
technical skills.

\item It was quite a challenge to get Sage to work in approximatively 10 minutes for 50
computers all together. The solution of the bootable USB key has been developed by Thierry
Monteil but is not well known nor well documented. This was an occasion to test this solution
in this particular setup and improve our experience for future events.

\item The bootable USB keys were very successful and many students brought their own keys to
get a copy of the software. During the sessions, we could also help students install
Sage on their own computers.

\item Many of the students, especially the younger ones from South America, had never used
Sage before. We proposed many different tutorials so that everyone could have something to
work on and we created exercises related to the class content of the two weeks. We received
 enthusiastic feedbacks for the sessions.

\item Some of the introduction tutorials of the Sage documentation were translated into 
Spanish for the sessions and will eventually be added to Sage.

\item The conference in general was a very rewarding event. It has been growing and successful
for the past ten years with a strong focus on inclusivity and impact. It was a great occasion
to be part of the event and learn from their experience. A blog post from Viviane Pons was
published by the \emph{AMS Blog on Teaching and learning mathematics} [CITE HERE].
\end{itemize}



\begin{figure}[ht]
\caption*{Participants of ECCO at the Sage sessions}
\includegraphics[scale=0.5]{pictures/ECCO-1.jpg}

\includegraphics[scale=0.5]{pictures/ECCO-2.jpg}
\end{figure}



\end{event}
\newpage

\subsection{Communication and participation to external events}

Dissemination activities also include the participation of \ODK
members to many different conferences of various size and topics
in computer science, mathematics, physics, and more. The goal is
to reach potential end-users, build bridges between communities and stay aware 
of current development in the scientific community.

We list here major events and communication. We have also put in place
a blog on our website: \url{http://opendreamkit.org/activities/} to track
these activities.

\begin{event}{PyConFr}{pyconfr}{Pau (France), 2015-10-17 to 2015-10-18}{PS}{around 200}{http://www.pycon.fr/2015/}

\textbf{Main goals.} PyConFr is the main gathering of the python community in France. It is a good place to meet the 
French open source Python community and to talk and learn about projects.

\textbf{ODK implication.} Viviane Pons was present at the meeting and she gave a talk
on her teaching experience using SageMathCloud \cite{15PonsSMC} (peer-reviewed submission). She was
also part of a panel on diversity. 

\textbf{Results and impact.} This event was an occasion to introduce ODK to the larger python community in France as well as to
keep active the link between Sage development and other Python open source projects. It is always a great occasion to discuss subjects
 such as user interfaces, cloud servers, best practices, communities, etc. The diversity panel in particular was a great success bringing
 together most of participants of the event. As this is a great concern for ODK, we were happy to be part of it.
\end{event}

\begin{event}{Finite Simple Groups: Thirty Years of the Atlas and
    Beyond}{tyatlas}{Princeton (US California), 2015-11-02 to 2015-11-05}{SA}{94
    }{http://math.arizona.edu/~grouptheory/princeton/}

\textbf{Main goals.} Gather contributors to and users of the ``Atlas of Finite
Simple Groups'' and other mathematical databases to learn about past, presence,
and future.

\textbf{ODK implication.} ODK was not the main organizer of this event, it was used to fund
one project member (Markus Pfeiffer).

\textbf{Event summary.} The event featured talks by high-profile mathematicians,
such as John H. Conway, John Thompson, Michael Aschbacher, and many more.

Discussion sessions highlighted the need for mathematical knowledge
stored in databases. Some major examples that were discussed are
\begin{itemize}
  \item The ``ATLAS of Finite Group Representations - Version 3''
    http://brauer.maths.qmul.ac.uk/Atlas/v3/
  \item The ``Online Encyclopedia of Integer Sequences (OEIS)''
    http://oeis.org
  \item The ``Modular forms and L-functions database (LMFDB)''
    http://lmfdb.org
  \item The ``Small Groups Database'' (small)
\end{itemize}

Most of these databases share common issues such as \emph{reliability} of the
data, the \emph{reliability} and \emph{longevity} of the storage,
\emph{maintenance}, and \emph{managing contributions}.
    
\textbf{Results and impact.} 
The attendance of this conference shed light on how some mathematicians view
mathematical databases, and what issues they see. This is an important
contribution for WP6. It also contributed to the attendee's understanding of the
needs of our potential users (WP4).

\begin{itemize}
\item \longtaskref{dksbases}{data-assessment} Survey of existing DKS bases, Formulation of Requirements
\item \longtaskref{dksbases}{mws} Math Search Engine

\end{itemize}

Further discussions with GAP users and developers about HPC-GAP were a
side-effect of the attendance of this event. 

\end{event}

\begin{event}{13th Joint Magnetism and Magnetic Materials (MMM) -
    Intermag Conference}{mmm2016}{San Diego (CA, USA), 2016-01-11 to
    2016-01-15}{SOUTHAMPTON}{30}{https://magnetism.org}

  \textbf{Main goals.} The main goals of presenting the project at
  this important international Joint Magnetism and Magnetic Materials
  and Intermag conference was to introduce the Micromagnetic Virtual
  Reasearch Environment (VRE) to our target user audience - the
  micromagnetic scientific community.


\textbf{ODK implication.} The OpenDreamKit project has sent the
speaker (Hans Fangohr), and paid expenses for the trip.

\textbf{Event summary.} Hans Fangohr submitted a talk [1] that was peer
reviewed and accepted for presentation. In the talk, he outlined the
vision for the project and invited feedback from the community.

The ODK project for computational micromagnetics was discussed with
various attendees informally throughout the conference.

In addition, we organised a meeting with the main developers of the
OOMMF micromagnetic simulation code, Dr M. Donahue and
Dr D. Porter, in order to discuss the project's vision, our plans for
interfaces to get early feedback.

\textbf{Demographic.} About 30 people were present, but no demographic survey was done.

\textbf{Results and impact.} We announced the project and its website
to the community and encouraged input to extend our vision, to make
sure the tool we develop can be as practical and efficient for as
large parts of the community as possible.

[1]
\href{http://joommf.github.io/assets/2016-01-12-MMM2016-AD02-Jupyter-OOMMF.pdf}{pdf
of slides}
\end{event}


\begin{event}{International Workshop on Software Engineering for
    Science}{se4science2016}{Austin (TX, USA),
    2016-05-16}{SOUTHAMPTON}{15}{http://se4science.org/workshops/se4science16/}

  \textbf{Main goals.} Spread recommendations to support better
  science in the area of software engineering for computational research.

  \textbf{ODK implication.} The work presented has been created with
  the upcoming Jupyter OOMMF integration in mind, and is of wider
  interest to the OpenDreamKit partners and users. The conference
  attendance was paid from a different grant.

  \textbf{Event summary.} Hans Fangohr delivered a talk on Software
  Engineering for Computational Science, in particular reviewing
  technical and social aspects of a computational science code that
  was developed about 10 years ago. The presentation [1], and associated
  publication [2] extracted lessons learned from the past and with the aim
  to enable the community to identify potential mistakes sooner. The
  presentation and work provides recommendations to enable better
  science in the field of computational science and engineering; in
  particular focusing on software engineering for computational
  science and research codes.

  The talk was the keynote presentation of the morning session in the
  workshop on Software Engineering for Science (30 minutes).

  \textbf{Demographic.} About 15 people were present, 3 female.

  \textbf{Results and impact.} We reported evidence from the
  effectiveness of particular sofware engineering techniques and
  provided recommendations for future projects (including user
  interface, testing, version control, documentation,
  installation).

  [1]
  \href{http://www.southampton.ac.uk/~fangohr/publications/talk/2016-05-16-ICSE-SE4Science-Austin-Texas-US.pdf}
  {slides as pdf}

  [2] Hans Fangohr etal, ``Maximilian Albert and Matteo Franchin Nmag
  micromagnetic simulation tool - software engineering lessons
  learned'', Proceedings of the International Workshop on Software
  Engineering for Science, at ICSE2016 SE4Science '16, Austin, Texas,
  US, 1-7 (2016)


\end{event}


\begin{event}{PyCon}{pycon}{Portland (Oregon), 2016-05-28 to 2016-06-02}{PS}{around 3000}{https://us.pycon.org/2016/}

\textbf{Main goals.} PyCon is the biggest Python conference in the world. It is the best place to learn
about the python community, open-source tools, new technologies, etc. It is also a good place
to grow a network in the software development community.

\textbf{ODK implication.} Viviane Pons was present at the meeting for the third time in a row,
consolidating her effort to build links between Sage and Python communities. In 2015, she had given
a talk and organized a parallel Sage Days event. It was not possible to do so this year but a
Sage sprint was still maintained.

\textbf{Results and impact.} The conference itself was very instructive as usual in thematics such as:
efficient programming, parallel computing, open-source community building, teaching, inclusivity. It
was a good occasion to discuss with other python programmers and introduce the ODK project. The academic
community did not seem as present as it had been in the previous years. In the future, we might want to
target smaller events such as SciPy and EuroScipy.
\end{event}

\begin{event}{Invited Seminar Talk at ``Universidade Nova de
    Lisboa''}{lisbon}{Lisbon, Portugal), 2016-07-19}{USO}{6 participants}{}

\textbf{Main goals.} Advertise capabilities of \ODK with a research talk
that showcases the GAP Jupyter interface.

\textbf{ODK implication.} Markus Pfeiffer showcased the GAP Jupyter notebook
interface as part of a research seminar talk on search in permutation groups.

\textbf{Results and impact.} 
This seminar talk was a good opportunity to advertise \ODK outside of
our core developer or user groups.

\end{event}


\begin{event}{EuroSciPy}{euroscipy}{Erlangen (Germany), 2016-08-24 to 2016-08-27}{SR}{around 100}{http://euroscipy.org/2016/}

\textbf{Main goals.} EuroSciPy is a gathering of the scientific Python community in Europe.
It brings together Scientific users and tools developers. Attending EuroSciPy
 is a great way to confront ideas about the future of scientific development
in relations with OpenDreamKit work plan.

\textbf{ODK implication.} Benjamin Ragan-Kelley presented on the Jupyter project as a whole.
Thomas Kluyver discussed Jupyter notebooks as academic publications.
Vidar Fauske presented on nbdime, a deliverable in Work Package 4.
A sprint was organized, gathering some new contributors for Jupyter projects,
and useful discussion was had on the prospects of nbdime in the scientific software community.

\textbf{Results and impact.} The conference produced good conversations on the future of the Jupyter project and how Work Package 4 can improve scientific and open source work.
There were many discussions on the prospect of open source practices improving the scientific process,
which inform how OpenDreamKit can have the most impact moving the scientific community forward.
\end{event}



\section{Upcoming events and plans for the future}

This past year, \ODK has been organizing or co-organizing
14 different events, thus helping and supporting the open-source
communities on which the project is built. Some of these events had
been announced in the proposal: the Sage development workshop (event \ref{event-sd77}) , the Atelier
Pari (event \ref{event-AtelierPARI2016}), the FPSAC Sage Days (event \ref{event-sd78}), as
well as the events in developing countries. We have also decided
to support extra events which were directly helping \ODK goals
or beneficial for the entire community. 
We believe that this persistent presence of the \ODK project within
the open-source community activities is a key part of our success 
and we plan on keeping it on. The next few years look just as much eventful
as this past year. 

\begin{itemize}
\item Some \textbf{development workshops} are already planned such as \textbf{Atelier Pari} in Lyon,
a \textbf{high performance computing} in Grenoble and probably at least one \textbf{Sage Days} in Orsay.

\item We are working on three \textbf{major dissemination events} that should tentatively take place
at CIRM, Dagstuhl, and ICMS. We have formed organizing committees and are working on proposals.

\item The first ever \textbf{Women in Sage} in Europe has been announced and will take place
in France in January.
\end{itemize}

\bibliographystyle{plain}
\bibliography{report}

\footnotesize{All referenced ODK talks are available as annexes}
\end{document}

%%% Local Variables:
%%% mode: latex
%%% TeX-master: t
%%% End:
