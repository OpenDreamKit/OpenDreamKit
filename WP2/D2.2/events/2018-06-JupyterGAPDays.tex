\begin{event}{GAPDays Jupyter in GAP and other CAS}{GAPDays2018}{St Andrews (UK),
2018-06-04 to 2018-06-08}{SA,PS}{19}{http://www.gapdays.de/gap-jupyter-days2018/}


\textbf{Main goals.}

The aim of this workshop was to bring together people who are developing, using,
or want to use Jupyter, MyBinder, and ThebeLab for research and teaching. The
main focus of the workshop was GAP and Jupyter, people developing Jupyter or
using Jupyter to access other software were welcome to attend and give input.

Topics that were worked on included:

\begin{itemize}
\item The GAP JupyterKernel
\item Javascript-Visualizations using Jupyter
\item Using Thebe for interactive manuals
\item Developing teaching materials with Jupyter, and publishing them using MyBinder
\item Writing academic publications using MyBinder and Docker images, to make
  all computational results and all examples fully and easily reproducible.
\item Discussing demands on software and hardware infrastructure in day to day
  use.
\end{itemize}

\textbf{ODK implication.} 
%Describe how ODK was involved and give a rough estimation of cost for ODK

GAPDays were mainly supported by \ODK funding travel, accommodation, and subsistence
for participants. The University of St Andrews kindly provided a lecture room
for the meeting free of charge.

\textbf{Event summary.} 
%Give a summary of your event

The event consisted of the following talks
\begin{itemize}
\item \emph{Introduction to the GAP-Jupyter kernel} -- Markus Pfeiffer (University of St Andrews),
\item \emph{Basic setup for Binder} -- Sebastian Gutsche (University of Siegen),
\item \emph{ThebeLab demo} -- Nicolas M. Thiéry (Paris Sud),
\item \emph{Experiencing with Jupyter GAP} -- Pedro Garcia-Sanchez (Universidad de Granada),
\item \emph{FSR - Feedback Shift Registers} -- Nusa Zidaric (University of Waterloo).
\end{itemize}

To structure programming work and discussions, we followed the well-established
practice of a stand-up meeting in the morning to coordinate work, and a meeting
in the afternoon to record achievements.

Discussions ranged from deepining the understanding of how Jupyter works, and
what needs the research and teaching community has to more technical topics
which reach into other workpackges such as improved integration between SageMath
and GAP.

\textbf{Results and impact.} 
% What did you achieve with this event? (If ever it impacted 
% other ODK tasks and deliverables, mention it here)

The workshop was very productive. The goal of a readily packaged JupyterKernel
for GAP was achieved, an improved Docker deployment of GAP has been provided.
Apart from work on WP4, we also achieved progress on WP3 and WP6 through
discussions and indentification of synergies between the software demands of
these workpackages.


\end{event}
