\begin{event}{EuroSciPy}{euroscipy}{Erlangen (Germany), 2016-08-24 to 2016-08-27}{SR}{around 100}{http://euroscipy.org/2016/}

\textbf{Main goals.} EuroSciPy is a gathering of the scientific Python community in Europe.
It brings together Scientific users and tools developers,
and is a great way

PyCon is the biggest python conference in the world. It is the best place to learn
about the python community, open-source tools, new technologies, etc. It is also a good place
to grow a network in the software development community.

\textbf{ODK implication.} Benjamin Ragan-Kelley presented on the Jupyter project as a whole.
Thomas Kluyver discussed Jupyter notebooks as academic publications.
Vidar Fauske presented on nbdime, a deliverable in Work Package 4.
A sprint was organized, gathering some new contributors for Jupyter projects,
and useful discussion was had on the prospects of nbdime in the scientific software community.

\textbf{Results and impact.} The conference produced good conversations on the future of the Jupyter project and how Work Package 4 can improve scientific and open source work.
There were many discussions on the prospect of open source practices improving the scientific process,
which inform how OpenDreamKit can have the most impact moving the scientific community forward.
\end{event}
