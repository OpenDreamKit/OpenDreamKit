\begin{event}{First Joint GAP-SageMath Days}{GAPSage2016}{St Andrews (UK),
2016-01-18 to 2016-01-22}{SA,PS,UV,UK,UO}{19}{http://gapdays.de/gap-sage-days2016/}

\textbf{Main goals.}

Both GAP and SageMath systems have traditions of regular developer meetings, where those
interested in these systems, from newcomers to contributors, are gathering together for
collaborative code writing, sharing best practices, advertising recent new features and
improvements, and discussing further developments. You can find the list of previous GAP 
days at \url{http://gapdays.de/} and of SageMath days at \url{https://wiki.sagemath.org/Workshops}.

Following these traditions, it was decided to organise the 1st Joint GAP-SageMath Days, with the
focus on improving GAP-SageMath integration and interaction between these systems and between
their developers.

\textbf{ODK implication.} 
%Describe how ODK was involved and give a rough estimation of cost for ODK

The 1st Joint GAP-SageMath Days were mainly supported by CoDiMa -- Collaborative Computational
Project (CCP) in the area of Computational Discrete Mathematics (EPSRC grant EP/M022641/1,
\url{http://www.codima.ac.uk/}). It was immediately followed by the WP7 Workshop ``Knowledge
representation in mathematical software and databases'' on January 25th-27th, 2016, therefore
\ODK participants involved in GAP and/or SageMath development could conveniently attend
both events. Accommodation, subsistence and travel expenses of partners from UPSud and UVSQ
were paid by the \ODK project, and those of partners from UNIKL,UOXF were reimbursed
by the CoDiMa project.

\textbf{Event summary.} 
%Give a summary of your event

A typical day of the workshop had one or two introductory talks to facilitate subsequent
discussions and coding sprints, in particular:
\begin{itemize}
\item \emph{Contributing to Sage} by Nicolas M. Thiéry and Volker Braun
\item \emph{Contributing to GAP } by Max Horn, Alex Konovalov, and Markus Pfeiffer
\item \emph{libGAP} by Volker Braun
\item \emph{GAP in the cloud} by Markus Pfeiffer
\end{itemize}
Other topics included, among others,
further integration of HPC-GAP into GAP;
working on the semantic-aware SageMath interface to GAP;
improving the installation of GAP in SageMath and in SageMath cloud;
creating and working with Docker containers;
development of GAP and SageMath teaching materials for Software Carpentry,
etc.

\textbf{Results and impact.} 
% What did you achieve with this event? (If ever it impacted 
% other ODK tasks and deliverables, mention it here)

The workshop was very productive. Only to the main GAP repository 
(\url{https://github.com/gap-system/gap/}) there were 51 new pull 
requests submitted, just 8 of which are still open; in addition,
28 new issues were created (9 of them are closed by now), and 
there was also progress achieved with GAP packages developed 
elsewhere; the work on converging GAP and HPC-GAP; discussing 
development workflows, etc. It helped to both GAP and SageMath 
teams to get further insights into each other's systems and was 
particularly beneficial to WP3 (component architecture), WP4 
(user interfaces) and WP5 (high-performance computing). 

\end{event}
