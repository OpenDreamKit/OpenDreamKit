\begin{event}{Modular Knowledge}{tetrapod2018}{Oxford, 2018-07-13}{FAU,PS}{20}{3}{https://kwarc.info/events/Tetrapod-2018/}

\textbf{Main goals.}
Expanding on the Tetrapod workshop at the conference on intelligent computer mathematics (CICM) 2016, this workshop brings together researchers from a diverse set of research areas in order to create a universal understanding of the challenges and solutions regarding highly structured knowledge bases.
Of particular interest are foundational principles such as theory graphs and colimits, interchange languages and module systems, languages and tools for representing, reasoning, computing, managing, and documenting modular knowledge bases.

\textbf{\ODK implication.}
This was co-organized by \ODK participants together with Jacques Carette, in order to connect with researchers in adjacent fields.
Cost to \ODK was limited to travel cost.

\textbf{Event summary.}
The event had 6 invited speakers, each of which introduced a specific topic.
This consisted of a 15-minute presentation, which was followed by a 30-minute free discussion session.

\textbf{Demographic.}
Researchers in logic, mostly male, mostly senior.

\textbf{Results and impact.}
No direct impact on \ODK tasks and deliverables but it informed \WPref{dksbases} in general.


\end{event}
