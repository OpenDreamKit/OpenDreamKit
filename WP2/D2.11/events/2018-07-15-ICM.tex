\begin{event}{ICM 2018 - Computational micromagnetics with JOOMMF workshop}{ICM2018}{San Francisco, CA, USA, 15-20 July 2018}{XFEL}{130}{2}{No url.}

\textbf{Main goals.} In this workshop we taught the participants to run micromagnetic simulations using JOOMMF.

\textbf{ODK implication.} JOOMMF was developed as a part of the ODK project and two participants from the ODK were present to deliver the workshop (Marijan Beg, and Ryan A. Pepper). The workshop was fully funded from the ODK project and the total costs were 7259.58 euros.

\textbf{Event summary.} We started this 3.5 hours workshop by helping the participants to install all required software on their laptops. After that we introduced some physics basics of micromagnetics as well as theoretically explained how does a micromagnetic simulation work. After that we went through the tutorials and explained all individual commands of JOOMMF required for participants to complete the exercises. Finally, participants were working on an exercise and managed to reproduce the results from already published work. We had a lot of opportunity to talk to the existing and potential users and get feedback from them.

\textbf{Demographic.} We had around 150 participants, but due to the data protection we could not obtain any demographics data from them.

\textbf{Results and impact.} During the workshop we received the feedback from the participants about our Python interface to OOMMF as well as gained experience which helped us to structure future workshops.

\begin{figure}[ht]
\include{ICMPhoto1.jpg}
\caption*{JOOMMF workshop at ICM2018 conference in San Francisco, CA, USA}
\end{figure}

\begin{figure}[ht]
\include{ICMPhoto2.jpg}
\caption*{JOOMMF workshop at ICM2018 conference in San Francisco, CA, USA}
\end{figure}

\begin{figure}[ht]
\include{ICMPhoto3.jpg}
\caption*{JOOMMF workshop at ICM2018 conference in San Francisco, CA, USA}
\end{figure}

\end{event}
