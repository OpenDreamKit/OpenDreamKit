\begin{event}{LinBox Days December 2017}{LinBoxDays17}{Camaret sur Aigue (FR),
2017-12-13 to 2017-12-15}{UJF}{9}{3}{https://github.com/linbox-team/fflas-ffpack/wiki/LinBox-developper-meeting-in-Camaret}

\textbf{Main goals.}

The LinBox developper meeting are meant to gather developper and users of the
LinBox ecosystem (composed of \texttt{givaro}, \texttt{fflas-ffpack} and
\texttt{LinBox}).

The main goals are to enable interraction between the several groups of
developpers and users in order to share experiences, take collegial design
decision and practice collaborative code writing.

\textbf{ODK implication.} 
%Describe how ODK was involved and give a rough estimation of cost for ODK

\ODK participants: C. Pernet, J.-G. Dumas, H. Zhu.

\ODK provided the main funding source for the workshop (accommodation,
subsistence and travel expenses), for about 2.1k\euro.

\textbf{Event summary.} 
%Give a summary of your event

The Camaret LinBox developper meeting took place in Grignan (France), from December
13th to 15th, 2017.

There were 9 participants from 3 different institutions (UGA, Université
Montpellier 2, US Naval Academy).
No registration fees were applied.

After a round table branstorming on developpment projects for the meeting,
developpers gathered in groups each adressing a given project.
The main achievements of the meeting include
\begin{itemize}
\item Progress on the distributed parallel ration solver (\delivref{hpc}{LinBox-distributed});
\item Improvement on the polynomial matrix algorithmic;
\item wider support of SIMD micro-architectures;
\item Cleanup and simplification of the build system;
\item Improvement of code robstness and release preparation (\delivref{hpc}{LinBox-algo}).
\end{itemize}


\textbf{Results and impact.} 
% What did you achieve with this event? (If ever it impacted 
% other ODK tasks and deliverables, mention it here)

The workshop was very productive and particularly beneficial to WP5
(\delivref{hpc}{LinBox-algo} and \delivref{hpc}{LinBox-distributed}).
\end{event}
