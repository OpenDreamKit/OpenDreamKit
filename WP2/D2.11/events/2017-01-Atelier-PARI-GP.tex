\begin{event}{Atelier PARI/GP 2017}{AtelierPARI2017}{Grenoble (FR),
2017-01-09 to 2017-01-13}{PS,UB,UV,UW}{36}{http://pari.math.u-bordeaux.fr/Events/PARI2017/}

\textbf{Main goals.}

The PARI/GP Ateliers were established in 2012 as a yearly meeting
between developers and users of the PARI/GP system.

The main goals are advertising new features and improvements,
discussing further developments, sharing best practices, and collaborative
code writing (hacking sessions, doc reviews, bug-squashing parties).

You can find the list of previous PARI Ateliers at
\url{http://pari.math.u-bordeaux.fr/ateliers.html}

\textbf{ODK implication.} 
%Describe how ODK was involved and give a rough estimation of cost for ODK

\ODK participants: B. Allombert, K. Belabas, V. Delecroix, J. Demeyer,
J.-P. Flori, L. de Feo.

\ODK provided the main funding source for the workshop (accommodation,
subsistence and travel expenses), for about XXXk\euro. The Lyon
mathematic instutite (Camille Jordan) co-funded the event.

\textbf{Event summary.} 
%Give a summary of your event

The 7th Atelier PARI/GP took place in Lyon (France) from january
9th to 13th.

There were 43 registered participants from 19 different institutions
(no registration fees).

A typical day of the workshop had introductory talks and tutorials
in the morning; afternoons allowed ample time for hacking sessions,
discussions and training.

The Atelier featured 10 morning talks on mathematical topics and
implementation projects including 4 talks by ODK members
\begin{itemize}
\item Karim Belabas "Using GIT with PARI", "L-functions" and "Dirichlet characters"
\item Bill Allombert "New GP features"
\end{itemize}

\textbf{Results and impact.} 
% What did you achieve with this event? (If ever it impacted 
% other ODK tasks and deliverables, mention it here)

The workshop was very productive and was particularly beneficial to
WP5 (high-performance computing).

\end{event}
