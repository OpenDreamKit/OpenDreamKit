\section{Follow-up of recommendations and Quality Management}

In this section, we will detail our actions in response to the recommendations and
comments of the reviewers and review the risk management and quality assurance procedures
adopted in \pn.

\subsection{Follow-up of recommendations}

We are extremely grateful for the very constructive comments and
recommendations that were provided during the review itself and in the
formal report.

\newtheorem{recommendation}{Recommendation}{}

\begin{recommendation}
  A minor aspect: In the deliverable D5.11, authors have to clarify
  the reason why the speedup with the use of cores is not so high when
  you increment the number of cores. The presentation has also to be
  improved.
\end{recommendation}

Deliverable 5.11 was polished, complemented with a clarification and
resubmitted after the review.

\begin{recommendation}
  To include the KPIs in a centralized way in the technical report (KPI table)
\end{recommendation}

All KPI's are presented within a single section of the Technical
Report for reporting period. We tried making this section into a
table; however many of our KPI take the form of qualitative narratives
and do not fit well in such a table. Instead, we made individual
tables for each quantitative KPI.

\begin{recommendation}
  Demonstration of capabilities due to project results is crucial,
  especially for test cases/show cases. Often, such demonstrations are
  extremely technical. A higher level approach to such demonstrations
  is needed, so that potential users are not taken aback by the many
  actions they need to undertake. It would be good if the project team
  discusses this, and takes action to make demonstrations more
  attractive and appealing. This is also vital for the sustainability
  of the project results.
\end{recommendation}

We have discussed the matter within the project and we will be trying
our best during the next review. We are facing an intrinsic difficulty
due to OpenDreamKit's toolkit strategy. We indeed have a long
experience of delivering very progressive demonstrations at our
training workshops; in fact we often introduce some of the technology
stack even to hundreds of freshmen students at the occasion of our
classes. However much of that technology is not a single product of
OpenDreamKit per se. Rather it is an ecosystem of products, to which
OpenDreamKit makes many contributions. Often the contributions are by
nature almost invisible from the end-user perspective, and it takes
some technical context to highlight their relevance and impact on
usability or sustainability.

\begin{recommendation}
  Financial statements must be made available to reviewers no later
  than 15 days before Review Meeting in final form and to the
  Commission much earlier.
\end{recommendation}

Our previous project manager had left the project in August 2018, and
despite a rehiring process started as early as May 2018 the position
remained vacant until December 2018. We are happy to report however
that we very lucky in the recruitment. Our new project manager Izabela
Faguet -- together with the project coordinator -- took early and
active steps to devise, advertise, and enforce a strict time line to
ensure that all partners -- including UPSud itself -- would submit
their financial reports well on time.

\begin{recommendation}
  It is to be hoped that spend can be accelerated in the next year to
  make best advantage of the funds available and to ensure maximum
  benefit to the communities.
\end{recommendation}

On the day after the review a brainstorm was run among the
participants to explore opportunities of funds reallocation. Practical
feasibility was then explored and selection of best opportunities were
then discussed in the following weeks.

The largest source of tentatively unused funds was from Leeds,
following the departure to industry of its members. Most of the
resources ($\approx$180 k euros) were redistributed to the other partners.
This proved extremely useful to organize many additional dissemination
events -- making up for Leeds departure -- add a new case study
exploring the MitM approach in the context of proof systems, and
generally speaking increase the participant involvement on existing
tasks. This is formalized and explained in the 5th grant agreement
amendment.

Another source of tentatively unused funds was from \site{UK},
Indeed, those funds were originally reserved for the organization of
conferences early in the project which could finally be covered from
other sources. It was decided that \site{UK} would cover some of
the expenses ($\approx$60 k euros) for the organisation of our large
dissemination event at CIRM, enabling \site{PS} to reallocate funds
for other dissemination events.

\begin{recommendation}
  Greater attention must be paid to acknowledgement of EU funding in
  all areas. For exemple, include the name of the project in Software
  Carpentry: Related projects:
  \url{https://software-carpentry.org/join/projects/}
\end{recommendation}

We made sure that EU funding was acknowledged by the projects we
developed or contributed to:
\href{https://jupyter.org/about}{Jupyter},
\href{http://joommf.github.io/}{JOOMMF},
\href{https://www.gap-system.org/Contacts/funding.html}{GAP},
%\href{https://github.com/K3D-tools/K3D-jupyter}{K3D}
\href{https://linalg.org/support.html}{LinBox},
\href{https://pypersist.readthedocs.io/}{pypersist},
\href{https://mathhub.info/}{MathHub},
\href{https://gap-packages.github.io/Memoisation/}{Memoisation},
\href{http://www.mpir.org/news.html}{MPIR},
\href{https://pari.math.u-bordeaux.fr/funding.html}{PARI/GP},
\href{https://www.sagemath.org/development-ack.html}{SageMath},
\href{https://github.com/sagemath/sage-combinat-widgets}{Sage Combinat widgets},
\href{https://github.com/sagemath/sage-explorer}{Sage Explorer},
\href{https://github.com/nthiery/sage-gap-semantic-interface}{Sage GAP Semantic Interface},
\href{https://www.singular.uni-kl.de/index.php/background/funding.html}{Singular},
\href{https://ubermag.github.io/}{Ubermag}.

% TODO MMT

We have reached to Software Carpentry; their site is under
reconstruction and the aforementioned page is about disappear. We are
working with them for a proper location for the acknowledgment.
Presumably this will be on the pages of the lessons that ODK
contributed to.

We also asked the partners to double check their publications.

% https://github.com/gap-packages/Memoisation
% https://github.com/mtorpey/pypersist

\begin{recommendation}
  To develop a comic explaining the MitM approach.
\end{recommendation}
The comic has been published on:
\url{https://github.com/OpenDreamKit/OpenDreamKit.github.io/blob/master/public/images/use-cases/MitM.png}. It
has already been used in the MitM use case description at
\url{https://opendreamkit.org/2018/05/16/lmfdb-usecase/}, in conference presentations and
posters.

\begin{recommendation}
  To disseminate the Adoption by Logipedia of the MitM principle of
  integrating (logical) systems by aligning concepts.
\end{recommendation}
We have made a blog post about this, see \url{https://opendreamkit.org/2019/01/24/logipedia/}.

\begin{recommendation}
  Some guidelines (set of recommendations) for using the different
  tools provided by OpenDreamKit would be recommendable.
\end{recommendation}
We have expanded our use case section on \url{opendreamkit.org} and
will keep doing so.

% , and
% wrote additional blog posts, notably on how to deploy custom VRE's; see e.g.
% \url{https://opendreamkit.org/2018/10/17/jupyterhub-docker/}.

\begin{recommendation}
  Some guidelines (set of recommendations) for using the different
  hardware architectures would be recommendable.
\end{recommendation}

\ednote{@ClementPernet: hardware architecture recommendations}

%%% Local Variables:
%%% mode: latex
%%% TeX-master: "report"
%%% End:

%  LocalWords:  newenvironment noindent textbf begingroup endgroup delivref emph WPref
%  LocalWords:  ipython-kernels specialized longlocaltaskref dissem hpc optimization
%  LocalWords:  MPIRsuperoptimiser parallelization sage-HPCcombi XKaapi OmpSS
%  LocalWords:  vectorization localtaskref sage-paral-tree
