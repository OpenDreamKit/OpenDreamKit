% ---------------------------------------------------------------------------
%  Section 2: Impact
% ---------------------------------------------------------------------------

\section{Impact}
\label{sec:impact}

\TODO{Orsay's grant services will help here in December}

\subsection{Expected Impacts}

\eucommentary{Please be specific, and provide only information that applies
to the proposal and its objectives. Wherever possible, use quantified
indicators and targets.\\
Describe how your project will contribute to:\\
-- the expected impacts set out in the work programme, under the relevant topic
(including key performance indicators/metrics for monitoring results and impacts);\\
-- improving innovation capacity and the integration of new knowledge
(strengthening the competitiveness and growth of companies by developing
innovations meeting the needs of European and global markets; and, where
relevant, by delivering such innovations to the markets;\\
-- any other environmental and socially important impacts (if not already
covered above).\\
Describe any barriers/obstacles, and any framework conditions (such as
regulation and standards), that may determine whether and to what extent
the expected impacts will be achieved. (This should not include any risk
factors concerning implementation, as covered in section 3.2.)}

\subsubsection{Impacts as listed in the work programme}

The following table shows how \TheProject  addresses the specific impacts
listed in the work programme.

\def\tmc#1{\hline\multicolumn{2}{|m{.94\textwidth}|}{\textbf{#1}}\\\hline}
\tablehead{}
\begin{supertabular}{|m{.47\textwidth}m{.47\textwidth}|}\hline
  \tmc{More effective collaborations between researchers}

  \TheProject will strengthen collaborations between European scientific community in
  different branches of mathematics, through a development of the common e-infrastracture
  by bringing together software and databases previously separated, and build links with
  scientific communities in other disciplines (biology, physics, astronomy) that will use
  this e-infrastructure. The scientific community will considerably increase, by
  integrating new actors both from academic and non academic sector. & Key performance
  indicators :
  \begin{compactenum}
  \item By the end of the project, the scientific community in mathematics using the tool
    will increase by X \%
  \item The scientific community in other disciplines (biology, physics) using the tool
    will increase by X\%
  \item Number of enterprises using the tool will increase by X\%
  \end{compactenum} \\\hline

  \tmc{Higher efficiency and creativity in research, higher productivity of researchers
    thanks to reliable and easy access to discovery, access and re-use of data}

  The development of a new integrated tool, replacing 3
    previously separated tools, a possibility of real time data sharing, data re-use and
    simultaneous working will allow an important gain in time and in efficiency, and, by
    consequence, higher productivity of researchers. Moreover, the exchange of best
    practice (such as regular audit of codes) will have an important impact on the quality
    of the research. Finally, the unique tool will allow considerably reducing the costs
    of further research.

 &

  Key performance indicators :
  \begin{compactenum}
\item The e-infrastructure developed by the end of the project will help reducing time of
  research by X\%\item The productivity of researchers will increase by X\%\item Thank to
  best practice exchange, the quality of research will considerably increase and number of
  errors will be reduced\item Better traceability of research \item Costs for research
  considerably reduced
\end{compactenum}
\\\hline
\tmc{Accelerated innovation in research via
an integrated access to digital research resources, tools and services
across disciplines and user communities}

An integrated access to digital research
resources and tools that \TheProject will provide will clearly help
accelerating innovation in research across disciplines and communities.

The integrated tool will meet the needs and help overcoming the
obstacles that are common to several disciplines impacted by the
project, and both to academic and non-academic research. Industrial
actors, actively involved into the project will directly benefit from
the project results. &

Key performance indicators :

\begin{compactenum}
\item Emergence of new, improved research methods in several disciplines
using the tool\item Resolution of series of problems proper to
industrial actors in several disciplines
\end{compactenum}
\\\hline

\tmc{Researchers able to process structured and qualitative data in virtual and/or ubiquitous workspaces}

\TheProject will enable researchers to process
different kind of  data thanks to an integrated tool interconnected
with databases. Efficient data storage will allow further exploitation
and  re-use of mathematics data for further calculations and thus make
data processing more efficient.

Le travail sur la sémantique de données est fait plus en amont.
 &
Key performance indicators :

Note E.S. – Nicolas, ici, j'ai du mal, car pas sure si on répond
vraiment à ce critère.\\\hline

\tmc{Increased take-up of collaborative research and data sharing by new disciplines,
  research communities and institutions}

The project will clearly enhance a take-up
of collaborative research and data sharing by and between new
disciplines and research communities. The communities already using the
parts of the tool, will be enlarged, involving more and more industrial
actors and young scientists.

By developing the tool, it will reinforce collaborations between
different branches of mathematics (both pure and applied). Once
developed, the tool will be opened for research in various disciplines,
such as biology, physics, astronomy etc. &
Key performance indicators :

\begin{compactenum}
\item Increased collaborations and data sharing between different communities in
  mathematics
\item Increased collaborations and data sharing between research communities in other
  disciplines
\item Increased number of users, enlarged research communities
\end{compactenum}
\\\hline
\end{supertabular}

\subsubsection{Improving innovation capacity and the integration of new knowledge}


 Innovations developed by \TheProject project will meet the needs of the
following ecosystem participants:

\begin{enumerate}
\item Device / module vendors: hardware manufacturers, equipment
manufacturers of smartphones, tablets, laptops;
\item Network providers: service providers, network infrastructure
vendors (Avaya, Juniper, Extreme, Cisco…)
\item Platform providers
\item Cloud service providers: Software-as-a-Service,
Platform-as-Service, Infrastructure-as-a-Service
\item Systeme integrators: end-to-end integration services and
value-added services (Accenture, HP, IBM…)
\item End users: research communities; IT, healthcare, education and
aeronautics stakeholders
\end{enumerate}
Industrial stakeholders will be directly involved into the project and
the VRE development, so that the tool will be exactly tailored to their
specific needs – that are the same that the scientific community ones.
Moreover, this will allow an early time-to-market and will facilitate
the technology uptake.

In the table below we have specified different market needs, and the
ways the project will address each of them :

Table XX

\begin{flushleft}
\tablehead{}
\begin{supertabular}{|m{.47\textwidth}|m{.47\textwidth}|}
\hline
\centering Needs of markets &
\centering\arraybslash How the project will address these needs\\\hline
Performance gain &
The tool will enable its users to combine functionality from XXX other
mathematical software programs and programming languages

{}-mainstream programming language Python?

{}-which compilers?\\\hline
Infrastructure capacity: newly built infrastructures with fast broadband
connections are well positioned for adoption &
\TheProject will allow different groups of users to work simultaneously, and
thus provide a considerably gain in efficiency.\\\hline
Lower cost of scaling  &
An open source architecture brings affordability: people and
organizations donate towards common goals, and small organizations can
gain access to equipment and research talent typically only afforded by
the largest firms. Resources integration will reduce considerably the
time and the costs of operations.\\\hline
Going beyond limitations of interconnect technology &
An open source architecture brings creativity (=the best minds to solve
a problem)\\\hline
Enabling new applications and features &
Through a series of links that will be created between previously
separated tools, and data interoperability, the VRE will enable new
applications and features\\\hline
Early time-to-market (TTM): companies are looking for solutions that
would improve the speed at which they can procure services to bypass
traditional information technology departments &
The speed of development will improve tremendously thanks to open
source. The implication of industrial stakeholders into the development
will allow to deliver a tool the suits the best to their needs, and
thus speed-up the time to market and technology uptake.\\\hline
Easy-to-use service: first-time experiences is crucial to gain
acceptance &
{}-Design? Ergonomie? Nice multiuser web-based graphical user interface

{}-integration with learning tools? (example : interactive whiteboard)

\TODO{E.S: Nicolas, il faut réflechir à la façon de rendre l'outil
plus attractif pour la jeune génération (=future génération des
chercheurs), car c'est la génération Y qui va définir des standards
pour la suite  {}- donc  plus d'impact et à plus long terme. Réfléchis
à la possibilité de rendre l'outil accessible sur mobiles, tablettes
etc., mais aussi à la façon ``~intéressante~'' de l'enseigner.}\\\hline
\end{supertabular}
\end{flushleft}

\subsubsection{Other impacts (environmental and socially important impacts)}

\TODO{Neil: Can we not be more ambitious than simply replacing the other alternative? By working in an open way we should be able to achieve much more rapid deployment of ideas (like in the RADIANT project I mentioned above).} 

Here, our mission statement will be the same as for \Sage: to provide
a viable free open source alternative to \Magma, \Maple, \Mathematica
and \Matlab.

To make \TheProject a new reference, it is important to focus on the young
generation, which will be the future users.

Generation Y phenomenon:

\begin{compactenum}
\item generation Y accounts for 30\% of the total projected population
in 2025
\item key influencers to change in workplace habits
\item easy adaptability to technology -{\textgreater} collaborative work
would make them more productive
\item compulsion to check smartphones for emails, texts or social media
updates -{\textgreater} adoption of social networks.
\end{compactenum}
2.1.4. Potential barriers and framework conditions (such as regulation
and standards), that may determine whether and to what extent the
expected impacts will be achieved. (This should not include any risk
factors concerning implementation, as covered in section 3.2.)

The project does not arise any specific regulation issues: if new norms
appear during the project, they will be accorded with relevant
standards agencies on national and EU level.

E.S. - Quant à la partie suivante, je me demande si elle ne serait pas
mieux dans 3.2 :

An open source architecture allows risk sharing: collaborating in the
early stages of research could help an early detection, and, by
consequence, reducing risks.

But:  since Open Source softwares are freely accessible, security and
privacy issues are a concern. Anytime a resource is shared, there is
greater risk of unauthorised access and contaminated data.  Providers
must demonstrate security solutions, which should include physical
security controlling access to the facility and protection of user data
from corruption and cyber attacks.




\subsection{Measures to Maximise Impact}

\subsubsection{Dissemination and Exploitation of Results}
\label{subsubsect:dissemination}

\eucommentary{-- Provide a draft 'plan for the dissemination and exploitation
of the project's results'. The plan, which should be proportionate to the
scale of the project, should contain measures to be implemented both during
and after the project.\\
Dissemination and exploitation measures should address the full range
of potential users and uses including research, commercial, investment,
social, environmental, policy making, setting standards, skills and
educational training.\\
The approach to innovation should be as comprehensive as possible,
and must be tailored to the specific technical, market and organisational
issues to be addressed\\
-- Explain how the proposed measures will help to achieve the expected impact of the
project . Provide a draft business plan for financial sustainability as stated in the Part
E of the Specific features for Research Infrastructures of the Horizon 2020 European
Research Infrastructures (including e-Infrastructures) Work Programme 2014-2015.\\
-- Where relevant, include information on how the participants will
manage the research data generated and/or collected during the
project, in particular addressing the following issues:
What types of data will the project generate/collect? What
standards will be used? How will this data be exploited and/or
shared/made accessible for verification and re-use (If data cannot
be made available, explain why)? How will this data be curated and preserved?\\
-- Include information about any open source software used or developed by the
project.\\
You will need an appropriate consortium agreement to manage (amongst other things)
the ownership and access to key knowledge (IPR, data etc.). Where relevant,
these will allow you, collectively and individually, to pursue market opportunities
arising from the project's results.\\
The appropriate structure of the consortium to support exploitation is addressed
in section 3.3. \\
-- Outline the strategy for knowledge management and protection. Include measures to
provide open access (free on-line access, such as the ``green'' or ``gold'' model) to
peer-reviewed scientific publications which might result from the project.\\
Open access publishing (also called 'gold' open access) means that an article is
immediately provided in open access mode by the scientific publisher. The associated costs
are usually shifted away from readers, and instead (for example) to the university or
research institute to which the researcher is affiliated, or to the funding agency supporting
the research.\\
Self-archiving (also called ``green'' open access) means that the published article or the
final peer-reviewed manuscript is archived by the researcher - or a representative - in an
online repository before, after or alongside its publication. Access to this article is often -
but not necessarily - delayed (``embargo period''), as some scientific publishers may wish to
recoup their investment by selling subscriptions and charging pay-per-download/view fees
during an exclusivity period.}

Three types of impact are possible with our dissemination and
communication activities: (1) people or organizations are informed
about \TheProject; (2) people or organizations act and use our conclusions
or results; (3) people or organizations contribute and help to develop
or improve the research infrastructure. The second form of impact
supposes that parties understand the messages. The third form supposes
learning, which is a very high level of impact. In the following table,
we have listed how the proposed measures will help to achieve the
expected impact among our stakeholders and target audiences.

Table X.  Draft 'plan for the dissemination and exploitation of the
project's results'.

\begin{flushleft}
\tablehead{}
\begin{supertabular}{|m{2cm}|m{8cm}|l|l|l|}
\hline
Target users &
Measures during the project &
\multicolumn{3}{m{3.9129999cm}|}{Expected impact}\\\hline
 &
 &
(1) &
(2) &
(3)\\\hline
Scientific community in mathematics

(experienced researchers and PhD students) &
\begin{compactenum}
\item Recruitment for the project of specialists from industrial sector
and PhD students that are already a part of the community ;\item 10
technical workshops organized in the frame of \TheProject,  \item 10
scientific trainings/year,  \item Participation to FPSac, Isaac
international conferences(4), international congress of mathematical
software (1)\item Workshop ``~Sage and women'' in USA (1?)\item 10
publications in (social aspects) \item Software demonstration during
the conferences{\textgreater}publication\item 100 PhD students trained
and accessing the infrastructure\item Workshops for PhD students in
Africa 
\end{compactenum}
 &
X

X

X

X &
X

X

X

X

X

X

X &
X

X

X

X

X

X\\\hline
Scientific community in other disciplines &
\begin{compactenum}
\item Direct implication of the representatives of those disciplines
into the project;\item Annual participation in Pycon international
conference\item X scientific trainings to other communities; \item News
in Nature and other editions in other disciplines (specify!)\item Up to
X PhD students trained on the tool in biology, physics etc.\item
Workshop ``~sage \& women~'' in USA{\textgreater} pour
IPython
\end{compactenum}
 &
X
 &
X
X
X
X
X
X
 &
X
X
X\\\hline
Policy makers &
Are not directly concerned by the tool, but can be informed via
international conferences and publications &
X &
 &
\\\hline
Industry &
\begin{compactenum}
\item Industrial stakeholders have common needs with academic
researchers. They bring to the project their specific competences and
human resources. They are actively involved into the workshops and
trainings (50\% of audience). The project aims the enlargement of the
community thanks notably to new industrial actors (including in other
disciplines and sectors). They will appropriate the tool by their
direct involvement into the project, by participation to the workshops,
trainings and conferences or by their usual information channels.\item
Annual participation in Pycon international conference\item
Certification by technology clusters
\end{compactenum}
 &
X &
X
X
X
 &
X\\\hline
Standartisation

agencies  &
\begin{compactenum}
\item At the end of the project,~after internal standartisation, the new
norms will be accorded with specialized agencies at national and EU
levels
\end{compactenum}
 &
 &
X &
\\\hline
Students &
\begin{compactenum}
\item 5 MooCs destined to master students.\item The tool will be used
for the elaboration of pedagogical documents, referenced on the
specific website\item The projects results will be integrated into
Master courses, and into teacher training courses
\end{compactenum}
 &
X
X
X
 &
X
X &
\\\hline
Civil society &
\begin{compactenum}
\item Results will be presented on the annual event “Worldwide meetings
of the free software”. This event generally touches upon all free
phenomena in the society, and involved various stakeholders, including
civil society actors.
\end{compactenum}
 &
X &
X &
\\\hline
Public at large &
\begin{compactenum}
\item Series of actions in high schools led by PhD students to raise
awareness of pupils, and especially girls, on mathematics research and
scientific careers\item Communication large public via annual events
like ``~Science holiday~'' etc.\item Vulgarization papers and
communication events addressed to people interested by ICT. \item Social
networks and platforms
\end{compactenum}
 &
X
X
X
X &
 &
\\\hline
\end{supertabular}
\end{flushleft}
Measures after the project:

\begin{compactenum}
\item Continue dissemination to scientific community and industrial
stakeholders through participation to international conferences (
FPSac, Isaac, Pycon, Sage and Women etc.) and publications
\item Software demonstration during the conferences
\item Training of  PhD students in mathematics, informatics and other
disciplines, both in Europe and all over the world
\item Large public communication  through regular vulgarization events
\end{compactenum}
Draft business plan for financial sustainability (as stated in the Part
E of the Specific features for Research Infrastructures of the Horizon
2020 European Research Infrastructures (including e-Infrastructures)
Work Programme 2014-2015).

\paragraph{Long term sustainability}

By design (Objective~\ref{objective:framework}), the VRE's promoted by
\TheProject will consist of a thin layer on top of an ecosystem of
components. Hence, the long term sustainability of those VRE is
guaranteed by the sustainability of the ecosystem of components, that
is by Objective~\ref{objective:sustainable}.

The needs in financial support after the end of the project are
therefore not very
important. We expect that a part of developers positions will be made
durable by the partners institutions, thanks to an increase of
awareness among them on the necessity of this infrastructure for their
own needs.

With the increase of the number of users, more and more research
laboratories, teaching institutions, and enterprises will get a need
for using this VRE – thus, additional funding will be possible through
access provision to other scientific communities, on projects base, or
via service delivery.
% This is already the business model chosen by \SMC.

\TODO{None of the two models below really match our situation;
  investigate for some good picture and language, e.g. in ``Economie
  du Logiciel Libre'' of François Élie}

Here we propose two possible models of this use:

\begin{figure}[ht]\centering
 \fbox{\includegraphics[width=.94\textwidth]{Impact-img1.png}}
 \caption{The GPL Model}\label{fig:gpl-model}
\end{figure}

\begin{figure}[ht]\centering
 \fbox{\includegraphics[width=.94\textwidth]{Impact-img2.png}}
 \caption{The Third-Party Services Model}\label{fig:tps-model}
\end{figure}

\begin{enumerate}
\item The \textbf{GPL model} (see Figure~\ref{fig:gpl-model}): With this model, the vendor
  is required to make the new code available in source form but it can choose to keep the
  new code as proprietary and charge for that proprietary software.  The vendor can
  provide the code commercially as part of a larger platform (hardware/software product)
  for which the companies receives revenue (license fee for the code + fees for technical
  support, updates and upgrades).
\item The \textbf{Third Party Service Model} (see Figure~\ref{fig:tps-model}) Many users
  may be willing to employ a third party service for distribution, modifications
  (debugging) and other support.
\end{enumerate}

\eucommentary{
Where relevant, include information on how the participants will manage
the research data generated and/or collected during the project, in
particular addressing the following issues: What types of data will the
project generate/-collect? What standards will be used? How will this
data be exploited and/or shared/made accessible for verification and
re-use (If data cannot be made available, explain why)? How will this
data be curated and preserved?}

\TODO{E.S. – Nicolas, ici je ne peux pas  écrire à ta place. Il faut juste que
tu répondes précisément aux questions posées ci-dessus.}

Open source software.


All software used and/or generated by the project will be Open Source.
This is a deliberate choice of the project consortium, as commercial
licenses (and patents) on this type of software only creates barriers
in our scientific domain.

Benefits of Open Source:

Acquisition and Costs: lower costs, easy access to the infrastructure,
lower risks of proprietary lock-in

Flexibility: picking up from Open Source projects, reduces dependence on
supplier, ability to view and modify the source code. Allows peer
reviewed modifications, community discussions. Open Source provides the
customer/end user the opportunity to innovate

Support: from developer community.

Besides being cost effective, Open Source software fosters
reuse, reliability, flexibility, and interoperability.

A consortium agreement will be established to manage ownership and
access to key knowledge, including software generated by the project.

Open access policy and data protection.

In line with the Horizon 2020 rules and current trends in IP
management and publication strategies, our consortium is committed to
open access publishing. This means that an article is immediately
provided in open access mode by the scientific publisher. As the
associated costs are usually shifted away from the readers, and
instead to the university or research institute to which the
researcher is affiliated, such costs have been accounted for by all
the research partner institutions. We have agreed to follow the
Guidelines on Open Access to Scientific Publications and Research Data
in Horizon 2020. In fact, following the now well established tradition
of the community, preprints for most publications will be posted on
\Arxiv.

Conforming to the call requirement, we will also participate in Open
research data pilot.  Following the Guidelines on Data Management in
Horizon 2020, we will establish a Data Management Plan, which first
version will be provided as an early deliverable in first six months of
the project. More developed versions of the plan will be provided as
additional deliverables at later stages.

Our strategy for knowledge management and protection includes measures
to provide open access (‘green' or ‘gold') to peer-reviewed scientific
publications and all data that may result from the project.

\subsubsection{Communication activities}
\label{subsubsect:communication}

\eucommentary{Describe the proposed communication measures for promoting the
project and its findings during the period of the grant. Where appropriate
these measures should include social media and public events with user
participation. Measures should be proportionate to the scale of the project,
with clear objectives. They should be tailored to the needs of various audiences,
including groups beyond the project's own community. Where relevant, include
measures for public/societal engagement on issues related to the project.}
%%% Local Variables:
%%% mode: latex
%%% TeX-master: "proposal"
%%% End:

%  LocalWords:  eucommentary programme subsubsection tablehead supertabular hline sur est
%  LocalWords:  e-infrastracture sémantique données amont j'ai mal si répond vraiment ce
%  LocalWords:  critère Systeme flushleft arraybslash Ergonomie il faut réflechir façon
%  LocalWords:  rendre l'outil attractif jeune génération génération des chercheurs va je
%  LocalWords:  définir donc terme Réfléchis possibilité tablettes mais aussi l'enseigner
%  LocalWords:  intéressante textgreater partie suivante demande elle ne serait mieux que
%  LocalWords:  unauthorised Maximise subsubsect organisational Pycon IPython Economie tu
%  LocalWords:  Standartisation Logiciel Libre includegraphics Impact-img1.png peux
%  LocalWords:  Impact-img2.png écrire répondes précisément posées ci-dessus
