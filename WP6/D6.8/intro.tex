The Math-in-the-Middle (MitM) Ontology is developed in the OpenDreamKit project as a
central information resource for the Math-in-the-Middle (MitM) approach to flexible and
knowledge-based integration of mathematical software systems. It serves as a reference and
pivotal point for translations between the various input languages of mathematical
software systems. This integration and interoperability has been described
in~\cite{DehKohKon:iop16,WieKohRab:vtuimkb17,KohMuePfe:kbimss17} and -- in great detail --
in \cite{ODK-D6.5}. In a nutshell, the MitM Ontology describes the mathematical objects,
concepts, and their relations in a general, system-agnostic way in an OMDoc/MMT theory
graph while the mathematical systems export API theories that describe the system
interface language in terms of types, classes, constructors, and functions -- again in
OMDoc/MMT. These two levels of descriptions are linked by OMDoc/MMT
alignments~\cite{MueGauKal:cacfms17} that allow the translation of expressions in the
interface language of system $A$ into the MitM-induced language, and from there to the
interface language of system $B$~\cite{MueRoYuRa:abtafs17}.

The Math-in-the-Middle Ontology is hosted on the MathHub.info system~\cite{IanJucKoh:sdm14,MathHub:on}, the sources can be obtained from \url{http://gl.mathhub.info}.
As the MathHub front-end is currently undergoing major re-write\footnote{The old MathHub interface was based on Drupal, which led to major system vulnerabilities and therefore maintenance hassles, because Drupal was targeted by hackers.
  We are currently working on a docker-based orchestration of services with a React.JS based front-end in the general spirit of the OpenDreamKit VRE toolkit; see \url{https://github.com/MathHubInfo/}.
  The new system can be accessed at \url{http://new.mathhub.info}.} we will reference the MitM ontology by its sources -- the semantically enhanced user interface can be accessed via the same URLs without the \url{gl.} part.

The development consists of four parts:
\begin{compactenum}[\em i\rm)]
\item The \textbf{MitM ontology} which has the formalization of mathematical background theories
  which expresses the knowledge in terms of mathematical concepts found in the
  mathematical literature, without concern -- and thus abstracting from -- for system
  requirements.
\item The \textbf{SMGloM Glossary} which has a human-oriented -- and thus informal --, but
  semantically structured versions of the same content, and can therefore act as a
  human-readable documentation.
\item \textbf{System API theories for the OpenDreamKit Systems and mathematical data bases}, these
  express the system-specific instances of the mathematical concepts, constructors,
  procedures, and types that make up the input/output language of the respective systems.
\item The \textbf{Alignments} between the OpenDreamKit System API theories and the MitM ontology. 
\end{compactenum}
We will discuss all four parts separately below, and conclude with a self assessment of the MitM at this stage.

%%% Local Variables:
%%% mode: visual-line
%%% fill-column: 5000
%%% mode: latex
%%% TeX-master: "report"
%%% End:

%  LocalWords:  DehKohKon:iop16,WieKohRab:vtuimkb17,KohMuePfe:kbimss17 MueGauKal:cacfms17
%  LocalWords:  MitM-induced MueRoYuRa:abtafs17 IanJucKoh:sdm14,MathHub:on compactenum
%  LocalWords:  textbf formalization
