\documentclass[book]{deliverablereport}
\usepackage[style=alphabetic,backend=bibtex]{biblatex}
\addbibresource{../../lib/kbibs/kwarcpubs.bib}
\addbibresource{../../lib/kbibs/extpubs.bib}
\addbibresource{../../lib/kbibs/kwarccrossrefs.bib}
\addbibresource{../../lib/kbibs/extcrossrefs.bib}
\addbibresource{../../lib/deliverables.bib}
\addbibresource{local.bib}
%\usepackage{local}
\usepackage[show]{ed}
\usepackage{tikzinput}
\usetikzlibrary{backgrounds,shadows,shapes,fit,arrows,mmt,backgrounds,positioning}
\usetikzlibrary{decorations.pathmorphing}
\def\defemph{\textbf}
\usepackage{systems}
%\usepackage{arydshln} % interacts negatively with proposal.cls
\usepackage{listings}
\lstset{columns=fullflexible,basicstyle=\sf}
\usepackage{wrapfig}

\title{GAP/SAGE/LMFDB Interface Theories and alignment in OMDoc/MMT for System Interoperability}
\def\shorttitle{GAP/SAGE/LMFDB Interface Theories}
\deliverable{dksbases}{psfoundation}
\deliverydate{1/07/2018}
\duedate{1/09/2017}

\author{John Cremona}
\author{Dennis M\"uller}
\author{Michael Kohlhase}
\author{Markus Pfeiffer}
\author{Florian Rabe}
\author{Nicolas Thierry}
\author{Tom Wiesing}


\begin{document}
\begin{abstract}
  There is a large ecosystem of mathematical software systems.  Individually, these are
  optimized for particular domains and functionalities, and together they cover many needs
  of practical and theoretical mathematics.  However, each system specializes on one
  particular area, and it remains very difficult to solve problems that need to involve
  multiple systems.  Some integrations exist, but the are ad-hoc and have scalability and
  maintainability issues.  In particular, there is not yet an interoperability layer that
  combines the various systems into a virtual research environment (VRE) for mathematics.
  
  The OpenDreamKit project aims at building a toolkit for such VREs.  It suggests using a
  central system-agnostic formalization of mathematics (Math-in-the-Middle, MitM) as the
  needed interoperability layer.  In this paper, we report on a case study that
  instantiates the MitM paradigm the systems \GAP, \Sage, and \Singular to perform
  computation in group and ring theory.
 
  Our work involves massive practical efforts, including a novel formalization of
  computational group theory, improvements to the involved software systems, and a novel
  mediating system that sits at the center of a star-shaped integration layout between
  mathematical software systems.
\end{abstract}

\maketitle
%\newpage\strut\githubissuedescription
\newpage\tableofcontents\newpage
\ednote{Plan for the Document: we will use the MACIS papers for this and include the LMFDB
  worked example~\cite{CreLow:mdcmds18}.}

In this report we present a prototypical integration of the Jupyter notebooks into the MathHub.info portal for active mathematical documents and a versioned hosting system for flexiformal mathematics.
MathHub.info offers a rich interface for reading, writing, and interacting with mathematical documents and knowledge. Jupyter offers a uniform interface to the computation facilities of the OpenDreamKit VRE toolkit in the form of a read-eval-print loop (REPL).

A mathematical Virtual Research Environment (VRE) needs both kinds of interface functionality: mathematical documents have been very successful for presenting mathematical knowledge, and while there have been efforts to make them modular and interactive they predominantly remain in the mode of archiving and transporting knowledge in Mathematics.
Notebook interfaces also use the document metaphor at the surface; however the REPL interaction
tends to take structural precedence, leading to documents consisting of a sequence of computational cells within which the mathematical discourse is interspersed in the form of ``rich comments''.

A ``literate computing'' version of notebooks which gives mathematical discourse structural precedence is possible in principle, but has not been supported consistently at the system level.\ednote{MK: put the following sentence somewhere: A ``literate programming'' version of notebooks which gives mathematical discourse structural precedence is possible in principle, but has not been supported consistently at the system level.}
This tension and trade-off has been explored in OpenDreamKit Deliverable D4.2~\cite{ODK-D4.2}, and the concept of in-document computation in OpenDreamKit Deliverable D4.9~\cite{ODK-D4.9}.
In both cases, the integration was incomplete, since it lacked a full integration of the
underlying knowledge/computation services.

Generally, the integration of MathHub and Jupyter consists of two parts:
\begin{inparaenum}[\em a\rm )]
\item the integration of the user interfaces (as reported previously) and
\item the integration of the knowledge/computation management services.
\end{inparaenum}
Here we report on progress in both; recall that MMT is the knowledge management service behind MathHub (and more generally for the Math-in-the-Middle based system integration; see OpenDreamKit Deliverable D6.5~\cite{ODK-D6.5}).
%
For the service integration we present an MMT kernel for Jupyter.
%
\ednote{specify what Jupyter widgets are; NT: you may want to reuse some of the language of the D4.16 report, around l21 of https://github.com/OpenDreamKit/OpenDreamKit/blob/master/WP4/D4.16/report.tex}
%
Reciprocally, for the user interface integration, we show how the Jupyter widgets can be deeply integrated within the MMT knowledge management facilities to give semantics-aware interaction facilities, extending the front-end capabilities of MathHub/Jupyter Notebooks by semantic widgets driven by the MMT in-document knowledge management services.

We show and evaluate the integration on two case studies: in-document computing facilities in active documents and a knowledge-based specification dialog for modeling and simulation. 

This report is structured as follows. In Section~\ref{sec:mmt-jp} we report on the MathHub/Jupyter integration at the system level: a Jupyter server as part of the MathHub system and a MMT kernel for Jupyter. Section~\ref{sec:nb-mh} presents the integration of Jupyter Notebooks as active documents in the (new) MathHub front-end, and Section~\ref{sec:mitm-nb} presents the two case studies. Section~\ref{sec:concl} concludes the report and discusses future work.

\ednote{this paragraph seems a bit out of place after the description of the structure of the document}
The goal of this report\ednote{of this deliverable?} is to integrate Jupyter notebooks into MathHub
and make them compatible with MMT, in a way that we can conveniently use 
MMT syntax in these notebooks and also a little bit of extra functionality
like e.g. the Jupyter widgets. The first step is setting up a Jupyter server,
which currently runs on \url{http://juypter.mathhub.info}. \ednote{KA: maybe show picture of it?}
For this server, we have developed a custom kernel, that forwards the input 
entered into the Jupyter notebook to the MMT backend. This then processes 
said input and sends the response back to the Jupyter frontend via the kernel.
We will cover the implementation of the Jupyter kernel and the MMT-backend,
later in this report.


\paragraph{Acknowledgements} The authors gratefully acknowledge the support of the Jupyter team and in particular the advice of Benjamin Ragan-Kelly. Also, the input of Theresa Pollinger and her work on the MoSIS system~\cite{PolKohKoe:kacse18} has shaped our perception of the integration reported here. 

%%% Local Variables:
%%% mode: latex
%%% mode: visual-line
%%% fill-column: 5000
%%% TeX-master: "report"
%%% End:


\section{Math-in-the-Middle Interoperability}\label{sec:mitm}
When integrating multiple systems we are mostly talking about using concrete algorithms
(implemented by these systems) to solve specific computational problems (the knowledge
about the problem). To integrate multiple systems with this knowledge we want to enable
users to write down a problem in one system and then solve it in another system. We want
to be independent of the implementation of the knowledge -- independent of the systems.

For this we make use of an approach we call ``Math-In-The-Middle'' paradigm
(see~\cite{DehKohKon:iop16} for details). Here the underlying mathematical knowledge, the
``real math'', is used as a reference ontology for system (in the ``middle'') -- hence the
name. Each system needs access to this knowledge. As each of them come with their own
particularities, they will need some interface to it.

We want to make use of the modular approach to mathematics provided by theory graphs, and
in particular \MMT as an implementation thereof, to first of all allow us translate
mathematical expressions between systems. We define a ``Math In The Middle'' theory as
well as interface theories for each system. With the help of \MMT and bi-views\footnote{A
  bi-view is a bidirectional view between two theories. } between the interface theories and
the central theory, we can translate objects from one system to the other.

\begin{figure}[ht]\centering
  \def\myxscale{3}\def\myyscale{1.2}
  \documentclass{standalone}
\usepackage[mh]{mikoslides}
% this file defines root path local repository
\defpath{MathHub}{/Users/kohlhase/localmh/MathHub}
\mhcurrentrepos{MiKoMH/talks}
\libinput{WApersons}
% we also set the base URI for the LaTeXML transformation
\baseURI[\MathHub{}]{https://mathhub.info/MiKoMH/talks}

\usetikzlibrary{backgrounds,shapes,fit,shadows,mmt}
\begin{document}
\begin{tikzpicture}[xscale=2.6,yscale=.9]
  \tikzstyle{withshadow}=[draw,drop shadow={opacity=.5},fill=white]
   \tikzstyle{database} = [cylinder,cylinder uses custom fill,
      cylinder body fill=yellow!50,cylinder end fill=yellow!50,
      shape border rotate=90,
      aspect=0.25,draw]
   \tikzstyle{human} = [red,dashed,thick]
   \tikzstyle{machine} = [green,dashed,thick]

\node[thy]  (mf) at (.2,5.3) {MathF};
\node[thy,dashed]  (compf) at (0,6) {CompF};
\node[thy,dashed]  (pf) at (-.9,5.5) {PyF};
\node[thy,dashed]  (cf) at (1,5.5) {C\textsuperscript{++}F};
\node[thy,dashed]  (sf) at (-0.9,4.6) {SAGE};
\node[thy,dashed]  (gf) at (1,4.6) {GAP};

\draw[include] (compf) -- (pf);
\draw[includeleft] (compf) -- (cf);
\draw[include] (pf) -- (sf);
\draw[includeleft] (cf) -- (gf);

\node[thy] (kec) at (0,3) {EC};
\node[thy,minimum height=.4cm] (kl) at (0,4) {\ldots};

\node[thy] (sec) at (-1,2) {SEC};
\node[thy,minimum height=.4cm] (sl) at (-1,3) {\ldots};

\node[thy] (gec) at (1,2) {GEC};
\node[thy,minimum height=.4cm] (gl) at (1,3) {\ldots};

\node[thy] (lec) at (-.3,1.2) {LEC};
\node[thy,minimum height=.4cm] (ll) at (.3,1.2) {\ldots};

\node (sc) at (-2,4) {SAGE};
\node[draw] (salg) at (-2,3.35) {Algo};
\node[database,dashed] (sdb) at (-2,2.4) {DB?};
\node[draw] (skr) at (-2,1.7) {KR};
\node[draw,machine] (sac) at (-2,1) {AbsClass};

\node (gc) at (2,4) {GAP};
\node[draw] (galg) at (2,3.35) {Algo};
\node[database,dashed] (gdb) at (2,2.4) {DB?};
\node[draw] (gkr) at (2,1.7) {KR};
\node[draw,machine] (gac) at (2,1) {AbsClass};

\node (lmfdb) at (0,0) {LMFDB};
\node[database] (ldb) at (1,-.4) {Mongo};
\node[draw] (knowls) at (-1,-.4) {Knowls};
\node[draw,machine] (lac) at (0,-.5) {AbsClass};

  \begin{pgfonlayer}{background}
    \node[draw,cloud,fit=(sec) (sl),aspect=.4,inner sep=-3pt,withshadow,purple!30] (st) {};
    \node[draw,cloud,fit=(gec) (gl),aspect=.4,inner sep=-4pt,withshadow,purple!30] (gt) {};
    \node[draw,cloud,fit=(kec) (kl),aspect=.4,inner sep=0pt,withshadow,blue!30] (kt) {};
    \node[draw,cloud,fit=(lec) (ll),aspect=2.5,inner sep=-7pt,withshadow,purple!30] (lt) {};
  \end{pgfonlayer}

\begin{pgfonlayer}{background}
  \node[draw,withshadow,fit=(sc) (skr) (sac) (sdb),inner sep=1pt] {};
  \node[draw,withshadow,fit=(gc) (gkr) (gac) (gdb),inner sep=1pt] {};
  \node[draw,withshadow,fit=(lmfdb) (lac) (ldb) (knowls),inner sep=1pt] {};
\end{pgfonlayer}

\draw[view] (kec) -- (sec);
\draw[view] (kec) -- (gec);
\draw[view] (kec) -- (lec);
\draw[include] (kec) -- (kl);
\draw[include] (gec) -- (gl);
\draw[include] (sec) -- (sl);
\draw[include] (lec) -- (ll);
\draw[view] (kl) -- (sl);
\draw[view] (kl) -- (gl);
\draw[view] (kl) to[bend left=5] (ll);

\draw[meta] (mf)  to [bend right=10] (st);
\draw[meta] (sf) -- (st);
\draw[meta] (mf)  to [bend left=10] (gt);
\draw[meta] (gf) -- (gt);
\draw[meta] (mf) -- (kt);
\draw[meta] (compf) to[bend right=15] (kt);

\draw[human,->] (skr) -- node[above]{\scriptsize induce} (st);
\draw[human,->] (gkr) -- node[above]{\scriptsize induce} (gt);
\draw[human,->] (knowls) -- node[left,near end]{\scriptsize induce} (lt);

\draw[machine,->] (gt) to[bend right=30] node[below,near start]{\scriptsize generate} (gac);
\draw[machine,->] (st) to[bend left=30] node[below,near start]{\scriptsize generate} (sac);
\draw[human,->] (st) to[bend left=20] node[below]{\scriptsize refactor} (kt);
\draw[human,->] (gt) to[bend right=20] node[below]{\scriptsize refactor} (kt);
\draw[human,->] (lt) -- node[right]{\scriptsize refactor} (kt);
\end{tikzpicture}
\end{document}
%%% Local Variables: 
%%% mode: latex
%%% TeX-master: t
%%% End: 

  \caption{The MitM paradigm in detail. PyF, C${}^{++}$F and CompF are (basic)
    foundational theories for \python, C${}^{++}$ and a generic computational model. SEC,
    LEC and GEC are theories for \SageMath, \LMFDB and \GAP elliptic curves.}\label{fig:mitm}
\end{figure}

A sketch of the theory graph based on the example of elliptic curves can be found in
Figure~\ref{sec:mitm}. We will not go into details here but show how this architecture
integrates the \emph{Software} and \emph{Knowledge Aspects}. Clearly, the (hand-curated)
MitM ontology -- the purple cloud in the middle -- is a specification of the underlying
mathematical knowledge as an OMDoc/MMT theory graph, while the system interface theories
-- the blue clouds around it -- formally specify the names and types (i.e. the argument
patterns) and intended behaviour of the interface functions of the systems (often
semi-formally to make the MitM approach scalable). The OMDoc/MMT views -- the wavy arrows
between the theories -- are interpretation morphisms; in this particular case where they
connect the mathematical specification to the system theories, they express the
``implementation relation''. Thus the OMDoc/MMT framework already allows to integrate the
knowledge and software aspects for system interoperability.

The restriction to formalizing the signature (i.e. names and types of the interface
functions) of the systems is sufficient to ensure system interoperability; integrating the
implementations -- e.g. C\textsuperscript{++} or Python code -- into the theories would
be overkill here, since the code can only be executed by the respective systems --
i.e. \GAP or \SageMath. Therefore we will base our foundation on OMDoc/MMT theory graphs
directly rather than on an extension of OMDoc/MMT with ``biform
theories''~\cite{KohManRab:aumftg13,Farmer:btc07} as envisioned in the proposal. Biform
theories would enable (partial) verification of mathematical software systems, but this is
not on the critical path towards a mathematical VRE. The MitM paradigm constitutes a
lightweight alternative; identifying and refining it has been one of the major
achievements of the first year in \WPref{dksbases}.
b

\section{The MitM Ontology for Computational Group Theory}\label{sec:cgt}
\section{The MitM Ontology for Computational Group Theory}\label{sec:cgt}
\begin{todolist}{MK@MP+DM: describe your work here}
\item talk about the levels (abstract, concrete subgroup theory, computational)
\item talk about alignments from the IFT to the CGT, how they work, building
  on~\cite{MueRoYuRa:abtafs17,MueGauKal:cacfms17} 
\end{todolist}

To create a working example, we turned towards one of the topics best
understood by GAP: Computation with (permutation) groups, and formalise it in
MMT.
This a first part of the MitM Ontology.


\subsection{Layers of Abstraction}\ednote{MP@DM this could do with a picture a
  bit like the one in your Alignments paper; if you have the source for it, I
  could adapt it}
The layers: abstract, representation, canonical(?), system dialect

We discovered that formalisation of CGT requires different levels of
abstraction\ednote{MP: This will probably be true for any type of object
  in this game}. At the highest level there is the theory of \emph{Groups}: the
group axioms, generating sets, homomorphisms, group actions, stabilisers,
and orbits. This also easily leads into definitions of
centralisers\footnote{stabilisers of elements under conjugation} and
normalisers\footnote{stabilisers of subgroups under
conjugation}, stabiliser chains,  Sylow-$p$ subgroups, Hall subroups, and many
other concepts. 

MMT also allows expressing that there are different equivalent definitions of a
concept: We defined group actions in two ways and used \emph{views} to show
their equivalence.

\ednote{MP: We need to be able to talk about elements/subsets of groups,
  elements/subsets of $S_n$ that generate groups}

\medskip

Abstract groups can be represented in many ways as more concrete mathematical
objects: as groups of permutations, groups of matrices, finitely presented
groups, or as a polycyclic presentation.\ednote{MP: Not sure this is relevant but the first
three of these are universal: every group has a representation as such an
object, whereas the last is a specialised representation for polycyclic groups}

Additionally, mathematicians often compute with canonical representatives of an
isomorphism class of groups: When a group theorist talks about the ``Dihedral
group of order 8'', they often have a particular canonical representation in
mind, for example as a permutation group that acts on the square by rotations
and reflections, but in GAP this group would be represented as a group of
permutations of (usually) the corners of the square, or a polycyclic
presentation.\ednote{MP: I think this needs better explanation}

These representations also arise naturally from \emph{group actions}: If we are
considering symmetry in a setting where we want to apply group theory, we start
with a group action.\ednote{MP: More concrete? More ``gripping''? I already
talked about the canonical example with the dihedral group}

The universal tool to bridge the gap between groups, representations and
canonical representatives are group homomorphisms.

\medskip

At the lowest level there are implementation details: Permutation groups in GAP
are considered as finite subgroups of the group $S_{\mathbb{N}+}$, and defined by
providing a set of generating permutations. GAP then computes a stabiliser chain
for a group that was defined this way, and naturally considers the group to be a
subgroup of $S_{[1..n]}$, where $n$ is the largest point moved.

\ednote{MP: There might have to be more layers, but these are the main ones I
  can think about right now}

\subsection{Alignments}

Alignments are currently produced by hand: \ednote{MP: potential for
  automated or semi-automated production of alignments from our exports}
For example the filter \texttt{IsGroup} is aligned with \texttt{Group}, and the
filter \texttt{IsPermGroup} is aligned with \texttt{Subgroup SymmetricGroup
  [1..n]}.
\ednote{MP: Need to be more concrete here, in particular we should maybe
  describe how GAP's notion of an action homomorphism translates through this?
  Also is this even correct?}

We formalised the theory of symmetric groups of a set; in GAP permutation groups
are represented as subgroups (with finite support) of the symmetric group of
$\mathbb{N}$, and often one concretely has an isomorphism between the group one
is interested in and a subgroup of $S_{\mathbb{N}}$, for example
via a group action.

\texttt{SylowSubgroup} are more difficult: They are special groups in their
own right, namely groups whose size is a prime-power, but we also want them
to be identified with a certain subgroup of the group we are working
with.\ednote{MP: While I believe this to be an excellent additional example
  for MMT formalisation, this could be going too far for this paper}


\ednote{MP+FR+DM: Mention that this attempt at formalising group theory
  lead to improvements in MMT?}
\ednote{MP@ALL: We might want to be a bit careful/mention implementations of group
  theory for example in COQ where they did the Odd-Order-Proof?}
%%% Local Variables:
%%% mode: latex
%%% TeX-master: "paper"
%%% End:

%  LocalWords:  sec:cgt MueRoYuRa:abtafs17,MueGauKal:cacfms17 emph Sylow subroups medskip
%  LocalWords:  mathbb


\section{The System Dialects of \GAP, \Sage, and \Singular}\label{sec:apit}
We now show how we produce \OMMT theory graphs that specify the system dialects of \GAP, \Singular, and \Sage.
The three systems are sufficiently different that we can consider the development presented in this section a meaningful case study in the methodology and difficulty of exposing the APIs of real-world systems as of formally described system dialects.

In each case, we had to overcome major implementation difficulties and invest significant manpower.
In fact, even the serialization of internal abstract syntax trees as \OMMT objects proved difficult, for different system-specific reasons.
In the following, we summarize these efforts.

\subsection{\Sage}

We first consider our previous work \cite{DehKohKon:iop16} regarding a direct (i.e., without MitM) integration of \Sage and \GAP.
Here \Sage's native interface to \GAP is upgraded from the \defemph{handle paradigm} to the \defemph{semantic handles} paradigm.
In the former, when a system $A$ delegates a calculation to a system $B$, the result $r$ of the calculation is not converted to a native $A$ object (unless it is of some basic type); instead $B$ just returns a handle $h$ (i.e., some kind of reference) to the $B$-object $r$.
Later, $A$ can run further calculations with $r$ by passing it as argument to functions or methods implemented by $B$.
Additionally, with a \defemph{semantic} handle, $h$ behaves in $A$ as if it was a native $A$ object.
In other words, one adapts the API satisfied by $r$ in $B$ to match the API for the same kind of objects in $A$.
For example, the method call \texttt{h.cardinality()} on a \Sage handle \texttt{h} to a \GAP group \texttt{G} triggers in \GAP the corresponding function call \texttt{Size(G)}. 

%The implementation of this paradigm builds on the classical \defemph{adapter pattern}.
%For conciseness, the adapters are generated automatically from \defemph{alignments} between the methods from \Sage's \defemph{categories} (Sage's hierarchy of abstract classes for the usual algebraic structures: sets, groups, algebras, ...) and their \GAP counterparts.
%In a first stage, the alignments are expressed using annotations in the \Sage categories.
%The second stage is to exploit MitM to manage the alignments in order to properly scale from custom one-to-one interfaces to interfaces between multiple systems.

This approach avoids the overhead of back and forth conversions between $A$ and $B$ and enables the manipulation of $B$-objects from $A$ even if they have no native representation in $A$.
However, if these $B$-objects need to be acted on by native operations of $A$ or other systems (as in Jane's scenario), we actually have to convert the objects $r$ between $A$ and $B$.

%This introduces the following additional challenges:
%\begin{compactenum}
%\item Both $A$ and $B$ need to have a native representation for $r$.
%\item $A$ and $B$ need to support the (de)serialization of $r$.
%\item The serialization format need to have a native representation
%  for $r$
%\item Alignments must be specified not only for the abstract methods
%  but also for constructors
%\item These alignments must be used at some point of the conversion.
%\end{compactenum}

%We can now refine our approach from Section~\ref{sec:mitm}. Recall
%that $A$ and $B$ define each a system-near dialect of OpenMath, and
%provide to MitM an API content dictionary describing that dialect. In
%addition, MitM maintains a curated common ontology and a database of
%alignments between the system-near dialects and that ontology. Then,
%for a conversion of $r$ from $B$ to $A$,
%\begin{enumerate}
%\item $B$ serializes $r$ in $B$'s dialect of OpenMath
%\item MitM exploits the alignments to translate this serialization
%  into $A's$ dialect of OpenMath
%\item $A$ deserializes it.
%\end{enumerate}
%
%In this section, we report on the ongoing work in the various systems
%to export the desired API content dictionaries and support
%(de)serialization.

\subsubsection{API}

In \cite{DehKohKon:iop16} we describe the extraction of some of \Sage's API from its \defemph{categories}.
This exploited the mathematical knowledge explicitly embedded in the code to cover a fairly large area  of mathematics (hundreds of kinds of algebraic structures such as groups, algebras, fields, ...), with little additional efforts or need to curate the output.
This extraction did not cover the constructors, knowledge about
which is critical for (de)serialization, nor other areas of
mathematics (graph theory, elliptic curves, ...) where \Sage
developers currently do not use categories (usually because the
involved hierarchies of abstract classes are shallow and easily maintained by hand).

To extract more APIs, we took the following approach:
\begin{compactenum}
\item We constructed a list of typical \Sage objects.
\item We used introspection to analyze those objects, crawling recursively through their hierarchy of classes to extract constructors and available methods together with some mathematical knowledge.
\end{compactenum}

At this stage, the list of objects was crafted by hand to cover Jane's scenarios and some others.
In a later stage, we plan to take advantage of one of \Sage's coding standards: every concrete type must be instantiated at least once in \Sage's tests and the instance passed
trough a generic test suite that runs sanity checks for its advertised
properties (e.g. associativity, ...).
Therefore, by a simple instrumentation of \Sage's test framework, we could run our exporter on a fairly complete collection of \Sage objects.

The process remains brittle and the export will eventually require much curation:
\begin{compactitem}
\item The signature of methods is incomplete: it specifies the number and names of the
  arguments, but only the type of the first argument.
\item For constructors, the type of all the arguments is known, but
  only for the specific call that led to the construction of the
  introspected object.
\item There is no distinction between mathematically relevant methods
  and purely technical ones like data structure manipulation helpers.
\item The export is very large and seems of limited use without
  alignments with the MitM ontology. At this stage we do not foresee
  much opportunities to produce such alignments other than manually.
\end{compactitem}

Nonetheless, we consider this an important first step toward fully automatic extraction of the \Sage API.
Moreover, we expect further improvements by code annotations in \Sage
(e.g., the ongoing porting of \Sage from \Python 2 to \Python
3 will enable \defemph{gradual typing}, which we hope to become widely
adopted by the community) or using type inference in \Sage and/or MitM.

%Here are some potential directions to refine the signatures:
%\begin{itemize}
%\item \Sage is being ported from \Python 2 to \Python 3 (tentative
%  horizon: 2020). The latter enables \defemph{gradual typing} in the form
%  of type annotations in the method parameters and output. Assuming
%  that the \Sage developers community perceives the added value and
%  adopts this programming style, and with some work to setup
%  mathematically relevant types, we can hope for the \Sage library to
%  be progressively annotated with mathematically rich semantic. We
%  will be pushing in this direction.
%\item Some amount of type inference, either at the \Sage or MitM levels.
%\end{itemize}

\subsubsection{Serialization and Deserialization}\label{lst:sagedihedral}

Because \Sage is based on \Python, it benefits from its native serialization support.
For example, the dihedral group $D_4$ is serialized as a binary string, which encodes the following straight line program to be executed upon deserialization:
\begin{lstlisting}
  pg_unreduce = unpickle_global('sage.structure.unique_representation', 'unreduce')
  pg_DihedralGroup = 
       unpickle_global('sage.groups.perm_gps.permgroup_named', 'DihedralGroup')
  pg_make_integer = unpickle_global('sage.rings.integer', 'make_integer')
  pg_unreduce(pg_DihedralGroup, (pg_make_integer('4'),), {})
\end{lstlisting}
The first three lines recover the constructors for integers and for dihedral groups from \Sage's library.
The last line applies them to construct successively the integer $4$ and $D_4$.

Up to concrete syntax, this serialization is already close to the desired \Sage system dialect.
We can therefore extend \Python's native (de)serializer to use \OMMT as an alternative serialization format (using the \Python library~\cite{py-openmath:on}).
The following shows the corresponding OpenMath syntax tree in Python and XML respectively:
\begin{lstlisting}
OMApplication(
  elem=OMSymbol(name='DihedralGroup',
                cd='sage.groups.perm_gps.permgroup_named', cdbase='http://python.org'),
  arguments=[OMApplication(
    elem=OMSymbol(name='make_integer',
                  cd='sage.rings.integer', cdbase='http://python.org'),
    arguments=[OMBytes(bytes='4')])])
\end{lstlisting}
\begin{lstlisting}
<OMA xmlns="http://www.openmath.org/OpenMath">
  <OMS name="DihedralGroup"
       cd="sage.groups.perm_gps.permgroup_named" cdbase="http://python.org"/>
  <OMA>
    <OMS name="make_integer" cd="sage.rings.integer" cdbase="http://python.org"/>
    <OMB>NA==</OMB>
  </OMA>
</OMA>
\end{lstlisting}
This approach has the additional advantage of benefiting from future optimizations implemented in \Python's serialization, like structure sharing for identical subexpressions.

% For example, a list of integers modulo 2 will be serialized as a
  % program like:

  %       Z2 = IntegerModRing(2)
  %       [Z2(0), Z2(1), Z2(0), Z2(0)]

  % Instead of recreating Z2 five times. In fact, it's even smarter in
  % this case, building Z2(0) and Z2(1) only once.

Still, systematically expanding \OMMT serialization to the \emph{entire} \Sage library requires significant manpower and can only be a long-term goal.
To increase community support, our design elegantly decouples the problem into (i) instrumenting the serialization to generate \OMMT as an alternative target format, and (ii) structural improvements of the serialization that benefit \Sage in general.

In particular, our serialization of \Sage objects is \defemph{by construction} rather than \defemph{by representation}, i.e., we serialize the constructor call that was used to build an object instead of the low-level \Python representation of the resulting object.
This is important to hide implementation details and allow for straightforward alignments.
From the origin, the \Sage community has internally promoted
good support for serialization as this is a fundamental building
block for communication between parallel processes, databases, etc.
Thus, it already values serialization by construction as
superior because it is usually more concise and more robust under
changes to \Sage. Therefore, independent of the purposes of this
\papertype, we expect a synergy with the \Sage community toward improving
serialization.

  % There are generic tests; it is an official part of the review
  % process, etc. All in all \Sage is doing relatively well (most objects
  % can be serialized and deserialized safely, often even so in a later
  % \Sage version).

% - There is a pure Python implementation of the serializer. With some
%   luck (we still have to dig a bit more into the code), there is not
%   much to do to derive a subclass of the serializer that would output
%   OpenMath expressions instead of binary strings. Variant: derive from
%   the deserializer to obtain a converter from binary strings to
%   OpenMath.

% As in \GAP, a large part of the mathematical knowledge embedded in the
% \Sage library is encoded using its type system. This library is
% written in the \Python programming language which comes with a
% traditional object oriented dynamic type system.
% For example The MiTM ontology of Figure~\ref{fig:cgtontology}
% translates into a hierarchy of four abstract classes (\texttt{Group},
% \texttt{PermutationGroup}, \texttt{MatrixGroup},
% \texttt{FinitelyPresentedGroup}) and concrete classes
% (\texttt{SymmetricGroup}, \texttt{MathieuGroup},
% \texttt{LinearMatrixGroup}, ...).

% Altogether, the hierarchy of classes of \Sage contains thousands of
% abstract and concrete classes, with heavy use of multiple inheritance.
% To tame code bloat and make such a deep and large hierarchy
% maintainable, \Python's type system is enriched with a category system
% that collects closely related abstract classes (e.g. \texttt{Group},
% \texttt{GroupElement}, \texttt{GroupMorphism}, \texttt{GroupHomset}),
% together with explicitly represented mathematical knowledge, in a
% so-called \defemph{category} (e.g that of \texttt{Groups}).
% See~\ref{Sage,Sage.Categories} for details.

% In \cite{DehKohKon:iop16} we describe the use of annotations in the code to enrich the
% mathematical knowledge in \Sage's categories with alignments with other systems, notably
% \MMT. This knowledge is then exported to generate interfaces theories. We also describe how
% this can be used to automatically generate \defemph{handle interfaces} with other systems
% like e.g. \GAP.
% \begin{todolist}{NT@NT}
% \item next step: also export constructors to enable non-handle interfaces where objects
%   are actually exchanged. Besides, by nature certain areas of \Sage (e.g. graph theory,
%   elliptic curves, ...) have shallow hierarchy of classes; there categories become
%   irrelevant and are not used. \ednote{statistics would be useful here}
% \item using introspection to export the information; instrument TestSuite to export all
%   objects; parents and unique representation objects have a constructor method. pickling
%   by construction, ...
% \end{todolist}

\subsection{\GAP}

In \cite{DehKohKon:iop16}, we already described our general approach to extract APIs from the \GAP system.
We have now improved on this work considerably.

Firstly, we improved the MitM foundation so that the primitives of \GAP's type system can be expressed in the MitM ontology.\footnote{In the future \MMT might even serve as an   external type-checker for \GAP.}  \GAP's type system heavily uses subtyping: \defemph{filters} express finer and finer subtypes of the universal type \lstinline|IsObject|.  
Moreover, an object in \GAP can learn about its properties, meaning its type is refined at runtime: a group can learn that it is Abelian or nilpotent and change its type accordingly.

Secondly, we devised and implemented a special treatment of \GAP's constructors during serialization.
As \GAP only has a weak notion of object construction, we achieved this by manually identifying and annotating all functions that create objects in the \GAP code base and then instrumenting them to store which arguments they were called with.
With the constructor annotation in place, it is possible to have \GAP represent any object in a running session as either a primitive type (integers, permutations, transformations, lists, floats, strings), or as a constructor applied to a list of arguments.

The instrumentation itself is minimal -- 57 lines of \GAP code, plus 100 lines for serializing and parsing.
The main -- and indeed considerable -- challenge was to identify the constructors and their arguments.
In \GAP, objects are created by calling the function \lstinline|Objectify| with a type and some arguments.
Hence we analyzed
all call-sites to this function and some light inference of the enclosing function.
This amounted to 665 call sites in the \GAP library and an additional 1664 in the standard package distribution.
The instrumentation will be released as part of a future version of \GAP, making \GAP fully MitM capable.
\ednote{MP: Put an example of OM\_Print here, maybe for a group, or for Cosets
  (as they are something that the standard OpenMath CDs in \GAP cannot do)}

As a major positive side-effect of our work, this instrumentation led to general improvements of the type infrastructure in \GAP.
For example, it enables static type analysis, which can be used to optimize the dynamic method dispatch and thus hopefully lead to efficiency gains in the system.

\subsection{\Singular}

As we only need a very small part of \Singular for our case study, we were able to use the existing OpenMath content dictionaries for polynomials~\cite{OMCD:poly:on} as the \Singular system dialect.
These are part of a standard group of content dictionaries that describe (some) mathematical objects at a high level of abstraction to be universally applicable.
\OMMT understands OpenMath, i.e., it can use these content dictionaries as \OMMT theories.

Building on the OpenMath toolkits for OpenMath phrasebooks~\cite{py-openmath:on} and SCSCP communication~\cite{py-scscp:on} in {\Python} -- which were developed for \Sage in the
OpenDreamKit project, we wrapped \Singular in a thin layer of \Python code that provides \SCSCP communication.
This work was undertaken by the sixth author as part of a summer internship in about a week without prior expert knowledge of the system. Of course, if we want to achieve a more comprehensive coverage of the \Singular dialect, we will have to either manually write a theory graph or instrument \Singular for extraction as we did for \Sage or \GAP above. 

\subsection{Alignments}\label{sec:integrating:alignments}

Finally we have to curate the \textbf{alignments} between the system dialects and the MitM ontology. 

To make the systems interoperable and translate objects and expressions, it is crucial to inform the system which symbols in the 
respective system API theories represent the same mathematical concepts. Then translation (as a first approximation) reduces to simply substituting symbols in an expression (see example below).\medskip

However, even when $A$ and $B$ deal with the ``same mathematical objects'', these may be constructed and represented differently, e.g., symbols can differ in name,
argument order/number, types, etc.
A major difficulty for system interoperability is correctly handling these subtle differences.
To formalize the details of this relation, \cite{MueGauKal:cacfms17} introduced \OMMT \textbf{alignments}.
Technically, these are pairs of \OMMT symbol identifiers decorated by a set of key-value pairs.
The alignments of $a$-symbols with the MitM ontology determine which $A$-objects correspond to MitM-objects.

The alignment of $a$-symbols to ontology symbols must be spelled out manually.
But this is usually straightforward and easy even for inexperienced users. For example, the following line aligns GAP's symbol \textsf{IsCyclic} (in the file \lstinline|lib/grp.gd|) with the corresponding symbol \textsf{cyclic} in the MitM ontology.
The key-value pairs are used to signify that this alignment is part of a group of alignments called ``VRE'' and can be used for translations in both directions.

\begin{verbatim}
gap:/lib?grp?IsCyclic  mitm:/smglom/algebra?group?cylic
    direction="both" type="VRE"
\end{verbatim}

Thus we can reduce the problem of interfacing $n$ systems to
\begin{inparaenum}[\em i\rm)]
\item curating the MitM ontology for the joint mathematical domain,
\item generating $n$ theory graphs for the system dialects,
\item maintaining $n$ collections of alignments with the MitM ontology.
\end{inparaenum}\medskip

For an example, consider again the dihedral group $D_4$ in \Sage (see Section \ref{lst:sagedihedral}). We can align the relevant symbol 
\verb+http://python.org?sage.groups.perm_gps.permgroup_named?DihedralGroup+
with an abstract representation of dihedral groups in the MitM ontology (say, \verb+mitm:?Groups?dihedralGroup+). The \MMT system, when translating from \Sage to e.g. \GAP, then knows to replace the sage-specific symbol \verb+?DihedralGroup+ by its system neutral equivalent in MitM, and the complex expression \verb+make_integer(4)+ by the plain OpenMath integer 4. To translate the resulting statement in MitM to \GAP, we only need to align \verb+mitm:?Groups?dihedralGroup+ with the constructor for dihedral groups in \GAP, namely \verb+gap:/grp?basic?DihedralGroupCons+. Translating to \GAP is then merely a matter of substituting this symbol in the OpenMath expression (\GAP already uses OpenMath integers for remote procedure calls via \SCSCP) to reconstruct the original \Sage object in \GAP.
\medskip

Alignments form an independent part of the MitM interoperability infrastructure.
Incidentally, they obey a separate development schedule: the MitM ontology is developed by the community as a whole as the understanding of a mathematical domain changes.
The system dialects are released together with the systems according to their respective development cycle.
The alignments bridge between them and have to mediate these cycles.

These alignments are currently produced and curated using the approach, repository, and syntax described in~\cite{MueGauKal:cacfms17,MueRoYuRa:abtafs17}.
In the future, we will also consider automatically extracting alignments from the existing ad-hoc \Sage-to-$X$ translations.
These are (mainly) given as \Sage code annotations that relate \Sage operations and constructors with those of system $X$.

Figure~\ref{fig:cgtontology} shows some example alignments between symbols in the GAP content dictionary and the MitM ontology.




%%, but from some of the initial alignments and the \GAP API theories we will be able to infer more alignments automatically.
%%For example, the filter \texttt{GAP:IsGroup} is aligned with
%%\texttt{mitm:Group}, and the filter \texttt{GAP:IsPermGroup} is aligned with
%%\texttt{mitm:Subgroup SymmetricGroup [1..n]}.
%
%We formalized the theory of symmetric groups of a set in the MitM ontology.
%In \GAP, permutation groups are represented as subgroups (with finite support) of the symmetric group of $\mathbb{N}+$, and often one concretely has an isomorphism between the group one is interested in and a subgroup of $S_{\mathbb{N}+}$, for example via a group action.

%\texttt{SylowSubgroup}s are more difficult: They are special groups in their own right, namely groups whose size is a prime-power, but we also want them to be identified with a certain subgroup of the group we are working with.
%\ednote{MP: While I believe this to be an excellent additional example for \OMMT formalisation, this could be going too far for this paper}

%\ednote{MK@MK: still to write: the alignment-based priorization and suggestion mechanism. }


%%% Local Variables:
%%% mode: latex
%%% TeX-master: "report"
%%% End:

%  LocalWords:  DehKohKon:iop16 emph serialization sec:mitm deserializes subsubsection MueGauKal:cacfms17 MueRoYuRa:abtafs17
%  LocalWords:  analyze itemize Deserialization serialized lstlisting pg_unreduce mathbb
%  LocalWords:  unpickle_global unreduce pg_DihedralGroup serializer py-openmath:on
%  LocalWords:  lstinline optimizations factorization serialize deserialized deserializer
%  LocalWords:  fig:cgtontology GroupHomset serializing analyzed py-scscp:on mitm:Group
%  LocalWords:  mitm:Subgroup SylowSubgroup newpart priorization summarize compactitem
%  LocalWords:  formalized


\section{Distributed Computational Group Theory}\label{sec:case}
\begin{figure}[ht]\centering
  \tikzinput{gap_singular_mitm_fig}
  \caption{MitM Interaction in Jane's Use Case}\label{fig:mitmpoc}
\end{figure}

Figure~\ref{fig:mitmpoc} shows the overall architecture with an MitM server as the central mediator.
All arrows represent the transfer of \OMMT ojbects via SCSCP.
Critically, the MitM server also implements alignments and uses them to convert between system dialects.

We have extended the \MMT system~\cite{Rabe:MAGMS13} with an SCSCP server/client so that it can receive objects from computation systems and generates calls to others.
For the \GAP server, we built on pre-existing \SCSCP support.
To obtain an \SCSCP server for \Singular, which does not have native \SCSCP support, we wrapped \Singular in a python script that includes the \lstinline|pyscscp| library~\cite{py-scscp:on}.
In \Sage, we directly programmed the client interface to the MitM server.

The numbers on the edges indicate the order of communications when processing Jane's use case.
Initially, Jane has already built in \Sage the ring $R=\mathbb{Z}[X_1,X_2,X_3,X_4]$, the group $G=D_4$, and the action $A$ of $G$ on $R$ that permutes the variables, and the polynomial $p = 3\cdot X_1 + 2\cdot X_2$.
She now calls \lstinline|MitM.Singular(MitM.Gap.orbit(G, A, p)).Ideal().Groebner().sage()|, which results in the following steps:
\begin{compactenum}
  \item Jane uses \Sage to call the MitM server with the respective \Sage object, along with metadata about which system should be used for which computation.
  \item The MitM server translates \lstinline|MitM.Gap.orbit(G, A, p)| to the \GAP system dialect and sends it to \GAP.
  \item \GAP returns the orbit $O$.
  \item The MitM server translates \lstinline|MitM.Singular(O).Ideal().Groebner()| to the \Singular system dialect and sends it to \Singular.
  \item \Singular returns the Gröbner base $B$.
  \item The MitM server translates \lstinline|B| to the \Sage system dialect and sends it to \Sage, where the result is shown to Jane.
\end{compactenum}

\paragraph{Another use-case}

Suppose Jon prefers working in \GAP, and she wants to compute the
Galois group of the rational polynomial $p = x^5 - 2$.

Jon discovers the \GAP package \texttt{radiroot}, which promises this
functionality, but the package does not work for this polynomial.
\ednote{MP: The radiroot part can go away, the jist is: This cannot be done
  with \GAP currently}

Jon hears from his colleague Jane that he should just use \Sage, because
computing Galois groups is a breeze.

% $p =x^4-x^3-x^2+x+1$ over $\mathbb{Q}$ would have D_8 as galois group again...

Jon calls \lstinline|G := MitM("Sage", "GaloisGroup", p)| in \GAP which yields
the desired Galois group as a \GAP permutation group.

Jon, being a proficient \GAP user, also knows that he can now install a \emph{method}
in \GAP that will compute the Galois group of any rational polynomial
transparently for him whenever he calls \lstinline|GaloisGroup| for a rational
polynomial in \GAP. \ednote{MP: And he submits a pull-request to \GAP to make
  that happen}



\ednote{FR@all: Does my description match what is happening? We have to discuss this and probably adapt the implementation accordingly.}

% \begin{oldpart}{MK: just copied here; Victor writes\\
%     ``\emph{A peer-to-peer connection must be made with the CAS servers, so that CAS
%       servers can, in turn, query MitM if during a computation they encounter a concept
%       that lies outside their field of knowledge. In application to this particular case,
%       it would be cleaner if, instead of asking MitM to produce permutations of a list,
%       the client simply queries MitM for the orbit of a polynomial by defining an action
%       of a member of the symmetric group on a polynomial. \GAP would then be able to
%       calculate the orbit by making the group act on the polynomial with the described
%       action and querying MitM for equality of polynomials, resulting in a linear-time
%       algorithm instead of quadratic-time behaviour displayed by the current client.}''  I
%     do not quite understand the maths here, maybe we can stillmake this happen?}
%   The control script follows the procedure:\ednote{MK: for this to make sense we would
%     have to describe what problem we want to solve.}
% \begin{enumerate}
%   \item Create an OpenMath polynomial.
%   \item Obtain a symmetric group of size that is equal to the number of variables 
%     in the polynomial from MitM.
%   \item Using the obtained group, query MitM for all permutations of the list 
%     of variables.
%   \item Create polynomials from the permutations of the list of variables.
%   \item Filter out the duplicate polynomials by querying MitM for equality of 
%     polynomials.
% \end{enumerate}
% While this is very much a brute-force algorithm to calculate an orbit of a
% polynomial, it showcases the ability of the client to query the MitM server that 
% is then forced to use multiple CAS without the client needing any knowledge of the
% underlying systems.
% \end{oldpart}

%The \Sage client behaves exactly as described in
%Section~\ref{sec:mitm:comms}\ednote{MK@NT/TW; it seems that we will have time to
%  implement this after the extension. So we should make it happen.}

%%% Local Variables:
%%% mode: latex
%%% TeX-master: "paper"
%%% End:

%  LocalWords:  sec:case fig:mitmpoc IanJucKoh:sdm14,MathHub:on summarize sec:cgt pyscscp
%  LocalWords:  twiesing:msc17 centering tikzinput gap_singular_mitm_fig lstinline emph
%  LocalWords:  py-scscp:on oldpart


\section{Conclusion}\label{sec:concl}
  We have implemented the MitM approach to integrating mathematical software systems based on formalizations of the underlying mathematical knowledge.
  The main investment here was the curation of an MitM Ontology, the generation of formal specifications of system APIs for \Sage, \GAP, and \Singular, identifying the alignments of these APIs with the ontology, implementing an MitM server that can use alignments to translate between systems, and implementing the \SCSCP protocol for all involved systems.

  We have also shown how to extend the Math-in-the-Middle framework for integrating systems to mathematical data bases like the \lmfdb. 
The main idea is to embed knowledge sources as virtual theories, i.e. theories that are not -- theoretically or in practice -- limited in the number of declarations and allow dynamic loading and processing. 
For accessing real-world knowledge sources, we have developed the notion of codecs and integrated them into the MitM ontology framework. 
These codecs (and their MitM types) lift knowledge source access to the MitM level and thus enable object-level interoperability and allow humans (mathematicians) access using the concepts they are familiar with. 
Finally, we have shown a prototypical query translation facility that allows to delegate some of the processing to the underlying knowledge source and thus avoid thrashing of virtual theories. 

\paragraph{Related Work} Most other integration schemes employ a \textbf{homogenous approach}, where there is a master system and all data is converted into that system. 
A paradigmatic example of this is the Wolfram Language~\cite{WolframLanguage:wikipedia} and the Wolfram Alpha search engine~\cite{WolframAlpha:on}, which are based on the Mathematica kernel. 
This is very flexible for anyone owning a Mathematica license and experienced in the Mathematica language and environment.

The MitM-based approach to interoperability of data sources and systems proposed in this paper is inherently a \textbf{heterogeneous approach}: systems and data sources are kept ``as is'', but their APIs are documented in a machine-actionable way that can be utilized for remote procedure calls, content format mediation, and service discovery. 
As a consequence, interaction between systems is very flexible.
For the data source integration via virtual theories presented in this paper this is important. 
For instance, we can just make an extension of \mmt or \Sage\ which just act as a programmatic interface for e.g. \lmfdb. 

Our case studies show that MitM-based integration is an achievable goal.
Delegation-based workflows can either be programmed directly or embedded into the interaction language of the mathematical software systems.

The main advantages and challenges claimed by the MitM framework come from its loosely coupled and knowledge-based nature.
Compared to ad-hoc translations, MitM-based interoperability is relatively expensive as objects have to be serialized into (possibly large) \OMMT objects, transferred via \SCSCP to \MMT, parsed, translated into another system dialect, serialized and transferred, and parsed again.
On the other hand, instead of implementing and maintaining $n^2$ translations, we only have to establish and maintain $n$ collections of system APIs and their alignments to the
MitM ontology.
This makes the management of interoperability much more tractable:
\begin{compactenum}
\item The MitM ontology is developed and maintained as a shared resource by the community.
We expect it to be well-maintained, since it can directly be used as a documentation of the functionality of the respective systems.
\item All the workflows are star-shaped: instead of requiring expert knowledge in two systems -- a rare commodity even in open-source projects, and even for the system experts involved in this \papertype -- and keeping up with their changes, the MitM approach only needs expertise and change management for single systems.
\end{compactenum}
All in all, these translate into a ``business model'' for MitM-based cooperation in terms of the necessary investment and achievable results, which is based on the well-known \emph{network effects}: the joining costs are in the size of the respective system, whereas the rewards -- i.e. the functionality available by delegation -- is in the size of the network.

This network effect can be enhanced by technical refinements we are currently studying:
For instance, if we annotate alignments with a ``priority'' value that specifies how canonically/efficiently/powerfully a given system implements a given MitM operation, then we can let
the \MMT mediator automatically choose a suitable target system for a requested computation (as opposed to our current setup where Jane specifies which systems she wants to use). On the other hand, for workflows where we do not need or want service-discovery, alignments can be ``compiled'' into $n^2$ transport-efficient direct translations that may even eliminate the need for serialization and parsing.

\paragraph{Future Work}\ednote{MK: this is essentially future work only for LMFDB, we need future work also for MitM here.}
\ednote{MK@FR: could you please describe the MMT Python bridge and how this could be used for SCSCP-less communication with Sage. The MitM-work here is to allow for compilation of the alignment based translations into code.}
We have discussed the MitM+virtual theories methodology on the elliptic curves sub-base of the \lmfdb, which we have fully integrated. 
We are currently working on additional \lmfdb sub-bases. 
The main problem to be solved is to elicit the information for the respective schema theories from the \lmfdb community. 
Once that is accomplished, specifying them in the format discussed in this paper and writing the respective codecs is straightforward. 

Moreover, we are working on integrating the the Online Encyclopedia of Integer Sequences (OEIS~\cite{Sloane:OEIS,oeis}). 
Here we have a different problem: the OEIS database is essentially a flat ASCII file with different slots (for initial segments of the sequences, references, comments, and formulae); all minimally marked up ASCII art. 
In~\cite{LuzKoh:fsarfo16} we have already (heuristically) flexiformalized OEIS contents in \ommt; the next step will be to come up with codecs based on this basis and develop schema theories for OEIS.


\subsubsection*{Acknowledgements}
The authors gratefully acknowledge the fruitful discussions with other participants of
work package WP6, in particular Alexander Konovalov on \SCSCP, Paul Dehaye on the \Sage
export and the organization of the MitM ontology, Luca de Feo on OpenMath phrasebooks
and the \SCSCP library in python, and David Lowry-Duda\ednote{MK: or make him a co-author?  JC: We should give him the option.}

We acknowledge financial support from the OpenDreamKit Horizon 2020 European Research
Infrastructures project (\#676541) and DFG project RA-18723-1 OAF.

%%% Local Variables:
%%% mode: latex
%%% mode: visual-line
%%% fill-column: 5000
%%% TeX-master: "report"
%%% End:

%  LocalWords:  sec:concl MitM-based itemize subsubsection Dehaye organization serialized
%  LocalWords:  math-savy emph serialization formalizations

\end{document}

%%% Local Variables:
%%% mode: latex
%%% TeX-master: t
%%% End:

%  LocalWords:  maketitle newpage tableofcontents newpage newcommand xspace ednote mathdb
%  LocalWords:  standardize dktheories concl printbibliography pn textit mmt mitm emph
%  LocalWords:  WPref dksbases prioritized taskref organized delivref dkstheories
%  LocalWords:  githubissuedescription
