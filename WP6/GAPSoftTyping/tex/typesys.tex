\subsection{Overview}

GAP uses a single flat namespace where every declared entity is identified by its \textbf{name}.

Three kinds of named \textbf{declarations} exists: categories, operations, and methods.
Additionally, constructors, attributes, and properties are distinguished special cases of operations.

Operations and categories introduce new objects and thus must have fresh names.
Methods, on the other hand, introduce unnamed implementations of a previously declared named operation; thus, the name of a method is the same as that of an operation.
(At the meta-level, a specific method may be referenced by combining the operation name with the method's documentation string.)

Three kinds of anonymous \textbf{complex} entities exist: objects, families, and filters.
GAP objects represent mathematical objects and are the primary interest.
Families and filters provides a type system on objects: a type consists of a family (the base type) and a filter $F$, which provides a unary predicate on objects of family $F$.
The family of an object $O$ is a hard type: it is unique, computable, and fixed.
The filter is a soft type: $O$ can satisfy any number of filters, filters may be undecidable, and the type of $O$ can be refined at run-time as more filters become known that $O$ satisfies.

\subsection{Complex Entities}

In deduction systems, it is possible, even typical to build all or most expressions ex nihilo, typically via inductive types or axiomatic specifications.
But such representations are usually efficient and therefore problematic in computation systems.
Therefore, GAP allows arbitrary primitive objects backed by concrete representations in the underlying run-time environment.
These are provided by the \textbf{families}: each family introduces a set of primitive objects.

Users can implement new families.
But the following families with their respective primitive objects are built into GAP:
\begin{compactitem}
  \item one each for a few types of built-in literals:
    \begin{compactitem}
      \item cyclotomic numbers (elements of the algebraic closure of the rationals),
      \item booleans,
      \item strings,
    \end{compactitem}
  \item one each for several built-in operators that form complex objects
    \begin{compactitem}
      \item homogeneous lists (called \textbf{collections}): lists of objects that have the same family,
      \item heterogeneous lists: lists of arbitrary objects,
      \item functions on objects.
    \end{compactitem}
\end{compactitem}

The objects are the primitive objects introduced by the families and any application of an operation to objects.

A \textbf{filter} is one of the following:
\begin{compactitem}
  \item the universal filter $\isobj$,
  \item a category $C$,
  \item a property $P$,
  \item a conjunction $F\wedge G$ of filters.
\end{compactitem}
We call categories and properties \textbf{atomic filters}.
By convention, their names are of the form \lstinline|IsXXX|.

Filters can be normalized into a set of atomic filters (with $\isobj$ corresponding to the empty set and $\wedge$ to union).
Therefore, types are essentially pairs of a family and a set of atomic filters, and a type can be efficiently stored as a bitvector indexed by the known atomic filters.
GAP stores this bitvector together with every object.

Because the family is an inherent property of an objects anyway, the \textbf{typing relation} reduces to a relation $\has{O}{F}$ between objects $O$ and filters $F$.
It is defined as follows:
\begin{compactitem}
  \item $\has{O}{\isobj}$ always holds
  \item $\has{O}{F\wedge G}$ holds if $\has{O}{F}$ and $\has{O}{G}$.
  \item $\has{O}{C}$ holds if $O$ was returned by a constructor of category $C$,
  \item $\has{O}{P}$ if evaluating property $P$ on $O$ returns \lstinline|true|,
  \item In addition to the above rules, $\has{O}{F}$ holds for an atomic filter $F$ if the corresponding bit was set when $O$ was constructed.
  This is used in particular by the constructors of categories (see below).
\end{compactitem}

Types are hyper-dynamic: whenever a property is evaluated for an object at run-time, its bit in the cached bitvector type is updated.
Thus, the type changes dynamically as more properties are evaluated.

\subsection{Declarations}

\paragraph{Categories}
A category declaration consists of
\begin{compactitem}
  \item a name,
  \item a filter (called the superfilter).
\end{compactitem}
The concrete syntax is \lstinline|DeclareCategory(name: String, superfilter: Filter)|.

A \textbf{category} declaration introduces a primitive filter.
All categories are created empty.
The objects satisfying this filter are introduced by declaring constructors.
These are operations whose implementation explicitly marks the returned objects as having the category as a filter.

When a constructor of category $C$ is run, the returned object automatically has all filter bits set that correspond to the atomic filters in the superfilter of $C$.

\paragraph{Operations}
An \textbf{operation} declaration introduces an $n$-ary\footnote{GAP has an implementation restriction of $n\leq 6$.} function on objects.
Operations are softly typed: each $n$-ary operations provides a list of length $n$ providing the input filters of the respective argument.
Operations may also carry an optional return type, which defaults to $\isobj$ if omitted.\footnote{This is a recent feature motivated by the discussions that also led to this paper.}

The concrete syntax is

 \lstinline|DeclareOperation(name: String, inputfilters: Filter*, outputfilter: Filter?)|.

An \textbf{attribute} declaration is the special case of an operation that is unary.
The special treatment of attributes is important only for efficiency reasons: The values of attributes are cached with each object.
The concrete syntax is \lstinline|DeclareAttribute(name: String, inputfilter: Filter, outputfilter: Filter?)|.

A \textbf{property} declaration is the special case of an attribute that returns a boolean.
The special treatment of properties is important only because properties can be used as filters.
The concrete syntax is \lstinline|DeclareProperty(name: String, inputfilter: Filter)|.

A \textbf{constructor} declaration is the special case of an operation that returns an object of a given category.
The concrete syntax is \lstinline|DeclareConstructor(name: String, inputfilters: Filter*, outputfilter: Filter?)|.%
\footnote{The return argument is a recent feature. More generally, the current implementation of constructors is somewhat awkward and may be subject to change. Currently, a constructor's first argument is special: It must be the expected return filter (rather than an object).
This is used to allow method selection to choose a different method for different special cases.
A more elegant solution would be to allow every operation to declare that some of its arguments must be filters.
This would yield an untyped version of bounded polymorphism with filter arguments corresponding to type arguments.}

Conceptually, all operations are defined.
But the definiens is never part of the declaration and instead provided separately in method declarations.

\paragraph{Methods}
Every operation can have multiple definitions, which are provided by methods.
A method declaration consists of
\begin{compactitem}
  \item the name of the operation,
  \item the input and output filters,
  \item the actual definition, as a function in the underlying programming language.
\end{compactitem}

The concrete syntax of a method declaration is
\lstinline|InstallMethod(operationname: String, inputfilters: Filter*, outputfilter: Filter?, definition: function)|.

The input and return filters of a method may be more restrictive than the filters used in the operation declaration.
More restrictive input filters can be used to represent overloading of operations or run-time polymorphism.
A more restrictive output filter can be used to indicate a sharper type than required by the operations.

When evaluating the application of an operation to arguments, GAP selects a specific method executes its definition.
If more than one method exists, whose input filters type the operation arguments, an internal ranking is used to disambiguate.\footnote{In particular, if a property of $O$ is evaluated in between two calls of the same operation on $O$, a different method may be selected the second time. This is often desirable, particularly when the second method is more efficient.}

%\subsection{Theory Level}
%
%There is no explicit theory level.
%Instead, theories are represented as categories, and theory morphisms as operations, and their relation is a special case of typing.
%
%Therefore, we can treat each source file as a theory.
%
%\subsection{Document Level}
%
%Source files are grouped into folders and \textbf{packages}.
%The package bundled with GAP is called the \textbf{library}.