\documentclass[a4paper]{article}

\usepackage[utf8]{inputenc}
\usepackage{url}
%\usepackage{wrapfig}
\usepackage{amsmath,amssymb,amsthm}

\usepackage{listings}

%\usepackage{array}
%\usepackage{xcolor}
\usepackage{paralist}

\newcommand{\cn}[1]{\ensuremath{\mathtt{#1}}}

\setlength{\hfuzz}{3pt} \hbadness=10001
\setcounter{tocdepth}{2} % for pdf bookmarks

\usepackage[bookmarks,linkcolor=red,citecolor=blue,urlcolor=gray,colorlinks,breaklinks,bookmarksopen,bookmarksnumbered]{hyperref}

%%%%%%%%%%%%%%%%%%%%%%%%%%%%%%%%%%%%%%%%%%%%%%%%%%%%%%%%%%%%%%%%%%%%%%%
% local macros and configurations

\usepackage{../../lib/florian_rabe/basics}
\usepackage{../../lib/florian_rabe/ed}
\usepackage{local}
\renewcommand{\bnf}[1]{{\color{red}#1}}

\pagestyle{plain} % remove for final version

\begin{document}

\lstset{basicstyle={\tt\footnotesize},breaklines}

\title{The Soft Type System of GAP}
\author{Florian Rabe}
\date{LRI Paris, University Erlangen-Nuremberg}
\maketitle

\begin{abstract}
The question of how to design a good type system for mathematics remains open and challenging.
Despite many proposals, no best solution has emerged.
Of particular interest are soft type systems, which can be a compromise between sophisticated type theories and untyped systems.
We contribute to the discussion by giving an easily accessible high-level overview of GAP's soft type system.
\end{abstract}

\section{Introduction}
  In this report we present a prototypical integration of the Jupyter notebooks into the MathHub.info portal for active mathematical documents and a versioned hosting system for flexiformal mathematics.
MathHub.info offers a rich interface for reading, writing, and interacting with mathematical documents and knowledge. Jupyter offers a uniform interface to the computation facilities of the OpenDreamKit VRE toolkit in the form of a read-eval-print loop (REPL).

A mathematical Virtual Research Environment (VRE) needs both kinds of interface functionality: mathematical documents have been very successful for presenting mathematical knowledge, and while there have been efforts to make them modular and interactive they predominantly remain in the mode of archiving and transporting knowledge in Mathematics.
Notebook interfaces also use the document metaphor at the surface; however the REPL interaction
tends to take structural precedence, leading to documents consisting of a sequence of computational cells within which the mathematical discourse is interspersed in the form of ``rich comments''.

A ``literate computing'' version of notebooks which gives mathematical discourse structural precedence is possible in principle, but has not been supported consistently at the system level.\ednote{MK: put the following sentence somewhere: A ``literate programming'' version of notebooks which gives mathematical discourse structural precedence is possible in principle, but has not been supported consistently at the system level.}
This tension and trade-off has been explored in OpenDreamKit Deliverable D4.2~\cite{ODK-D4.2}, and the concept of in-document computation in OpenDreamKit Deliverable D4.9~\cite{ODK-D4.9}.
In both cases, the integration was incomplete, since it lacked a full integration of the
underlying knowledge/computation services.

Generally, the integration of MathHub and Jupyter consists of two parts:
\begin{inparaenum}[\em a\rm )]
\item the integration of the user interfaces (as reported previously) and
\item the integration of the knowledge/computation management services.
\end{inparaenum}
Here we report on progress in both; recall that MMT is the knowledge management service behind MathHub (and more generally for the Math-in-the-Middle based system integration; see OpenDreamKit Deliverable D6.5~\cite{ODK-D6.5}).
%
For the service integration we present an MMT kernel for Jupyter.
%
\ednote{specify what Jupyter widgets are; NT: you may want to reuse some of the language of the D4.16 report, around l21 of https://github.com/OpenDreamKit/OpenDreamKit/blob/master/WP4/D4.16/report.tex}
%
Reciprocally, for the user interface integration, we show how the Jupyter widgets can be deeply integrated within the MMT knowledge management facilities to give semantics-aware interaction facilities, extending the front-end capabilities of MathHub/Jupyter Notebooks by semantic widgets driven by the MMT in-document knowledge management services.

We show and evaluate the integration on two case studies: in-document computing facilities in active documents and a knowledge-based specification dialog for modeling and simulation. 

This report is structured as follows. In Section~\ref{sec:mmt-jp} we report on the MathHub/Jupyter integration at the system level: a Jupyter server as part of the MathHub system and a MMT kernel for Jupyter. Section~\ref{sec:nb-mh} presents the integration of Jupyter Notebooks as active documents in the (new) MathHub front-end, and Section~\ref{sec:mitm-nb} presents the two case studies. Section~\ref{sec:concl} concludes the report and discusses future work.

\ednote{this paragraph seems a bit out of place after the description of the structure of the document}
The goal of this report\ednote{of this deliverable?} is to integrate Jupyter notebooks into MathHub
and make them compatible with MMT, in a way that we can conveniently use 
MMT syntax in these notebooks and also a little bit of extra functionality
like e.g. the Jupyter widgets. The first step is setting up a Jupyter server,
which currently runs on \url{http://juypter.mathhub.info}. \ednote{KA: maybe show picture of it?}
For this server, we have developed a custom kernel, that forwards the input 
entered into the Jupyter notebook to the MMT backend. This then processes 
said input and sends the response back to the Jupyter frontend via the kernel.
We will cover the implementation of the Jupyter kernel and the MMT-backend,
later in this report.


\paragraph{Acknowledgements} The authors gratefully acknowledge the support of the Jupyter team and in particular the advice of Benjamin Ragan-Kelly. Also, the input of Theresa Pollinger and her work on the MoSIS system~\cite{PolKohKoe:kacse18} has shaped our perception of the integration reported here. 

%%% Local Variables:
%%% mode: latex
%%% mode: visual-line
%%% fill-column: 5000
%%% TeX-master: "report"
%%% End:


\section{Concepts}
  \subsection{Overview}

GAP uses a single flat namespace where every declared entity is identified by its \textbf{name}.

Three kinds of named \textbf{declarations} exists: categories, operations, and methods.
Additionally, constructors, attributes, and properties are distinguished special cases of operations.

Operations and categories introduce new objects and thus must have fresh names.
Methods, on the other hand, introduce unnamed implementations of a previously declared named operation; thus, the name of a method is the same as that of an operation.
(At the meta-level, a specific method may be referenced by combining the operation name with the method's documentation string.)

Three kinds of anonymous \textbf{complex} entities exist: objects, families, and filters.
GAP objects represent mathematical objects and are the primary interest.
Families and filters provides a type system on objects: a type consists of a family (the base type) and a filter $F$, which provides a unary predicate on objects of family $F$.
The family of an object $O$ is a hard type: it is unique, computable, and fixed.
The filter is a soft type: $O$ can satisfy any number of filters, filters may be undecidable, and the type of $O$ can be refined at run-time as more filters become known that $O$ satisfies.

\subsection{Complex Entities}

In deduction systems, it is possible, even typical to build all or most expressions ex nihilo, typically via inductive types or axiomatic specifications.
But such representations are usually efficient and therefore problematic in computation systems.
Therefore, GAP allows arbitrary primitive objects backed by concrete representations in the underlying run-time environment.
These are provided by the \textbf{families}: each family introduces a set of primitive objects.

Users can implement new families.
But the following families with their respective primitive objects are built into GAP:
\begin{compactitem}
  \item one each for a few types of built-in literals:
    \begin{compactitem}
      \item cyclotomic numbers (elements of the algebraic closure of the rationals),
      \item booleans,
      \item strings,
    \end{compactitem}
  \item one each for several built-in operators that form complex objects
    \begin{compactitem}
      \item homogeneous lists (called \textbf{collections}): lists of objects that have the same family,
      \item heterogeneous lists: lists of arbitrary objects,
      \item functions on objects.
    \end{compactitem}
\end{compactitem}

The objects are the primitive objects introduced by the families and any application of an operation to objects.

A \textbf{filter} is one of the following:
\begin{compactitem}
  \item the universal filter $\isobj$,
  \item a category $C$,
  \item a property $P$,
  \item a conjunction $F\wedge G$ of filters.
\end{compactitem}
We call categories and properties \textbf{atomic filters}.
By convention, their names are of the form \lstinline|IsXXX|.

Filters can be normalized into a set of atomic filters (with $\isobj$ corresponding to the empty set and $\wedge$ to union).
Therefore, types are essentially pairs of a family and a set of atomic filters, and a type can be efficiently stored as a bitvector indexed by the known atomic filters.
GAP stores this bitvector together with every object.

Because the family is an inherent property of an objects anyway, the \textbf{typing relation} reduces to a relation $\has{O}{F}$ between objects $O$ and filters $F$.
It is defined as follows:
\begin{compactitem}
  \item $\has{O}{\isobj}$ always holds
  \item $\has{O}{F\wedge G}$ holds if $\has{O}{F}$ and $\has{O}{G}$.
  \item $\has{O}{C}$ holds if $O$ was returned by a constructor of category $C$,
  \item $\has{O}{P}$ if evaluating property $P$ on $O$ returns \lstinline|true|,
  \item In addition to the above rules, $\has{O}{F}$ holds for an atomic filter $F$ if the corresponding bit was set when $O$ was constructed.
  This is used in particular by the constructors of categories (see below).
\end{compactitem}

Types are hyper-dynamic: whenever a property is evaluated for an object at run-time, its bit in the cached bitvector type is updated.
Thus, the type changes dynamically as more properties are evaluated.

\subsection{Declarations}

\paragraph{Categories}
A category declaration consists of
\begin{compactitem}
  \item a name,
  \item a filter (called the superfilter).
\end{compactitem}
The concrete syntax is \lstinline|DeclareCategory(name: String, superfilter: Filter)|.

A \textbf{category} declaration introduces a primitive filter.
All categories are created empty.
The objects satisfying this filter are introduced by declaring constructors.
These are operations whose implementation explicitly marks the returned objects as having the category as a filter.

When a constructor of category $C$ is run, the returned object automatically has all filter bits set that correspond to the atomic filters in the superfilter of $C$.

\paragraph{Operations}
An \textbf{operation} declaration introduces an $n$-ary\footnote{GAP has an implementation restriction of $n\leq 6$.} function on objects.
Operations are softly typed: each $n$-ary operations provides a list of length $n$ providing the input filters of the respective argument.
Operations may also carry an optional return type, which defaults to $\isobj$ if omitted.\footnote{This is a recent feature motivated by the discussions that also led to this paper.}

The concrete syntax is

 \lstinline|DeclareOperation(name: String, inputfilters: Filter*, outputfilter: Filter?)|.

An \textbf{attribute} declaration is the special case of an operation that is unary.
The special treatment of attributes is important only for efficiency reasons: The values of attributes are cached with each object.
The concrete syntax is \lstinline|DeclareAttribute(name: String, inputfilter: Filter, outputfilter: Filter?)|.

A \textbf{property} declaration is the special case of an attribute that returns a boolean.
The special treatment of properties is important only because properties can be used as filters.
The concrete syntax is \lstinline|DeclareProperty(name: String, inputfilter: Filter)|.

A \textbf{constructor} declaration is the special case of an operation that returns an object of a given category.
The concrete syntax is \lstinline|DeclareConstructor(name: String, inputfilters: Filter*, outputfilter: Filter?)|.%
\footnote{The return argument is a recent feature. More generally, the current implementation of constructors is somewhat awkward and may be subject to change. Currently, a constructor's first argument is special: It must be the expected return filter (rather than an object).
This is used to allow method selection to choose a different method for different special cases.
A more elegant solution would be to allow every operation to declare that some of its arguments must be filters.
This would yield an untyped version of bounded polymorphism with filter arguments corresponding to type arguments.}

Conceptually, all operations are defined.
But the definiens is never part of the declaration and instead provided separately in method declarations.

\paragraph{Methods}
Every operation can have multiple definitions, which are provided by methods.
A method declaration consists of
\begin{compactitem}
  \item the name of the operation,
  \item the input and output filters,
  \item the actual definition, as a function in the underlying programming language.
\end{compactitem}

The concrete syntax of a method declaration is
\lstinline|InstallMethod(operationname: String, inputfilters: Filter*, outputfilter: Filter?, definition: function)|.

The input and return filters of a method may be more restrictive than the filters used in the operation declaration.
More restrictive input filters can be used to represent overloading of operations or run-time polymorphism.
A more restrictive output filter can be used to indicate a sharper type than required by the operations.

When evaluating the application of an operation to arguments, GAP selects a specific method executes its definition.
If more than one method exists, whose input filters type the operation arguments, an internal ranking is used to disambiguate.\footnote{In particular, if a property of $O$ is evaluated in between two calls of the same operation on $O$, a different method may be selected the second time. This is often desirable, particularly when the second method is more efficient.}

%\subsection{Theory Level}
%
%There is no explicit theory level.
%Instead, theories are represented as categories, and theory morphisms as operations, and their relation is a special case of typing.
%
%Therefore, we can treat each source file as a theory.
%
%\subsection{Document Level}
%
%Source files are grouped into folders and \textbf{packages}.
%The package bundled with GAP is called the \textbf{library}.
  
%\section{Relationships to Other Type Theories}
%  \ednote{This section does not work yet. I forgot to quiz Markus on how the binary operations of algebraic structures are treated. I remember GAP hard-codes two binary operations.}

%\paragraph{A Logic for GAP Theories}
%We can identify a logic and a group of theories that can be naturally embedded into GAP's type system.
%
%Any GAP filter can be used as a type.
%
%Every theory implicitly declares a fixed base type $u$ for the universe.
%
%Then it may have two kinds of declarations:
% \begin{compactitem}
%   \item includes of another theory,
%   \item function symbols $f:a_1\times\ldots\times a_n\to a_0$ where each $a_i$ is a type (either $u$ or some GAP type),
%   \item potential axioms: code in GAP's underlying programming language that evaluates to a boolean
% \end{compactitem}

\paragraph{Theories and Types of Models}
Categories can be used to represent the type of models of a logical theory.
Consider a theory of sorted first-order logic with name $T$ that includes theories $T_1,\ldots,T_k$, declares declares function symbols $f_i:A_i$ for $i=1,\ldots,l$, and declares axioms $p_i$ asserting $F_i$ for $i=1,\ldots,m$.

We can represent this in GAP using the following declarations:
\begin{compactitem}
 \item a category with name $T$ and superfilter $T_1\wedge \ldots \wedge T_n$,
 \item for every $f_i:A_i$ with $A_i=a_1\times\ldots\times a_n\to a$, an $n+1$-ary operation $f_i$ with input filters $T,a_1,\ldots,a_n$ and output filter $a$,
 \item for every $p_i$, a property with input filter $T$,
 \item for every $p_i$, a method implementing $F_i$ (which may be impossible or only possible for restricted case, e.g., only for finite models).
\end{compactitem}

Now the filter $T\wedge p_1\wedge\ldots\wedge p_m$ represents the type of models of the theory.

A concrete model of the theory is represented as follows: \ednote{unclear how to do that}

%\paragraph{Extracting Explicit Theories from GAP's}
%Because GAP does not enforce an abstraction boundary between theories and types, it is not generally feasible to extract explicit theories from GAP.
%
%A heuristic extraction might be possible by trying to identity groups of GAP declarations for which the above operation can be inverted to yield a theory.
  
\section{Conclusion}
  This report summarizes the achievements in Work Package 6 over the last year of the OpenDreamKit project. The main achievements were
\begin{compactenum}
\item The re-conceptualization of ``doing mathematics'' which leads to a better understanding of the nature and intended semantics of VRE components (see Section~\ref{sec:tetrapod}).
\item The integration of (an exemplary) formal knowledge base into the MitM Ontology that provides the pivotal point for system integration and service discovery (see Section~\ref{sec:knowledge})  
\item the development of a semantic model for mathematical datasets (see Section~\ref{sec:data}) which has been used in \WPref{dksbases} in two ways: 
  \begin{compactitem}
  \item The Warwick group led a move to inventory all the data sets, and to (manually) recover their specifications at the mathematical and data base level (schema information).
    In essence this retrofits the existing LFMDB project with the a more semantic level and has led to a vastly improved and more semantic API for LMFDB (see \url{http://www.lmfdb.org/api2/}) that has recently come online.
    A \Sage interface based on API2 is currently under development.
  \item a from-scratch implementation \dmh  that is described in Section~\ref{sec:hub}.
  \end{compactitem}
\item Special and adapted search facilities for all kinds of mathematical data and VRE components; see Section~\ref{sec:software}
\item A standalone implementation of persistent memoization in Python and GAP (see \delivref{dksbases}{persistent-memoization}).
\end{compactenum}
We will go over the various parts in detail in the rest of this section:

\paragraph{Knowledge}
We have introduced an upper ontology for formal mathematical libraries (ULO), which we propose as a community standard, and we exemplified its usefulness at a large scale.
We posit ULO as an interface layer that enables a separation of concerns between library maintainers and users/application developers.
Regarding the former, we have shown how ULO data can be extracted from formal knowledge libraries such as Isabelle.
We encourage other library maintainers to build similar extractors.
Regarding the latter, we have shown how powerful, scalable applications like querying can be built with relative ease on top of ULO datasets.
We encourage other users and library-near developers to build similar ULO applications, or using future datasets provided for other libraries.

Finally, we expect our own and other researchers' applications to generate feedback on the specific design of ULO, most likely identifying various omissions and ambiguities.
We will collect these and make them available for a future release of ULO 1.0, which should culminate in a standardization process.

\paragraph{Data}
We have analyzed the state of research data in mathematics with a focus on the instantiation of the general FAIR principles to mathematical data.
Realizing FAIR mathematical data is much more difficult than for other disciplines because mathematical data is inherently complex, so much so that datasets can only be understood (both by humans or machines) if their semantics is not only evident but itself suitable for automated processing.
Thus, the accessibility of the mathematical meaning of the data in all its depth becomes a prerequisite to any strong infrastructure for FAIR mathematical data.

Based on these observations, we developed the concept of Deep FAIR research data in mathematics.
As a first step towards developing a Deep FAIR--enabling standard for mathematical datasets, we focused on relational datasets.
We presented the prototypical \dmh system that lets mathematicians integrate a dataset by specifying its semantics using a central knowledge and codec collection.
We expect that \dmh also helps alleviate the problem of \emph{disappearing datasets}:
Many datasets are created in the scope of small, underfunded or unfunded research projects, often by junior researchers or PhD students, and are often abandoned when developer change research areas or pursue a non-academic career.

\paragraph{Software: computational mathematical documents}
For the S aspect of D/K/S structures from the \pn proposal or the \textbf{narration} and \textbf{computation} aspects of the finer tetrapod model from Figure~\ref{fig:tetrapod} we have developed a formula harvester for Jupyter notebooks and a formula search engine that builds on them.

To make this possible, we had to invest a heavy dose of software engineering into the MathWebSearch system: Even though the system has successfully been used as a formula search engine in zbMATH publication information system (see \url{https://zbmath.org/formulae/}), the deployment of the system required a lot of domain-specific development and workflow integration.
To this end we have developed Go bindings for the MathWebSearch daemon, documented the interfaces, and provide a web application wrapper.
With this, specific applications only need a domain-specific harvester and front-end.
We have exercised that for the Jupyter Search engine  (as envisioned in task \taskref{dksbases}{mws}) and analogously for a formula search engine for the $n$ category Cafe (nLab, see \url{https://nlabsearch.mathweb.org/}).


\paragraph{Persistent Memoization}
This can also be seen as a source (and implementation) of mathematical datasets, but with an eye on computation rather than than semantics.
Even though it is called ``persistent'' memoization, the temporal scope of the memoized data is less than the ``eternity-scope'' of datasets in LMFDB and \dmh. Indeed, the characteristic innovation in \delivref{dksbases}{persistent-memoization} is that mathematical objects and data can be shared ``across multiple computations''.
In the use cases studied in \pn so far, the semantic level can be left implicit to the implementations, e.g. between \Sage and \GAP, and does not need a uniform search or query interface.
But it is clear that the borders between persistent memoization and mathematical datasets are fluent; indeed many datasets we have now started out as computation caches which ended up being shared in the community. 
We will study the inter-conversion of memoization caches and full datasets on \dmh in the future. 

%%% Local Variables:
%%% mode: latex
%%% mode: visual-line
%%% fill-column: 5000
%%% TeX-master: "report"
%%% End:

%  LocalWords:  standardization analyzed Realizing emph ednote summarizes re-conceptualization WPref dksbases compactitem dmh delivref textbf textbf Jupyter zbMATH taskref mws



\bibliographystyle{alpha}
\bibliography{../../lib/florian_rabe/bib/rabe,../../lib/florian_rabe/bib/systems,../../lib/florian_rabe/bib/pub_rabe}

\end{document}

%%% Local Variables:
%%% mode: latex
%%% TeX-master: t
%%% End:

%  LocalWords:  Cezary Kaliszyk maketitle conc
