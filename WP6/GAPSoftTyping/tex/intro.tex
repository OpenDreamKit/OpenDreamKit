Both computation-oriented systems (like computer algebra system) and deduction-oriented systems (like proof assistants) are developed in adjacent but almost disjoint communities.
Not surprisingly, the distinction between computation and deduction causes many differences between these systems.
But more interestingly, we find major differences also at the level of \emph{type systems} even though these are needed in both kinds of systems.

As a particular example, we consider the type systems of GAP \cite{gap}.
It uses a \emph{soft} type system, where --- akin to the elementhood relation of mathematics --- typing is an undecidable binary relation on a collection of not-inherently-typed objects.
But with the notable exception of Mizar \cite{mizar_types}, deduction-based languages for mathematics tend to use \emph{hard} type systems, where the type is an inherent property of an expression.
Moreover, GAP uses a \emph{hyper-dynamic} type system, where the type of an object is not only \emph{determined} at run time (as in all dynamic type systems) but can even \emph{change} at run time as more information is discovered about the object.
But deduction-based languages tend to use \textbf{static} type systems where the type is a decidable property of the expression.

The above factors have led to fundamental differences between GAP and other type systems, already starting with the GAP keywords being unintuitive to many type theorists.
It is not obvious at all whether and to what extent these differences are due to (i) a lack of awareness by the GAP community of formal type theories, or (ii) the inadequacy of the latter for practical computational applications.

The author suspects the truth is somewhere in between.
This paper is meant to contribute to help inform the discussion by providing an easily accessible overview of the essential qualities of the GAP type system presented from the perspective of type theory.
It does not provide an authoritative account of GAP's type system:
Firstly, some subtleties and possibly even a few important features remain beyond the scope of this paper.
Secondly, the author has --- knowingly or unknowingly --- abstracted away some idiosyncrasies in order to make this account more accessible.

Additionally, this paper can be seen as a working document that can be refined over time with the goal of obtaining a rigorous, formal description of the entire GAP type system.
The author is hopeful that this process may even feed back into the GAP design when discussing whether inconsistencies between the author's understanding and the actual implementation should result in changes to the former or the latter.

In any case, none of the type system design or the GAP implementation is due to the author, and this paper should not be construed as claiming such a contribution.
The author's sole contribution is describing the existing type system from a specific outsider's perspective.
This, however, is still difficult enough to be worthwhile.
