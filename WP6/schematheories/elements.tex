The elements of most types are straightforward.
In any case, for the purposes of this document, we do not need to fix concrete syntax for the elements of most types.

However, some types have multiple representations that are different in meaningful ways (e.g., because converting between representations is expensive or imprecise).
Moreover, some types have multiple constructors similar to an inductive type.
In the sequel, we describe which representations are supported in those cases.

\paragraph{Integers Modulo}
The elements of $Z(m)$ are represented by $0,\ldots,m-1$.

\paragraph{Real Numbers}
A real number can be one of the following:
\begin{compactitem}
 \item a rational number
 \item an IEEE double precision float
 \item a root $\sqrt[n]{x}$ for $n\in N$ and $x\in Z$
 \item the strings "pi" and "e"
\end{compactitem}

\paragraph{Complex Numbers}
A complex number can be one of the following:
\begin{compactitem}
 \item Cartesian form $x+yi$
 \item polar form $r e^{i\phi}$
 \item root of unity $\zeta_n$
\end{compactitem}

\paragraph{p-Adic Numbers}
A $p$-adic number $x$ consists of unit $u\in N$ ($u,p$ coprime), valuation $v\in Z$, and precision $r\in N$ (for $u<p^r$).

\paragraph{Polynomials}
For $r\in Ring$ and distinct strings $x_i$, we consider polynomials $p\in Polynomial(r,[x_1,\ldots,x_n])$ to be of the form $p=\Sigma_{\vec{i}\in N^n} a_{\vec{i}} \vec{x}^{\vec{i}}$ where $a_{\vec{i}}\in r$ and $(x_1,\ldots,x^n)^{(i_1,\ldots,i_n)}$ abbreviates $x_1^{i_1}\cdot \ldots\cdot x_n^{i_n}$.

\paragraph{Rings}
A ring can be one of the following:
\begin{compactitem}
 \item a field
 \item $Polynomial(r,[x_1,\ldots,x_n])$ for $r\in Ring$
\end{compactitem}

\paragraph{Fields}
A field can be one of the following:
\begin{compactitem}
 \item base fields $Q$, $R$, and $C$
 \item finite fields $Z(p)$ for $p\in Prime$ (same type as integers modulo $p$)
 \item polynomial field extensions $FieldExtension(F,p,a)$ of $F\in Field$ for a polynomial $p\in Polynomial(F,[x])$ (for any variable name $x$)
 \item named fields identified by a string
\end{compactitem}

We define some abbreviations for common fields:
\begin{compactitem}
 \item $Q(p,a)=FieldExtension(Q,p,a)$
 \item $Qsqrt(n,a)=Q(x^2-n,a)$
 \item $Qzeta(n,a)=Q(y_n,a)$ where $y_n$ is the $n$-th cyclotomic polynomial
\end{compactitem}

We do not define $GF(q)$ for $q=p^n$ as an abbreviation for $FieldExtension(Z(p),g)$ for some irreducible polynomial $g\in Polynomial(Z(p))$ of degree $n$ because there is no way to choose $g$ canonically and it is necessary to know $g$ to represent the elements of $GF(q)$.

\paragraph{Structure Elements}
Every structure has an underlying type, which is used to represent its elements.

The underlying types of fields are defined as follows:
For $Q$, $R$, $C$, and $Z(p)$, the underlying type is the field itself.
The underlying type of $FieldExtension(F,p,a)$ is $Polynomial(F,[a])$ ($a=x$ is allowed).
