\section{Conclusion}\label{sec:concl}

We have shown how to extend the Math-in-the-Middle framework for integrating systems to mathematical data bases like the \lmfdb. 
The main idea is to embed knowledge sources as virtual theories, i.e. theories that are not -- theoretically or in practice -- limited in the number of declarations and allow dynamic loading and processing. 
For accessing real-world knowledge sources, we have developed the notion of CoDecs and integrated them into the MitM ontology framework. 
These CoDec's (and their MitM types) lift knowledge source access to the MitM level and thus enable object-level interoperability and allow humans (mathematicians) access using the concepts they are familiar with. 
Finally, we have shown a prototypical query translation facility that allows to delegate some of the processing to the underlying knowledge source and thus avoid thrashing of virtual theories. 
We have discussed this methodology on the elliptic curves sub-base of the \lmfdb, which we have fully integrated. 
We are currently working on additional \lmfdb sub-bases and the Online Encyclopedia of Integer Sequences. 

% We have shown how to extend the data model for theories of the \mmt system, and implemented a generic approach that does no longer require the theories to reside in the main memory; instead knowledge is retrieved from an external system and declarations are generated on demand.
% The main conceptual leap here was to make a difference between the representations of objects in the database, and the underlying mathematical objects, and to translate between them using generic codecs. 
% We have demonstrated that this approach is functional using the example of \lmfdb. 

% The main advantages of this approach are, unlike existing interfaces to knowledges bases, its' generic design and easy extensibility. 
% \begin{itemize}
% \item To add a new \lmfdb database to the set of represented theories one in practice only
%   has to add a new schema theory.  This schema theory will have to contain the sets of
%   fields within this new database, along with their mathematical types and codecs.  All of
%   these should be known to the database maintainers, albeit not directly as types and
%   \identifier{codecs}, and are thus easy to find.  This schema theory can then be used to
%   automatically generate a new database theory.
% \item To add a knowledge base that uses a different data base, we also have to implement a
%   new \mmt backend and -- possibly -- extend the set of CoDecs.
% \end{itemize}

% There are several other aspects which we have not detailed here as addition of Virtual Theories to \mmt impacts several other aspects of the system. 
% For example, \mmt has a Query Language allowing users to query information available; the previous implementation relied on theories being concrete, which is no longer true for all theories.  
% Furthermore, we are also planning to extend our implementation of Virtual Theories; e.g. we want to extend it to more than a few \lmfdb databases. 
% We also have concrete plans for a second example based on OEIS\ednote{We had at some point at least}. 

\subsubsection*{Acknowledgements}
The authors gratefully acknowledge the fruitful discussions with other participants of
work package WP6, in particular John Cremona on the LMFDB and J\"org Arndt on the OEIS. We
acknowledge financial support from the OpenDreamKit Horizon 2020 European Research
Infrastructures project (\#676541).

%%% Local Variables:
%%% mode: latex
%%% TeX-master: "paper"
%%% End:

%  LocalWords:  sec:concl subsubsection ommt lmfdb itemize
