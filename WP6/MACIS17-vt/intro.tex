\section{Introduction}\label{sec:intro}

There are various mathematical knowledge collections and information systems available. 
These range from generic information systems -- like Wikipedia -- via informal mathematical databases -- like the Cornell arXiv, zbMath, or MathSciNet -- and semi-formal object collections -- like the GAP group libraries, the Online Encyclopedia of Integer sequences (OEIS), and the L-functions and Modular Forms Database (LMFDB), to fully formal theorem prover libraries like those of Mizar, Coq, PVS, and the HOL systems. 
We commonly refer to all these as Mathematical Knowledge Bases. 

These Mathematical Knowledge Bases can be very useful in mathematical research and practice. 
Commonly these systems are only accessible via a dedicated web interface that allows humans to query or browse the databases. 
A programatic interface, if it exists at all, is system specific, meaning that to use it mathematicians need to be familiar both with the mathematical background and internal structure of the system in question. 
No predominant standard exists, and these interfaces usually only expose the low-level raw database content. 

In this paper, we focus on addressing this problem. 
We ask the question of what mathematicians desire from a ``programmatic, mathematical API''. 
Such an API would give access to the knowledge-bases programmatically via their mathematical constructions and properties. 

In this paper, we take it a step further and discuss our implementation of such an approach. 
We interpret mathematical knowledge bases as \omdocmmt\ theory graphs -- modular, flexi-formal representations of mathematical objects, their properties, and relations. 
We update \omdocmmt\ theories to ``virtual theories'' and update knowledge management algorithms so that they can cope with theories that do not fit into main memory but directly deal with the underlying databases as backends employing a modular system of codecs to bridge the gap between the database schema and the mathematical construction of objects.

\subsection*{Structure}\label{sec:intro:struct}

In this paper we continue as follows:
In Section~\ref{sec:mmtmitm} we give the reader a short overview of \omdocmmt\ theory graphs along with OpenDreamKit, and the Math-In-The-Middle approach, our primary use-case for Virtual Theories. 
We then continue in Section~\ref{sec:sota} by giving an example a State-Of-The-Art Mathematical Databases along with its' interface by discussing the L-functions and Modular Forms Database. 
In Section~\ref{sec:vt} we describe and discuss our Virtual Theory approach by going into this example further and show-casing how our implementation of Virtual Theories using Codecs works. 
Finally, we conclude the paper in Section~\ref{sec:concl}. 

%%% Local Variables:
%%% mode: latex
%%% TeX-master: "paper"
%%% End:

%  LocalWords:  sec:intro oeis lmfdb sagemath wrapfigure textwidth tikzpicture textbf
%  LocalWords:  fig:classicalconnect LuzKoh:fsarfo16 DehKohKon:iop16 omdocmmt mechanized
