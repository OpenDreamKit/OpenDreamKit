\section{Example: The API and Structure of LMFDB}\label{sec:sota}

The ``L-functions and Modular Forms Database'' (\lmfdb~\cite{lmfdb}) is a Python web application with a MongoDB backend. 
The project contains several thousand L-Functions and curves along with their properties. 
We use this as an example of a Virtual Theory. 
Before we go into this in more detail, we first have a closer look at the structure and existing APIs to communicate with it.

\subsection{The Structure of LMFDB}\label{sec:sota:struct}

\lmfdb has several sub-databases -- each of which contains different kinds of objects.

These databases include e.g. a database of elliptic curves or a database of transitive groups. 
Within each database, each curve is stored as a single JSON record with common keys, Figure~\ref{fig:lmfdbexample} shows one: each property of this JSON object corresponds to a property of the underlying mathematical object. 
For example, the \identifier{degree} property -- here $1$ -- of the JSON objects corresponds to the degree of the underlying elliptic curve. 

\begin{figure}[ht]\centering
      \begin{lstlisting}[language=json]
{
    "degree": 1,
    "non-maximal_primes": [5],
    "torsion_structure": ["5"],
    "ainvs": ["0","-1","1","-10","-20"],
    "x-coordinates_of_integral_points": "[5,16]",
    "real_period": 1.26920930427955,
    "min_quad_twist": {"disc": 1,"label": "11a1"},
    "sha_an": 1.0,
    "conductor": 11,
    "iwp0": 7,
    "2adic_gens": [],
    "torsion_primes": [5],
    "signD": -1,
    "tamagawa_product": 5,
    "isogeny_matrix": [[1,5,25],[5,1,5],[25,5,1]],
    "non-surjective_primes": [5],
    "lmfdb_label": "11.a2",
    "2adic_index": 1,
    "equation": "\\( y^2 + y = x^{3} -  x^{2} - 10 x - 20  \\)",
    "label": "11a1",
    "regulator": 1.0,
    "anlist": [0,1,-2,-1,2,1,2,-2,0,-2,-2,1,-2,4,4,-1,-4,-2,4,0,2],
    "iso": "11a",
    "_id": "ObjectId('4f71d4304d47869291435e6e')"
}
      \end{lstlisting}\vspace*{-1.5em}
  \caption[An elliptic curve from \lmfdb]{
    An elliptic curve, as found within \lmfdb. 
    Some key-value pairs are omitted for readability. 
  }
  \label{fig:lmfdbexample}
\end{figure}

Other properties are more complex.
Whereas the value of the \identifier{degree} property is a simple integer, the value of the \identifier{isogeny\_matrix} property is a list of lists, which represents a matrix. 
This can become even more technical. 
For example the \identifier{x-coordinates\_of\_integral\_points} field, \lmfdb represents a list of integers as a list of strings as the integers can exceed the range of MongoDB system integers. 
This already shows that it is non-trivial to get from a MongoDB encoding of an elliptic curve in \lmfdb to the representation of a mathematical object. 


\subsection{An API for \lmfdb Objects}\label{sec:sota:api}

As \lmfdb is is a mathematical knowledge base, one important use case is to find elliptic curves subject to specific criteria. 
Consider for example a mathematician that wants to find all abelian elliptic curves in \lmfdb. 
How can this be achieved using the \lmfdb API located at \cite{lmfdbapi}? 

\begin{figure}[ht]\centering
  \includegraphics[width=\textwidth]{APIScreenshot.png}
  \caption[The Web-Interface for the \lmfdb API. ]{
    The Web-Interface for the \lmfdb API. 
  }
  \label{fig:apiscreenshot}
\end{figure}

Queries can be sent to the API by making appropriate GET requests. 
The \lmfdb API can present results in two different ways, either using a web-based interface or programmatically by returning a set of JSON objects. 
A screenshot of the former can be seen in Figure~\ref{fig:apiscreenshot}. 
The mode can be decided upon by adding an appropriate parameter to the query. 
In the following we will focus on the latter mode only, however all links will not include this format parameter so that readers can follow along in the web browser\ednote{Since the API is broken, I'm not sure what to do here }. 

Queries must be formulated in terms of the underlying MongoDB schema, are sub-database specific, and should consist of a set of key value pairs. 
To solve the example given here we need to send the key-value pair \identifier{commutative}$ = $ \identifier{true}, finding all elements for which the commutative property is true. 
However, these values need to be encoded to be understood by MongoDB. 
We need to realize that the \identifier{ab} key corresponds to the commutativity property, has boolean values, and that MongoDB encodes \inlinecode{true} as \inlinecode{1}, and \inlinecode{false} as \inlinecode{0} in this \lmfdb sub-database. 
This information can then be used to make a query by sending a request to \url{http://www.lmfdb.org/api/transitivegroups/groups/?ab=1}. 

In this example, each of the steps are relatively straightforward. 
In a general setting, e.g. when searching for all elliptic curves with a specific isogeny matrix, this not only requires a good familiarity with the mathematical background but also with the system internals of the particular \lmfdb sub-database; a skillset commonly found in neither research programmers nor average mathematicians.   

To summarize: while \lmfdb offers a programmable API for accessing its contents, the content API is at the MongoDB level, and not the level of mathematical objects. 
Our Diagnosis is that \lmfdb -- and most other mathematical knowledge databases -- suffer from a double impedance mismatch problem.
\begin{compactenum}[\bf {I}1]
\item \emph{human/computer impedance mismatch}: Humans have problems interacting with \lmfdb, since they must speak the system language instead \lmfdb speaking mathematics
\item \emph{computer/computer impedance mismatch}: mathematical computer systems cannot interoperate, since their system languages differ.
\end{compactenum}
In the MitM approach we have presented in Section~\ref{sec:mmtmitm}, we can solve both problems at the same time by lifting the communication to the level of \ommt-encoded MitM objects, which both MitM-compatible software systems and humans speak -- this is the central assumption of the MitM approach.
%%% Local Variables:
%%% mode: latex
%%% TeX-master: "paper"
%%% End:

%  LocalWords:  sec:sota lmfdb lmfdb lstlisting json ainvs iwp0 2adic_gens isogeny_matrix
%  LocalWords:  tamagawa_product 2adic_index anlist 4f71d4304d47869291435e6e vspace emph
%  LocalWords:  fig:lmfdbexample isogeny includegraphics textwidth fig:apiscreenshot
%  LocalWords:  centering summarize sec:mmtmitm ommt-encoded
