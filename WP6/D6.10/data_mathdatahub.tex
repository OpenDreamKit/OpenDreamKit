We present a unified infrastructure to support Deep FAIR for relational mathematical data.
It builds on our MathHub system, a portal for narrative and symbolic mathematical data.
MathDataHub is a part of the MathHub portal and provides storage and hosting with integrated support for Deep FAIR.
In the future, this will also allow for the development of mathematical query languages (i.e., queries that abstract from the encoding) and mathematical validation (e.g., type-checking relative to the mathematical types, not the database types).

To that end, we developed a mathematical data description language MDDL in~\cite{BerKohRab:tumdi19} (Math Data Description Language) that uses symbolic data to specify the semantics of relational data.
MDDL schemas combine the low-level schemas of relational database with high-level descriptions (which critically use symbolic mathematical data) of the mathematical types of the data in the tables.

\begin{figure}[ht]
  \includegraphics[width=.48\textwidth]{data_joe-schema}
  \caption{Schema theory for Joe's dataset}\label{fig:joe-schema}
\end{figure}

To fortify our intuition let us assume that Joe has collected a set of integer matrices together with their trace
and the Boolean property whether they are orthogonal.
Figure~\ref{fig:joe-schema} shows a MDDL theory that describes his database schema.
For example, the mathematical type of the field $\mathsf{mat}$ is integer $2\times2$ matrices;
the $\mathsf{codec}$ annotation specifies how this mathematical type is be encoded as a low-level database type (in this case: arrays of integers).
Concretely, the codec is $\mathsf{MatrixAsArray}$ codec operator applied to the identity codec for integers.
These codec annotations capture the representation theorem that allows representing the mathematical objects as ground data that can be stored in databases. 
%The tag \textsf{opaque} specifies that matrices cannot be used for filtering in the user interface. 

The information is sufficient to generate a database schema -- here one table with columns $\mathsf{mat}$, $\mathsf{trace}$, and $\mathsf{orthogonal}$ -- as well as a database browser-like website frontend (see Figure~\ref{fig:joe}).
The generation of APIs for computational software such as computer algebra systems is also possible and currently under development. 

\begin{figure}[ht]
  \includegraphics{data_joe.png}
  \caption{Website for Joe's dataset}\label{fig:joe}
\end{figure}

Crucially,  the codec-based setup transparently connects the mathematical level of specification with the database level -- a critical prerequisite for the deep FAIR properties postulated above.
Moreover, in Figure~\ref{fig:joe-schema}, the mathematical background knowledge is imported from a theory $\mathsf{IntegerMatrix}$ in the Math In The Middle ontology (MitM)~\cite{MitM:on}, which supplies the full mathematical specification and thus the basis for \emph{Interoperability} and \emph{Reusability}; see~\cite{BerKohRab:tumdi19,WieKohRab:vtuimkb17,KohMuePfe:kbimss17} for details.
The overhead of having to specify the semantics of the mathematical data is offset by the fact that we can reuse central resources like the MitM ontology and codec collection. 
Thus, MitM and MDDL form the nucleus of a common vocabulary for typical mathematical relational datasets. 
%These can and should eventually be linked to representation standards in other domains. 
%For mathematical datasets, the math-specific aspects attacked by our work are the dominant factor.




%%% Local Variables:
%%% mode: latex
%%% mode: visual-line
%%% fill-column: 5000
%%% TeX-master: "report"
%%% End:

%  LocalWords:  flexiformal BerKohRab:tumdi19 includegraphics textwidth textbf textsf textsf textsf ednote BerKohRab:tumdi19,WieKohRab:vtuimkb17,KohMuePfe:kbimss17 externalize
