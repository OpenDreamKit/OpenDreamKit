\paragraph{Formal Knowledge Bases}
For many decades, the development of a universal database of all mathematical knowledge, as envisioned, e.g., in the QED manifesto \cite{qed}, has been a major driving force of computer mathematics.
Today a variety of such libraries are available.
These are most prominently developed in proof assistants such as Coq \cite{coq} or Isabelle \cite{isabelle} and are treasure troves of detailed mathematical knowledge.
However, despite the enormous potential for many applications of and in virtual research environments, this treasure is usually locked into system- and logic-specific representations that can only be understood by the respective theorem prover system.
For example, this precludes applications such as finding related object in knowledge bases from within computation-oriented systems as used in OpenDreamKit.


Therefore, we have developed interface standards that allow maintainers of formal libraries to make their content available to outside systems.
In this deliverable, we report on complementing our existing OMDoc/MMT standard for representing entire knowledge bases with a new Upper Library Ontology (ULO).
ULO is a standard ontology for exchanging high-level information about mathematical libraries that systematically abstracts all symbolic knowledge away and only retain what can be easily represented relationally.
That allows for semantic web-style representations, for which simple and standardized formalisms such as OWL2 \cite{w3c:owl2-xml}, RDF \cite{rdf}, and SPARQL~\cite{w3c:SPARQL-Rec:13} as well as highly scalable tools are readily available.
While it is well-known that ontology language--based relational formalisms are inappropriate for symbolic knowledge like formulas, algorithms, and proofs, it is this high-level information that is often critical important for integration into virtual research environments, e.g., to realize benefits like search.

We report on this in Section~\ref{sec:knowledge}.
The bulk of this section is taken by a report on a new task (D6.11) \ednote{maybe insert taskref} that we have added to \pn.
In this task, we export the large Isabelle knowledge bases \ednote{insert size estimate} as both OMDoc/MMT and ULO format.
We show the utility of the generated ULO data by setting up a relational query engine that provides easy access to certain library information that was previously hard or impossible to determine.

\paragraph{Acknowledgments and Dissemination}
Task D6.10 \ednote{insert task ref} was carried out as a subcontract by Makarius Wenzel.
Some parts of Section~\ref{sec:isabelle} are adapted from his descriptions of his work in the context of this subcontract.

This work has been published as \ednote{cite ULO paper} for ULO.\ednote{add other papers}
The export facilities of Isabelle and the integration with MMT have been integrated with Isabelle in its June 2019 release.
It was reported on in two blog posts by Wenzel%
\footnote{\url{https://sketis.net/2018/isabelle-mmt-export-of-isabelle-theories-and-import-as-omdoc-content}, and \url{https://sketis.net/2019/mmt-as-component-for-isabelle2019}}
and a paper is forthcoming.