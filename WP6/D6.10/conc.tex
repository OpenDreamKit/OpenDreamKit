\paragraph{Knowledge}
We have introduced an upper ontology for formal mathematical libraries (ULO), which we propose as a community standard, and we exemplified its usefulness at a large scale.
We posit ULO as an interface layer that enables a separation of concerns between library maintainers and users/application developers.
Regarding the former, we have shown how ULO data can be extracted from formal knowledge libraries such as Isabelle.
We encourage other library maintainers to build similar extractors.
Regarding the latter, we have shown how powerful, scalable applications like querying can be built with relative ease on top of ULO datasets.
We encourage other users and library-near developers to build similar ULO applications, or using future datasets provided for other libraries.

Finally, we expect our own and other researchers' applications to generate feedback on the specific design of ULO, most likely identifying various omissions and ambiguities.
We will collect these and make them available for a future release of ULO 1.0, which should culminate in a standardization process.

\paragraph{Data}
We have analyzed the state of research data in mathematics with a focus on the instantiation of the general FAIR principles to mathematical data.
Realizing FAIR mathematical data is much more difficult than for other disciplines because mathematical data is inherently complex, so much so that datasets can only be understood (both by humans or machines) if their semantics is not only evident but itself suitable for automated processing.
Thus, the accessibility of the mathematical meaning of the data in all its depth becomes a prerequisite to any strong infrastructure for FAIR mathematical data.

Based on these observations, we developed the concept of Deep FAIR research data in mathematics.
As a first step towards developing a Deep FAIR--enabling standard for mathematical datasets, we focused on relational datasets
We presented the prototypical MathDataHub system that lets mathematicians integrate a dataset by specifying its semantics using a central knowledge and codec collection.
We expect that MathDataHub also helps alleviate the problem of \emph{disappearing datasets}:
Many datasets are created in the scope of small, underfunded or unfunded research projects, often by junior researchers or PhD students, and are often abandoned when developer change research areas or pursue a non-academic career.
