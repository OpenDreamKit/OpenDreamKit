\ednote{Write a general notebook introduction, cite D4.2, D4.3}

\subsection{Formula Search Engine}

\ednote{
    Recycle some existing text from somewhere
    Spin: MathWebSearch was an experimental tool that required a lot of work to setup per application
}

MathWebSearch \ednote{cite} is a search engine enabling semantic formula search. 
For example, given a query like $a^2 + b^2 = c^2$\ednote{Make formulae red} it will find documents containing $3^2 + 4^2 = 5^2$ or $x^2 + y^2 = z^2$.
It was originally developed by \ednote{TODO}. 

The MathWebSearch daemon takes in a set of Content MathML\ednote{cite original}-encoded formulae.
It then builds an index using a technique called Substitution Tree Indexing
Having generated and loaded this index into memory, it is then capable of answering queries.
The queries are also ContentMathML with a custom extension for marking query variables.
The answer to a query consists of the set of matching formulae and the appropriate variable subsiutions. 
Additionally meta-information, such as the URLs of the documents that the formulae come from, is returned. 
For details, we refer the interested reader to \ednote{cite}.  

However, the daemon itself is not sufficient to fully use MathWebSearch. 
Conceptually, an instance of MathWebSearch consists of four components:
\ednote{Do we want to have a picture here?}
\begin{itemize}
    \item A program called a \textbf{Harvester} which is given a provided with a corpus of documents and extracts the set of formulae contained in it;
    \item the \textbf{MathWebSearch daemon} itself, which based on the harvested formulae, generates and maintains an index as described above;
    \item an \textbf{query input parser}\ednote{Name?} which converts user input into a query the daemon can understand
    \item a \textbf{frontend} which sends user input to the query parser and daemon, and presents the query result to the user.  
\end{itemize}

\ednote{
    - ArXiV search
    - Zentralblatt search
}

\subsection{Enabling Formula Search Deployments}

To use MathWebSearch inside OpenDreamKit, we need to be able to flexibly use it as a new component. 
To achieve this, we developed deployment infrastructure in the form of several components on top of the core MathWebSearch daemon. 
Each component typically exists as a single Docker Container\ednote{Reference other ODK Docker stuff here}. 
The components are composed using a Docker Compose file. The structure of our infrastructure can be seen in Figure~\ref{fig:mwsdeployment}. 

\begin{figure}[ht]
  \includegraphics[width=0.8\textwidth]{mws_layout.pdf}
  \caption{Structure of a typical MathWebSearch Deployment}\label{fig:mwsdeployment}
\end{figure}

We describe the components of our infrastructure. 

\paragraph{The MathWebSearch Daemon}
The central component is the \textit{MathWebSearch daemon}, which can be found in the bottom left. 
As previously, it uses an \textit{Index} and exposes an XML-based API for queries. 
The only change we made to the core daemon is that it now exists inside a Docker Container. 

\paragraph{The Harvesters and The Indexer}
In order to generate an index from a set of documents, as before, we need two components. 
First we extract a set of ContentMathML formulae using a corpus-specific \textit{Harvester} (bottom right). 
The generated \textit{Harvest} is then sent to the \textit{Indexer} (bottom center), which generates or updates the \textit{Index}. 

Additionally, we introduced a scheduler component called \textit{MWS cron}. 
This periodically sends updated harvests to the \textit{indexer}, which then in turn updates the index. 
This process ensures that the Index remains up-to-date. 

\paragraph{Frontend, mwsapi and the query syntax parser}

In order to process end user queries, we introduced several additional components.

The most important component of these is the \textit{frontend}, which runs inside the users' browser. 
It allows users to enter a formula search query, and view the results. 
The frontend contains corpus-specific branding and text, and is otherwise not corpus-specific. 
In particular the querying code which interacts with the backend components, does not require specialization. 
\ednote{Screenshot here?}

While the MathWebSearch daemon only understands formulae in Content MathML, users often enter them using different representations, such as \LaTeX. 
For this purpose, the frontend allows entering queries in human-writable, corpus-specific syntax. 
In order to transform the user query into a system-understood query, we make use of a new component called the \textit{Query Syntax Parser}. 
For {\LaTeX} syntax this is achieved using {\LaTeX}ML along with a custom MathWebSearch extension, but this component is fully interchangable if the user desires other syntaxes. 

The frontend does not directly send MathML queries to the \textit{Daemon}.
Instead, it sends them to a thin API layer on top called \textit{mwsapi}. 
This layer forwards the queries to the daemon and, upon receiving a response, performs some post-processing. 
This process includes transforming substiutions returned by MathWebSearch into a format that can be directly presented to the user by the frontend. 

As the frontend, the mwsapi server and the Query Syntax Presenter are all exposed to the end-user under the same url, a proxy delegating requests accordingly was also neccessary. 
This is using an nginx \ednote{cite} server. 

\subsection{Building a Notebook Search}

\ednote{
- describe how this new setup allowed us to flexibly create new frontends
- describe sage notebooks
- describe how we implemented the harvester
- describe the dedidcated frontend
- describe the simplicity of this implementation wrt the previous parts
}

\subsection{Future Plans for MathWebSearch}

- describe the custom component / frontends for NLAB
- temasearch
- python ast search + input syntax?

%%% Local Variables:
%%% mode: latex
%%% mode: visual-line
%%% fill-column: 5000
%%% TeX-master: "report"
%%% End:
