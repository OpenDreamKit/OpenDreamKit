\section{The \ODK project}\label{sec:odk}
The \ODK project involves around 50 researchers and developers at 15 European institutions (with on average 15 FTE funded concurrently to work on the project). It will run for four years, starting in September 2015.

\ODK's funding comes to work on \emph{Virtual Research Environments} (VRE), that is online services enabling groups of researchers, typically widely dispersed, to work collaboratively on a per project basis. Rather than constructing a large monolithic VRE, we have designed our proposal to instead make the long-term investments listed in the previous section, and work on the large scale yet modular modular integration of mathematical software. Our endgoal is thus to have a modular, interoperable and customisable toolkit of relatively modest components, and our approach to work on the grease to make this work. Since the funding scheme's focus is a bit wider than just technical goals, the project will also address other aspects, such as outreach and tools to support teaching. 

More precisely, the \ODK work plan consists consists of several work packages, with wide breadth: component architecture (modularity, packaging, distribution, deployment), user interfaces (\Jupyter notebook interfaces to interactive components, 3D visualization, documentation tools), high performance mathematical computing (especially multicore/parallel architectures), a study of social aspects of collaborative software development, and a package focused on data/knowledge/software-bases. 

The latter package is core to this paper, and cross-cutting to the \ODK project. It focuses on the identification and extension of ontologies and standards to facilitate safe and efficient storage, reuse, interoperation and sharing of rich mathematical data, whilst taking into account of provenance and citability. It will build a component architecture for data archiving and sharing in a semantically sound way and integrate computational software and databases. We want to  enable researchers to seamlessly manipulate mathematical objects across computational engines (switch algorithm implementations from one computer algebra system to another, for instance), front end interaction modes (database queries, notebooks, web, etc) and even backends (distributed vs.~local, for instance). 

% The \ODK project is committed to working openly. Deliverables are tracked using public GitHub issues (see~\cite{OpenDreamKit:on}), which tightens the loop for early community feedback.

%%% Local Variables:
%%% mode: latex
%%% TeX-master: "paper"
%%% End:

%  LocalWords:  specialized Arxiv Jupyter IPython ldots compactitem emph compactenum odk
%  LocalWords:  ODKproposal organization standardization visualization citability oldpart
%  LocalWords:  organizing Swinnerton-Dyer resentation desingularisation Hironaka ednote
%  LocalWords:  Hironaka algorithmisation Villamayor
