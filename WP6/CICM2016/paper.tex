\documentclass{llncs}
\pagestyle{plain}
\usepackage[show]{ed}
\usepackage[utf8]{inputenc}
\usepackage{lststex}
\usepackage{paralist}
\lstset{basicstyle=\sf,columns=fullflexible,showstringspaces=false}
% \usepackage{url}
% \usepackage{wrapfig}
\usepackage{tikz,standalone}
\usetikzlibrary{backgrounds,shapes,fit,shadows,mmt}
\usepackage{wrapfig}
\usepackage{xspace}
\usepackage{hyperref}
\usepackage{stex-logo}
\usepackage{odksystems}
\usepackage[style=alphabetic,backend=bibtex]{biblatex}
\addbibresource{kwarc.bib}% do not change
\addbibresource{rest.bib}% add bibs here!
\renewbibmacro*{event+venue+date}{}
\renewbibmacro*{doi+eprint+url}{%
  \iftoggle{bbx:doi}
    {\printfield{doi}\iffieldundef{doi}{}{\clearfield{url}}}
    {}%
  \newunit\newblock
  \iftoggle{bbx:eprint}
    {\usebibmacro{eprint}}
    {}%
  \newunit\newblock
  \iftoggle{bbx:url}
    {\usebibmacro{url+urldate}}
    {}}
\title{Interoperability in the \ODK Project:\\
The Math-in-the-Middle Approach}
\author{Paul-Olivier Dehaye\inst{3} Michael Kohlhase\inst{2} Alexander
  Konovalov\inst{4} Samuel Lelièvre\inst{1} Markus
  Pfeiffer\inst{4} Nicolas M. Thiéry\inst{1}}
\institute{
  Universit\'e Paris-Sud\and
Jacobs University \and 
  University of Z\"urich \and
  University of St~Andrews
}
\providecommand\ifprefchar[2]{}% fix biblatex
\newcommand\defemph[1]{\textbf{#1}}
%\renewcommand{\ednote}[1]{}

\begin{document}
\maketitle
\begin{abstract}
  \ODK --- ``Open Digital Research Environment Tool\-kit for the Advancement of
  Mathematics'' --- is an H2020 EU Research Infrastructure project that aims at
  supporting, over the period 2015--2019, the ecosystem of open-source mathematical
  software systems. From that, \ODK will deliver a flexible toolkit enabling research
  groups to set up Virtual Research Environments, customised to meet the varied needs of
  research projects in pure mathematics and applications.

  An important step in the \ODK endeavor is to foster the interoperability between a
  variety of systems, ranging from computer algebra systems over mathematical databases to
  front-ends. This is the mission of the integration work package (WP6). We report on
  experiments and future plans with the \emph{Math-in-the-Middle} approach. This
  information architecture consists in a central mathematical ontology that documents the
  domain and fixes a joint vocabulary, combined with specifications of the functionalities
  of the various systems. Interaction between systems can then be enriched by pivoting off
  this information architecture.
\end{abstract}


\ednote{NT: fix the over-capitalization of section titles (matter of taste)}
\ednote{NT: explain how this relates to other efforts in the same direction, in particular OpenMath}

In this report we present a prototypical integration of the Jupyter notebooks into the MathHub.info portal for active mathematical documents and a versioned hosting system for flexiformal mathematics.
MathHub.info offers a rich interface for reading, writing, and interacting with mathematical documents and knowledge. Jupyter offers a uniform interface to the computation facilities of the OpenDreamKit VRE toolkit in the form of a read-eval-print loop (REPL).

A mathematical Virtual Research Environment (VRE) needs both kinds of interface functionality: mathematical documents have been very successful for presenting mathematical knowledge, and while there have been efforts to make them modular and interactive they predominantly remain in the mode of archiving and transporting knowledge in Mathematics.
Notebook interfaces also use the document metaphor at the surface; however the REPL interaction
tends to take structural precedence, leading to documents consisting of a sequence of computational cells within which the mathematical discourse is interspersed in the form of ``rich comments''.

A ``literate computing'' version of notebooks which gives mathematical discourse structural precedence is possible in principle, but has not been supported consistently at the system level.\ednote{MK: put the following sentence somewhere: A ``literate programming'' version of notebooks which gives mathematical discourse structural precedence is possible in principle, but has not been supported consistently at the system level.}
This tension and trade-off has been explored in OpenDreamKit Deliverable D4.2~\cite{ODK-D4.2}, and the concept of in-document computation in OpenDreamKit Deliverable D4.9~\cite{ODK-D4.9}.
In both cases, the integration was incomplete, since it lacked a full integration of the
underlying knowledge/computation services.

Generally, the integration of MathHub and Jupyter consists of two parts:
\begin{inparaenum}[\em a\rm )]
\item the integration of the user interfaces (as reported previously) and
\item the integration of the knowledge/computation management services.
\end{inparaenum}
Here we report on progress in both; recall that MMT is the knowledge management service behind MathHub (and more generally for the Math-in-the-Middle based system integration; see OpenDreamKit Deliverable D6.5~\cite{ODK-D6.5}).
%
For the service integration we present an MMT kernel for Jupyter.
%
\ednote{specify what Jupyter widgets are; NT: you may want to reuse some of the language of the D4.16 report, around l21 of https://github.com/OpenDreamKit/OpenDreamKit/blob/master/WP4/D4.16/report.tex}
%
Reciprocally, for the user interface integration, we show how the Jupyter widgets can be deeply integrated within the MMT knowledge management facilities to give semantics-aware interaction facilities, extending the front-end capabilities of MathHub/Jupyter Notebooks by semantic widgets driven by the MMT in-document knowledge management services.

We show and evaluate the integration on two case studies: in-document computing facilities in active documents and a knowledge-based specification dialog for modeling and simulation. 

This report is structured as follows. In Section~\ref{sec:mmt-jp} we report on the MathHub/Jupyter integration at the system level: a Jupyter server as part of the MathHub system and a MMT kernel for Jupyter. Section~\ref{sec:nb-mh} presents the integration of Jupyter Notebooks as active documents in the (new) MathHub front-end, and Section~\ref{sec:mitm-nb} presents the two case studies. Section~\ref{sec:concl} concludes the report and discusses future work.

\ednote{this paragraph seems a bit out of place after the description of the structure of the document}
The goal of this report\ednote{of this deliverable?} is to integrate Jupyter notebooks into MathHub
and make them compatible with MMT, in a way that we can conveniently use 
MMT syntax in these notebooks and also a little bit of extra functionality
like e.g. the Jupyter widgets. The first step is setting up a Jupyter server,
which currently runs on \url{http://juypter.mathhub.info}. \ednote{KA: maybe show picture of it?}
For this server, we have developed a custom kernel, that forwards the input 
entered into the Jupyter notebook to the MMT backend. This then processes 
said input and sends the response back to the Jupyter frontend via the kernel.
We will cover the implementation of the Jupyter kernel and the MMT-backend,
later in this report.


\paragraph{Acknowledgements} The authors gratefully acknowledge the support of the Jupyter team and in particular the advice of Benjamin Ragan-Kelly. Also, the input of Theresa Pollinger and her work on the MoSIS system~\cite{PolKohKoe:kacse18} has shaped our perception of the integration reported here. 

%%% Local Variables:
%%% mode: latex
%%% mode: visual-line
%%% fill-column: 5000
%%% TeX-master: "report"
%%% End:

\section{The \ODK project}\label{sec:odk}

The project ``Open Digital Research Environment Toolkit for the Advancement of
Mathematics'''~\cite{OpenDreamKit:on} is a European H2020 project funded under the
EINFRA-9 call~\cite{EINFRA-9}. 

The \ODK consortium consists of core European developers of the aforementioned systems
for pure mathematics, and reaching toward the numerical community, and in particular the
\Jupyter community, to work together on joint needs.

The project aims to improve the productivity of researchers in pure mathematics and
applications by promoting collaborations on \emph{Data}, \emph{Knowledge}, and
\emph{Software}, to make it easy for teams of researchers of any size to set up custom,
\emph{Virtual Research Environments} tailored to their specific needs, and
to support the entire life-cycle of computational work in mathematical research, from
\emph{initial exploration} to \emph{publication}, \emph{teaching}, and \emph{outreach}.

The acceptance of the proposal~\cite{ODKproposal:on} in May 2015 was a strong sign of
recognition, at the highest level of funding agencies, of the values of open science and
the strength and maturity of the ecosystem.

The \ODK projects~\cite{ODKproposal:on} will run for four years, starting from September
2015. It involves about 50 people spread over 15 sites in Europe, with a total budget of
about 7.6 million euros.
The \ODK work plan consists of 58 concrete tasks split in seven work packages. The goals
include creating a component architecture, user interfaces, making high performance
computing accessible, interoperation with a wealth of data sources, and social aspects
of collaborative research.

%\ednote{R3: The technical and formal sections of the paper seem quite well put together.  
%The exposition is clear, and the topic is relevant to CICM.  My only objection was to a 
%couple of early sections that give high-level details on the OpenDreamKit project.  
%For this reviewer, these sections were not high-level enough.  I would recommend to 
%summarise the OpenDreamKit project much more briefly in these early sections, which 
%could save 1 or 2 pages which could then be used to expand the technical and formal 
%points in the rest of the paper.}
%\ednote{R3: After reading Section 2, "The OpenDreamKit Project", I wonder how much of 
%this background could be skipped or compressed so that we could get on with Section 3.  
%It seems the key idea needed is "VRE".  I'm not sure that Section 2 provides adequate 
%background about what you mean by that.}


%%% Local Variables:
%%% mode: latex
%%% TeX-master: "paper"
%%% End:

%  LocalWords:  specialized Arxiv Jupyter IPython ldots compactitem emph compactenum odk
%  LocalWords:  ODKproposal organization standardization visualization citability oldpart
%  LocalWords:  organizing Swinnerton-Dyer resentation desingularisation Hironaka ednote
%  LocalWords:  Hironaka algorithmisation Villamayor

When integrating multiple systems we are mostly talking about using concrete algorithms
(implemented by these systems) to solve specific computational problems (the knowledge
about the problem). To integrate multiple systems with this knowledge we want to enable
users to write down a problem in one system and then solve it in another system. We want
to be independent of the implementation of the knowledge -- independent of the systems.

For this we make use of an approach we call ``Math-In-The-Middle'' paradigm
(see~\cite{DehKohKon:iop16} for details). Here the underlying mathematical knowledge, the
``real math'', is used as a reference ontology for system (in the ``middle'') -- hence the
name. Each system needs access to this knowledge. As each of them come with their own
particularities, they will need some interface to it.

We want to make use of the modular approach to mathematics provided by theory graphs, and
in particular \MMT as an implementation thereof, to first of all allow us translate
mathematical expressions between systems. We define a ``Math In The Middle'' theory as
well as interface theories for each system. With the help of \MMT and bi-views\footnote{A
  bi-view is a bidirectional view between two theories. } between the interface theories and
the central theory, we can translate objects from one system to the other.

\begin{figure}[ht]\centering
  \def\myxscale{3}\def\myyscale{1.2}
  \documentclass{standalone}
\usepackage[mh]{mikoslides}
% this file defines root path local repository
\defpath{MathHub}{/Users/kohlhase/localmh/MathHub}
\mhcurrentrepos{MiKoMH/talks}
\libinput{WApersons}
% we also set the base URI for the LaTeXML transformation
\baseURI[\MathHub{}]{https://mathhub.info/MiKoMH/talks}

\usetikzlibrary{backgrounds,shapes,fit,shadows,mmt}
\begin{document}
\begin{tikzpicture}[xscale=2.6,yscale=.9]
  \tikzstyle{withshadow}=[draw,drop shadow={opacity=.5},fill=white]
   \tikzstyle{database} = [cylinder,cylinder uses custom fill,
      cylinder body fill=yellow!50,cylinder end fill=yellow!50,
      shape border rotate=90,
      aspect=0.25,draw]
   \tikzstyle{human} = [red,dashed,thick]
   \tikzstyle{machine} = [green,dashed,thick]

\node[thy]  (mf) at (.2,5.3) {MathF};
\node[thy,dashed]  (compf) at (0,6) {CompF};
\node[thy,dashed]  (pf) at (-.9,5.5) {PyF};
\node[thy,dashed]  (cf) at (1,5.5) {C\textsuperscript{++}F};
\node[thy,dashed]  (sf) at (-0.9,4.6) {SAGE};
\node[thy,dashed]  (gf) at (1,4.6) {GAP};

\draw[include] (compf) -- (pf);
\draw[includeleft] (compf) -- (cf);
\draw[include] (pf) -- (sf);
\draw[includeleft] (cf) -- (gf);

\node[thy] (kec) at (0,3) {EC};
\node[thy,minimum height=.4cm] (kl) at (0,4) {\ldots};

\node[thy] (sec) at (-1,2) {SEC};
\node[thy,minimum height=.4cm] (sl) at (-1,3) {\ldots};

\node[thy] (gec) at (1,2) {GEC};
\node[thy,minimum height=.4cm] (gl) at (1,3) {\ldots};

\node[thy] (lec) at (-.3,1.2) {LEC};
\node[thy,minimum height=.4cm] (ll) at (.3,1.2) {\ldots};

\node (sc) at (-2,4) {SAGE};
\node[draw] (salg) at (-2,3.35) {Algo};
\node[database,dashed] (sdb) at (-2,2.4) {DB?};
\node[draw] (skr) at (-2,1.7) {KR};
\node[draw,machine] (sac) at (-2,1) {AbsClass};

\node (gc) at (2,4) {GAP};
\node[draw] (galg) at (2,3.35) {Algo};
\node[database,dashed] (gdb) at (2,2.4) {DB?};
\node[draw] (gkr) at (2,1.7) {KR};
\node[draw,machine] (gac) at (2,1) {AbsClass};

\node (lmfdb) at (0,0) {LMFDB};
\node[database] (ldb) at (1,-.4) {Mongo};
\node[draw] (knowls) at (-1,-.4) {Knowls};
\node[draw,machine] (lac) at (0,-.5) {AbsClass};

  \begin{pgfonlayer}{background}
    \node[draw,cloud,fit=(sec) (sl),aspect=.4,inner sep=-3pt,withshadow,purple!30] (st) {};
    \node[draw,cloud,fit=(gec) (gl),aspect=.4,inner sep=-4pt,withshadow,purple!30] (gt) {};
    \node[draw,cloud,fit=(kec) (kl),aspect=.4,inner sep=0pt,withshadow,blue!30] (kt) {};
    \node[draw,cloud,fit=(lec) (ll),aspect=2.5,inner sep=-7pt,withshadow,purple!30] (lt) {};
  \end{pgfonlayer}

\begin{pgfonlayer}{background}
  \node[draw,withshadow,fit=(sc) (skr) (sac) (sdb),inner sep=1pt] {};
  \node[draw,withshadow,fit=(gc) (gkr) (gac) (gdb),inner sep=1pt] {};
  \node[draw,withshadow,fit=(lmfdb) (lac) (ldb) (knowls),inner sep=1pt] {};
\end{pgfonlayer}

\draw[view] (kec) -- (sec);
\draw[view] (kec) -- (gec);
\draw[view] (kec) -- (lec);
\draw[include] (kec) -- (kl);
\draw[include] (gec) -- (gl);
\draw[include] (sec) -- (sl);
\draw[include] (lec) -- (ll);
\draw[view] (kl) -- (sl);
\draw[view] (kl) -- (gl);
\draw[view] (kl) to[bend left=5] (ll);

\draw[meta] (mf)  to [bend right=10] (st);
\draw[meta] (sf) -- (st);
\draw[meta] (mf)  to [bend left=10] (gt);
\draw[meta] (gf) -- (gt);
\draw[meta] (mf) -- (kt);
\draw[meta] (compf) to[bend right=15] (kt);

\draw[human,->] (skr) -- node[above]{\scriptsize induce} (st);
\draw[human,->] (gkr) -- node[above]{\scriptsize induce} (gt);
\draw[human,->] (knowls) -- node[left,near end]{\scriptsize induce} (lt);

\draw[machine,->] (gt) to[bend right=30] node[below,near start]{\scriptsize generate} (gac);
\draw[machine,->] (st) to[bend left=30] node[below,near start]{\scriptsize generate} (sac);
\draw[human,->] (st) to[bend left=20] node[below]{\scriptsize refactor} (kt);
\draw[human,->] (gt) to[bend right=20] node[below]{\scriptsize refactor} (kt);
\draw[human,->] (lt) -- node[right]{\scriptsize refactor} (kt);
\end{tikzpicture}
\end{document}
%%% Local Variables: 
%%% mode: latex
%%% TeX-master: t
%%% End: 

  \caption{The MitM paradigm in detail. PyF, C${}^{++}$F and CompF are (basic)
    foundational theories for \python, C${}^{++}$ and a generic computational model. SEC,
    LEC and GEC are theories for \SageMath, \LMFDB and \GAP elliptic curves.}\label{fig:mitm}
\end{figure}

A sketch of the theory graph based on the example of elliptic curves can be found in
Figure~\ref{sec:mitm}. We will not go into details here but show how this architecture
integrates the \emph{Software} and \emph{Knowledge Aspects}. Clearly, the (hand-curated)
MitM ontology -- the purple cloud in the middle -- is a specification of the underlying
mathematical knowledge as an OMDoc/MMT theory graph, while the system interface theories
-- the blue clouds around it -- formally specify the names and types (i.e. the argument
patterns) and intended behaviour of the interface functions of the systems (often
semi-formally to make the MitM approach scalable). The OMDoc/MMT views -- the wavy arrows
between the theories -- are interpretation morphisms; in this particular case where they
connect the mathematical specification to the system theories, they express the
``implementation relation''. Thus the OMDoc/MMT framework already allows to integrate the
knowledge and software aspects for system interoperability.

The restriction to formalizing the signature (i.e. names and types of the interface
functions) of the systems is sufficient to ensure system interoperability; integrating the
implementations -- e.g. C\textsuperscript{++} or Python code -- into the theories would
be overkill here, since the code can only be executed by the respective systems --
i.e. \GAP or \SageMath. Therefore we will base our foundation on OMDoc/MMT theory graphs
directly rather than on an extension of OMDoc/MMT with ``biform
theories''~\cite{KohManRab:aumftg13,Farmer:btc07} as envisioned in the proposal. Biform
theories would enable (partial) verification of mathematical software systems, but this is
not on the critical path towards a mathematical VRE. The MitM paradigm constitutes a
lightweight alternative; identifying and refining it has been one of the major
achievements of the first year in \WPref{dksbases}.

\section{Presenting LMFDB data}
\subsection{Short description of LMFDB's goals and implementation}
The \emph{$L$-functions and modular forms database} is a project involving dozens of  mathematicians, who assemble computational data about $L$-functions, modular  forms and related number theoretic objects. The main output of the project is a website, hosted at \url{http://www.lmfdb.org}, that presents this data in a way that could serve as a reference for research efforts and should be accessible at the graduate student level.  The mathematical concepts underlying the \LMFDB are extremely complex and varied, so part of the effort has been focused on how to relay mathematical definitions and their relationships to data and software. For this purpose, the \LMFDB has developed so-called \emph{knowls}, which are am technical solution to present \LaTeX-encoded information interactively, heavily exploiting the concept of transclusion. The end result there is a very modular and highly interlinked set of definitions in mathematical natural language. 
 
The \LMFDB code is primarily written in \textsf{python}, with some reliance on \Sage for the business logic. The frontend is written in the web framework Flask, while the backend uses the NoSQL document database system \Mongo \cite{lmfdb-repo}. Again, due to the complexity of the objects considered, many idiosyncratic encodings are used for the data. This makes the whole data management lifecycle particularly tricky, and dependent on different select groups of individuals for each component. 

\subsection{Tentative approach to MMT semantic layers for the LMFDB}
Since the LMFDB spans the whole "vertical", from writing software, to producing new data, up to presenting this new knowledge,  it is a perfect test case for a large scale try-out of the "knowledge in the middle": a semantic layer would be beneficial to its activities across data, knowledge and software, which it would help integrate more cohesively and systematically. Among the components of the LMFDB, elliptic curves stand out in the best shape, and a source of best practices for other areas. For this reason, the \ODK collaboration targeted that relatively small subset of the LMFDB for its prototype. We plan to extend coverage in a second phase, and expect it to be relatively easy if we manage to demonstrate the added value of a semantic layer for our prototype. 

In practice, our first step was to integrate into the MMT system the mathematical knowledge that had already been made explicit in the knowls (see Fig.~\ref{stex-ec}). For this, we used the \stex interpreter, which was able to crudely type-check the definitions, avoiding circularity and ensuring some level of consistency in their scope. Knowledge translated into this format immediately became browsable through \textsf{MathHub.info}, a project developed in parallel to MMT to host such formalisations. 

\lstinputlisting[language={[sTeX]TeX},label={stex-ec},
   caption= {Sample \protect\stex implementation of \LMFDB knowl content for elliptic curves
    and their Weierstrass models.}]{examples/elliptic-curve.tex}

The second step consisted of translating these informal natural language definitions into
progressively more exhaustive MMT formalisations of mathematical concepts (see
Listing~\ref{lst:mmt-ec}). This is possibly the hardest step, as that layer will have to
eventually interoperate with the informal \stex definitions, the data that has been
previously computed, the software used to compute more, and the software needed to view all
this information. We found that the best approach was to push for "small theories", again
emphasizing modularity.

\lstinputlisting[morekeywords={namespace,theory,include},mathescape,
caption= {MMT formalisation of elliptic curves and their Weierstrass models},
label=lst:mmt-ec]{examples/elliptic-curve.mmt}


The next step was in integrating computational data into MMT. We first needed to expand
the capabilities of MMT with respect to literals. For this purpose, the MMT core team
constructed an MMT interface with the \LMFDB API, which delivered data in JSON
format. This is not fully satisfactory, and a more generic solution could be found,
interfacing for instance directly with the \Mongo database.  Once JSON documents were
obtained, we still needed to define explicit specifications for them. There are recent
ongoing efforts \cite{lmfdb-formats} to document in natural language the formats used by
the \LMFDB, and this provided an easy starting point. Actually, the scope of those
specifications had to be extended, as they were not focused only on the format used to
store the data fields (string \emph{vs.} float \emph{vs.} integer tuple, for instance),
but also the types these data fields were meant to represent (rational \emph{vs.}
polynomial, for instance). It turns out that many patterns exist there: often different
people will encode the same semantic type with the same data format, following the same
conventions. Therefore, to maximize reusability, we again emphasized modularity and
developed \emph{MMT codecs}. Codecs integrate \emph{some} information about semantics and
some about data formats, and are composable. For instance, one could compose two MMT
codecs, say \emph{polynomial-as-reversed-list} and \emph{rational-as-tuple-of-int}, to
signify that the data $[(2,3),(0,1),(4,1)]$ is meant to represent the polynomial
$4x^2+2/3$. Of course, these codecs could be further decomposed (signalling which variable
name to use, for instance). The initial costs of developing these codecs are high, but the
clarity gained in documentation is valuable, they are highly reusable, and they
drastically expand the range of tooling that can be built around data management. Our
efforts also fit neatly alongside similar efforts underway across the sciences to
standardize metadata formats (for instance through the Research Data Alliance's Typing
Registry Working Group\cite{rda-typing}).

% emphasis on typing systems, which make sense in each individual context and come with
% associated goals and processes for maintenance.  MMT as above that, by necessity

%  LocalWords:  subsubsection emph knowls textsf lmfdb-repo stex-ec stex knowl mmt-ec odk
%  LocalWords:  Weierstrass emphasizing alltt tp rightarrow vdash doteq vdash doteq lst
%  LocalWords:  polynomial_equation_injectivity a_invariants_factors_injectivity colorbox
%  LocalWords:  minimality_idempotence lmfdb-formats emphasized composable standardize
%  LocalWords:  rda-typing lstinputlisting mathescape elliptic-curve.mmt

%%% Local Variables:
%%% mode: latex
%%% TeX-master: "paper"
%%% End:

\section{Distributed Collaboration with GAP/Sage}\label{sec:gapsage}
\begin{newpart}{MK@MK: completely rewrite from a high-level perspective the material that
    is now in \texttt{deleted-scenes.tex} and should be extended to a technical paper}
Another aspect of interoperability in a mathematical VRE is the possibility of distributed
computation, where one the systems delegates a sub-computation to a specialist system or
multiple system share components of a larger computation or reasoning task.

There are already a variety of peer-to-peer integrations (see Figure~\ref{fig:interop})
based on the ``handle paradigm'' between systems in the \ODK project, e.g. between \Sage a
the master and \GAP, \Singular, and \Pari. 

In the ``handle paradigm'', when a system $A$ delegates a calculation to a system $B$, the
result $r$ of the calculation is not converted to a native $A$ object; instead $B$ just
returns a handle (or reference) to the object $r$. Later $A$ can run further calculations
with $r$ by passing it as argument to $B$ functions or methods. The advantages of this
approach include that we can avoid the overhead of back and forth conversions between $A$
and $B$ and that we can manipulate objects of $B$ from $A$ even if they have no native
representation in $A$.

Given a mapping of corresponding methods in the systems, we can use the adaptor pattern to
implement this. For example, calling the method \texttt{h.cardinality()} on a \Sage handle
\texttt{h} to a \GAP object \texttt{G}, triggers in \GAP a call to \texttt{Size(G)} if
\texttt{cardinality} and \texttt{Size} are marked as corresponding. But this dispatch
depends on an alignment of the type systems in \Sage and \GAP. For example, if \texttt{h}
is a handle to a set \texttt{S}, \Sage only knows that \texttt{h.cardinality()} can be
computed by \texttt{Size(S)} in \GAP if \texttt{S} is a group; in fact if \texttt{h} has
been constructed through the \texttt{PermutationGroup} or \texttt{MatrixGroup}
constructors. Whereas we would want this method to be available as soon as \texttt{S} is a
set.

To get around this problem we have worked on a more semantic integration, where adaptor
methods are made aware of the type hierarchies of the respective other system, see
Listing~\ref{lst:adaptor} below. But for scaling this up to a mathematical VRE with a
douzen systems, we want to use the MitM paradigm.
 
\begin{lstlisting}[language=Python,label=lst:adaptor,
  caption=A Semantic Adaptor Mehthod in \Sage]
class Sets: # Everything about sets in Sage
    class GAP: # The adapter methods relevant to Sets in the Sage-Gap interface
         class ParentMethods: # Adapter methods for sets
             def cardinality(self): # The adapter for the cardinality method
                 return self.gap().Size().sage()
         class ElementMethods: # Adapter methods for set elements
             ...
         class MorphismMethods: # Adapter methods for set morphisms
             ...
\end{lstlisting}
The main problem now becomes in generating interface theories and interviews into the core
MitM ontology so that the adaptor pattern above still applies, but can be made generic in
terms of the MitM ontology structure instead of the concrete structure of the respective
type systems. In our example, the correspondence between \texttt{cardinality} and
\texttt{Size} still holds, if the MitM interviews link the \texttt{cardinality} function
in the \Sage interface theory on sets with the \texttt{Size} function in the corresponding
interface theory for \GAP.

We will now show first results of our experiments of generating interface theories that
support distributed computation for \Sage and \GAP. 
\end{newpart}

\subsection{Semantics in the \Sage Category System}

The \Sage library includes 40k functions and allows for manipulating thousands of
different kinds of objects. As usual in such large systems, it’s critical for taming code
bloat to
\begin{inparaenum}[\em i\rm)]
\item identify the core concepts describing common behavior among the objects; 
\item exploit this to implement generic operations that apply on all object having a given
  behavior, with appropriate specializations when performance calls for it.  
\item design or choose a process for selecting the best implementation available when
  calling an operation on one or several objects.
\end{inparaenum}
Following mathematical tradition, \Sage has developed a category-theory-inspired
``category system'' and found a way to implement it in terms of the underlying \Python
object system\ednote{MK@NT; cite your paper on this here}. This cagegory system moddels
taxonomic knowledge from mathematics explicity and uses it to support genericity, control
the method selection process, structure the code and documentation, enforce consistency,
and provide generic tests.

\begin{wrapfigure}r{7cm}\vspace*{-2.5em}
\begin{lstlisting}[language=Python]
@semantic(mmt="sets")
class Sets:
    class ParentMethods:
         @semantic(mmt="o", gap="Size")
         @abstractmethod
         def cardinality(self):
             r"""
             Return the cardinality of ``self``.
             """
\end{lstlisting}
\vspace*{-.5em}
\caption{An anotated Category in \Sage}\label{fig:anncat}\vspace*{-1.5em}
\end{wrapfigure}
To generate interface theories from the \Sage category system, we are experimenting with a
system of annotations in the \Sage source files. Consider for instance the situtation in
Figure~\ref{fig:anncat} where we have annotated the \texttt{Magmas()} category in \Sage
with \texttt{@semantic} lines that state correspondences to other interface theories. From
these the \Sage-to-MMT exporter can generate the respective interface theories and views.

Several variants of the annotations exist to allow for adding annotations on existing
categories without touching their file, and also for specifying directly the corresponding
method names in other systems when this has not yet been formalized elsewhere. Similarly,
one could provide directly the signature information in case that is not yet modelled in
MMT.

\subsection{Exporting the \GAP Type System}
\ednote{summarize the \GAP stuff here, also add the picture}

%%% Local Variables:
%%% mode: latex
%%% TeX-master: "paper"
%%% End:

  We have implemented the MitM approach to integrating mathematical software systems based on formalizations of the underlying mathematical knowledge.
  The main investment here was the curation of an MitM Ontology, the generation of formal specifications of system APIs for \Sage, \GAP, and \Singular, identifying the alignments of these APIs with the ontology, implementing an MitM server that can use alignments to translate between systems, and implementing the \SCSCP protocol for all involved systems.

  We have also shown how to extend the Math-in-the-Middle framework for integrating systems to mathematical data bases like the \lmfdb. 
The main idea is to embed knowledge sources as virtual theories, i.e. theories that are not -- theoretically or in practice -- limited in the number of declarations and allow dynamic loading and processing. 
For accessing real-world knowledge sources, we have developed the notion of codecs and integrated them into the MitM ontology framework. 
These codecs (and their MitM types) lift knowledge source access to the MitM level and thus enable object-level interoperability and allow humans (mathematicians) access using the concepts they are familiar with. 
Finally, we have shown a prototypical query translation facility that allows to delegate some of the processing to the underlying knowledge source and thus avoid thrashing of virtual theories. 

\paragraph{Related Work} Most other integration schemes employ a \textbf{homogenous approach}, where there is a master system and all data is converted into that system. 
A paradigmatic example of this is the Wolfram Language~\cite{WolframLanguage:wikipedia} and the Wolfram Alpha search engine~\cite{WolframAlpha:on}, which are based on the Mathematica kernel. 
This is very flexible for anyone owning a Mathematica license and experienced in the Mathematica language and environment.

The MitM-based approach to interoperability of data sources and systems proposed in this paper is inherently a \textbf{heterogeneous approach}: systems and data sources are kept ``as is'', but their APIs are documented in a machine-actionable way that can be utilized for remote procedure calls, content format mediation, and service discovery. 
As a consequence, interaction between systems is very flexible.
For the data source integration via virtual theories presented in this paper this is important. 
For instance, we can just make an extension of \mmt or \Sage\ which just act as a programmatic interface for e.g. \lmfdb. 

Our case studies show that MitM-based integration is an achievable goal.
Delegation-based workflows can either be programmed directly or embedded into the interaction language of the mathematical software systems.

The main advantages and challenges claimed by the MitM framework come from its loosely coupled and knowledge-based nature.
Compared to ad-hoc translations, MitM-based interoperability is relatively expensive as objects have to be serialized into (possibly large) \OMMT objects, transferred via \SCSCP to \MMT, parsed, translated into another system dialect, serialized and transferred, and parsed again.
On the other hand, instead of implementing and maintaining $n^2$ translations, we only have to establish and maintain $n$ collections of system APIs and their alignments to the
MitM ontology.
This makes the management of interoperability much more tractable:
\begin{compactenum}
\item The MitM ontology is developed and maintained as a shared resource by the community.
We expect it to be well-maintained, since it can directly be used as a documentation of the functionality of the respective systems.
\item All the workflows are star-shaped: instead of requiring expert knowledge in two systems -- a rare commodity even in open-source projects, and even for the system experts involved in this \papertype -- and keeping up with their changes, the MitM approach only needs expertise and change management for single systems.
\end{compactenum}
All in all, these translate into a ``business model'' for MitM-based cooperation in terms of the necessary investment and achievable results, which is based on the well-known \emph{network effects}: the joining costs are in the size of the respective system, whereas the rewards -- i.e. the functionality available by delegation -- is in the size of the network.

This network effect can be enhanced by technical refinements we are currently studying:
For instance, if we annotate alignments with a ``priority'' value that specifies how canonically/efficiently/powerfully a given system implements a given MitM operation, then we can let
the \MMT mediator automatically choose a suitable target system for a requested computation (as opposed to our current setup where Jane specifies which systems she wants to use). On the other hand, for workflows where we do not need or want service-discovery, alignments can be ``compiled'' into $n^2$ transport-efficient direct translations that may even eliminate the need for serialization and parsing.

\paragraph{Future Work}\ednote{MK: this is essentially future work only for LMFDB, we need future work also for MitM here.}
\ednote{MK@FR: could you please describe the MMT Python bridge and how this could be used for SCSCP-less communication with Sage. The MitM-work here is to allow for compilation of the alignment based translations into code.}
We have discussed the MitM+virtual theories methodology on the elliptic curves sub-base of the \lmfdb, which we have fully integrated. 
We are currently working on additional \lmfdb sub-bases. 
The main problem to be solved is to elicit the information for the respective schema theories from the \lmfdb community. 
Once that is accomplished, specifying them in the format discussed in this paper and writing the respective codecs is straightforward. 

Moreover, we are working on integrating the the Online Encyclopedia of Integer Sequences (OEIS~\cite{Sloane:OEIS,oeis}). 
Here we have a different problem: the OEIS database is essentially a flat ASCII file with different slots (for initial segments of the sequences, references, comments, and formulae); all minimally marked up ASCII art. 
In~\cite{LuzKoh:fsarfo16} we have already (heuristically) flexiformalized OEIS contents in \ommt; the next step will be to come up with codecs based on this basis and develop schema theories for OEIS.


\subsubsection*{Acknowledgements}
The authors gratefully acknowledge the fruitful discussions with other participants of
work package WP6, in particular Alexander Konovalov on \SCSCP, Paul Dehaye on the \Sage
export and the organization of the MitM ontology, Luca de Feo on OpenMath phrasebooks
and the \SCSCP library in python, and David Lowry-Duda\ednote{MK: or make him a co-author?  JC: We should give him the option.}

We acknowledge financial support from the OpenDreamKit Horizon 2020 European Research
Infrastructures project (\#676541) and DFG project RA-18723-1 OAF.

%%% Local Variables:
%%% mode: latex
%%% mode: visual-line
%%% fill-column: 5000
%%% TeX-master: "report"
%%% End:

%  LocalWords:  sec:concl MitM-based itemize subsubsection Dehaye organization serialized
%  LocalWords:  math-savy emph serialization formalizations


\printbibliography
\end{document}
%%% Local Variables:
%%% mode: latex
%%% TeX-master: t
%%% End:

%  LocalWords:  maketitle endeavor ednote ODKproposal printbibliography subsubsection pn
%  LocalWords:  IPython Jupyter emph itemize specialized Arxiv oldpart organization ec
%  LocalWords:  citability github Swinnerton-Dyer resentation desingularisation Hironaka
%  LocalWords:  Hironaka algorithmisation Villamayor visualization lstinline omtext lec
%  LocalWords:  multisystem-semantic-handle-interfaces.tex presenting-lmfdb.tex Dehaye
%  LocalWords:  Mihnea Iancu Konovalov Leli evre Wiesing odk mitm presenting-lmfdb concl
%  LocalWords:  multisystem-semantic-handle-interfaces
