\documentclass{llncs}
\pagestyle{plain}
\usepackage[show]{ed}
\usepackage[utf8]{inputenc}
\usepackage{xspace}
\usepackage[style=alphabetic,backend=bibtex,isbn=false]{biblatex}
\addbibresource{../../lib/kbibs/kwarcpubs.bib}
\addbibresource{../../lib/kbibs/extpubs.bib}
\addbibresource{../../lib/kbibs/kwarccrossrefs.bib}
\addbibresource{../../lib/kbibs/extcrossrefs.bib}
\addbibresource{rest.bib}% add bibs here!
\renewbibmacro*{event+venue+date}{}
\renewbibmacro*{doi+eprint+url}{%
  \iftoggle{bbx:doi}
    {\printfield{doi}\iffieldundef{doi}{}{\clearfield{url}}}
    {}%
  \newunit\newblock
  \iftoggle{bbx:eprint}
    {\usebibmacro{eprint}}
    {}%
  \newunit\newblock
  \iftoggle{bbx:url}
    {\usebibmacro{url+urldate}}
    {}}

\usepackage{hyperref}
\title{REGULAR-T1: Knowledge-Based Interoperability for Mathematical Software Systems}
\author{
Victor Aether\inst{2} 
 Michael Kohlhase\inst{1} 
Dennis M\"uller\inst{1} 
Markus Pfeiffer\inst{2} 
Florian Rabe\inst{2} 
Nicolas~M.~Thiéry\inst{3} 
Tom Wiesing\inst{2}
}

\institute{
   FAU Erlangen-N\"urnberg
   \and University of St~Andrews 
   \and Universit\'e Paris-Sud
}
\begin{document}
\maketitle
\begin{abstract}
  There is a large, open-source ecosystem of mathematical software systems that collect and
  classify, compute with, prove statements about, and visualize mathematical objects and
  models. Individually, these systems are optimized for particular domains and
  functionalities and together they cover many, but no system covers all needs of
  practical and theoretical mathematics. System integrations exist, but are ad-hoc and
  have scalability and maintainability issues. In particular, there is not yet an
  interoperability layer that combines them into a virtual research environment (VRE) for
  mathematics.
  
  The OpenDreamKit project, which aims at a mathematical VRE toolkit, proposes the
  Math-in-the-Middle (MitM) Paradigm, an interoperability framework based on a flexiformal
  representation of mathematical knowledge and aligns this with system-generated interface
  theories. In this paper we instantiate the MitM paradigm with a concrete domain
  development and evaluate it on a distributed computing case study involving GAP,
  SageMath and Singular. 
\end{abstract}

\section{Introduction}\label{sec:intro}

\begin{newpart}{MK: adapted from Tom's Thesis}
There is a large and vibrant ecosystem of open-source mathematical software systems.
These systems can range from calculators, which are only capable of performing simple
computations, via mathematical databases (curating collections of a mathematical objects)
to powerful modeling tools and computer algebra systems (CAS). 

Most of these systems are very specific -- they focus on one or very few aspects of
mathematics.  For example, the ``Online Encyclopedia of Integer Sequences''
(OEIS~\cite{Sloane:oeis12,oeis}) focuses on sequences over $\mathbb{Z}$ an their
properties and the ``L-Functions and Modular Forms Database''
(LMFDB)~\cite{Cremona:LMFDB16,lmfdb:on} objects in number theory pertaining to Langland's
program.  GAP~\cite{GAP:on} excels at discrete algebra, whereas
SageMath~\cite{SageMath:on} focuses on Algebra and Geometry in general, and
Singular~\cite{singular:on} on polynomial computations, with special emphasis on
commutative and non-commutative algebra, algebraic geometry, and singularity theory.

For a mathematician however (a user; let us call her Jane) the systems themselves are not relevant, instead she only cares about being able to solve problems. 
Typically, it is not possible to solve a mathematical problem using only a single problem. 
Thus Jane needs to work with multiple systems and combine the results to reach a solution. 
Currently there is very little help with this practice, so Jane has to isolate sub-problems the respective systems are amenable to, formulate them into the respective input language, collect results, and reformulate them for the next system a tedious and error-prone process at best, a significant impediment to scientific progress in its overall effect. 
Solutions for some situations certainly exist, which can help get Jane unstuck, but these are ad-hoc and for specific, often-used system combinations only. 
Each of these requires a lot of maintenance and does not scale to a larger set of specialist systems. 

The OpenDreamKit project, which aims at a mathematical VRE toolkit, proposes the Math-in-the-Middle (MitM~\cite{DehKohKon:iop16}) Paradigm, an interoperability framework based on a flexiformal
representation of mathematical knowledge and aligns this with system-generated interface
theories. 

In this paper we instantiate the MitM paradigm with a concrete domain development and
evaluate it on a distributed computing GAP, SageMath and Singular.\ednote{ we generally we
  want to show that the promises in the CICM paper become reality.}

 \ednote{MK: continue with the structure} 
\end{newpart}

\begin{todolist}{the structure should follow this}
\item show the generated system ontologies taking GAP and SageMath as examples (and
  compare them)
\item with the example of GAP group theory make the MitM ontology part , in the levels
  (abstract, concrete subgroup theory, computational), show 
  \begin{itemize}
  \item star-formed alignments and how we come by them, 
  \item talk about SCSCP and GAP/Sage/ dialects and intra-MMT translation
  \item running example use case is to act on singular polynomials with GAP permutation
    groups
  \end{itemize}
\item what is the ``business model'' for the general MitM-based cooperation model? 
\end{todolist}

\section{Conclusion}\label{sec:concl}
\ednote{how can YOU help extend the MitM?}\ednote{investment: small network joining
  costs in the size of the system, rewards: network effect in the size of the network.}
\ednote{details in ~\cite{twiesing:msc17}}
\printbibliography
\end{document}
%%% Local Variables:
%%% mode: latex
%%% TeX-master: t
%%% End:

%  LocalWords:  maketitle twb sec:intro DehKohKon:iop16 MueGauKal:cacfms17 optimized
%  LocalWords:  itemize MitM-based sec:concl twiesing:msc17 printbibliography
