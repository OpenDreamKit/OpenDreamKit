\documentclass{llncs}
\pagestyle{plain}
\usepackage[show]{ed}
\usepackage[utf8]{inputenc}
\usepackage{xspace}
\usepackage{amsfonts}
\usepackage[style=alphabetic,backend=bibtex,isbn=false]{biblatex}
\usepackage{tikz}
\addbibresource{../../lib/kbibs/kwarcpubs.bib}
\addbibresource{../../lib/kbibs/extpubs.bib}
\addbibresource{../../lib/kbibs/kwarccrossrefs.bib}
\addbibresource{../../lib/kbibs/extcrossrefs.bib}
\addbibresource{rest.bib}% add bibs here!
\renewbibmacro*{event+venue+date}{}
\renewbibmacro*{doi+eprint+url}{%
  \iftoggle{bbx:doi}
    {\printfield{doi}\iffieldundef{doi}{}{\clearfield{url}}}
    {}%
  \newunit\newblock
  \iftoggle{bbx:eprint}
    {\usebibmacro{eprint}}
    {}%
  \newunit\newblock
  \iftoggle{bbx:url}
    {\usebibmacro{url+urldate}}
    {}}

\usepackage{hyperref}
\title{REGULAR-T1: Knowledge-Based Interoperability for Mathematical Software Systems}
\author{
Michael Kohlhase\inst{1} 
Dennis M\"uller\inst{1} 
Markus Pfeiffer\inst{3} 
Florian Rabe\inst{2} 
Nicolas~M.~Thiéry\inst{4} 
Victor Vasilyev\inst{3} 
Tom Wiesing\inst{1}
}

\institute{
   FAU Erlangen-N\"urnberg
   \and Jacobs University Bremen
   \and University of St~Andrews 
   \and Universit\'e Paris-Sud
}
\begin{document}
\maketitle
\begin{abstract}
  There is a large, open-source ecosystem of mathematical software systems that collect and
  classify, compute with, prove statements about, and visualize mathematical objects and
  models. Individually, these systems are optimized for particular domains and
  functionalities and together they cover many, but no system covers all needs of
  practical and theoretical mathematics. System integrations exist, but are ad-hoc and
  have scalability and maintainability issues. In particular, there is not yet an
  interoperability layer that combines them into a virtual research environment (VRE) for
  mathematics.
  
  The OpenDreamKit project, which aims at a mathematical VRE toolkit, proposes the
  Math-in-the-Middle (MitM) Paradigm, an interoperability framework based on a flexiformal
  representation of mathematical knowledge and aligns this with system-generated interface
  theories. In this paper we instantiate the MitM paradigm with a concrete domain
  development and evaluate it on a distributed computing case study involving GAP,
  SageMath and Singular. 
\end{abstract}

In this report we present a prototypical integration of the Jupyter notebooks into the MathHub.info portal for active mathematical documents and a versioned hosting system for flexiformal mathematics.
MathHub.info offers a rich interface for reading, writing, and interacting with mathematical documents and knowledge. Jupyter offers a uniform interface to the computation facilities of the OpenDreamKit VRE toolkit in the form of a read-eval-print loop (REPL).

A mathematical Virtual Research Environment (VRE) needs both kinds of interface functionality: mathematical documents have been very successful for presenting mathematical knowledge, and while there have been efforts to make them modular and interactive they predominantly remain in the mode of archiving and transporting knowledge in Mathematics.
Notebook interfaces also use the document metaphor at the surface; however the REPL interaction
tends to take structural precedence, leading to documents consisting of a sequence of computational cells within which the mathematical discourse is interspersed in the form of ``rich comments''.

A ``literate computing'' version of notebooks which gives mathematical discourse structural precedence is possible in principle, but has not been supported consistently at the system level.\ednote{MK: put the following sentence somewhere: A ``literate programming'' version of notebooks which gives mathematical discourse structural precedence is possible in principle, but has not been supported consistently at the system level.}
This tension and trade-off has been explored in OpenDreamKit Deliverable D4.2~\cite{ODK-D4.2}, and the concept of in-document computation in OpenDreamKit Deliverable D4.9~\cite{ODK-D4.9}.
In both cases, the integration was incomplete, since it lacked a full integration of the
underlying knowledge/computation services.

Generally, the integration of MathHub and Jupyter consists of two parts:
\begin{inparaenum}[\em a\rm )]
\item the integration of the user interfaces (as reported previously) and
\item the integration of the knowledge/computation management services.
\end{inparaenum}
Here we report on progress in both; recall that MMT is the knowledge management service behind MathHub (and more generally for the Math-in-the-Middle based system integration; see OpenDreamKit Deliverable D6.5~\cite{ODK-D6.5}).
%
For the service integration we present an MMT kernel for Jupyter.
%
\ednote{specify what Jupyter widgets are; NT: you may want to reuse some of the language of the D4.16 report, around l21 of https://github.com/OpenDreamKit/OpenDreamKit/blob/master/WP4/D4.16/report.tex}
%
Reciprocally, for the user interface integration, we show how the Jupyter widgets can be deeply integrated within the MMT knowledge management facilities to give semantics-aware interaction facilities, extending the front-end capabilities of MathHub/Jupyter Notebooks by semantic widgets driven by the MMT in-document knowledge management services.

We show and evaluate the integration on two case studies: in-document computing facilities in active documents and a knowledge-based specification dialog for modeling and simulation. 

This report is structured as follows. In Section~\ref{sec:mmt-jp} we report on the MathHub/Jupyter integration at the system level: a Jupyter server as part of the MathHub system and a MMT kernel for Jupyter. Section~\ref{sec:nb-mh} presents the integration of Jupyter Notebooks as active documents in the (new) MathHub front-end, and Section~\ref{sec:mitm-nb} presents the two case studies. Section~\ref{sec:concl} concludes the report and discusses future work.

\ednote{this paragraph seems a bit out of place after the description of the structure of the document}
The goal of this report\ednote{of this deliverable?} is to integrate Jupyter notebooks into MathHub
and make them compatible with MMT, in a way that we can conveniently use 
MMT syntax in these notebooks and also a little bit of extra functionality
like e.g. the Jupyter widgets. The first step is setting up a Jupyter server,
which currently runs on \url{http://juypter.mathhub.info}. \ednote{KA: maybe show picture of it?}
For this server, we have developed a custom kernel, that forwards the input 
entered into the Jupyter notebook to the MMT backend. This then processes 
said input and sends the response back to the Jupyter frontend via the kernel.
We will cover the implementation of the Jupyter kernel and the MMT-backend,
later in this report.


\paragraph{Acknowledgements} The authors gratefully acknowledge the support of the Jupyter team and in particular the advice of Benjamin Ragan-Kelly. Also, the input of Theresa Pollinger and her work on the MoSIS system~\cite{PolKohKoe:kacse18} has shaped our perception of the integration reported here. 

%%% Local Variables:
%%% mode: latex
%%% mode: visual-line
%%% fill-column: 5000
%%% TeX-master: "report"
%%% End:

When integrating multiple systems we are mostly talking about using concrete algorithms
(implemented by these systems) to solve specific computational problems (the knowledge
about the problem). To integrate multiple systems with this knowledge we want to enable
users to write down a problem in one system and then solve it in another system. We want
to be independent of the implementation of the knowledge -- independent of the systems.

For this we make use of an approach we call ``Math-In-The-Middle'' paradigm
(see~\cite{DehKohKon:iop16} for details). Here the underlying mathematical knowledge, the
``real math'', is used as a reference ontology for system (in the ``middle'') -- hence the
name. Each system needs access to this knowledge. As each of them come with their own
particularities, they will need some interface to it.

We want to make use of the modular approach to mathematics provided by theory graphs, and
in particular \MMT as an implementation thereof, to first of all allow us translate
mathematical expressions between systems. We define a ``Math In The Middle'' theory as
well as interface theories for each system. With the help of \MMT and bi-views\footnote{A
  bi-view is a bidirectional view between two theories. } between the interface theories and
the central theory, we can translate objects from one system to the other.

\begin{figure}[ht]\centering
  \def\myxscale{3}\def\myyscale{1.2}
  \documentclass{standalone}
\usepackage[mh]{mikoslides}
% this file defines root path local repository
\defpath{MathHub}{/Users/kohlhase/localmh/MathHub}
\mhcurrentrepos{MiKoMH/talks}
\libinput{WApersons}
% we also set the base URI for the LaTeXML transformation
\baseURI[\MathHub{}]{https://mathhub.info/MiKoMH/talks}

\usetikzlibrary{backgrounds,shapes,fit,shadows,mmt}
\begin{document}
\begin{tikzpicture}[xscale=2.6,yscale=.9]
  \tikzstyle{withshadow}=[draw,drop shadow={opacity=.5},fill=white]
   \tikzstyle{database} = [cylinder,cylinder uses custom fill,
      cylinder body fill=yellow!50,cylinder end fill=yellow!50,
      shape border rotate=90,
      aspect=0.25,draw]
   \tikzstyle{human} = [red,dashed,thick]
   \tikzstyle{machine} = [green,dashed,thick]

\node[thy]  (mf) at (.2,5.3) {MathF};
\node[thy,dashed]  (compf) at (0,6) {CompF};
\node[thy,dashed]  (pf) at (-.9,5.5) {PyF};
\node[thy,dashed]  (cf) at (1,5.5) {C\textsuperscript{++}F};
\node[thy,dashed]  (sf) at (-0.9,4.6) {SAGE};
\node[thy,dashed]  (gf) at (1,4.6) {GAP};

\draw[include] (compf) -- (pf);
\draw[includeleft] (compf) -- (cf);
\draw[include] (pf) -- (sf);
\draw[includeleft] (cf) -- (gf);

\node[thy] (kec) at (0,3) {EC};
\node[thy,minimum height=.4cm] (kl) at (0,4) {\ldots};

\node[thy] (sec) at (-1,2) {SEC};
\node[thy,minimum height=.4cm] (sl) at (-1,3) {\ldots};

\node[thy] (gec) at (1,2) {GEC};
\node[thy,minimum height=.4cm] (gl) at (1,3) {\ldots};

\node[thy] (lec) at (-.3,1.2) {LEC};
\node[thy,minimum height=.4cm] (ll) at (.3,1.2) {\ldots};

\node (sc) at (-2,4) {SAGE};
\node[draw] (salg) at (-2,3.35) {Algo};
\node[database,dashed] (sdb) at (-2,2.4) {DB?};
\node[draw] (skr) at (-2,1.7) {KR};
\node[draw,machine] (sac) at (-2,1) {AbsClass};

\node (gc) at (2,4) {GAP};
\node[draw] (galg) at (2,3.35) {Algo};
\node[database,dashed] (gdb) at (2,2.4) {DB?};
\node[draw] (gkr) at (2,1.7) {KR};
\node[draw,machine] (gac) at (2,1) {AbsClass};

\node (lmfdb) at (0,0) {LMFDB};
\node[database] (ldb) at (1,-.4) {Mongo};
\node[draw] (knowls) at (-1,-.4) {Knowls};
\node[draw,machine] (lac) at (0,-.5) {AbsClass};

  \begin{pgfonlayer}{background}
    \node[draw,cloud,fit=(sec) (sl),aspect=.4,inner sep=-3pt,withshadow,purple!30] (st) {};
    \node[draw,cloud,fit=(gec) (gl),aspect=.4,inner sep=-4pt,withshadow,purple!30] (gt) {};
    \node[draw,cloud,fit=(kec) (kl),aspect=.4,inner sep=0pt,withshadow,blue!30] (kt) {};
    \node[draw,cloud,fit=(lec) (ll),aspect=2.5,inner sep=-7pt,withshadow,purple!30] (lt) {};
  \end{pgfonlayer}

\begin{pgfonlayer}{background}
  \node[draw,withshadow,fit=(sc) (skr) (sac) (sdb),inner sep=1pt] {};
  \node[draw,withshadow,fit=(gc) (gkr) (gac) (gdb),inner sep=1pt] {};
  \node[draw,withshadow,fit=(lmfdb) (lac) (ldb) (knowls),inner sep=1pt] {};
\end{pgfonlayer}

\draw[view] (kec) -- (sec);
\draw[view] (kec) -- (gec);
\draw[view] (kec) -- (lec);
\draw[include] (kec) -- (kl);
\draw[include] (gec) -- (gl);
\draw[include] (sec) -- (sl);
\draw[include] (lec) -- (ll);
\draw[view] (kl) -- (sl);
\draw[view] (kl) -- (gl);
\draw[view] (kl) to[bend left=5] (ll);

\draw[meta] (mf)  to [bend right=10] (st);
\draw[meta] (sf) -- (st);
\draw[meta] (mf)  to [bend left=10] (gt);
\draw[meta] (gf) -- (gt);
\draw[meta] (mf) -- (kt);
\draw[meta] (compf) to[bend right=15] (kt);

\draw[human,->] (skr) -- node[above]{\scriptsize induce} (st);
\draw[human,->] (gkr) -- node[above]{\scriptsize induce} (gt);
\draw[human,->] (knowls) -- node[left,near end]{\scriptsize induce} (lt);

\draw[machine,->] (gt) to[bend right=30] node[below,near start]{\scriptsize generate} (gac);
\draw[machine,->] (st) to[bend left=30] node[below,near start]{\scriptsize generate} (sac);
\draw[human,->] (st) to[bend left=20] node[below]{\scriptsize refactor} (kt);
\draw[human,->] (gt) to[bend right=20] node[below]{\scriptsize refactor} (kt);
\draw[human,->] (lt) -- node[right]{\scriptsize refactor} (kt);
\end{tikzpicture}
\end{document}
%%% Local Variables: 
%%% mode: latex
%%% TeX-master: t
%%% End: 

  \caption{The MitM paradigm in detail. PyF, C${}^{++}$F and CompF are (basic)
    foundational theories for \python, C${}^{++}$ and a generic computational model. SEC,
    LEC and GEC are theories for \SageMath, \LMFDB and \GAP elliptic curves.}\label{fig:mitm}
\end{figure}

A sketch of the theory graph based on the example of elliptic curves can be found in
Figure~\ref{sec:mitm}. We will not go into details here but show how this architecture
integrates the \emph{Software} and \emph{Knowledge Aspects}. Clearly, the (hand-curated)
MitM ontology -- the purple cloud in the middle -- is a specification of the underlying
mathematical knowledge as an OMDoc/MMT theory graph, while the system interface theories
-- the blue clouds around it -- formally specify the names and types (i.e. the argument
patterns) and intended behaviour of the interface functions of the systems (often
semi-formally to make the MitM approach scalable). The OMDoc/MMT views -- the wavy arrows
between the theories -- are interpretation morphisms; in this particular case where they
connect the mathematical specification to the system theories, they express the
``implementation relation''. Thus the OMDoc/MMT framework already allows to integrate the
knowledge and software aspects for system interoperability.

The restriction to formalizing the signature (i.e. names and types of the interface
functions) of the systems is sufficient to ensure system interoperability; integrating the
implementations -- e.g. C\textsuperscript{++} or Python code -- into the theories would
be overkill here, since the code can only be executed by the respective systems --
i.e. \GAP or \SageMath. Therefore we will base our foundation on OMDoc/MMT theory graphs
directly rather than on an extension of OMDoc/MMT with ``biform
theories''~\cite{KohManRab:aumftg13,Farmer:btc07} as envisioned in the proposal. Biform
theories would enable (partial) verification of mathematical software systems, but this is
not on the critical path towards a mathematical VRE. The MitM paradigm constitutes a
lightweight alternative; identifying and refining it has been one of the major
achievements of the first year in \WPref{dksbases}.

\section{Interface Theories for GAP and SageMath}\label{sec:ift}
\begin{todolist}{MK: some of this has already been discussed in the CICM16 paper, }
\item MK@MP+DM: describe the GAP interface theory and how they are generated; give the
numbers, but only give the diffs to the CICM16 paper.
\item MK@NT+DM: describe the SageMath interface theory and how it is generated; this is
  new. 
\item MK@MK+NT+MP: compare and discuss the different generation approaches\ednote{MK: do
    we have a IFT for Singular yet?}
\item MK@MK: conclude the section by a discussion about OpenMath and Dialects.
\end{todolist}

%%% Local Variables:
%%% mode: latex
%%% TeX-master: "paper"
%%% End:

\section{The MitM Ontology for Computational Group Theory}\label{sec:cgt}
\begin{todolist}{MK@MP+DM: describe your work here}
\item talk about the levels (abstract, concrete subgroup theory, computational)
\item talk about alignments from the IFT to the CGT, how they work, building
  on~\cite{MueRoYuRa:abtafs17,MueGauKal:cacfms17} 
\end{todolist}

To create a working example, we turned towards one of the topics best
understood by GAP: Computation with (permutation) groups, and formalise it in
MMT.
This a first part of the MitM Ontology.


\subsection{Layers of Abstraction}\ednote{MP@DM this could do with a picture a
  bit like the one in your Alignments paper; if you have the source for it, I
  could adapt it}
The layers: abstract, representation, canonical(?), system dialect

We discovered that formalisation of CGT requires different levels of
abstraction\ednote{MP: This will probably be true for any type of object
  in this game}. At the highest level there is the theory of \emph{Groups}: the
group axioms, generating sets, homomorphisms, group actions, stabilisers,
and orbits. This also easily leads into definitions of
centralisers\footnote{stabilisers of elements under conjugation} and
normalisers\footnote{stabilisers of subgroups under
conjugation}, stabiliser chains,  Sylow-$p$ subgroups, Hall subroups, and many
other concepts. 

MMT also allows expressing that there are different equivalent definitions of a
concept: We defined group actions in two ways and used \emph{views} to show
their equivalence.

\ednote{MP: We need to be able to talk about elements/subsets of groups,
  elements/subsets of $S_n$ that generate groups}

\medskip

Abstract groups can be represented in many ways as more concrete mathematical
objects: as groups of permutations, groups of matrices, finitely presented
groups, or as a polycyclic presentation.\ednote{MP: Not sure this is relevant but the first
three of these are universal: every group has a representation as such an
object, whereas the last is a specialised representation for polycyclic groups}

Additionally, mathematicians often compute with canonical representatives of an
isomorphism class of groups: When a group theorist talks about the ``Dihedral
group of order 8'', they often have a particular canonical representation in
mind, for example as a permutation group that acts on the square by rotations
and reflections, but in GAP this group would be represented as a group of
permutations of (usually) the corners of the square, or a polycyclic
presentation.\ednote{MP: I think this needs better explanation}

These representations also arise naturally from \emph{group actions}: If we are
considering symmetry in a setting where we want to apply group theory, we start
with a group action.\ednote{MP: More concrete? More ``gripping''? I already
talked about the canonical example with the dihedral group}

The universal tool to bridge the gap between groups, representations and
canonical representatives are group homomorphisms.

\medskip

At the lowest level there are implementation details: Permutation groups in GAP
are considered as finite subgroups of the group $S_{\mathbb{N}+}$, and defined by
providing a set of generating permutations. GAP then computes a stabiliser chain
for a group that was defined this way, and naturally considers the group to be a
subgroup of $S_{[1..n]}$, where $n$ is the largest point moved.

\ednote{MP: There might have to be more layers, but these are the main ones I
  can think about right now}

\subsection{Alignments}

Alignments are currently produced by hand: \ednote{MP: potential for
  automated or semi-automated production of alignments from our exports}
For example the filter \texttt{IsGroup} is aligned with \texttt{Group}, and the
filter \texttt{IsPermGroup} is aligned with \texttt{Subgroup SymmetricGroup
  [1..n]}.
\ednote{MP: Need to be more concrete here, in particular we should maybe
  describe how GAP's notion of an action homomorphism translates through this?
  Also is this even correct?}

We formalised the theory of symmetric groups of a set; in GAP permutation groups
are represented as subgroups (with finite support) of the symmetric group of
$\mathbb{N}$, and often one concretely has an isomorphism between the group one
is interested in and a subgroup of $S_{\mathbb{N}}$, for example
via a group action.

\texttt{SylowSubgroup} are more difficult: They are special groups in their
own right, namely groups whose size is a prime-power, but we also want them
to be identified with a certain subgroup of the group we are working
with.\ednote{MP: While I believe this to be an excellent additional example
  for MMT formalisation, this could be going too far for this paper}


\ednote{MP+FR+DM: Mention that this attempt at formalising group theory
  lead to improvements in MMT?}
\ednote{MP@ALL: We might want to be a bit careful/mention implementations of group
  theory for example in COQ where they did the Odd-Order-Proof?}
%%% Local Variables:
%%% mode: latex
%%% TeX-master: "paper"
%%% End:

%  LocalWords:  sec:cgt MueRoYuRa:abtafs17,MueGauKal:cacfms17 emph Sylow subroups medskip
%  LocalWords:  mathbb

\begin{figure}[ht]\centering
  \tikzinput{gap_singular_mitm_fig}
  \caption{MitM Interaction in Jane's Use Case}\label{fig:mitmpoc}
\end{figure}

Figure~\ref{fig:mitmpoc} shows the overall architecture with an MitM server as the central mediator.
All arrows represent the transfer of \OMMT ojbects via SCSCP.
Critically, the MitM server also implements alignments and uses them to convert between system dialects.

We have extended the \MMT system~\cite{Rabe:MAGMS13} with an SCSCP server/client so that it can receive objects from computation systems and generates calls to others.
For the \GAP server, we built on pre-existing \SCSCP support.
To obtain an \SCSCP server for \Singular, which does not have native \SCSCP support, we wrapped \Singular in a python script that includes the \lstinline|pyscscp| library~\cite{py-scscp:on}.
In \Sage, we directly programmed the client interface to the MitM server.

The numbers on the edges indicate the order of communications when processing Jane's use case.
Initially, Jane has already built in \Sage the ring $R=\mathbb{Z}[X_1,X_2,X_3,X_4]$, the group $G=D_4$, and the action $A$ of $G$ on $R$ that permutes the variables, and the polynomial $p = 3\cdot X_1 + 2\cdot X_2$.
She now calls \lstinline|MitM.Singular(MitM.Gap.orbit(G, A, p)).Ideal().Groebner().sage()|, which results in the following steps:
\begin{compactenum}
  \item Jane uses \Sage to call the MitM server with the respective \Sage object, along with metadata about which system should be used for which computation.
  \item The MitM server translates \lstinline|MitM.Gap.orbit(G, A, p)| to the \GAP system dialect and sends it to \GAP.
  \item \GAP returns the orbit $O$.
  \item The MitM server translates \lstinline|MitM.Singular(O).Ideal().Groebner()| to the \Singular system dialect and sends it to \Singular.
  \item \Singular returns the Gröbner base $B$.
  \item The MitM server translates \lstinline|B| to the \Sage system dialect and sends it to \Sage, where the result is shown to Jane.
\end{compactenum}

\paragraph{Another use-case}

Suppose Jon prefers working in \GAP, and she wants to compute the
Galois group of the rational polynomial $p = x^5 - 2$.

Jon discovers the \GAP package \texttt{radiroot}, which promises this
functionality, but the package does not work for this polynomial.
\ednote{MP: The radiroot part can go away, the jist is: This cannot be done
  with \GAP currently}

Jon hears from his colleague Jane that he should just use \Sage, because
computing Galois groups is a breeze.

% $p =x^4-x^3-x^2+x+1$ over $\mathbb{Q}$ would have D_8 as galois group again...

Jon calls \lstinline|G := MitM("Sage", "GaloisGroup", p)| in \GAP which yields
the desired Galois group as a \GAP permutation group.

Jon, being a proficient \GAP user, also knows that he can now install a \emph{method}
in \GAP that will compute the Galois group of any rational polynomial
transparently for him whenever he calls \lstinline|GaloisGroup| for a rational
polynomial in \GAP. \ednote{MP: And he submits a pull-request to \GAP to make
  that happen}



\ednote{FR@all: Does my description match what is happening? We have to discuss this and probably adapt the implementation accordingly.}

% \begin{oldpart}{MK: just copied here; Victor writes\\
%     ``\emph{A peer-to-peer connection must be made with the CAS servers, so that CAS
%       servers can, in turn, query MitM if during a computation they encounter a concept
%       that lies outside their field of knowledge. In application to this particular case,
%       it would be cleaner if, instead of asking MitM to produce permutations of a list,
%       the client simply queries MitM for the orbit of a polynomial by defining an action
%       of a member of the symmetric group on a polynomial. \GAP would then be able to
%       calculate the orbit by making the group act on the polynomial with the described
%       action and querying MitM for equality of polynomials, resulting in a linear-time
%       algorithm instead of quadratic-time behaviour displayed by the current client.}''  I
%     do not quite understand the maths here, maybe we can stillmake this happen?}
%   The control script follows the procedure:\ednote{MK: for this to make sense we would
%     have to describe what problem we want to solve.}
% \begin{enumerate}
%   \item Create an OpenMath polynomial.
%   \item Obtain a symmetric group of size that is equal to the number of variables 
%     in the polynomial from MitM.
%   \item Using the obtained group, query MitM for all permutations of the list 
%     of variables.
%   \item Create polynomials from the permutations of the list of variables.
%   \item Filter out the duplicate polynomials by querying MitM for equality of 
%     polynomials.
% \end{enumerate}
% While this is very much a brute-force algorithm to calculate an orbit of a
% polynomial, it showcases the ability of the client to query the MitM server that 
% is then forced to use multiple CAS without the client needing any knowledge of the
% underlying systems.
% \end{oldpart}

%The \Sage client behaves exactly as described in
%Section~\ref{sec:mitm:comms}\ednote{MK@NT/TW; it seems that we will have time to
%  implement this after the extension. So we should make it happen.}

%%% Local Variables:
%%% mode: latex
%%% TeX-master: "paper"
%%% End:

%  LocalWords:  sec:case fig:mitmpoc IanJucKoh:sdm14,MathHub:on summarize sec:cgt pyscscp
%  LocalWords:  twiesing:msc17 centering tikzinput gap_singular_mitm_fig lstinline emph
%  LocalWords:  py-scscp:on oldpart

  We have implemented the MitM approach to integrating mathematical software systems based on formalizations of the underlying mathematical knowledge.
  The main investment here was the curation of an MitM Ontology, the generation of formal specifications of system APIs for \Sage, \GAP, and \Singular, identifying the alignments of these APIs with the ontology, implementing an MitM server that can use alignments to translate between systems, and implementing the \SCSCP protocol for all involved systems.

  We have also shown how to extend the Math-in-the-Middle framework for integrating systems to mathematical data bases like the \lmfdb. 
The main idea is to embed knowledge sources as virtual theories, i.e. theories that are not -- theoretically or in practice -- limited in the number of declarations and allow dynamic loading and processing. 
For accessing real-world knowledge sources, we have developed the notion of codecs and integrated them into the MitM ontology framework. 
These codecs (and their MitM types) lift knowledge source access to the MitM level and thus enable object-level interoperability and allow humans (mathematicians) access using the concepts they are familiar with. 
Finally, we have shown a prototypical query translation facility that allows to delegate some of the processing to the underlying knowledge source and thus avoid thrashing of virtual theories. 

\paragraph{Related Work} Most other integration schemes employ a \textbf{homogenous approach}, where there is a master system and all data is converted into that system. 
A paradigmatic example of this is the Wolfram Language~\cite{WolframLanguage:wikipedia} and the Wolfram Alpha search engine~\cite{WolframAlpha:on}, which are based on the Mathematica kernel. 
This is very flexible for anyone owning a Mathematica license and experienced in the Mathematica language and environment.

The MitM-based approach to interoperability of data sources and systems proposed in this paper is inherently a \textbf{heterogeneous approach}: systems and data sources are kept ``as is'', but their APIs are documented in a machine-actionable way that can be utilized for remote procedure calls, content format mediation, and service discovery. 
As a consequence, interaction between systems is very flexible.
For the data source integration via virtual theories presented in this paper this is important. 
For instance, we can just make an extension of \mmt or \Sage\ which just act as a programmatic interface for e.g. \lmfdb. 

Our case studies show that MitM-based integration is an achievable goal.
Delegation-based workflows can either be programmed directly or embedded into the interaction language of the mathematical software systems.

The main advantages and challenges claimed by the MitM framework come from its loosely coupled and knowledge-based nature.
Compared to ad-hoc translations, MitM-based interoperability is relatively expensive as objects have to be serialized into (possibly large) \OMMT objects, transferred via \SCSCP to \MMT, parsed, translated into another system dialect, serialized and transferred, and parsed again.
On the other hand, instead of implementing and maintaining $n^2$ translations, we only have to establish and maintain $n$ collections of system APIs and their alignments to the
MitM ontology.
This makes the management of interoperability much more tractable:
\begin{compactenum}
\item The MitM ontology is developed and maintained as a shared resource by the community.
We expect it to be well-maintained, since it can directly be used as a documentation of the functionality of the respective systems.
\item All the workflows are star-shaped: instead of requiring expert knowledge in two systems -- a rare commodity even in open-source projects, and even for the system experts involved in this \papertype -- and keeping up with their changes, the MitM approach only needs expertise and change management for single systems.
\end{compactenum}
All in all, these translate into a ``business model'' for MitM-based cooperation in terms of the necessary investment and achievable results, which is based on the well-known \emph{network effects}: the joining costs are in the size of the respective system, whereas the rewards -- i.e. the functionality available by delegation -- is in the size of the network.

This network effect can be enhanced by technical refinements we are currently studying:
For instance, if we annotate alignments with a ``priority'' value that specifies how canonically/efficiently/powerfully a given system implements a given MitM operation, then we can let
the \MMT mediator automatically choose a suitable target system for a requested computation (as opposed to our current setup where Jane specifies which systems she wants to use). On the other hand, for workflows where we do not need or want service-discovery, alignments can be ``compiled'' into $n^2$ transport-efficient direct translations that may even eliminate the need for serialization and parsing.

\paragraph{Future Work}\ednote{MK: this is essentially future work only for LMFDB, we need future work also for MitM here.}
\ednote{MK@FR: could you please describe the MMT Python bridge and how this could be used for SCSCP-less communication with Sage. The MitM-work here is to allow for compilation of the alignment based translations into code.}
We have discussed the MitM+virtual theories methodology on the elliptic curves sub-base of the \lmfdb, which we have fully integrated. 
We are currently working on additional \lmfdb sub-bases. 
The main problem to be solved is to elicit the information for the respective schema theories from the \lmfdb community. 
Once that is accomplished, specifying them in the format discussed in this paper and writing the respective codecs is straightforward. 

Moreover, we are working on integrating the the Online Encyclopedia of Integer Sequences (OEIS~\cite{Sloane:OEIS,oeis}). 
Here we have a different problem: the OEIS database is essentially a flat ASCII file with different slots (for initial segments of the sequences, references, comments, and formulae); all minimally marked up ASCII art. 
In~\cite{LuzKoh:fsarfo16} we have already (heuristically) flexiformalized OEIS contents in \ommt; the next step will be to come up with codecs based on this basis and develop schema theories for OEIS.


\subsubsection*{Acknowledgements}
The authors gratefully acknowledge the fruitful discussions with other participants of
work package WP6, in particular Alexander Konovalov on \SCSCP, Paul Dehaye on the \Sage
export and the organization of the MitM ontology, Luca de Feo on OpenMath phrasebooks
and the \SCSCP library in python, and David Lowry-Duda\ednote{MK: or make him a co-author?  JC: We should give him the option.}

We acknowledge financial support from the OpenDreamKit Horizon 2020 European Research
Infrastructures project (\#676541) and DFG project RA-18723-1 OAF.

%%% Local Variables:
%%% mode: latex
%%% mode: visual-line
%%% fill-column: 5000
%%% TeX-master: "report"
%%% End:

%  LocalWords:  sec:concl MitM-based itemize subsubsection Dehaye organization serialized
%  LocalWords:  math-savy emph serialization formalizations
  
  
\printbibliography
\end{document}
%%% Local Variables:
%%% mode: latex
%%% TeX-master: t
%%% End:

%  LocalWords:  maketitle twb sec:intro DehKohKon:iop16 MueGauKal:cacfms17 optimized
%  LocalWords:  itemize MitM-based sec:concl twiesing:msc17 printbibliography sec:mitm
%  LocalWords:  sec:mmt summarize sec:cgt sec:ift MueRoYuRa:abtafs17,MueGauKal:cacfms17
