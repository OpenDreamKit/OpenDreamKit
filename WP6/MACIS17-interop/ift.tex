\section{Interface Theories for GAP and SageMath}\label{sec:ift}
\begin{todolist}{MK: some of this has already been discussed in the CICM16 paper, }
\item MK@MP+DM: describe the GAP interface theory and how they are generated; give the
numbers, but only give the diffs to the CICM16 paper.

\item MK@NT+DM: describe the SageMath interface theory and how it is generated; this is
  new. 
\item MK@MK+NT+MP: compare and discuss the different generation approaches\ednote{MK: do
    we have a IFT for Singular yet?}
\item MK@MK: conclude the section by a discussion about OpenMath and Dialects.
\end{todolist}


\subsection{GAP Interface Theory}

In \cite{DehKohKon:iop16} we describe our approach to export
knowledge in the form of type information from a running GAP session.

For a successful implementation of MitM system it is also necessary to
know how any given object in a running session has been constructed
(Constructor annotation). For this GAP was instrumented with code to store
this information.

In combination with the static type system export this enables an MitM session
to recreate an object exactly from given input data.



%%% Local Variables:
%%% mode: latex
%%% TeX-master: "paper"
%%% End:
