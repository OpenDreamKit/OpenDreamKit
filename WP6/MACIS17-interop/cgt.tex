\section{The MitM Ontology for Computational Group Theory}\label{sec:cgt}
\begin{todolist}{MK@MP+DM: describe your work here}
\item talk about the levels (abstract, concrete subgroup theory, computational)
\item talk about alignments from the IFT to the CGT, how they work, building
  on~\cite{MueRoYuRa:abtafs17,MueGauKal:cacfms17} 
\end{todolist}

To create a working example, we turned towards one of the topics best
understood by GAP: Computation with (permutation) groups, and formalise them in
MMT, forming a first part of the MitM Ontology, and serving as a case study in
creating such.

Starting from the theory of groups, we formalised the notions of
groups, subgroups, homomorphisms, two different, but equivalent definitions of
group actions, stabilisers, orbits, and some more advanced concepts such as
centralisers\footnote{stabilisers of elements under
conjugation} normalisers\footnote{stabilisers of subgroups under
conjugation} and and p-Sylow-subgroups.

We formalised the theory of symmetric groups of a set; in GAP permutation groups
are represented as subgroups (with finite support) of the symmetric group of
$\mathbb{N}$, and often one concretely has an isomorphism between the group one
is interested in and a subgroup of $S_{\mathbb{N}}$, for example
via a group action.

As a test of the power of MMT we als created views that prove equivalence of the
two two different formal definitions of group actions.

There are also
``canonical'' representations of certain groups; thinking about for example
the dihedral group on $4$ points (which is sometimes called $D_8$ and sometimes
called $D_4$) and sometimes the user only wants to know properties of these canonical
representations, but sometimes the user has a concrete group and wants to
determine such a canonical representation (i.e. identify the group)

Since GAP\footnote{any computational system needs to}
knows about different
concrete representations of groups, for example universal representations such
as permutation groups, matrix groups (over finite fields), or finitely presented
groups, or specialised representations such as polyclically presented groups
there needs to be a layer of formalisation that
understands these representations.

Most of this information is encoded in GAP as well, but some of it is much less
explicit. A lot of potential for cross-checking results from GAP computations
needs to be explored.

Alignments are currently produced by hand\texttt{There is probably potential for
automated or semi-automated production of alignments from our exports}: For
example the filter \texttt{IsGroup} is aligned with \texttt{Group}, and the
filter \texttt{IsPermGroup} should be aligned with \texttt{Subgroup SymmetricGroup
  [1..n]}.

\texttt{SylowSubgroup} are more difficult: They are special groups in their
own right, namely groups whose size is a prime-power, but we also want them
to be identified with a certain subgroup of the group we are working
with.\footnote{MP: While I believe this to be an excellent additional example
  for MMT formalisation, this could be going too far for this paper}

\footnote{Do we mention that this attempt at formalising group theory lead to
  improvements in MMT?}
\footnote{We might want to be a bit careful/mention implementations of group
  theory for example in COQ where they did the Odd-Order-Proof?}
\footnote{We need to describe how we'd do alignments now; by hand?
  semi-automated? automated? what do we actually align!}
%%% Local Variables:
%%% mode: latex
%%% TeX-master: "paper"
%%% End:
