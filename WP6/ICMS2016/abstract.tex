OpenDreamKit ``Open Digital Research Environment Toolkit for the Advancement of
Mathematics'' --- is an H2020 EU Research Infrastructure project that aims at supporting,
the ecosystem of open-source mathematical software systems. From that, OpenDreamKit will
deliver a flexible toolkit enabling research groups to set up Virtual Research
Environments, customised to meet the varied needs of research projects in pure mathematics
and applications.

An important step in the OpenDreamKit endeavor is to foster the interoperability and
distributed computing between a variety of systems, ranging from computer algebra systems
over mathematical databases to front-ends. This is the mission of the integration work
package (WP6). We report on experiments and future plans with the
\emph{Math-in-the-Middle} approach. This information architecture consists in a central
mathematical ontology that documents the domain and fixes a joint vocabulary, combined
with specifications of the functionalities of the various systems. Interaction between
systems can then be enriched by pivoting off this information architecture.

%%% Local Variables:
%%% mode: latex
%%% TeX-master: t
%%% End:
