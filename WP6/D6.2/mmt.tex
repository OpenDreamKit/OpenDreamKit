An OMDoc/MMT theory graph consists of theories and the relations between
them~\cite{RabKoh:WSMSML13}. An OMDoc/MMT theory is a set of declarations -- a set of
declared symbols. In addition to the declarations, each theory has a name (which together
with its namespace forms the global URI for the theory) and a meta-theory. A meta-theory
is commonly the logical framework that is used to model the content of the theory. Each
declared symbol has a name and can additionally have a type, a definition and different
kinds of meta-data. In each theory these symbols can then be used to form terms that can
be used to express more advanced knowledge. Here terms are effectively OpenMath 2.0
\cite{BusCapCar:2oms04} objects -- they mostly consist of literal values, symbols and
applications of terms to other terms.

There are two basic kinds of relations between theories: imports and views. An import is a
way to declare symbols from one theory in another theory -- to import the symbols from a
source theory to a target theory. This can for example be used to extend an existing
theory without re-declaring all symbols or to combine two theories. Furthermore the
concept of imports allows to modularise knowledge. On top of imports there are also
Structures which are imports and additional renamings of the imported symbols. The second
type of relation, the view, is a mapping from one theory to another -- a way to ``view''
one theory as another. This mapping allows terms from one theory to be translated into
another theory. In the case where terms represent boolean statements or proofs, the
mapping given by the view is truth preserving -- \emph{i.e.}~if a statement is true in the
source theory, it is be true in the target theory after translation.

Theory graphs are implemented inside the \MMT system~\cite{Rabe:MAGMS13,uniformal:on}. The
system allows for the declaration of theories along with symbols, imports and
views. Furthermore it is possible to create terms over these theories and translate them
along views. The \MMT system also provides a type checker that can be used to type check
declarations.