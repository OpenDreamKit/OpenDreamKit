We only sketch \MMT here to the extent that we need it and refer to \cite{RabKoh:WSMSML13} for details.
An OMDoc/MMT theory graph is a diagram in the category of \MMT theories and theory morphisms.
Theories represent all kinds of languages such as mathematical foundations and type systems as well as individual mathematical theories.
Theory morphisms represent relations between them including translations, imports, and representation theorems.

\paragraph{Theories}
In the simplest case (which is sufficient for our purposes here) an \MMT \textbf{theory} is a set of symbol declarations.
Each \textbf{symbol declaration} must have a name and may optionally have a type object, a defining object, and arbitrary meta-data (such as tags, cross-references, comments, etc).
The \textbf{objects} are OpenMath 2.0 objects \cite{BusCapCar:2oms04} --- these are complex expressions formed from application, binding, variables, literals, and symbol references.
\ednote{@TW: give example of a theory, e.g., a fragment of the ODK math theory}

Critically, \MMT enforces structural validity: Every object may only reference symbols that are declared in or imported into the containing theory.
But \MMT abstracts from the foundational semantics such as type and logic systems that specify exactly which objects are meaningful.
This puts \MMT into a powerful intermediate position, where enough structure is guaranteed for knowledge management while not committing to a particular foundation.

Above we have skipped the \MMT module system, which is not essential for our results here.
However, one detail of the \MMT module system \emph{is} critical: Every theory may have a \textbf{meta-theory}.
Practically, the meta-theory mostly behaves like a plain import.
But conceptually, the meta-theory of a theory $T$ is the language that provides the foundational background to understand $T$.
The most common use case employs three meta-levels: Firstly, an \MMT theory introduces a logical framework $F$.
Secondly, $F$ serves as the meta-theory of a foundation $L$, which uses the symbols of $F$ to define a particular type system and logic.
Thirdly, a library of mathematical knowledge is developed as a theory graph in which all theories have meta-theory $L$.
\ednote{@TW: give example graph involving some theories involved in this report}

Most systems (including most systems involved in \pn) focus on the third level only.
The second level is usually left implicit, e.g., because it is hard-coded in the implementation of computation system.
And because the second level is hard-coded, there is no need for the first level.
For system integration, it is important to make the second level explicit so that the semantics of the exchanged knowledge can be specified.
Thus, \MMT already anticipates much of the high-level concepts needed for the system-spanning \DKS-bases in \pn.

\paragraph{Theory Morphisms}
There are two kinds of theory morphisms: imports and views.
Imports are the central for building theories modularly by pulling together symbols introduced in other theories.
Practically, imports can be seen amount to copying symbol declarations while instantiating, renaming, or otherwise modifying them.
The semantics of theories with imports is defined via colimits.

A view from theory $S$ to theory $T$, is a translation of $S$-objects to $T$-objects.
Contrary to imports, views are given after the involved theories have been defined.
Thus, they have the character of theorems rather than definitions, and indeed views usually induce proof obligations that must be discharged for a view to be accepted.
In the presence of appropriate logical frameworks, \MMT guarantees a critical property: The translation of objects preserves all properties of the type system and the logic.
In particular, all $S$-theorems are translated to $T$-theorems.

\paragraph{Implementation}
Theory graphs are implemented in the \MMT system~\cite{Rabe:MAGMS13,uniformal:on}.
At its core, this system allows for the declaration of theories along with symbols, imports and views, to build objects and translate them along views.

On top of this, the \MMT system provides a number of logical and knowledge management services.
The former includes computing the respective colimits, translating objects along morphisms, or type checking and proving objects relative to the respective meta-theory.
The latter include import/export of libraries, editing, browsing, and middleware for system integration.