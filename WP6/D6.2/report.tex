\documentclass{../../Proposal/LaTeX-proposal/deliverablereport}
\usepackage[style=alphabetic,backend=biber]{biblatex}
\addbibresource{../../lib/kbibs/kwarc.bib}
\addbibresource{../../lib/deliverables.bib}
\addbibresource{rest.bib}
% temporary fix due to http://tex.stackexchange.com/questions/311426/bibliography-error-use-of-blxbblverbaddi-doesnt-match-its-definition-ve
\makeatletter\def\blx@maxline{77}\makeatother

% TIkZ magic
\usepackage{tikz,standalone}
\usetikzlibrary{backgrounds,shapes,fit,shadows}
\makeatletter../../WP6/tikz/pgflibrarytikzmmt.code.tex\makeatother

\usepackage{standalone}
\usepackage[show]{ed}

\usepackage{graphicx}
\usepackage{float}

\usepackage{color}
\definecolor{gray}{rgb}{0.4,0.4,0.4}
\definecolor{darkblue}{rgb}{0.0,0.0,0.6}
\definecolor{cyan}{rgb}{0.0,0.6,0.6}

\hypersetup{colorlinks}

\deliverable{dksbases}{design}
\issue{136}
\deliverydate{31/08/2016}
\duedate{31/08/2016 (Month 12)}
\def\pn{OpenDreamKit\xspace}

\author{Michael Kohlhase, Florian Rabe, Tom Wiesing, Paul-Olivier Dehaye, Dennis M\"uller}

\usepackage[parfill]{parskip}

% for definitions
\usepackage{amsthm}
\newtheorem{mydef}{Definition}

\usepackage{xspace}

%%%%%%%% fix %%%%%%%%%%%
\makeatletter\gdef\deliv@lead{JU}\makeatother
%%%%%%%% endfix %%%%%%%%%%%


% Keep all these presented in a standard fashion
\newcommand{\SageMath}{SageMath\xspace}
\newcommand{\GAP}{GAP\xspace}
\newcommand{\LMFDB}{LMFDB\xspace}
\newcommand{\OEIS}{OEIS\xspace}
\newcommand{\python}{Python\xspace}
\newcommand{\MMT}{MMT\xspace}
\newcommand{\FindStat}{FindStat\xspace}
\newcommand{\Pari}{Pari\xspace}
\newcommand{\PariGP}{Pari/GP\xspace}

\newcommand{\Magma}{Magma\xspace}

\newcommand{\DKS}{$\mathcal{DKS}$}

\begin{document}


\begin{abstract}
  To build Virtual Research environments (VREs) we need to integrate three different aspects, Data (D), Knowledge (K) and Systems (S). We want to achieve this using the well-established framework of theory graphs. In this setting we need to expand the current model of theories.

  In this report we present the design process towards \textit{DKS} theories including the overall architecture, a survey of the systems involved, our current implementation of \textit{DK} theories as well as our plans for the future.
\end{abstract}
\maketitle 


\newpage\tableofcontents\newpage

In this report we present a prototypical integration of the Jupyter notebooks into the MathHub.info portal for active mathematical documents and a versioned hosting system for flexiformal mathematics.
MathHub.info offers a rich interface for reading, writing, and interacting with mathematical documents and knowledge. Jupyter offers a uniform interface to the computation facilities of the OpenDreamKit VRE toolkit in the form of a read-eval-print loop (REPL).

A mathematical Virtual Research Environment (VRE) needs both kinds of interface functionality: mathematical documents have been very successful for presenting mathematical knowledge, and while there have been efforts to make them modular and interactive they predominantly remain in the mode of archiving and transporting knowledge in Mathematics.
Notebook interfaces also use the document metaphor at the surface; however the REPL interaction
tends to take structural precedence, leading to documents consisting of a sequence of computational cells within which the mathematical discourse is interspersed in the form of ``rich comments''.

A ``literate computing'' version of notebooks which gives mathematical discourse structural precedence is possible in principle, but has not been supported consistently at the system level.\ednote{MK: put the following sentence somewhere: A ``literate programming'' version of notebooks which gives mathematical discourse structural precedence is possible in principle, but has not been supported consistently at the system level.}
This tension and trade-off has been explored in OpenDreamKit Deliverable D4.2~\cite{ODK-D4.2}, and the concept of in-document computation in OpenDreamKit Deliverable D4.9~\cite{ODK-D4.9}.
In both cases, the integration was incomplete, since it lacked a full integration of the
underlying knowledge/computation services.

Generally, the integration of MathHub and Jupyter consists of two parts:
\begin{inparaenum}[\em a\rm )]
\item the integration of the user interfaces (as reported previously) and
\item the integration of the knowledge/computation management services.
\end{inparaenum}
Here we report on progress in both; recall that MMT is the knowledge management service behind MathHub (and more generally for the Math-in-the-Middle based system integration; see OpenDreamKit Deliverable D6.5~\cite{ODK-D6.5}).
%
For the service integration we present an MMT kernel for Jupyter.
%
\ednote{specify what Jupyter widgets are; NT: you may want to reuse some of the language of the D4.16 report, around l21 of https://github.com/OpenDreamKit/OpenDreamKit/blob/master/WP4/D4.16/report.tex}
%
Reciprocally, for the user interface integration, we show how the Jupyter widgets can be deeply integrated within the MMT knowledge management facilities to give semantics-aware interaction facilities, extending the front-end capabilities of MathHub/Jupyter Notebooks by semantic widgets driven by the MMT in-document knowledge management services.

We show and evaluate the integration on two case studies: in-document computing facilities in active documents and a knowledge-based specification dialog for modeling and simulation. 

This report is structured as follows. In Section~\ref{sec:mmt-jp} we report on the MathHub/Jupyter integration at the system level: a Jupyter server as part of the MathHub system and a MMT kernel for Jupyter. Section~\ref{sec:nb-mh} presents the integration of Jupyter Notebooks as active documents in the (new) MathHub front-end, and Section~\ref{sec:mitm-nb} presents the two case studies. Section~\ref{sec:concl} concludes the report and discusses future work.

\ednote{this paragraph seems a bit out of place after the description of the structure of the document}
The goal of this report\ednote{of this deliverable?} is to integrate Jupyter notebooks into MathHub
and make them compatible with MMT, in a way that we can conveniently use 
MMT syntax in these notebooks and also a little bit of extra functionality
like e.g. the Jupyter widgets. The first step is setting up a Jupyter server,
which currently runs on \url{http://juypter.mathhub.info}. \ednote{KA: maybe show picture of it?}
For this server, we have developed a custom kernel, that forwards the input 
entered into the Jupyter notebook to the MMT backend. This then processes 
said input and sends the response back to the Jupyter frontend via the kernel.
We will cover the implementation of the Jupyter kernel and the MMT-backend,
later in this report.


\paragraph{Acknowledgements} The authors gratefully acknowledge the support of the Jupyter team and in particular the advice of Benjamin Ragan-Kelly. Also, the input of Theresa Pollinger and her work on the MoSIS system~\cite{PolKohKoe:kacse18} has shaped our perception of the integration reported here. 

%%% Local Variables:
%%% mode: latex
%%% mode: visual-line
%%% fill-column: 5000
%%% TeX-master: "report"
%%% End:

\section{Report and Case-Study}\label{sec:survey}
\ednote{@Paul: give a general overview of the results of the case-study}

%%% Local Variables:
%%% mode: latex
%%% TeX-master: "report"
%%% End:

%  LocalWords:  ednote


\section{Theory Graphs: The Knowledge Aspect}\label{sec:MMT}
 An OMDoc/MMT theory graph consists of theories and the relations between
them~\cite{RabKoh:WSMSML13}. An OMDoc/MMT theory is a set of declarations -- a set of
declared symbols. In addition to the declarations, each theory has a name (which together
with its namespace forms the global URI for the theory) and a meta-theory. A meta-theory
is commonly the logical framework that is used to model the content of the theory. Each
declared symbol has a name and can additionally have a type, a definition and different
kinds of meta-data. In each theory these symbols can then be used to form terms that can
be used to express more advanced knowledge. Here terms are effectively OpenMath 2.0
\cite{BusCapCar:2oms04} objects -- they mostly consist of literal values, symbols and
applications of terms to other terms.

There are two basic kinds of relations between theories: imports and views. An import is a
way to declare symbols from one theory in another theory -- to import the symbols from a
source theory to a target theory. This can for example be used to extend an existing
theory without re-declaring all symbols or to combine two theories. Furthermore the
concept of imports allows to modularise knowledge. On top of imports there are also
Structures which are imports and additional renamings of the imported symbols. The second
type of relation, the view, is a mapping from one theory to another -- a way to ``view''
one theory as another. This mapping allows terms from one theory to be translated into
another theory. In the case where terms represent boolean statements or proofs, the
mapping given by the view is truth preserving -- \emph{i.e.}~if a statement is true in the
source theory, it is be true in the target theory after translation.

Theory graphs are implemented inside the \MMT system~\cite{Rabe:MAGMS13,uniformal:on}. The
system allows for the declaration of theories along with symbols, imports and
views. Furthermore it is possible to create terms over these theories and translate them
along views. The \MMT system also provides a type checker that can be used to type check
declarations.
 
\section{Math in the Middle: The Software Aspect}\label{sec:mitm}
  When integrating multiple systems we are mostly talking about using concrete algorithms
(implemented by these systems) to solve specific computational problems (the knowledge
about the problem). To integrate multiple systems with this knowledge we want to enable
users to write down a problem in one system and then solve it in another system. We want
to be independent of the implementation of the knowledge -- independent of the systems.

For this we make use of an approach we call ``Math-In-The-Middle'' paradigm
(see~\cite{DehKohKon:iop16} for details). Here the underlying mathematical knowledge, the
``real math'', is used as a reference ontology for system (in the ``middle'') -- hence the
name. Each system needs access to this knowledge. As each of them come with their own
particularities, they will need some interface to it.

We want to make use of the modular approach to mathematics provided by theory graphs, and
in particular \MMT as an implementation thereof, to first of all allow us translate
mathematical expressions between systems. We define a ``Math In The Middle'' theory as
well as interface theories for each system. With the help of \MMT and bi-views\footnote{A
  bi-view is a bidirectional view between two theories. } between the interface theories and
the central theory, we can translate objects from one system to the other.

\begin{figure}[ht]\centering
  \def\myxscale{3}\def\myyscale{1.2}
  \documentclass{standalone}
\usepackage[mh]{mikoslides}
% this file defines root path local repository
\defpath{MathHub}{/Users/kohlhase/localmh/MathHub}
\mhcurrentrepos{MiKoMH/talks}
\libinput{WApersons}
% we also set the base URI for the LaTeXML transformation
\baseURI[\MathHub{}]{https://mathhub.info/MiKoMH/talks}

\usetikzlibrary{backgrounds,shapes,fit,shadows,mmt}
\begin{document}
\begin{tikzpicture}[xscale=2.6,yscale=.9]
  \tikzstyle{withshadow}=[draw,drop shadow={opacity=.5},fill=white]
   \tikzstyle{database} = [cylinder,cylinder uses custom fill,
      cylinder body fill=yellow!50,cylinder end fill=yellow!50,
      shape border rotate=90,
      aspect=0.25,draw]
   \tikzstyle{human} = [red,dashed,thick]
   \tikzstyle{machine} = [green,dashed,thick]

\node[thy]  (mf) at (.2,5.3) {MathF};
\node[thy,dashed]  (compf) at (0,6) {CompF};
\node[thy,dashed]  (pf) at (-.9,5.5) {PyF};
\node[thy,dashed]  (cf) at (1,5.5) {C\textsuperscript{++}F};
\node[thy,dashed]  (sf) at (-0.9,4.6) {SAGE};
\node[thy,dashed]  (gf) at (1,4.6) {GAP};

\draw[include] (compf) -- (pf);
\draw[includeleft] (compf) -- (cf);
\draw[include] (pf) -- (sf);
\draw[includeleft] (cf) -- (gf);

\node[thy] (kec) at (0,3) {EC};
\node[thy,minimum height=.4cm] (kl) at (0,4) {\ldots};

\node[thy] (sec) at (-1,2) {SEC};
\node[thy,minimum height=.4cm] (sl) at (-1,3) {\ldots};

\node[thy] (gec) at (1,2) {GEC};
\node[thy,minimum height=.4cm] (gl) at (1,3) {\ldots};

\node[thy] (lec) at (-.3,1.2) {LEC};
\node[thy,minimum height=.4cm] (ll) at (.3,1.2) {\ldots};

\node (sc) at (-2,4) {SAGE};
\node[draw] (salg) at (-2,3.35) {Algo};
\node[database,dashed] (sdb) at (-2,2.4) {DB?};
\node[draw] (skr) at (-2,1.7) {KR};
\node[draw,machine] (sac) at (-2,1) {AbsClass};

\node (gc) at (2,4) {GAP};
\node[draw] (galg) at (2,3.35) {Algo};
\node[database,dashed] (gdb) at (2,2.4) {DB?};
\node[draw] (gkr) at (2,1.7) {KR};
\node[draw,machine] (gac) at (2,1) {AbsClass};

\node (lmfdb) at (0,0) {LMFDB};
\node[database] (ldb) at (1,-.4) {Mongo};
\node[draw] (knowls) at (-1,-.4) {Knowls};
\node[draw,machine] (lac) at (0,-.5) {AbsClass};

  \begin{pgfonlayer}{background}
    \node[draw,cloud,fit=(sec) (sl),aspect=.4,inner sep=-3pt,withshadow,purple!30] (st) {};
    \node[draw,cloud,fit=(gec) (gl),aspect=.4,inner sep=-4pt,withshadow,purple!30] (gt) {};
    \node[draw,cloud,fit=(kec) (kl),aspect=.4,inner sep=0pt,withshadow,blue!30] (kt) {};
    \node[draw,cloud,fit=(lec) (ll),aspect=2.5,inner sep=-7pt,withshadow,purple!30] (lt) {};
  \end{pgfonlayer}

\begin{pgfonlayer}{background}
  \node[draw,withshadow,fit=(sc) (skr) (sac) (sdb),inner sep=1pt] {};
  \node[draw,withshadow,fit=(gc) (gkr) (gac) (gdb),inner sep=1pt] {};
  \node[draw,withshadow,fit=(lmfdb) (lac) (ldb) (knowls),inner sep=1pt] {};
\end{pgfonlayer}

\draw[view] (kec) -- (sec);
\draw[view] (kec) -- (gec);
\draw[view] (kec) -- (lec);
\draw[include] (kec) -- (kl);
\draw[include] (gec) -- (gl);
\draw[include] (sec) -- (sl);
\draw[include] (lec) -- (ll);
\draw[view] (kl) -- (sl);
\draw[view] (kl) -- (gl);
\draw[view] (kl) to[bend left=5] (ll);

\draw[meta] (mf)  to [bend right=10] (st);
\draw[meta] (sf) -- (st);
\draw[meta] (mf)  to [bend left=10] (gt);
\draw[meta] (gf) -- (gt);
\draw[meta] (mf) -- (kt);
\draw[meta] (compf) to[bend right=15] (kt);

\draw[human,->] (skr) -- node[above]{\scriptsize induce} (st);
\draw[human,->] (gkr) -- node[above]{\scriptsize induce} (gt);
\draw[human,->] (knowls) -- node[left,near end]{\scriptsize induce} (lt);

\draw[machine,->] (gt) to[bend right=30] node[below,near start]{\scriptsize generate} (gac);
\draw[machine,->] (st) to[bend left=30] node[below,near start]{\scriptsize generate} (sac);
\draw[human,->] (st) to[bend left=20] node[below]{\scriptsize refactor} (kt);
\draw[human,->] (gt) to[bend right=20] node[below]{\scriptsize refactor} (kt);
\draw[human,->] (lt) -- node[right]{\scriptsize refactor} (kt);
\end{tikzpicture}
\end{document}
%%% Local Variables: 
%%% mode: latex
%%% TeX-master: t
%%% End: 

  \caption{The MitM paradigm in detail. PyF, C${}^{++}$F and CompF are (basic)
    foundational theories for \python, C${}^{++}$ and a generic computational model. SEC,
    LEC and GEC are theories for \SageMath, \LMFDB and \GAP elliptic curves.}\label{fig:mitm}
\end{figure}

A sketch of the theory graph based on the example of elliptic curves can be found in
Figure~\ref{sec:mitm}. We will not go into details here but show how this architecture
integrates the \emph{Software} and \emph{Knowledge Aspects}. Clearly, the (hand-curated)
MitM ontology -- the purple cloud in the middle -- is a specification of the underlying
mathematical knowledge as an OMDoc/MMT theory graph, while the system interface theories
-- the blue clouds around it -- formally specify the names and types (i.e. the argument
patterns) and intended behaviour of the interface functions of the systems (often
semi-formally to make the MitM approach scalable). The OMDoc/MMT views -- the wavy arrows
between the theories -- are interpretation morphisms; in this particular case where they
connect the mathematical specification to the system theories, they express the
``implementation relation''. Thus the OMDoc/MMT framework already allows to integrate the
knowledge and software aspects for system interoperability.

The restriction to formalizing the signature (i.e. names and types of the interface
functions) of the systems is sufficient to ensure system interoperability; integrating the
implementations -- e.g. C\textsuperscript{++} or Python code -- into the theories would
be overkill here, since the code can only be executed by the respective systems --
i.e. \GAP or \SageMath. Therefore we will base our foundation on OMDoc/MMT theory graphs
directly rather than on an extension of OMDoc/MMT with ``biform
theories''~\cite{KohManRab:aumftg13,Farmer:btc07} as envisioned in the proposal. Biform
theories would enable (partial) verification of mathematical software systems, but this is
not on the critical path towards a mathematical VRE. The MitM paradigm constitutes a
lightweight alternative; identifying and refining it has been one of the major
achievements of the first year in \WPref{dksbases}.

  
\section{Virtual Theories: The Data Aspect}\label{sec:data}
OMDoc/MMT theories are limited when it comes to representing large amounts of data.
Conceptually, every database should be represented as one theory, but this can easily lead to very large theories.
For example, the theory for elliptic curves in the \LMFDB would contain \ednote{@TW: add number} symbol declarations: one definition for every curve.
Prior to \pn, \MMT could only load who{e theories into main memory, which made it insufficient as a basis for \DKS-bases.

Moreover, many data-driven theories are technically infinite collections of which only finite fragments have been explored so far.
For example, this applies to all databases in the \LFMBD (each of which enumerates a certain infinite class of mathematical objects) and \OEIS (each of which enumerates a certain integer sequence).
As the explored fragments grow, the set of symbol declarations in the corresponding \MMT theory must grow accordingly.

Therefore, we generalize \MMT theories to allow for a virtual, possibly infinite set of declarations, that is explored dynamically.
The combination of virtual theories with the Math-in-the-Middle approach yields our desired \DKS-bases.

  \subsection{Virtual Theories}\label{sec:data:def}
   OMDoc/MMT theories are limited when it comes to representing large amounts of data.
Conceptually, every database should be represented as one theory, but this can easily lead to very large theories.
For example, the theory for elliptic curves in the \LMFDB would contain $319,882$ symbol declarations: one definition for every curve.
Prior to \pn, \MMT could only load who{e theories into main memory, which made it insufficient as a basis for \DKS-bases.

Moreover, many data-driven theories are technically infinite collections of which only finite fragments have been explored so far.
For example, this applies to all databases in the \LMFDB (each of which enumerates a certain infinite class of mathematical objects) and \OEIS (each of which enumerates a certain integer sequence).
As the explored fragments grow, the set of symbol declarations in the corresponding \MMT theory must grow accordingly.

Therefore, we generalize \MMT theories to allow for a virtual, possibly infinite set of declarations, that is explored dynamically.
The combination of virtual theories with the Math-in-the-Middle approach yields our desired \DKS-bases.

\begin{mydef}[Virtual Theory]
  A \textbf{virtual theory} is like an \MMT theory but with a (possibly infinite) partially ordered set of declarations.
\end{mydef}

We give a trivial example of an infinite virtual theory for the natural numbers:
Besides the usual symbols for $0$ and $\mathtt{succ}$ as well as the Peano axioms, it contains the totally ordered set of one declaration for every natural number.
For example, we might have a declaration
 \[5:\mathtt{nat}=\mathtt{succ}(4)\]
to introduce a symbol for the number $5$.
In the presence of an addition operator, this theory might also contain one axiom for every pair of natural numbers, e.g., to state the truth of $2+2=4$.

This is a typical situation: We have an infinite (or very big) set of declarations that are generated systematically.
In some cases (as for the natural numbers above), every declaration can be easily generated on-demand.
Thus, one might think that virtual theories can be easily represented in a finitary way by storing the algorithm that produces the generations.

But this falls short in general.
For example, the generation of the declarations may be so expensive that it is only practical if they are precomputed and stored in a database.
This is what happens in the \LMFDB (and why the \LMFDB was introduced in the first place).
It is also possible that there are multiple algorithms enumerating different fragments of the virtual theory, or that no generating algorithm is known (e.g., for some integer sequences in the \OEIS) and individual declarations must be collected manually.

For example, consider the database of elliptic curves in the \LMFDB.
\ednote{@TW: give a detailed example of a declaration in the virtual theory here}

\paragraph{Implementation}
In practice we never need to access all of these declarations at once --- in most
scenarios we only need a very small subset of them, usually small enough to hold in memory.
This motivates the main idea behind how we have implemented virtual theories in the \MMT system.

\MMT already abstracts from physical storage backends (working copies, databases, etc.), from which theories are loaded.
We have extended this storage abstraction to allow loading not only theories but individual declarations on demand.
This is more difficult than it sounds because while theories have a self-contained semantics, declarations only make sense in the context of the containing theory.
Thus, we had to comb through the \MMT code base and generalize all processing to the case where a theory's declarations are only partially known.

We have also built an \LMFDB-specific implementation of this this generalized storage abstraction.
This instance dynamically queries \LMFDB for the appropriate entry, computes the corresponding declarations, and adds it to the in-memory representation of the virtual theory.
(Additionally, if that virtual theory is not in memory yet, it first creates it.)
We were able to retain an important feature of \MMT: The loading of declarations is transparent to the user.
The backend loads a declaration automatically when and if it is needed by some operation.
\ednote{@TW: give details on example here}



%%% Local Variables:
%%% mode: latex
%%% TeX-master: "report"
%%% End:

  \subsection{Relating Database Objects and Mathematical Objects}\label{sec:data:impl}
   The storages sketched in Section~\ref{sec:virtual} are not as simple as one might think.
A major complication is that scalable databases are only provide relatively low-level data types.
For example, a typical relational database provides primitive types for, e.g., integers and strings, and tables contain records built from these.
JSON databases (as used in MongoDb~\cite{Chodorow:mdg10}, which is used in \LMFDB) or XML databases are slightly better by providing structured types like trees and lists.
But the sets of mathematical objects stored in mathematical data systems use much richer data types such as matrices and polynomials and arbitrarily more complex types.

Therefore, any data system must employ encodings that translate the actual mathematical objects into database objects.
This has been done ad hoc in the past and has proved both very difficult and --- due to differing or undocumented encodings --- an obstacle for system interoperability.
Therefore, we have developed systematic method for encoding/decoding mathematical objects as database objects.
This allows formally specifying the schema of a database in such a way that \MMT storages can use it to encapsulate the encoding and provide users with a high-level view of a mathematical database.
A sketch of our method can be found in Figure~\ref{fig:codec_arch} -- we will give a
detailed explanation below.

\begin{figure}[ht]\centering
  \providecommand\myxscale{3.2}
  \providecommand\myyscale{1.5}
  \providecommand\myfontsize{\footnotesize}
  \documentclass{standalone}
\usepackage{../lstjson}
\usepackage{../lstmmt}
\usepackage{tikz}
\usetikzlibrary{backgrounds,shapes,fit,shadows,arrows,shapes.geometric}
\makeatletter../../WP6/tikz/pgflibrarytikzmmt.code.tex\makeatother

\begin{document}
\def\thmo#1#2{\mathsf{#1}\colon\kern-.15em{#2}}
\providecommand\myxscale{2.6}
\providecommand\myyscale{1.2}
\providecommand\myfontsize{\footnotesize}
\lstset{language=MMT,mathescape}

\setbox0\hbox{\myfontsize
\begin{lstlisting}[linewidth=.3\textwidth]
conductor: int $\US$
 ?codec standardInt $\US$
 ?implements cond
 $\RS$
$\dots$
\end{lstlisting}
  }

\setbox1\hbox{\myfontsize
\begin{lstlisting}[linewidth=.4\textwidth]
elliptic\_curve: type $\RS$
cond: elliptic\_curve $\rightarrow$ int $\RS$
$\dots$
\end{lstlisting}
  }

  \setbox2\hbox{\myfontsize
\begin{lstlisting}[linewidth=.4\textwidth]
11a1: elliptic_curve $\US$ = $\dots$ $\RS$
$\dots$
\end{lstlisting}
  }

\begin{tikzpicture}[xscale=\myxscale,yscale=\myyscale]\myfontsize
  \tikzstyle{database}=[cylinder,shape border rotate=90,aspect=0.25,draw,cylinder uses custom fill,cylinder body fill=white!30,cylinder end fill=white!3]
  \node[database] (db) at (1,4) {
    \begin{tabular}{c}Database\\\\ LMFDB\\ (MongoDB)\end{tabular}
  };

  \node[draw] (record) at (2.5,4) {
    \begin{minipage}{3cm}
\begin{lstlisting}[language=json,mathescape,frame=none,aboveskip=2pt,belowskip=2pt]
label: "11a1",
conductor: 5,
$\dots$
\end{lstlisting}
\end{minipage}};

  \draw[->] (db) -- node[above]{contains} (record);

  \node[thy] (schema) at (2.5,2) {$\mmtthy{Record Schema Theory}{\box0}{}$};
  \draw[->] (schema) -- node[left]{describes} (record);

  \node[thy] (mitm) at (0,2) {$\mmtthy{Math-In-The-Middle Theory}{\box1}{}$};

  \draw[view] (schema) -- node[above]{implements} (mitm);

  \node[thy] (virtual) at (1,0) {$\mmtthy{Virtual Theory}{\box2}{}$};

  \draw[includeleft] (virtual) -- node[left]{includes} (mitm);
\end{tikzpicture}
\end{document}

%%% Local Variables:
%%% mode: latex
%%% TeX-master: t
%%% End:

  \caption[Translation between Database and Mathematical Objects]{ Illustration of the
    method we developed to translate between database objects and mathematical objects.
    As an example we use Elliptic Curves from \LMFDB.  }
  \label{fig:codec_arch}
\end{figure}

\ednote{we should find a better name for the \enc and \dec macros}

\subsubsection{Codecs}

We fix an arbitrary data type of \textbf{codes}.  These are the primitive values of the
database.  Typical examples are strings or JSON objects.  We will use JSON for the purpose
of giving concrete examples.

Our goal is to define codecs that define the relation between \MMT objects and codes.

\ednote{we should find a better name for the \enc and \dec macros /TW}

\begin{mydef}[Codec]
  For an \MMT object $t$, a \textbf{$t$-codec} is a pair $(\enc, \dec)$ where
  \begin{compactitem}
   \item $\enc$ is a partial function from \MMT objects of type $t$ to codes
   \item $\dec$ is a partial function from codes to $\MMT$ objects of type $t$
  \end{compactitem}
  such that $\dec$ and $\enc$ are inverse to each other whenever defined.
\end{mydef}

Both encoding and decoding are partial functions.
This is to be expected for decoding: only certain codes are the result of encoding objects of type $t$.
It may be surprising for encoding.
Here we need partiality because \MMT objects may be arbitrarily complex expressions.
This can include non-normalized expressions or expressions with free variables or uninterpreted symbols.
This will become clear in the examples below.

Recall the Math-in-the-Middle theory from Section~\ref{sec:MMT}.
A trivial example of an $\Int$-codec is essentially the identity function: $\enc$ maps integer literals to the corresponding JSON integer, and $\dec$ inverts it.
$\enc$ is partial because it only encodes literals, e.g., it does not encode expressions like $x+1$ or $\min \{x^4+x^3+x^2+x+1|x\in \mathbb{N}\}$.
(An object like $1+1$ can be encoded by first simplifying it to a literal $2$.)
More subtly, it cannot encode integer literals that do not fit into the $64$ bit integers of a typical JSON implementation.

But this is not the only reasonable $\Int$-codec.  In \LMFDB databases, we often encounter
integers that do not fit into $64$ bits, and we have specify two additional codecs.
$\intAsString$ encodes integer literal as strings in decimal representation.
$\intAsString$ has the advantage of easily encoding all integers.  But it is not
convenient for computations.  Therefore, \LMFDB users have occasionally used a smarter
encoding: $\intAsList$ encodes integer literals as lists of JSON integers such that the
list $[n,d_1,\ldots,d_n]$ corresponds to the integer whose base $2^{64}$ representation is
$(d_1\ldots d_n)$.  $\intAsList$ has the advantage that the lexicographic ordering can be
used for size comparisons without having to decode.

Along the same lines, we can define codecs for other simple data types.
For example, we can define a codec for matrices of integers that encodes pairs of pairs of integers as a JSON list of $4$ JSON integers.
This example is interesting because it comes up in the \LMFDB and has confused some programmers: because the encoding was not fully documented, it was mistakenly assumed that the respective mathematical type of the value is that of lists of integers.

\lstset{
  morecomment=[l]{/T}
}

\lstinputlisting[
  morekeywords={namespace,theory,include,rule},
  mathescape,
  caption={Fragment of the special Codec Theory in MMT},
  label=lst:mmt-codec
]{examples/odk-codec.mmt}

We collect and document all available codecs in a special \MMT theory, a fraction of which can be seen in Listing~\ref{lst:mmt-codec}.
Here, we first declare a type of codecs for each MMT type.

Note that our codec theory does not include any implementations (which usually require highly programming language--specific function calls.
Instead, it standardizes names for the codecs and documents their behavior so that each codec can be implemented faithfully in multiple programming languages.
This is in line with the Math-in-the-Middle approach of centrally storing the shared knowledge.

We have seeded this theory with a few important codecs.  And for every codec we describe,
we have already given a reference implementation in Scala~\cite{scala:webpage} so that the
\MMT system (which is written in Scala) can use all codecs.  In the future, \pn will
gradually build a library of relevant codecs and implement them in multiple programming
languages.

\paragraph{Codec Operators}
$\Int$ is an atomic type.
But mathematical types are usually very complex types.
Already, types like $\lst(\Int)$ provide substantial encoding problems because both integers and lists can be encoded differently.
To systematize these choices, we introduce codec operators.

\begin{mydef}[Codec Operator]
  For an \MMT symbol $t$ and an arity $n$, a $t$-codec operator takes an $o_1$-codec $C_1$, \ldots, an $o_n$-codec $C_n$ and returns a $t(o_1,\ldots,o_n)$ codec.
\end{mydef}

For example, let us define a $\lst$-codec operator $\standardList$ of arity $1$.
It takes a type $t$-codec $C$ and returns the following codec: $\enc$ maps the object $\cons(a_1,\cons(a_2,\ldots,\cons(a_n,\nil)\ldots))$ to the JSON list $[e_1,\ldots,e_n]$ where each $e_i$ is the result or encoding $a_i$ with $C$.
$\dec$ is defined accordingly.

Similar to the codecs, the special \MMT codec theory (Listing~\ref{lst:mmt-codec} above) also includes standardised names for these.
Apart from $\standardList$, we also have straightforward codec operators for vectors (as lists of fixed length) and matrices (as lists of lists of fixed lengths).

\subsubsection{Schema Theories}

\ednote{@DM: Can you check/prettify/detailify the descriptions in this section. Maybe add the URIs that we use LMFDb or add details.}

Codecs can transform between \MMT objects and codes, but we still have to specify which codecs to use for which types.
We use special theories for this purpose, which we call \emph{schema theories}.
These satisfy two functions.

Firstly, a schema theory describes the database schema: many databases (including the ones in \LMFDB) can be seen as sets of records conforming to a certain schema.
We represent these schemas as \MMT theories with one symbol declaration for each field.
The meta-theory of these theories is the math-in-the-middle theory so that the types of the symbols can be the intended mathematical types.
Secondly, we annotate meta-data to each declaration, providing the information which concept in the math-in-the-middel theory a field implements and which codecs to use for converting between the two.

Thus, the schema documents both the mathematical meaning of the fields and the physical encoding used when exchanging records conforming to the schema.
By giving the schema theory for each database, we can capture all knowledge necessary to automatically interface with it.

As an example, the schema theory for the database of \emph{elliptic curves} in LMFDB
has meta-data linking it to the \emph{type} of elliptic curves in the corresponding math-in-the-middle theory, as well as meta-data telling \MMT which field in the database to use as \emph{names} for the resulting \MMT declarations - in this case the field \texttt{label}, which corresponds to a unique LMFDB-internal naming scheme. The schema theory contains e.g. a declaration \texttt{degree} of type \texttt{pos} (for positive  integers), corresponding to a field in the database by the same name, which is annotated with meta-data telling \MMT
\begin{enumerate}
\item to use the codec \texttt{standardPos} to convert from (and to) an LMFDB entry, and
\item that the field \texttt{degree} implements the function \texttt{modular\_degree} in the math-in-the-middle theory.
\end{enumerate}

We have implemented a new component of the \MMT system that takes an expression $c$ of type $\codec t$ and builds the appropriate $\tm t$-codec by traversing $c$.
We use this in the storage instance for \LMFDB as follows:\ednote{@DM add details here, e.g., explain theory names in general and/or for elliptic curve example}
\begin{compactenum}
 \item A declaration with name $n$ in theory $T$ is requested (e.g. the elliptic curve with label \texttt{11a1} in the theory \texttt{lmfdb:db/elliptic\_curves?curves}).
 \item We load the schema theory $S$ for $T$ (e.g. \texttt{lmfdb:schema/elliptic\_curves?curves})
 \item We connect to \LMFDB and retrieve the corresponding record (e.g. the database entry for the curve with label \texttt{11a1}).
 \item We decode every field of the record according to the codec specified in $S$.
 \item We collect the decoded \MMT object in an \MMT record $r$, the mathematical representation of the requested object.
 \item We add the declaration $n=r$ to the corresponding virtual theory.
\end{compactenum}

\ednote{FR: add description of handling of constructors/assessors for a future paper; I
  think we can omit it for the deliverable as unnecessarily detailed} \ednote{MK@TW: We
  need one or two diagrams above that show how the theories hold together and how the
  information flow works. This is non-trivial, but will help us clarify the presentation}

%%% Local Variables:
%%% mode: latex
%%% TeX-master: "report"
%%% End:

  \subsection{Towards Mathematical Querying of Databases}\label{sec:querying}
   \section{Querying}\label{sec:querying}

So far we have only concerned ourselves with accessing DK theories one declaration at a
time. This shows that DK theories are indeed useful, however in general one wants to be
able to access multiple declarations at once. Even though it is not the focus of this
report, we have had some thoughts about this.

Querying is the act of finding all declarations subject to some arbitary
criterion. Practically relevant queries can range from very simple queries -- such as find
all OEIS sequences containing a certain number -- to computationally intensive tasks, such
as find all elliptic curves with a conductor divisible by five\footnote{This particular
  example was given to us by John Cremona when asked for queries that the current
  architecture can not solve. }.

There gives a wide range of interesting questions that one might want to implement a query
engine for. Since most of the declarations that one wants to query over a big, one does
not want to evaluate each query by iterating over the entire set -- this is far to
slow. Thus a naive approach consists of iterating over all declarations beforehand,
finding all occuring values, and storing all of these inside a hash table. Such as index
can easily answer the first question from above. On top of this someone might want to find
sequences starting with a given integer. In this case we should either build a second
index that only contains starting numbers, or expand the first index in a smart
fashion. Another interesting query can be to find OEIS sequences containing arbitary
subsequences -- in which case the index no needs to contain any possible
subsequence. Eventually the index will explode -- the last addition would already scale
exponentially.

To be able to answer the second question above, one might want to take all occuring values
(in this case integers) and factorise them. Again we can store this in some form of
index. This can also be extended to polynomials -- which would allow users to search
polynomials based on their roots. Also in this situation it is very easy to underestimate
the complexity of the index.

It comes down to finding a good balance between interesting and useful queries and size
and scalability of the index. This in and of itself is a non-trvivial research task. In
the scope of K-theories we have solved parts of this question already. For example we have
built a general purpose query language for \MMT. We have also built MathWebSearch that
allows users to search for certain mathematical expressions within document corpera. We
will not discuss this here -- interested readers should take a look at \cite{Rabe:qlfml12}
and \cite{ODK-D6.1} for details.

In the future of the OpenDreamKit project we want to investigate this question for
DK-theories further. We want to take a deeper look at useful query languages. This could
consist of extending codecs from a value-translating mechanism to a query-translating
mechanism. That is we take a query from inside \MMT and then compile this into a database
query -- which can then be evaluated efficiently. It could also take another direction
entirely.

%%% Local Variables:
%%% mode: latex
%%% TeX-master: "report"
%%% End:


\section{Case Studies}\label{sec:cases}

While our theoretical model of DK theories and our architectural design of virtual
theories are applicable to a wide variety of databases, in the scope of the OpenDreamKit
project we want to conduct a few case studies and connect to some databases in
particular. Our efforts so far are based on two of these case studies of which we want to
give a short overview below.

\subsection{GAP}\label{sec:gap}
\GAP \cite{gap} is a computer algebra system with a particular emphasis on group theory and discrete mathematics in general. Its fundamental
ontology consists of \emph{objects} (e.g. a monoid) satisfying various (composite or elementary) \emph{filters} (e.g. \emph{isAbelian}) which
can be thought of as the \emph{types} of objects. Filters can \emph{imply} other filters.
On top of these filters, operations are defined (e.g. computing the \emph{degree} of a group) which are implemented by arbitrarily many concrete implementations
called \emph{methods}. The user only ever applies operations - the \GAP system then uses a sophisticated method selection algorithm based on
the specific (additional) filters satisfied by a given object.

We have a working specification of \GAP's ontology in \MMT containing declarations for all of the above concepts. This serves as a meta-theory for a working \GAP import. The result is currently an import
of 4097 filters and operations as \MMT symbols collected in approximately 200 theories, which can be used as the \emph{interface theories} for communicating with \GAP. Filters
are imported as \MMT declarations, the implications between which can be displayed as a graph in \MMT.

So far, all the imported operations have no information about their return types (i.e. the filters that apply to the returned object). Currently, work is done
on the \GAP system to make that information available within \GAP in general and for the export to \MMT specifically.

\subsection{Sage}\label{sec:sage}
\SageMath \cite{sagemath} is a \python-based computer algebra system. It is based on the notion of a \emph{category} (which is related to, but not equivalent to the category theoretical notion),
which provides \emph{methods} on its \emph{elements} (e.g. elements of a group), its \emph{parents} (e.g. groups themselves) and its \emph{morphisms}
(e.g. the group homomorphisms). Each category can add new \emph{axioms} and inherit from other categories - e.g. the category \emph{AbelianGroups} inherits from \emph{Groups} and adds the axiom \emph{abelian}.

As with \GAP, we have a specification theory in \MMT defining all the above concepts, as well as a working import of 382 categories using 25 axioms and 808 methods. Each category corresponds to one \MMT theory declaring its methods and axioms as well as the corresponding documentation. The theory graph of the resulting theories mirrors exactly the inheritance graph of the original categories in Sage. The imported theories can again serve as \emph{interface theories} to the SageMath system.

\subsection{LMFDB}\label{sec:lmfdb}

\LMFDB \cite{lmfdb} is a database of objects from number theory. It mostly consists of
L-functions, but also has a number of other sub-databases. It is built on top of MongoDB
and as such uses JSON to model all of its data.

We have already implemented the schema and codec architecture above to build a virtual
theory of elliptic curves in \LMFDB. Even though this only is a very small part of \LMFDB,
this can serve as a template for future implementations of the remaining parts of
LMFDB. We started out with having just a few fields of the curves available inside \MMT,
however adding the relevant codecs and making more fields available proofed to be a quick
and easy job. This has shown us that we are heading in the right direction with our ideas
so far.

In the future we want to extend the coverage of the existing set of theories. This will
include writing more schema theories and possibly introducing more codecs. This will
likely also lead to some refactoring inside \LMFDB itself, as the community for the first
time will try to semantically describe its entire dataset.

\subsection{OEIS}

\OEIS \cite{oeis} stands for On-Line Encyclopedia of Integer Sequences. It is a collection
of around 250 thousand integer sequences that are stored as pure text form. The \OEIS is
licensed under Creative Commons and thus freely accessible.

We have already semantified the pure text format and, among other things, this has helped
us finding new relations between the existing sequences. A more detailed look at our
previous work can be found in \cite{LuzKoh:fsarfo16} and we will not go into details
here. So far these efforts have helped us to understand how the OEIS database is
structured.

Similar to \LMFDB we plan on integrating this into our virtual theories architecture. We
are considering building one DK theory per sequence, where the declarations in each theory
contain the known elements of the sequence. We also plan to integrate this with our
infrastructure on knowledge management services, such as MathHub and MathWebSearch.


%%% Local Variables:
%%% mode: latex
%%% TeX-master: "report"
%%% End:

%  LocalWords:  lmfdb oeis LuzKoh fsarfo16

  We have implemented the MitM approach to integrating mathematical software systems based on formalizations of the underlying mathematical knowledge.
  The main investment here was the curation of an MitM Ontology, the generation of formal specifications of system APIs for \Sage, \GAP, and \Singular, identifying the alignments of these APIs with the ontology, implementing an MitM server that can use alignments to translate between systems, and implementing the \SCSCP protocol for all involved systems.

  We have also shown how to extend the Math-in-the-Middle framework for integrating systems to mathematical data bases like the \lmfdb. 
The main idea is to embed knowledge sources as virtual theories, i.e. theories that are not -- theoretically or in practice -- limited in the number of declarations and allow dynamic loading and processing. 
For accessing real-world knowledge sources, we have developed the notion of codecs and integrated them into the MitM ontology framework. 
These codecs (and their MitM types) lift knowledge source access to the MitM level and thus enable object-level interoperability and allow humans (mathematicians) access using the concepts they are familiar with. 
Finally, we have shown a prototypical query translation facility that allows to delegate some of the processing to the underlying knowledge source and thus avoid thrashing of virtual theories. 

\paragraph{Related Work} Most other integration schemes employ a \textbf{homogenous approach}, where there is a master system and all data is converted into that system. 
A paradigmatic example of this is the Wolfram Language~\cite{WolframLanguage:wikipedia} and the Wolfram Alpha search engine~\cite{WolframAlpha:on}, which are based on the Mathematica kernel. 
This is very flexible for anyone owning a Mathematica license and experienced in the Mathematica language and environment.

The MitM-based approach to interoperability of data sources and systems proposed in this paper is inherently a \textbf{heterogeneous approach}: systems and data sources are kept ``as is'', but their APIs are documented in a machine-actionable way that can be utilized for remote procedure calls, content format mediation, and service discovery. 
As a consequence, interaction between systems is very flexible.
For the data source integration via virtual theories presented in this paper this is important. 
For instance, we can just make an extension of \mmt or \Sage\ which just act as a programmatic interface for e.g. \lmfdb. 

Our case studies show that MitM-based integration is an achievable goal.
Delegation-based workflows can either be programmed directly or embedded into the interaction language of the mathematical software systems.

The main advantages and challenges claimed by the MitM framework come from its loosely coupled and knowledge-based nature.
Compared to ad-hoc translations, MitM-based interoperability is relatively expensive as objects have to be serialized into (possibly large) \OMMT objects, transferred via \SCSCP to \MMT, parsed, translated into another system dialect, serialized and transferred, and parsed again.
On the other hand, instead of implementing and maintaining $n^2$ translations, we only have to establish and maintain $n$ collections of system APIs and their alignments to the
MitM ontology.
This makes the management of interoperability much more tractable:
\begin{compactenum}
\item The MitM ontology is developed and maintained as a shared resource by the community.
We expect it to be well-maintained, since it can directly be used as a documentation of the functionality of the respective systems.
\item All the workflows are star-shaped: instead of requiring expert knowledge in two systems -- a rare commodity even in open-source projects, and even for the system experts involved in this \papertype -- and keeping up with their changes, the MitM approach only needs expertise and change management for single systems.
\end{compactenum}
All in all, these translate into a ``business model'' for MitM-based cooperation in terms of the necessary investment and achievable results, which is based on the well-known \emph{network effects}: the joining costs are in the size of the respective system, whereas the rewards -- i.e. the functionality available by delegation -- is in the size of the network.

This network effect can be enhanced by technical refinements we are currently studying:
For instance, if we annotate alignments with a ``priority'' value that specifies how canonically/efficiently/powerfully a given system implements a given MitM operation, then we can let
the \MMT mediator automatically choose a suitable target system for a requested computation (as opposed to our current setup where Jane specifies which systems she wants to use). On the other hand, for workflows where we do not need or want service-discovery, alignments can be ``compiled'' into $n^2$ transport-efficient direct translations that may even eliminate the need for serialization and parsing.

\paragraph{Future Work}\ednote{MK: this is essentially future work only for LMFDB, we need future work also for MitM here.}
\ednote{MK@FR: could you please describe the MMT Python bridge and how this could be used for SCSCP-less communication with Sage. The MitM-work here is to allow for compilation of the alignment based translations into code.}
We have discussed the MitM+virtual theories methodology on the elliptic curves sub-base of the \lmfdb, which we have fully integrated. 
We are currently working on additional \lmfdb sub-bases. 
The main problem to be solved is to elicit the information for the respective schema theories from the \lmfdb community. 
Once that is accomplished, specifying them in the format discussed in this paper and writing the respective codecs is straightforward. 

Moreover, we are working on integrating the the Online Encyclopedia of Integer Sequences (OEIS~\cite{Sloane:OEIS,oeis}). 
Here we have a different problem: the OEIS database is essentially a flat ASCII file with different slots (for initial segments of the sequences, references, comments, and formulae); all minimally marked up ASCII art. 
In~\cite{LuzKoh:fsarfo16} we have already (heuristically) flexiformalized OEIS contents in \ommt; the next step will be to come up with codecs based on this basis and develop schema theories for OEIS.


\subsubsection*{Acknowledgements}
The authors gratefully acknowledge the fruitful discussions with other participants of
work package WP6, in particular Alexander Konovalov on \SCSCP, Paul Dehaye on the \Sage
export and the organization of the MitM ontology, Luca de Feo on OpenMath phrasebooks
and the \SCSCP library in python, and David Lowry-Duda\ednote{MK: or make him a co-author?  JC: We should give him the option.}

We acknowledge financial support from the OpenDreamKit Horizon 2020 European Research
Infrastructures project (\#676541) and DFG project RA-18723-1 OAF.

%%% Local Variables:
%%% mode: latex
%%% mode: visual-line
%%% fill-column: 5000
%%% TeX-master: "report"
%%% End:

%  LocalWords:  sec:concl MitM-based itemize subsubsection Dehaye organization serialized
%  LocalWords:  math-savy emph serialization formalizations


\newpage\printbibliography

\begin{appendix}
\section{Raw Case Study Results}\label{sec:raw-survey}

\setcounter{secnumdepth}{4}

\makeatletter
\def\subsubsection{\@startsection{subsubsection}{3}%
  \z@{.2\linespacing\@plus.7\linespacing}{.1\linespacing}%
  {\normalfont\itshape}}
\makeatother
\makeatletter
\def\paragraph{\@startsection{paragraph}{4}%
  \z@{.2\linespacing\@plus.7\linespacing}{.1\linespacing}%
  {\normalfont}}
\makeatother

The authors are grateful to John Cremona,  Alex Konovalov, Markus Pfeiffer, Viviane Pons, and Nicolas M. Thi\'{e}ry for their time in answering our survey. Their responses and the structure of this survey have been adapted to fit this presentation format. 

\subsection{\FindStat}
\subsubsection{Overview at a high level of the system}

\FindStat \cite{FindStat} is a database and a web interface accessing the database. It is designed by and for combinatorists. The purpose is to store and search information on \emph{statistics} over \emph{combinatorial objects}. A statistic is mostly a map between a set of combinatorial objects to the natural numbers. As an example, the number of edges is a statistic on graphs. The main purpose of \FindStat is to give an interface for one to \emph{search} for some statistics the same way one would search for integer sequences on the OEIS \cite{oeis}.

\subsubsection{Available data}

\paragraph{Structure of the data}

\FindStat has basically 3 categories of objects.

\begin{description}
\item[The combinatorial collections] \FindStat stores a list of combinatorial collections: 18 as of today (January 2016). All these combinatorial collections are actually linked to a \SageMath combinatorial collection. We only store the minimal  information needed to print the collection on the website and to recreate the collection in \SageMath.

For every collection, we store a list of combinatorial objects. More precisely, we use \SageMath to generate the list of objects,
but we store a standardised version of the printout of the object. This standardised version is homemade: it has to be
\begin{description}
\item[standardized] a single given graph will always be printed the same way,
\item[unique] two different graphs will never be printed the same way,
\item[human readable] when possible, it should be easy to understand for a human reader and not only a machine (so no hash-key or anything like this).
When possible, we keep the default printout of \SageMath object. Sometimes, we have to store a little bit of code to convert this printout into a
\SageMath entry.
\end{description}

\item[The combinatorial statistics] A statistic is a list of couples : combinatorial object from a certain collection or value. As of now, we have 364 statistics,
each of them containing between 200 and 1000 entries. For each statistic, we store some metadata: name, identifier
(specific to \FindStat, can be referenced from outside), combinatorial collection, description, code, references, etc. And we also store the data itself: a list of entries,
each entry is made from combinatorial object (as a string, by its standardised printout) and a integer value. As an example, the values of "The number of edges of a graph"
St000081 is a list of all graphs up to size 6 with their associated number of edges.

\item[The combinatorial maps] A combinatorial map is a mathematical function from a combinatorial collection to another combinatorial collection. For example: binary search
tree insertion turns permutations into binary trees. As of now, we store 107 maps each of them containing between 200 and 1000 entries. We store the metadata of the map: domain, codomain, description, code, etc. And we store the map-data as a list of (value, image)
where value and image are combinatorial objects stored as strings through their standardised printout.
\end{description}

As an addition, \FindStat also provides some wiki pages with information on combinatorial objects, maps and statistics in a less formalized way.

The low level data format is a SQL database where we store everything we need. Most of the data described above is accessible through the website in HTML. All information about combinatorial statistics and combinatorial maps can be accessed through JSON files that have standardised URLs depending on the identifier of said statistics or maps. It is possible that the URL changes if the website organisation is changed in the future but it will always be related to the identifiers which are set once and for all. The format of the JSON files are also likely to change but we try to limit those changes and keep backward compatibility as much as possible. Those JSON files are the closest we have to an external API, they are used by the \SageMath-\FindStat interface.

All our data are distributed under Creative Commons Attribution-ShareAlike 3.0 Unported License.

\paragraph{How is this data produced?  How is it changed?}

The data are produced and changed through user contributions. As for now, 55 people are listed as contributors. We have an HTML form to submit statistics where the user receives many information on what should be submitted and in what format. Once a new statistic is submitted or a change is proposed, it has to be validated by one of the day developers. We don't receive that much data so the process is usually very quick. Each change is stored and so we have access to the full history of the statistic information with authors.

For maps, we don't have yet the "Add Map" form. Each map has to be added by one of the \FindStat developers. The reason is just that the maps are a more recent addition and so the adding process has not been finalized yet.

\paragraph{How do you document it?}

We have a very basic documentation for statistic data that we provide to the user who which to contribute. We don't have any documentation for our dataformat (JSON files).

\subsubsection{What knowledge do you have in the system?}
\paragraph{ What are the sources of external knowledge?}

We rely on the knowledge of our contributors about statistics and maps and try to store it. We also depend on some \SageMath algorithms, for example to generate the combinatorial
objects.

 \paragraph{Can you point to implicit knowledge? Is it common knowledge?}

Our website is targeted at combinatorists. Even though we try to give all the basic definitions and information, it might be difficult to use for someone who has no
knowledge of these objects.

 \paragraph{What would you gain if it was made explicit/machine actionable?}

At the moment, our infrastructure is really \SageMath oriented (object printouts, names, etc). A language-neutral description of our objects might make it easier for interfaces
from other system to appear. The gain for us is that the more user we have (from different background), the more contributors we might get and so the more mathematically
pertinent our database is.

 \paragraph{Have you gone in this direction? How did you represent the knowledge then?}

Giving access to the statistics and maps data as JSON files was a first step in this direction.

 \paragraph{How do you collaborate on knowledge representation?}

By referencing those data (statistics and maps) and proposing unique identifiers that can be referenced from the outside (the same way the OEIS identifies integer sequences with a unique number).

\subsubsection{What software do you have?}
 \paragraph{What custom software are you running?}

We need the software \SageMath to run some computations: basically, generating the objects, printing them, etc. The statistic and maps code are usually integrated into \SageMath for consistency but it is not mandatory.

There is also some \FindStat specific code to run the website. Most of this code is just basic web-programming views of our database.

The database search is the heart of the service. It is a small algorithm that takes a user-given statistics and compares it to the database up to some maps.

\paragraph{In which language is your system written?}

Our server runs on \SageMath with some imported web packages, so it is written in \python. We use the \python wiki server \textsf{MoinMoin} as a backend and have written some customized \textsf{MoinMoin} plugins to run our service.

 \paragraph{How does it use the data and the knowledge?}

The data is stored in a SQL database. It is preloaded and precomputed when we launch the server then all computations are made on this preloaded data. We don't use the knowledge at this stage, we just basically request the database and compare numbers using some parameters. In the future, we might want to  use the knowledge we have on the maps (bijection, injection, surjection, involution, etc) to improve our algorithm.

\paragraph{How does your software act on represented knowledge?}

The software might put into light some mathematical relations between combinatorial objects but doesn't store them or anything like this.


\subsection{\SageMath}
\subsubsection{Overview at a high level of the system}
\SageMath \cite{sage} is  general purpose computational  (pure) mathematics software. It has 300 contributors and consists of 1.5 million lines of \python/Cython code, around 40000 function, and 4000 classes. It is distributed with hundreds of open source (math) software and libraries.

Most of the survey answers are given specifically for the \SageMath \textsf{category} framework, which is used to structure a lot of \SageMath code by exploiting as much as possible of the underlying mathematical structure.

\subsubsection{Available data}
\SageMath interfaces with a large collection of (optional) databases, usually coming from external software and possibly repackaged from external databases. Examples include \GAP databases, the OEIS \cite{oeis}, various databases of elliptic curves, etc.

The data format is heavily reliant on pickling (\python protocol for serialization).  Objects can be converted to strings and reconstructed. This is used for persistence, for storing in \SageMath databases, for data exchange between \SageMath instances. \SageMath comes with code to reconstruct the object and perform sanity checks. By default, pickling is done by class and stores plain data (no encapsulation). We aim for pickling by construction (which would require more semantics).

\subsubsection{What knowledge do you have in the system?}
The system effectively knows many mathematical properties and theorems, algorithms, ...

In designing the system, a few key points conditioned the design:
\begin{enumerate}
\item There are only a handful of fundamental concepts: operations (*, +, ...),  axioms (associativity, commutativity, ...), ...;
\item The richness arises in the combinations of these concepts (\emph{e.g.}~fields);
\item We use an existing language and its object oriented features for modelling and method selection.
\end{enumerate}

\paragraph{Sources of external knowledge?}
Each \SageMath contributor brings on specific mathematical knowledge about the objects studied, which might not be available to others in the collaboration.

\paragraph{Can you point to implicit knowledge?}
The algorithms rely heavily on the mathematical properties of the objects they manipulate.
\SageMath uses the Object Oriented features of \python.
The properties of a \SageMath object are specified by its \python class:
\begin{itemize}
\item what mathematical object does it represent?
\item how is it represented?
\item the class information is often of technical flavor, and complemented
  by additional information on its universe (parent, category)
\end{itemize}

\SageMath strives to model mathematics closely: not only matrices are instances of a specific classes and not plain list of lists,  but linear maps themselves are instances of specific classes and not just represented by matrices. This reduces the  risk of calling a meaningless function.

The abstract algebraic properties of an object (\emph{e.g.}~being a group or a field) are modelled relatively explicitly: objects know the names of their categories and axioms.  The meaning is essentially implicit except, in the good cases, informally in the documentation and as testing methods. The names of the operations are hardcoded, which leads to duplication (for instance between additive and multiplicative structures). The size of the code is linear in the number of methods, which in the current setup ought to grow exponentially as the complexity of the modeled objects increases.

It is not always made explicit which methods an object in a given category should implement: methods and operations are documented, but their exact specifications is not always completely defined or defined consistently across the class hierarchy.

Some theorems (\emph{e.g.}~Wedderburn) are embedded in actionable form,  but that information cannot be extracted or operated on.

\paragraph{Is it common knowledge?}

The meaning of the relevant categories and axioms (\emph{e.g.}~ring or associativity) is relatively well known by the users and developers.


\paragraph{What would you gain if it was made explicit/machine actionable?}
\begin{itemize}
\item Dynamic generation of documentation that the user can navigate
\item Sanity/correctness checks; proofs?
\item Semantic handles to communicate with other systems
\item Avoiding duplication (\emph{e.g.}~additive magmas / multiplicative magmas)?
\end{itemize}

\paragraph{Have you gone in this direction? How did you represent the knowledge then?}
The \textsf{category} framework for \SageMath goes in this direction.

\paragraph{How do you collaborate on knowledge representation?}
This is done through collaborative development of code, documentation and tests in the \SageMath sources.

\subsubsection{Available software}
The \SageMath library consists of 1.5 M lines of code (\python/Cython), and relies on hundreds of other software packages, in a myriad of languages.

The software, and particularly the \textsf{category} framework, builds on the available data and knowledge to construct a hierarchy of classes mirroring the categorical properties.  Those are used for code factorization, documentation, and generic testing. For instance, a computation relying on the lattice of categories would be helped by the conclusion that if X is a division ring and X is a finite set, then X is a finite field.


\subsection{\GAP}
\subsubsection{Overview at a high level of the \GAP system}

\GAP \cite{gap} is an open-source system for computational discrete algebra, with particular emphasis on Computational Group Theory. \GAP provides a programming language, a library of thousands of functions implementing algebraic algorithms written in the \GAP language as well as large data libraries of algebraic objects. It is used in research and teaching for studying groups and their representations, rings, vector spaces, algebras, combinatorial structures, and more.

\subsubsection{Available data}
\GAP includes a number of data libraries listed online\footnote{\url{http://www.GAP-system.org/Datalib/datalib.html}}. Some of them are part of the core \GAP system, while some others belong to \GAP packages. Their exact format may vary, but in all cases there are some text files with data and there is certain code responsible for processing particular pieces of information from those files. In some cases, the data library may only consist of the \GAP code which will construct \GAP objects on demand. Documentation is contained in the manual of the \GAP system or relevant packages; however, it may not contain technical details which in the best case will be placed in README files or as comments in the code. Usually once produced, the data libraries are only changed when new data are added to them. Existing data may be altered only in case of discovered errors.

\subsubsection{What knowledge do you have?}
Apart from the knowledge that is stored in data libraries as explained above, there is a wealth of knowledge about properties of algebraic objects, or how to compute them, encoded in method installation and code. This knowledge can often not easily be extracted from the system.

However, the \GAP type system has a number of advantages over a "standard" object-oriented model for algebraic computation. Among the most important are:

\begin{itemize}
\item Method selection based equally on the types of all arguments. Thus, in implementing an extension field $K$ of an existing field $L$, new methods for multiplying $kl$ and $lk$ can be added without any special support. Similarly, inheritance applies to all arguments equally.

\item Method selection can take account of information accumulated during the lifetime of an object. For instance, as soon as a group is found to be abelian, special methods for abelian groups will be applied to it. Similarly, when the size of a group has been determined once, not only is it remembered in case it is needed again, but different methods for other computations may be selected to take account of this information.
\end{itemize}

A central idea in the design of \GAP is that as much of possible of the core functionality should be polymorphic, so that it can be applied to any mathematical object with appropriate properties, without knowing the underlying representation. Thus if you create some new kind of \GAP object, supply a method for multiplying such objects, and claim that it is associative, then you should be able to make semigroups from your objects. With additional methods and some additional claims of algebraic properties, you can make groups, rings or algebras.

\subsubsection{What software do you have?}
\GAP has a kernel written in C. It implements:
\begin{itemize}
\item the \GAP language,
\item  an interactive environment for developing and using \GAP programs,
\item  memory management, and
\item  fast versions of time critical operations for various data types.
\end{itemize}
All the rest of the library of functions is written in the \GAP language. Packages (user contributed extensions) are mainly written in the \GAP language, but some also involve standalone executables. Some packages, for example, extend mathematical functionality of the system or add data libraries, while some others add infrastructural capabilities or links to other systems.

\subsection{\LMFDB}

\subsubsection{Overview at a high level of the \LMFDB system}

 The \emph{L-functions and Modular Forms DataBase} \cite{lmfdb} aims to aggregate and integrate computational and mathematical knowledge about L-functions and other number theoretic objects, and to present these complex and interconnected objects reliably while maintaining accessibility. At a mathematical level, this could help provide a uniform view of the concept of L-function, objects which can (sometimes conjecturally) be produced out of very different mathematical constructions. The collaboration involves around 50 mathematicians of varying coding skills and with different mathematical expertise.

\subsubsection{Available data}
The entirety of the data held by the \LMFDB is accessible through an API. One counts around 30 different types of objects stored, for a total of a few Tb. The data is downloadable directly.

The data is held in a MongoDB database server, holding around 30 or so databases, each with their own collections. It is held there as BSON (binary JSON), the internal format of Mongo documents.

Data that ends up in the \LMFDB has many different origins. Some are historical computations. Most are done in either \GAP, \Pari, \SageMath, \Magma, etc, with the person who coded these original sources a member of the \LMFDB who aims to make their data more accessible to their peers. Some of the data shown on the website is actually computed on the fly.

Data comes in through a variety of ad hoc ways, but essentially always transits through a JSON format before upload to the Mongo database. Updating is mostly done through some form of overwriting, but it is not uniform across the \LMFDB. In the best cases, the data is stored completely separately from the \LMFDB's own (Mongo) database, \emph{e.g.}~in a GitHub repository, under full revision control of text files, and there are scripts to populate the database from that.
 At some point there was discussion of allowing anyone to upload their data through an online form. This option has never been used seriously, and is not currently supported.


In general, proper referencing of data sources and documentation of its quality is a struggle, but there is recent improvement.
Progress has been made on these through a new collection of "data quality" pages. The intention is to have source, extent and quality reliably documented for the main sections of the database.

In addition, the various formats are in the process of being formalised\footnote{See \url{https://github.com/LMFDB/LMFDB-inventory}, with the most advanced example (for elliptic curves) at \url{https://github.com/LMFDB/LMFDB-inventory/blob/master/db-elliptic_curves.md}. The formalisation format itself does not have a spec.}.


\subsubsection{What knowledge do you have?}
\paragraph{What are the sources of external knowledge?}
Each participant in the \LMFDB brings on specific mathematical knowledge about the objects studied, which might not be available to others in the collaboration. The \LMFDB has the concept of \emph{knowls}, which are encyclopaedic bits of content integrated alongside the data, and editable collaboratively. These help converge on common definitions of the objects described.


\paragraph{Can you point to implicit knowledge? Is it common knowledge?}
There is a lot of implicit knowledge in the encodings chosen for the data (ad hoc formats and references), some of it is made explicit\footnote{emph{e.g.}~ at \url{http://www.LMFDB.org/knowledge/show/ec.conductor_label}}.
There is also a lot of implicit knowledge in the source code. There is little common knowledge across the collaboration, or at least there is a lot that is not common.

\paragraph{What would you gain if it was made explicit/machine actionable?}
The development process could probably be made more robust and efficient. The knowls currently serve as entry points for users and crucially also for onboarding future collaborators, as a stable basis for further collaboration. \LMFDB could gain in productivity, robustness and ultimately utility if this process could be extended a bit further along the chain of contributions.

\paragraph{Have you gone in this direction? How did you represent the knowledge then?}

The furthest the \LMFDB has gone into the direction of formalising knowledge is in modularising as much as possible of the mathematical knowledge through knowls, creating an ad hoc ontology to classify them, and aligning it to the mathematical data objects that are presented. The \LMFDB also tries to adhere to the concept of "one URL per object".

\paragraph{How do you collaborate on knowledge representation?}

Edition of the knowls requires an account, which the \LMFDB intends to offer to anyone who wishes to contribute. There is some versioning in place for knowls.

\subsubsection{What software do you have?}

 The \LMFDB is mostly written in \python, relies on \SageMath and \PariGP as libraries. It uses the database MongoDB (and possibly also an SQL one), uses the web framework Flask, and the templating engine Jinja.

 \paragraph{What custom software are you running?}

In a way \SageMath is custom, since lots of \LMFDB developers also contribute the relevant functionality to \SageMath. Otherwise a whole lot of the logic is embedded in the website code.

 \paragraph{How does the system use the data and the knowledge?}
Generally, a URL path will be associated to a Jinja template, requiring simultaneous fetching of pre-entered knowledge (knowls, Mongo DB), precomputed data (Mongo DB), and on-the-fly computation based on this precomputed data or existing functions implemented in some of the Computer Algebra Software already used.

\paragraph{Which knowledge is implicit in the data you have?}
A lot of information about the data encoding is implicit in the data itself. For instance, even if $[0,4,5,1]$ is known to represent a polynomial, depending on the context it might represent $4*x+5*x^2+x^3$ or $x(x-4)(x-5)(x-1)$.

\paragraph{ Which knowledge is implicit in the software you have?}

When populating templates, some of the mathematical knowledge might be really entered through the code, by completing the template in different ways according to the calling class (\emph{e.g.}~elliptic curve L-functions are of degree 2).

I don't know if it is relevant here but we are also accumulating a set of
bibliographic references and have already instituted a system for making
citations within knowls very easy.

%%% Local Variables:
%%% mode: latex
%%% TeX-master: "report"
%%% End:

%  LocalWords:  setcounter secnumdepth makeatletter subsubsection subsubsection itshape
%  LocalWords:  makeatother ednote combinatorists emph emph emph standardized formalized
%  LocalWords:  Unported finalized textsf textsf Cython serialization itemize flavor oeis
%  LocalWords:  Wedderburn factorization knowls onboarding templating lmfdb

\end{appendix}
\end{document}

%%% Local Variables:
%%% mode: latex
%%% TeX-master: t
%%% End:

%  LocalWords:  maketitle newpage tableofcontents newpage newcommand xspace ednote mathdb
%  LocalWords:  standardize dktheories concl printbibliography
