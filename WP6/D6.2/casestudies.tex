\section{Case Studies of Existing Databases}\label{sec:cases}

While our theoretical model of DK theories and our architectural design of virtual
theories are applicable to a wide variety of databases, in the scope of the OpenDreamKit
project we want to conduct a few case studies and connect to some databases in
particular. Our efforts so far are based on two of these case studies of which we want to
give a short overview below.

\subsection{LMFDB}\label{sec:lmfdb}

LMFDB \cite{lmfdb} is a database of objects from number theory. It mostly consists of
L-functions, but also has a number of other sub-databases. It is built on top of MongoDB
and as such uses JSON to model all of its data.

We have already implemented the schema and codec architecture above to build a virtual
theory of elliptic curves in LMFDB. Even though this only is a very small part of LMFDB,
this can serve as a template for future implementations of the remaining parts of
LMFDB. We started out with having just a few fields of the curves available inside \MMT,
however adding the relevant codecs and making more fields available proofed to be a quick
and easy job. This has shown us that we are heading in the right direction with our ideas
so far.

In the future we want to extend the coverage of the existing set of theories. This will
include writing more schema theories and possibly introducing more codecs. This will
likely also lead to some refactoring inside LMFDB itself, as the community for the first
time will try to semantically describe its entire dataset.

\subsection{OEIS}

OEIS \cite{oeis} stands for On-Line Encyclopedia of Integer Sequences. It is a collection
of around 250 thousand integer sequences that are stored as pure text form. The OEIS is
licensed under Creative Commons and thus freely accessible.

We have already semantified the pure text format and, among other things, this has helped
us finding new relations between the existing sequences. A more detailed look at our
previous work can be found in \cite{LuzKoh:fsarfo16} and we will not go into details
here. So far these efforts have helped us to understand how the OEIS database is
structured.

Similar to lmfdb we plan on integrating this into our virtual theories architecture. We
are considering building one DK theory per sequence, where the declarations in each theory
contain the known elements of the sequence. We also plan to integrate this with our
infrastructure on knowledge management services, such as MathHub and MathWebSearch.

\section{Querying}\label{sec:querying}

So far we have only concerned ourselves with accessing DK theories one declaration at a
time. This shows that DK theories are indeed useful, however in general one wants to be
able to access multiple declarations at once. Even though it is not the focus of this
report, we have had some thoughts about this.

Querying is the act of finding all declarations subject to some arbitary
criterion. Practically relevant queries can range from very simple queries -- such as find
all OEIS sequences containing a certain number -- to computationally intensive tasks, such
as find all elliptic curves with a conductor divisible by five\footnote{This particular
  example was given to us by John Cremona when asked for queries that the current
  architecture can not solve. }.

There gives a wide range of interesting questions that one might want to implement a query
engine for. Since most of the declarations that one wants to query over a big, one does
not want to evaluate each query by iterating over the entire set -- this is far to
slow. Thus a naive approach consists of iterating over all declarations beforehand,
finding all occuring values, and storing all of these inside a hash table. Such as index
can easily answer the first question from above. On top of this someone might want to find
sequences starting with a given integer. In this case we should either build a second
index that only contains starting numbers, or expand the first index in a smart
fashion. Another interesting query can be to find OEIS sequences containing arbitary
subsequences -- in which case the index no needs to contain any possible
subsequence. Eventually the index will explode -- the last addition would already scale
exponentially.

To be able to answer the second question above, one might want to take all occuring values
(in this case integers) and factorise them. Again we can store this in some form of
index. This can also be extended to polynomials -- which would allow users to search
polynomials based on their roots. Also in this situation it is very easy to underestimate
the complexity of the index.

It comes down to finding a good balance between interesting and useful queries and size
and scalability of the index. This in and of itself is a non-trvivial research task. In
the scope of K-theories we have solved parts of this question already. For example we have
built a general purpose query language for \MMT. We have also built MathWebSearch that
allows users to search for certain mathematical expressions within document corpera. We
will not discuss this here -- interested readers should take a look at \cite{Rabe:qlfml12}
and \cite{ODK-D6.1} for details.

In the future of the OpenDreamKit project we want to investigate this question for
DK-theories further. We want to take a deeper look at useful query languages. This could
consist of extending codecs from a value-translating mechanism to a query-translating
mechanism. That is we take a query from inside \MMT and then compile this into a database
query -- which can then be evaluated efficiently. It could also take another direction
entirely.

%%% Local Variables:
%%% mode: latex
%%% TeX-master: "report"
%%% End:
