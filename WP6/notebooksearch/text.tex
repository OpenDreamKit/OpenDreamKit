\documentclass[a4paper,twoside,12pt]{article}
\usepackage[hide]{ed}

\begin{document}
MathWebSearch \ednote{cite} is a search engine enabling semantic formula search. 
For example, given a query like $a^2 + b^2 = c^2$\ednote{Make formulae red} it will find documents containing $3^2 + 4^2 = 5^2$ or $x^2 + y^2 = z^2$.
It was originally developed by \ednote{TODO}. 

Conceptually, an instance of MathWebSearch consists of four components:
\ednote{Do we want to have a picture here?}
\begin{itemize}
    \item A program called a \textbf{Harvester} which is given a provided with a corpus of documents and extracts the set of formulae contained in it;
    \item the \textbf{MathWebSearch daemon} itself, which based on the harvested formulae, generates and maintains an index to  be able to answer search queries;
    \item an \textbf{query input parser}\ednote{Name?} which converts user input into a query the daemon can understand
    \item a \textbf{frontend} which sends user input to the query parser and daemon, and presents the query result to the user.  
\end{itemize}

- for this task: built infrastructure for managing large harvests and building interfaces
    - auto-updating of index
    - deployment setup => being able to spin up new instances more easily
    - api wrapper (more flexible frontends)
    - flexible way of dropping in harvesters
- use this for formulae in Jupyter notebooks
    - generated as output from sage (using github api + latexml)
    - build a query language using native latex formulae
- additionally built an NLab Harvester

\ednote{@Nicolas: how does this actually perform?}
\end{document}