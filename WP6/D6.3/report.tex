\documentclass{deliverablereport}

\usepackage[style=alphabetic,backend=biber]{biblatex}
\addbibresource{../../lib/kbibs/kwarcpubs.bib}
\addbibresource{../../lib/kbibs/extpubs.bib}
\addbibresource{../../lib/kbibs/kwarccrossrefs.bib}
\addbibresource{../../lib/kbibs/extcrossrefs.bib}
\addbibresource{../../lib/deliverables.bib}
% temporary fix due to http://tex.stackexchange.com/questions/311426/bibliography-error-use-of-blxbblverbaddi-doesnt-match-its-definition-ve
\makeatletter\def\blx@maxline{77}\makeatother

% \usepackage{local}

\deliverable{dksbases}{dkstheories}
\deliverydate{10/09/2016}
\duedate{30/11/2016 (Month 15)}

\author{Michael Kohlhase, Florian Rabe, Tom Wiesing, Paul-Olivier Dehaye, Dennis M\"uller}

\begin{document}
\begin{abstract}
  \TODO{short explanation: this was not anticipated at the time of
    writing the proposal and why}
  The participants of \WPref{dksbases} identified the interoperability of \pn systems to
  be one of the most critical steps in creating a VRE toolkit and consequently prioritized
  tasks \taskref{dksbases}{data-assessment}, \taskref{dksbases}{data-triform},
  \taskref{dksbases}{data-design} and organized a series of workshops and code-maratons to
  develop a semantic foundation for system interoperability and simultaneously test it in
  implementations.

  As a consequence, we have completed in parallel the initial design
  of D/K/S-bases (for deliverable \delivref{dksbases}{design}), the
  initial implementation of a \DKS base format based on OMDoc/MMT
  together and the implementation \TODO{of???} based on the MMT system
  (both for \delivref{dksbases}{dkstheories}), all activities fueling
  each other.  \delivref{dksbases}{dkstheories} was thus completed
  three months ahead of schedule.

  Due to the resulting tight coupling between
  \delivref{dksbases}{design} and \delivref{dksbases}{dkstheories},
  and for the convenience of the reader, we have decided to report on
  both deliverables together; see the report for deliverable
  \delivref{dksbases}{design}.
\end{abstract}

\maketitle
\githubissuedescription

\end{document}

%%% Local Variables:
%%% mode: latex
%%% TeX-master: t
%%% End:

%  LocalWords:  maketitle newpage tableofcontents newpage newcommand xspace ednote mathdb
%  LocalWords:  standardize dktheories concl printbibliography pn textit mmt mitm emph
%  LocalWords:  WPref dksbases prioritized taskref organized delivref dkstheories
%  LocalWords:  githubissuedescription
