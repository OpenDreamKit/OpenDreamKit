
% maths at the core of technology and innovation
We live in an innovation-driven society. The key enabling tool for our
advances is mathematics. For just a few examples, the global
positioning system (GPS) needs relativistic mathematics, Computer
Assisted Tomography (CAT scanning) in health care is based on solving
mathematical inverse problems, mobile phone connectivity depends on
combinatorial optimization algorithms, while the modern communications
infrastructure relies on cryptographic algorithms derived from number
theory. At the core of each of these innovations there is the
underpinning mathematics that is implemented through algorithms. Such
innovations have been made possible by investments into pure and
applied mathematics research. Engineering, Science and business
innovation than enrich society and menkind are based on these
mathematical foundations.

% recent developments
Modern mathematical research is increasingly accelerated by and
enabled through computational software, such as Sage --- an open source
mathematics software system --- and Jupyter, an open source browser-based
notebook with support for code, text, mathematical expressions, inline
plots and other media. These tools have the potential to
revolutionise the way computational research is conducted. 

% what is this proposal about - aim
In this project, we will provide mathematicians and scientists with a
generic unified toolkit, the Open Digital Research Environment Toolkit
for the Advancement of Mathematics (\TheProject), that allows
(i)~building of specific Virtual Research Environments (VREs) and (ii)
more effective communication of research.


% How will we achieve this?
We will achieve this by investing into creation of a \emph{toolkit of
  software components} from which \emph{tailored VREs can be assembled
  flexibly} to cater for a variety of needs in maths, science and
engineering.  We are at a critical point providing an opportunity to
do so: emerging collaboration tools for code sharing, such as github,
allow to bring together very large sets of open source code developers
working on the same codebase. Throughout this project we will use and
extend open source code, thus exploiting existing coding effort,
benefitting from future open source contributions and delivering a
step change in computational science by creating \TheProject.

% other things we should say somewhere
In more detail, VREs based on \TheProject can combine symbolic
mathematics, automatic code generation, numerical computation, data
bases, post-processing and visualisation in a single document. The
document consists of a number of executable cells which contain code
that can be interactively executed, and the output of the code,
combined with text and equations as necessary to describe/document the
results. These executable documents can be shared with others and
fully define and document a computational study -- providing step
changes in effective research, research communication and
reproducibility in computational science.

The VREs build on \TheProject will provide end-to-end toolchains that
link fundamental mathematics to domain specific specialised
computation, this bridging the gap between fundamental research and
technology, and paving the way towards faster commercialisation of
basic research.

% other things we do to make this a holistic project [maybe expand here]
As part of this project, we will also study the social challenges
associated with large-scale open source code development, and develop
demonstrator VREs based on \TheProject.

% about the team
The \TheProject team a Europe-wide collaboration that assimilates a
leading body mathematicians and transdisciplinary computational
researchers with a track record of delivering innovative open source
software solution. All partners are code developers and end-users of the
toolkik.

% conclusion
By focusing on a toolkit rather than a monolithic VRE, and by
concentrating the efforts on improving and unifying existing general
purpose building blocks, and in the forefront \Jupyter, \TheProject
will simultaneously maximize sustainability and broad impact. Indeed,
even if the primary target users are \emph{researchers in
  mathematics}, the set of beneficiaries extends to scientific
computing, physics, chemistry, biology, engineering, medicine, earth
sciences and geography, and include researchers as well as teachers
and practitioners in the industry. \TheProject will further foster
development models that are mutually beneficial to academia and highly
innovative SME's.






\clearpage


%%% Local Variables:
%%% mode: latex
%%% TeX-master: "proposal"
%%% End:
