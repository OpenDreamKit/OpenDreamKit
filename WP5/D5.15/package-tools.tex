\subsection{Improvements to \GAP Package Tools}\label{sec:package-tools}

Sections \ref{sec:packages} and \ref{meataxe64} among others have shown
that developments important to our goal of a flexible, widely portable high performance
mathematical software system can increasingly occur in packages, as
well as in the core \GAP system. For that reason, as well as pursuit
of the broader goals of \ODK to support a user/developer community, we have attached
considerable priority to a supporting and promoting a healthy \GAP
package ecosystem. We have supported this through the developments
described in subsection~\ref{testing}, together with community
supporting activities described in D2.15, \TODO{check D2.15} with
good effect. During the \ODK
project we have observed rapid growth of the number of \GAP packages
and of the number of active package
authors as well as steady improvements in the quality of package testing and more frequent
package updates. This has both been supported by, and encouraged, the
development of a wide range of tools to help package authors.

In this section we will first introduce these tools, and then discuss the
impact which they had on the package ecosystem together with the increased
adoption of the open development model by package authors,

\TODO{Maybe use footnotes or href to move or hide the long URLs}
\subsubsection{ReleaseTools and GitHubPagesForGAP}
Our collaborator Max Horn (University of Siegen) developed {\sf ReleaseTools}
and {\sf GitHubPagesForGAP} (\url{https://github.com/gap-system/ReleaseTools/}
and \url{https://github.com/gap-system/GitHubPagesForGAP}) which allow package
authors to fully automate the release procedure of a \GAP package hosted on GitHub
and the maintainance of a website for it hosted on GitHub pages,
reducing the time needed to publish it to a matter of minutes.

\subsubsection{The Example Package}
The \GAP package {\sf Example} by Werner Nickel, Greg Gamble and
Alexander Konovalov \cite{example}, which acts as a package template,
has been consistently maintained to track the development of
the packages infrastructure and serves to demonstrate a model use of modern
development tools.
%It switched to using {\tt TestDirectory} to demonstrate how to use 
%multiple test files in a GAP package, and migrated
%to GitHub, using {\sf ReleaseTools/GitHubPagesForGAP}
%for new releases. The guidance for package authors, formerly offered
%as an appendix to the {\sf Example} package manual, was moved to the GAP
%Reference manual in \GAP~4.9 release. Finally, the {\sf Example} package
%demonstrates how to set up Travis CI and CodeCov. Package authors
%may copy its configuration files and adapt them with minimal changes.

\subsubsection{PackageMaker}
To offer an alternative way to create a package,
Max Horn developed a GAP package called {\sf PackageMaker}
(\url{https://github.com/gap-system/PackageMaker}) (not yet redistributed with \GAP),
which provides an interactive  ``package wizard''. This allows a
user to fill in the basic details 
of the intended package and then creates all of the files making up
the basic structure of the package. A particularly relevant
strength of this tool is that it hides the additional complexities of
packages which include C or C++ code, removing a barrier to faster implementation
of key kernels.

\subsubsection{AutoDoc}
\GAP packages are required to provide documentation.  The key central
format for \GAP documentation is XML-based, defined in the {\sf
  GAPDoc} package, and while convenient in many ways, can be rather
cumbersome and error prone to write by hand.
In 2012 Sebastian Gutsche 
and Max Horn (both now at the University of Siegen) developed the 
{\sf AutoDoc} package \cite{autodoc} which creates
documentation from simple structured comments in the source code,
similat to pydoc or javadoc.

By now, {\sf AutoDoc} has become very reliable and configurable, and
is in use by 89 GAP packages.
% for building some parts or their whole manuals.
% the number 89 was determined by 
%
% grep 'AutoDoc(' */makedoc.g | wc -l
%
% a note "parts or their whole manuals" is essential - 
% it is possible to build a GAPDoc manual using AutoDoc
% and automatically generate title page, updating the
% metadata from those in PackageInfo.g, avoiding the need
% to duplicate details in two places. The rest of the
% manuals for such packages may still remain in XML.

\subsubsection{Continuous Integration Support for Package Authors}
Using Docker containers described in Section~\ref{sec:gap-infra}
and shown on Figure~\ref{fig:gap-docker} enabled us to make the
results of regression tests for \GAP packages publicly
available via Travis CI. Using a prebuilt docker image 
greatly reduced the runtime of the test and allowed us to
have a separate test configuration for each package,
which is easily discoverable via the Travis CI interface.

Figure~\ref{fig:gap-docker-pkg-tests} shows a fragment of the dashboard
from \url{https://github.com/gap-system/gap-distribution} which 
indicates tests status. In this snapshot, it happens that a package
was depending on undefined behaviour of a \GAP library function, which
was changed, resulting in package tests failing. The prompt detection
of this interaction made it easy to diagnose and 
the author was immediately notified and reacted with a quick
fix. Previously this could have required many weeks of tedious
correspondence to resolve.


% \comment{this must be clear from picture}
%We separate passing and failing tests into different
%Travis jobs, and use that to monitor their status and check that
%changes in GAP do not break package tests. 

\begin{figure}[!ht]
    \centering
    \includegraphics[width=\textwidth]{images/gap-docker-pkg-tests}
    \caption{Dashboard for Travis tests of \GAP packages (GitHub view)}
    \label{fig:gap-docker-pkg-tests}
\end{figure}


%% \comment{not sure if  Figure~\ref{fig:gap-infra-travis} adds anything new
%% - may be deleted?}
%% The setup for such tests is organized in the gap-infra GitHub organisation
%% (\url{https://github.com/gap-infra}. Figure~\ref{fig:gap-infra-travis} shows a 
%% screenshot of a dashboard which indicates tests status.
%% \comment{All these screenshots may be superfluous - I made them to show
%% what could be included, we can remove some or cut parts of them out later}

%% \begin{figure}[!ht]
%%     \centering
%%     \includegraphics[width=\textwidth]{images/gap-infra-travis}
%%     \caption{Dashboard for Travis tests of \GAP packages (Travis CI view)}
%%     \label{fig:gap-infra-travis}
%% \end{figure}

% \comment{this may be too technical -- no I like it SL}
One of the extensions of the package testing framework, greatly 
facilitated by packages migrating to GitHub, was adding tests not
only for the latest official releases of GAP packages, but also
for their \emph{development} versions.
% https://travis-ci.org/gap-infra/gap-docker-pkg-tests-stable-4.10-devel
% https://travis-ci.org/gap-infra/gap-docker-pkg-tests-master-devel
In this case, the onus to check test outcomes is on package authors,
but they are able to quickly see how the changes that they have made to
a package but not released yet work in different settings (the
released and development versions of \GAP, with various different
combinations of other packages, for instance). Again this leads to
prompt diagnosis and easy fixing of problems, and improves the quality
of released packages.

%% Figure~\ref{fig:gap-packages-travis} shows a screenshot of a 
%% report on the activity in the gap-packages organization 
%% for the last month (for the current version of 
%% this report, check \url{https://travis-ci.org/gap-packages?tab=insights}).

%% \begin{figure}[!ht]
%%     \centering
%%     \includegraphics[width=\textwidth]{images/gap-packages-travis}
%%     \caption{Activity of Travis tests in gap-packages GitHub organization}
%%     \label{fig:gap-packages-travis}
%% \end{figure}

Finally, Figure~\ref{fig:gap-packages-codecov} show the code coverage
leaderboard. We aim at ensuring that code coverage for packages is 
not worse than for the core GAP system, but packages presented there 
put a lot of efforts in ensuring that their coverage is close to 100\%. 

\begin{figure}[!ht]
    \centering
    \includegraphics[width=\textwidth]{images/gap-packages-codecov}
    \caption{Codecov leaderboard for in gap-packages GitHub organization}
    \label{fig:gap-packages-codecov}
\end{figure}

\subsection{Open Development of \GAP Packages -- Measure of Uptake
  and Quality}
\TODO{Explain why this is not in a WP2 report?}
\comment{}

As an evidence of the widespread adoption of the open development model, 
at the time of writing only 4 packages redistributed with GAP do not have a public 
source code repository (to compare, in Month 38 there were still 23 such 
packages). In some cases, that involved non-trivial efforts to establish 
contacts with authors who left academia, getting their permission and 
establishing package license, populating a new repository with releases 
history using archives of past releases, and adopting the packages to
be collectively maintained by the GAP Group. This helps to preserve the
code in usable state, necessary to reproduce computational results, and
may eventually result in finding new maintainers for these packages.
Moving to GitHub made it easier to get new contributors to existing
packages. As Figure~\ref{fig:gap-package-tests}
shows, the number of individuals involved in package development
increased from 158 in Month 3 to 196 in Month 46. 

Figure~\ref{fig:gap-package-releases} shows the number of \GAP packages
included in some \GAP release and their distribution by the year of their
release. At the beginning of the project, GAP 4.7.9 published in November
2015 included 119 packages, 52\% of which were released in 2014--2015.
Now, GAP 4.10.2 included 145 packages, 88\% of which were released in
2018--2019. The ``long tail'' of packages that haven't been updated for
a long time is gradually reducing.
%(we take measures of not breaking
%existing functionality, but old packages are usually not picking up
%new improvements; and may prevent us from withdrawing obsolete
%functionality). 
Larger proportion of recent releases means that 
packages are updated to make use of the recent \GAP developments,
and are responsive to reports about detected problems.
This is greatly facilitated by hosting packages on GitHub which 
allows GAP developers a very efficient way to provide help by submitting 
pull requests and even publishing releases on behalf of package authors.
%who prefer to overlook mathematical functionality and general development of the
%package but do not want to dive into technicalities of release wrapping.

\begin{figure}[!ht]
    \centering
    \includegraphics[width=\textwidth]{images/gap-package-releases}
    \caption{Number of \GAP packages and their release year}
    \label{fig:gap-package-releases}
\end{figure}


Over the same time, not only package releases became more frequent, but also their quality
improved. If in November 52\% of packages did not provide a standard 
test at all, presently 78\% of packages do have a standard test 
% https://github.com/gap-packages/gap-packages.github.io/issues/9
so their authors are responsible for regular testing them 
with GAP releases and development versions.
Some of them use their own setup for continuous integration
(like e.g. the \href{https://homalg-project.github.io/}{\sf homalg} project,
which uses \href{https://circleci.com/}{CircleCI}).
At the moment of writing, in the stable-4.10 branch,
standard tests pass for 102 packages
% https://travis-ci.org/gap-infra/gap-docker-pkg-tests-stable-4.10
and fail for 12.
% https://travis-ci.org/gap-infra/gap-docker-pkg-tests-stable-4.10-staging 
(it may be worth to note what "fail" means - not serious bugs
but different output, or incompatibility with some other package
not loaded by default).

\begin{figure}[!ht]
    \centering
    \includegraphics[width=\textwidth]{images/gap-package-tests}
    \caption{Number of \GAP packages, their authors, and packages with tests.}
    \label{fig:gap-package-tests}
\end{figure}

% grep github currentPackageInfoURLList | wc -l
Another illustrative statistics is the number of \GAP packages
with the websites hosted on GitHub pages (some of them use
their own setup, different from {\sf GitHubPagesForGAP}.
The following table gives this number for August of each respective year:

\begin{center}
\begin{tabular}{| c | c | c | c | c | c |} 
\hline
Year & 2015 & 2016 & 2017 & 2018 & 2019 \\
\hline
Number of packages & 14 & 45 & 61 & 92 & 124 \\
\hline
\end{tabular}
\end{center}

\subsection{The \GAP package manager}\label{pkg-manager}

\subsubsection{Background}

For many years, the installation of packages in \GAP has been an entirely manual
process.  If a user requires a package that is not currently installed on their
system, they are required to visit the \GAP website, or the website of the
required package, download an archive or repository and extract it into their
package directory.  In many cases, they must also compile the package before
use, in a process that is not consistent between packages.  Worst of all, any
dependencies must also be installed individually, including all these steps for
each one.

This awkward installation procedure is mitigated by the inclusion of a large
number of packages with the main release archive (145 packages were included
with \GAP 4.10.2).  This solution masks the problem for the majority of users,
who do not require any more specialized packages.  However, it also results in a
1.6 GB standard installation, of which the packages make up 1.4 GB.  
\TODO{maybe add how many lines of GAP code and C code are there}
While some packages are used by almost every \GAP user because they
enhance the performance of core \GAP features, 
only a subset of the others will be needed by any specific end
user. This may be explicit (when they load a package relevant to their
research interests) or implicit (to satisfy package dependencies).
\TODO{Adding packages as a user to a central installation. Installing development
  versions of packages}

The possibility of a more sophisticated package management system for \GAP has
been discussed for some time, and was realized with the creation of {\sf
  PackageManager} \cite{PackageManager} in
Month 37, by Michael Torpey.  Development started after a discussion at
\GAP Days Fall 2018 (\url{https://www.gapdays.de/gapdays2018-fall/}) and an
initial release was made before the end of that meeting.  New features have been
added in various further releases across the past year.

\TODO{Give some kind of justification for why this is being reported
  in WP5}

\subsubsection{Design}

{\sf PackageManager} is a \GAP package itself, which is operated from inside a
\GAP session.  This approach to package management was chosen over an external
manager, since it allows users to install and remove packages without ever
leaving the \GAP prompt.  The first advantage of this is that
users have a comfortable environment, without needing to learn how to use
external tools; and the second is that, in most cases, packages can be installed
and loaded without restarting the session, as shown in Figure \ref{fig:pkgman-sample-b}.
This means that users' workflow is interrupted as little as possible, and they do
not need to save and reload their work.

\begin{figure}[!ht]
    \centering
    {\tiny
\begin{verbatim}
gap> InstallPackage("digraphs");
#I  Getting PackageInfo URLs...
#I  Retrieving PackageInfo.g from https://gap-packages.github.io/Digraphs/PackageInfo.g ...
#I  Downloading archive from URL https://github.com/gap-packages/Digraphs/releases/download\
/v0.15.4/digraphs-0.15.4.tar.gz ...
#I  Saved archive to /tmp/tmcLM2FI/digraphs-0.15.4.tar.gz
#I  Extracting to /home/user/.gap/pkg/digraphs-0.15.4 ...
#I  Checking dependencies for Digraphs...
#I    io >=4.5.1: true
#I    orb >=4.8.2: true
#I  Running compilation script on /home/user/.gap/pkg/digraphs-0.15.4 ...
true
gap> LoadPackage("digraphs", false);
true
\end{verbatim}
    }
    \caption{Installing a package using {\sf PackageManager}.}
    \label{fig:pkgman-sample-b}
\end{figure}

The main functions provided by {\sf PackageManager} are \texttt{InstallPackage},
\texttt{UpdatePackage}, and \texttt{RemovePackage}.  The first,
\texttt{InstallPackage}, takes a string which can be in any of several different
forms: the URL of an archive, repository, or package info file, or simply the
name of a package.  In the case of a URL, the appropriate file is retrieved
using an internet connection, and used to install the package; if a package name
is given, it is looked up in a pre-defined list of package URLs and installed
from there.  In most cases, the latest released version of a package is
installed; a user can specify an older version by giving an explicit archive
URL, or a development version by specifying a branch in a
version control repository.  % currentPackageInfoURLList

The ability to install a package given only its name is useful, but requires
some important design decisions: how do we decide which packages to support, and
which versions do we install by default?  One option would be to have a
centralized repository (as is common for the {\sf APT} package manager) which
contains a carefully checked set of package versions that are known to work with
the installed version of \GAP and with each other; this has the advantage of
stability and control, but may make it harder to get the latest software
versions.  The other option is to store links to appropriate locations on the
package websites, allowing the most recently released version to be installed
without explicit approval from the \GAP authors; this approach is closer to the
more liberal {\sf pip} package manager for the Python language.
In the end, the latter approach was taken for the following reasons:
\begin{itemize}
\item approved package versions for a given release are likely to be included
  with the \GAP installation anyway, and it is unlikely that users would need
  to reinstall these;
\item the latter approach requires no additional infrastructure, since a
  \texttt{currentPackageInfoURLList} is already maintained for testing purposes;
\item this approach encourages active use, and therefore testing, of recent releases.
\end{itemize}
This decentralized approach has therefore been adopted for the moment, though
future development could move easily towards a different approach.

\subsubsection{Features}

Over the course of {\sf PackageManager}'s development, it has accumulated
various features that automate as much of the installation process as possible.
During installation, the following operations are carried out automatically:
\begin{itemize}
\item any kernel module is configured and compiled, if present;
\item package documentation is built from source if necessary;
\item prerequisites are installed using \texttt{prerequisites.sh} if present;
\item all missing package dependencies are installed, or recompiled if broken;
\item basic checks are performed to verify that installation was successful.
\end{itemize}
This has resulted in a reasonably smooth installation process, with very little
friction encountered by a user who tries to get a new package working.  It is
now a deposited package in \GAP, and is therefore distributed in the main
archive with each release.

\TODO{Mention that it uses ~/.gap/pkg ?  Or is that too technical? AK: in some
non-technical words explain that the location will be visible from multiple
GAP installations, and packages will remain even if the main GAP installation
is deleted (e.g. during upgrade)}

\subsubsection{Best practice}

{\sf PackageManager} was written using modern open-source software technologies
and practices.  In the report on OpenDreamKit D1.5, in Section 4.3.2, there are
two checklists for best practices: one for software engineering, and one for
dissemination.  The package fulfills all the requirements in both lists, as summarized in
Tables \ref{tab:pkgman-se-check} and \ref{tab:pkgman-diss-check}.

\begin{table}[ht]
  \renewcommand{\arraystretch}{1.2}
  \begin{tabular}{|p{5.1cm}|c|p{9.5cm}|}\hline
    Version control & \checkmark & Git \\ \hline
    Tests & \checkmark & \multirow{2}{*}{\GAP test suite with 99\% code coverage} \\ \cline{1-2}
    Automated tests & \checkmark & \\ \hline
    Continuous integration & \checkmark & Travis runs test suite for every commit \\ \hline
    Automatic building of releases & \checkmark & {\sf ReleaseTools} (see above) \\ \hline
  \end{tabular}
  \vspace{0pt}
  \caption{Software engineering good practice checklist for {\sf PackageManager}}
  \label{tab:pkgman-se-check}
\end{table}

\begin{table}[ht]
  \renewcommand{\arraystretch}{1.2}
  \begin{tabular}{|p{5.1cm}|c|p{9.5cm}|}\hline
    Host code publicly & \checkmark & \url{https://github.com/gap-packages/PackageManager} \\ \hline
    Reference Manual (APIs) & \checkmark & GAPDoc/Autodoc, and hosted on package website \\ \hline
    Tutorial (for beginning users) & \checkmark & \multirow{2}{9.5cm}{Quick examples in readme and manual, before more detailed documentation} \\ \cline{1-2}
    Examples & \checkmark & \\ \hline
    Live interactive online demos & \checkmark & Binder link included in readme \\ \hline
    Support mechanisms & \checkmark & Github issues \\ \hline
    How to cite the output? & \checkmark & Explained on readme and website \\ \hline
    Installation mechanism & \checkmark & Distributed with \GAP, and can update itself \\ \hline
    High level description accessible to non-experts & \checkmark & In readme, and chapter 1 of manual \\ \hline
    URLs/Blog/etc to and from OpenDreamKit project & \checkmark & OpenDreamKit link and logo in readme \\ \hline
    Grant acknowledgments & \checkmark & Acknowledged in readme \\ \hline
    Open Source license & \checkmark & GPL v2 or later \\ \hline
    Workshop & \checkmark & \multirow{2}{9.6cm}{Demos in \textit{CIRCA Seminar} (USTAN, Month 43) and \textit{Workshop on Mathematical Data} (Cernay, Month 48)} \\ \cline{1-2}
    Engaging users & \checkmark & \\ \hline
  \end{tabular}
  \vspace{0pt}
  \caption{Dissemination good practice checklist for {\sf PackageManager}}
  \label{tab:pkgman-diss-check}
\end{table}

\subsubsection{Other uses}

Aside from making the on-demand installation of individual packages easier, {\sf
  PackageManager} has implications for the future of how \GAP is distributed.
There are currently no plans to abandon the established approach of distributing
deposited packages with the \GAP archive by default, but the presence of {\sf
  PackageManager} would also make it possible to distribute an archive of
minimal size, containing only the four required packages and {\sf
  PackageManager} itself, installing others as needed.  Furthermore, it
can be used by other systems that interact with \GAP in order to manage
installations more easily.  The \cocalc system now contains {\sf PackageManager}
as part of its \GAP 4.10.2 installation, enabling each user to install any
packages they need without \cocalc operators being forced to manage these
installations separately.  This is an example of positive two-way interaction
with \cocalc, and sets as a good precedent for interaction with other
systems.

