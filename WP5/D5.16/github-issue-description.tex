\section*{\texorpdfstring{Deliverable description, as taken from GitHub
issue
\href{https://github.com/OpenDreamKit/OpenDreamKit/issues/114}{\#114} on
2019-07-01}{Deliverable description, as taken from GitHub issue \#114 on 2019-07-01}}\label{deliverable-description-as-taken-from-github-issues-114-on-2019-07-01}

\begin{itemize}
\tightlist
\item
  \textbf{WP5:}
  \href{https://github.com/OpenDreamKit/OpenDreamKit/tree/master/WP5}{High
  Performance Mathematical Computing}
\item
  \textbf{Lead Institution:} CNRS
\item
  \textbf{Due:} 2019-08-31 (month 48)
\item
  \textbf{Released:} 2019-06-03
\item
  \textbf{Task}: T5.1
  (\href{https://github.com/OpenDreamKit/OpenDreamKit/issues/99}{\#99})
\item
  \textbf{Nature:} Demonstrator
\item
  \textbf{Proposal:}
  \href{https://github.com/OpenDreamKit/OpenDreamKit/raw/master/Proposal/proposal-www.pdf}{p.52}
\item Links to
  \underline{\textbf{\href{https://github.com/OpenDreamKit/OpenDreamKit/raw/master/WP5/D5.16/report-final.pdf}{Final report}}} and
    \underline{\textbf{\href{http://pari.math.u-bordeaux.fr/pub/pari/unstable/pari-2.12.0.alpha.tar.gz}{PARI-2.12.0\vphantom{p}}}}.
\end{itemize}
\medskip

\textbf{The \Pari library} is a state-of-the-art library for number theory and
an important
component of the \Sage computational system. Together with the \texttt{gp}
command line interface and the \texttt{gp2c} compiler, it forms
the \PariGP package. This deliverable implements a generic parallel engine in
the \PariGP system, uses it inside the system to implement fast parallel
variants of existing sequential code and exports it for library users. The
released \PariGP suite (PARI-2.12) makes those improvements and new features
available for the community, in particular \Sage users and all softwares
using the \Pari library.

The MultiThread engine transparently supports: 1) sequential computation, 2)
POSIX threads (for a single multicore machine) and 3) Message Passing Interface
(MPI, for clusters). It is used throughout the library to improve a large
number of high-level mathematical algorithms, including fast linear algebra over
the rationals or cyclotomic fields, fast Chinese remainders, resultants,
primality proofs, discrete logarithms, modular polymomials and isogeny-based
algorithms, motivic $L$-functions\dots Those implementations scale
transparently between single core, multicore and massively parallel machines.
