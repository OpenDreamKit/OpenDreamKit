\documentclass{deliverablereport}

\usepackage{doi}
\usepackage{hyperref}
\hypersetup{
  colorlinks=true,
  urlcolor=cyan,
  linkcolor=blue,
  citecolor=blue,
}

\usepackage{natbib}

\deliverable{hpc}{LinBox-algo}
\deliverydate{31/08/2018}
\duedate{31/08/2018 (M36)}
\author{Cl\'ement Pernet and Jean-Guillaume Dumas}

\begin{document}
\maketitle
% This will be the abstract, fetched from the github description
\githubissuedescription

% write the report here
%%%%%%%%%%%%%%%%%%%%%%%%%%%%%%%%%ù
\section{Algorithmic innovations}

\subsection{Triangular factorization}

\cite{DuPe18}
\cite{DPS17}

\subsection{Quasiseparable matrices}

Exploiting some structure in a matrix to speed up computations is a whole field
in linear algebra algorithmic. Quasiseparable matrices are  structured by a
bounding condition on the rank of any of their submatrices below or above
the main diagonal. It is a well studied field in numerical linear algebra, as
these matrices occur sevral major applications, such as solving particule
interraction, or generalized eigenvalues problems.
In exact linear algebra this class of structured matrices seemed to be absent
from the litterature and software ecosystem.

We introduced this class to the field in~\cite{Per16} and~\cite{PeSt18} where we
contributed with two new storages for these matrices and the related algorithmic
to compute with. The key innovations there are the following:
\begin{enumerate}
\item the first reduction in time complexity for the basic arithmetic with these
  matrices to the fast matrix multiplication complexity: $O(ns^{\omega-1})$
  where $\omega$ is the exponent of matrix multiplication, and $s$ is the order
  of quasiseparability.
\item the first flat (i.e. non-hierarchical) compact representation for these
  matrices reaching the best space and time complexities. This was made possible
  thanks to a non-trivial conncection with the notion of rank profile matrix,
  which we developped in~\cite{DPS17}.
\end{enumerate}

\subsection{Outsourced computing security}

\subsubsection{Certificates}
\cite{DLP17}
\cite{LNPRR18}
\subsubsection{Secure multiparty computation}

\cite{DFLLOPP18}
%%%%%%%%%%%%%%%%%%%%%%%%%%%%%%%%%ù
\section{Software release and integration}

\bibliographystyle{plainnat}
\bibliography{linbox}

\end{document}

%%% Local Variables:
%%% mode: latex
%%% TeX-master: t
%%% End:

