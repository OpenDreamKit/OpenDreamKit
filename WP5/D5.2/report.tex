\documentclass{deliverablereport}

\usepackage[style=alphabetic,backend=bibtex]{biblatex}
\addbibresource{../../lib/kbibs/kwarcpubs.bib}
\addbibresource{../../lib/kbibs/extpubs.bib}
\addbibresource{../../lib/kbibs/kwarccrossrefs.bib}
\addbibresource{../../lib/kbibs/extcrossrefs.bib}
\addbibresource{../../lib/deliverables.bib}
%\addbibresource{../../lib/publications.bib}
\addbibresource{rest.bib}
% temporary fix due to http://tex.stackexchange.com/questions/311426/bibliography-error-use-of-blxbblverbaddi-doesnt-match-its-definition-ve
\makeatletter\def\blx@maxline{77}\makeatother

\deliverable{hpc}{pythran-sage}
\deliverydate{02/27/2017}
\duedate{02/27/2017 (Month 18)}
\def\pn{OpenDreamKit}
\author{}

\begin{document}
\maketitle
%There is a large ecosystem of mathematical software systems and knowledge bases.
Individually, these are optimized for particular domains and functionalities, and together they cover many needs of practical and theoretical mathematics.
However, each system specializes on one particular area, and it remains very difficult to solve problems that need to involve multiple systems.
Some integrations exist, but they are ad-hoc and have scalability and maintainability issues.
In particular, there is not yet an interoperability layer that combines the various systems into a virtual research environment (VRE) for mathematics.
  
The OpenDreamKit project aims at building a toolkit for such VREs.
It suggests using a central system-agnostic formalization of mathematics (Math-in-the-Middle, MitM) as a mathematical pivot point for semantic-preserving translations in the needed interoperability layer.
In this \papertype, we report on a series of case studies that instantiates the MitM paradigm with the systems \GAP, \Sage, \LMFDB, and \Singular to perform distributed computation in group, ring, and number theory.
 
Our work involves massive practical efforts, including a novel formalization of computational group theory, improvements to the involved software systems, an extension of the underlying knowledge management system to cope with large theories, and a novel mediating system that sits at the center of a star-shaped integration layout between mathematical software systems and knowledge bases.

Together with deliverable report\textbf{D6.8}, this report describes the implementation and initial evaluation of the MitM integration and interoperability paradigm initially envisioned in deliverables \textbf{D6.2} and \textbf{D6.3}.
The MitM paradigm constitutes the core development goal of \textbf{WP6} and the curated content described in this report enables running non-trivial integration case studies.
In the future we hope to further consolidate content, increase coverage of alignment, and greatly extend the reach of the integration both interms of OpenDreamKit systems covered as well as knowledge available in the MitM Core ontology.

%%% Local Variables:
%%% mode: visual-line
%%% fill-column: 5000
%%% mode: latex 
%%% TeX-master: "report"
%%% End:

%  LocalWords:  optimized formalization papertype textbf textbf textbf

\strut\githubissuedescription
\newpage\tableofcontents\newpage

\section{Introduction}

The aim of this project is to optimize the overall performance of Cython code
that uses Numpy arrays.

Cython is a compiler for the Python and Cython language . It basically converts
Python code to a C/C++ module that calls directly the C Python API. The Cython
language is an extension of Python that allows to write C-like code within a
Python module. This code is directly inserted into the generated module. This
allows to write C python module with the flexibility of a Python script. This
is used in numerous Python project, with SageMath being one of them.

In Cython, when operations are done on Numpy arrays, Cython relies on the
original Numpy package to compute them. This involves a fall back to the Python
interpreter. It thus misses several optimisation opportunities, espacially with
complex expressions. Optimizing these Numpy calls can improve the overall
performances of applications that rely on Numpy to do their computation work.
The interest for the OpenDreamKit project is the overall performance gain that
could be achieve within various SageMath workload.

The Pythran project is a Python to C++ compiler, that aims at optimizing
scientific Python. It thus supports only a subset of the Python language.
It also has a full C++ implementation of a major set of the Numpy API. Some of
the advantage of this implementation is that it supports expression templates
and SIMD instructions. Expression templates allow to "fuse" loops that can
occurs when expressions with multiple operators are computed. For instance,
the expression $a+b*c$ will originally be transformed by Cython in two calls: one
for the multiplication of $b$ by $c$, and one for the addition of the result of
this multiplication and the addition by $a$. Each call will end-up in one loop,
that will read memory, compute the operation and write back to memory. The
second loop will have the same pattern. In nowadays architecture, memory
bandwidth is often the limiting factor in this kind of operation. It is thus
really interesting to merge these loops, and load/store the memory only once.
Expression templating is a C++ technique that allows to evaluate expressions
only when they are stored to memory. Thus, in this case, the two loops will be
automatically "merged" by the C++ compiler, and we'll get an optimized version
of this code. Note that this technique is used for instance by the C++ wrapper
of the GMP library.

\section{Cython and Pythran integration}

The project has been focused on using this Pythran backend for numpy arrays in
Cython when possible. Indeed, Pythran has a few limitations regarding the numpy
arrays it can handle:

\begin{itemize}
  \item array "views" are not supported. That means that arrays must be stored
    in contiguous memory. Fortran and C-style formats are supported.
  \item the endianess of the integers must be the same as the one of the
    targeted architecture (note that Cython has the same limitation)
\end{itemize}

We thus need to be able to fall-back to the Cython implementation if we are not
in one of those cases.

The integration within Cython works this way:

\begin{itemize}
  \item at the function level, for every argument that is a numpy array and
    supported by Pythran, we change its type by a fused type
    \footnote{\url{http://cython.readthedocs.io/en/latest/src/userguide/fusedtypes.html}}.
    This fused type is either a Pythran numpy buffer or the original Cython
    buffer type
  \item for variables defined as numpy array, we change them directly to a
    Pythran buffer, if their type and endianess are supported.
\end{itemize}

Cython has a comprehensive suite test regarding the Numpy features it supports.
This test suite is still valid after this integration.

\section{Usage}

A new flag {\tt --np-pythran} has been added to Cython that enables the usage
of Pythran for Numpy operations. It will generate a C++ file that uses the
optimized Numpy functions that are in Pythran.

Then, when compiling the final extension, a path to an existing Pythran
installation must be provided.

Here is an example of usage using {\tt distutils}:

\begin{lstlisting}[language=python]
from distutils.core import setup
from Cython.Build import cythonize

setup(
    name = "My hello app",
    ext_modules = cythonize('hello_pythran.pyx', np_pythran=True)
)
\end{lstlisting}

Basically, a flag {\tt np\_pythran} has been added to the {\tt cythonize} call
that will enable the usage of the Pythran backend. It needs to have the Pythran
module installed.

\section{Benchmarks}

We did some benchmarks to see the benefits of the Pythran integration for Numpy
operations. These benchmarks have been done using an Intel Core i7-6700HQ.

\subsection{Float computation}

Here is a code snippet that does some computation on floating-point values:

\begin{lstlisting}[language=python]
import numpy
cimport numpy
def float_comp(numpy.ndarray[numpy.float_t, ndim=1] a, \
             numpy.ndarray[numpy.float_t, ndim=1] b):
    return numpy.sqrt(numpy.sqrt(a*a+b*b))
\end{lstlisting}

The figure~\ref{fig:float_bench} shows the compute time for various sizes for
{\tt a} and {\tt b}, using the original Cython mode, then Cython with the
Pythran backend, and finally Cython with the Pythran backend and SIMD
instructions.

\begin{figure}[h]
  \caption{\label{fig:float_bench} Float computation benchmark (logarithmic scales)}
  \includegraphics{benchs/float/graph.pdf}
\end{figure}

For instance, for $N = 1000000$, we have a speedup of ~2.4x using the Pythran
backend (against the original Cython version), and ~3.7x using SIMD
instructions (still against the original Cython version).

\subsection{Harris}

This benchmark is based on the code available here:
\url{https://raw.githubusercontent.com/serge-sans-paille/numpy-benchmarks/master/benchmarks/harris.py}.

The adapted Cython code is the following:

\begin{lstlisting}[language=python]
import numpy
cimport numpy

def harris(numpy.ndarray[numpy.float_t, ndim=2] I):
  cdef int m = I.shape[0]
  cdef int n = I.shape[1]
  cdef numpy.ndarray[numpy.float_t, ndim=2] dx = \
    (I[1:, :] - I[:m-1, :])[:, 1:]
  cdef numpy.ndarray[numpy.float_t, ndim=2] dy = \
    (I[:, 1:] - I[:, :n-1])[1:, :]

  cdef numpy.ndarray[numpy.float_t, ndim=2] A = dx * dx
  cdef numpy.ndarray[numpy.float_t, ndim=2] B = dy * dy
  cdef numpy.ndarray[numpy.float_t, ndim=2] C = dx * dy
  cdef numpy.ndarray[numpy.float_t, ndim=2] tr = A + B
  cdef numpy.ndarray[numpy.float_t, ndim=2] det = A * B - C * C
  return det - tr * tr
\end{lstlisting}

The figure~\ref{fig:harris_bench} shows the compute time for various sizes of
the 2D array I ($N \times N$). We've got a ~25 \% improvement with an array of
$4096 \times 4096$ floats as input using the Pythran backend. Moreover, in this
case, it seems that this code does not really benefits from SIMD instructions.

\begin{figure}[h]
  \caption{\label{fig:harris_bench} Harris computation benchmark}
  \includegraphics{benchs/harris/graph.pdf}
\end{figure}

\subsection{Convolution}

We used this example from the Cython documentation:
\url{http://cython.readthedocs.io/en/latest/src/tutorial/numpy.html}.

The input we use is the following:

\begin{lstlisting}[language=python]
import numpy as np
N = 1000
f = np.arange(N*N, dtype=np.int).reshape((N,N))
g = np.arange(81, dtype=np.int).reshape((9, 9))
\end{lstlisting}

The results are the following:

\begin{itemize}
  \item for the classical Cython version: 155ms
  \item for the Cython version using the Pythran backend: 150ms
  \item for the Cython version using the Pythran backend using SIMD instructions: 149ms
\end{itemize}

It seems the Pythran backend don't manage to benefit a lot from SIMD
instructions in this case. We thus still got an average speedup of ~3 \%.

\section{Links}
\label{sec:links}

Modifications that were necessary to the Pythran project have been accepted and
merged into its master branch (see
\url{https://github.com/serge-sans-paille/pythran/pull/629},
\url{https://github.com/serge-sans-paille/pythran/pull/628},
\url{https://github.com/serge-sans-paille/pythran/pull/616} and
\url{https://github.com/serge-sans-paille/pythran/pull/614}).

The modifications necessary within the Cython project are currently being reviewed
(see \url{https://github.com/cython/cython/pull/1607}).

\printbibliography
\end{document}

%%% Local Variables:
%%% mode: latex
%%% TeX-master: t
%%% End:

%  LocalWords:  githubissuedescription newpage tableofcontents newpage printbibliography
