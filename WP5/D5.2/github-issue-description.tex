\section*{\texorpdfstring{Deliverable description, as taken from Github
issue
\href{https://github.com/OpenDreamKit/OpenDreamKit/issues/115}{\#115} on
2017-02-27}{Deliverable description, as taken from Github issue \#115 on 2017-02-27}}\label{deliverable-description-as-taken-from-github-issue-115-on-2017-02-27}

\begin{itemize}
\tightlist
\item
  \textbf{WP5:}
  \href{https://github.com/OpenDreamKit/OpenDreamKit/tree/master/WP5}{High
  Performance Mathematical Computing}
\item
  \textbf{Lead Institution:} Université Joseph Fourier (UJF)
\item
  \textbf{Due:} 2017-02-28 (month 18)
\item
  \textbf{Nature:} Demonstrator
\item
  \textbf{Task:} T5.7
  (\href{https://github.com/OpenDreamKit/OpenDreamKit/issues/105}{\#105})
  Pythran
\item
  \textbf{Proposal:}
  \href{https://github.com/OpenDreamKit/OpenDreamKit/raw/master/Proposal/proposal-www.pdf}{p.51}
\item
  \textbf{\href{https://github.com/OpenDreamKit/OpenDreamKit/raw/master/WP5/D5.2/report-final.pdf}{Final
  report}}
\end{itemize}

The \href{http://python.org}{\texttt{Python}} programming language is
widely used in the development of computational mathematics systems like
\href{http://sagemath.org}{\texttt{SageMath}} for its expressiveness and
flexibility. Yet, as an interpreted language, it suffers from inherent
inefficiencies.

Over the years several tools have been developed to overcome this
barrier. A major player is \href{http://cython.org}{\texttt{Cython}},
which is both an extension of the Python language, and a compiler
generating compilable \texttt{C} code. At the cost of additional work
from the developer (e.g.~type annotations), \texttt{Cython} can deliver
performances similar to that of a compiled language. It's being used
intensively in \texttt{SageMath}. Another emerging player is
\href{https://pythonhosted.org/pythran/}{\texttt{Pythran}}, a
\texttt{Python} to \texttt{C++} compiler for a subset of the
\texttt{Python} language, with a focus on scientific/numerical
computing. It takes advantage of type inference features of \texttt{C++}
as well as multi-cores and SIMD instruction units to deliver high
performance without the need for additional work from the developer. In
particular, it includes a \texttt{C++} implementation of a major subset
of the \texttt{Numpy} API, optimized for speed, with support for
expression templates that minimize the number of memory transfers needed
to compute complex expressions (\href{http://numpy.org}{\texttt{Numpy}}
is the fundamental package for scientific computing with
\texttt{Python}). However, unlike \texttt{Cython}, \texttt{Pythran} does
not support user defined classes, a key feature in systems like
\texttt{SageMath}.

This deliverable is a step toward taking the best of both worlds, and
helping bridge the gap between numerical and exact computing. It
proposes to incorporate \texttt{Pythran} support for \texttt{Numpy}
within \texttt{Cython}, which consequently provides high performance
numerical linear algebra to high level mathematical software developers,
especially within \texttt{SageMath}.

As an illustration, the new \texttt{Pythran} backend in \texttt{Cython}
achieves a speed-up of about 4 on the following typical \texttt{Numpy}
based function:

\begin{verbatim}
def sqrt_sum (numpy.ndarray[FLOATTYPE_t, ndim=1] a,
              numpy.ndarray[FLOATTYPE_t, ndim=1] b):
    return numpy.sqrt(numpy.sqrt(a*a+b*b))
\end{verbatim}
