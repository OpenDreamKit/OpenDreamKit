\hypertarget{deliverable-description-as-taken-from-github-issue-112-on-2019-08-27}{%
\section*{\texorpdfstring{Deliverable description, as taken from Github
issue
\href{https://github.com/OpenDreamKit/OpenDreamKit/issues/112}{\#112} on
2019-08-27}{Deliverable description, as taken from Github issue \#112 on 2019-08-27}}\label{deliverable-description-as-taken-from-github-issue-112-on-2019-08-27}}

\begin{itemize}
\tightlist
\item
  \textbf{WP5:}
  \href{https://github.com/OpenDreamKit/OpenDreamKit/tree/master/WP5}{High
  Performance Mathematical Computing}
\item
  \textbf{Lead Institution:} Université Joseph Fourier
\item
  \textbf{Due:} 2019-08-31 (month 48)
\item
  \textbf{Nature:} Demonstrator
\item
  \textbf{Task:} T5.3
  (\href{https://github.com/OpenDreamKit/OpenDreamKit/issues/101}{\#101})
\item
  \textbf{Proposal:}
  \href{https://github.com/OpenDreamKit/OpenDreamKit/raw/master/Proposal/proposal-www.pdf}{p.
  51}
\item
  \textbf{\href{https://github.com/OpenDreamKit/OpenDreamKit/raw/master/WP5/D5.14/report-final.pdf}{Upcoming
  report}}
  (\href{https://github.com/OpenDreamKit/OpenDreamKit/raw/master/WP5/D5.14/}{sources})
\end{itemize}

\hypertarget{context}{%
\section*{Context}\label{context}}

Computational linear algebra is a key tool delivering high computing
throughput to applications requiring large scale computations. In
numerical computing, dealing with floating point arithmetic and
approximations, a long history of efforts has lead to the design of a
full stack of technology for numerical HPC: from the design of stable
and fast algorithms, to their implementation in standardized libraries
such as LAPACK and BLAS, and their parallelization on shared memory
servers or supercomputers with distributed memory.

On the other hand, computational mathematics relies on linear algebra
with exact arithmetic, i.e. multiprecision integers and rationals,
finite fields, etc. This leads to significant differences in the
algorithmic and implementations approaches. Over the last 20 years, a
continuous stream of research has improved the exact linear algebra
algorithmic; simultaneously, software projects such as LinBox and
fflas-ffpack were created to deliver a similar set of kernel linear
algebra routines as LAPACK but for exact arithmetic.

The parallelization of these kernels has only been partially addressed
in the past, and was mostly focused on shared memory architectures.

\hypertarget{goal-of-the-deliverable}{%
\section*{Goal of the deliverable}\label{goal-of-the-deliverable}}

This deliverable aims at enhancing these libraries so that they can
exploit various large scale parallel architectures, including large
multi-cores, clusters, and accelerators. The target application is the
solver of linear systems of the field of multi-precision rational
numbers. For this application, several algorithmic approaches will be
studied and experimented, namely a Chinese Remainder based solver and a
p-adic lifting solver. The former exposes a much higher level of
parallelism in its tasks, while the latter requires asymptotically much
fewer operations.
