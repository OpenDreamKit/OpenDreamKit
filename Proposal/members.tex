\TOWRITE{ALL}{Proofread 4. Members of the consortium pass 2}

\subsection{Participants}

\eucommentary{Please provide, for each participant, the following (if available):\\
\begin{compactitem}
\item
a description of the legal entity and its main tasks,
with an explanation of how its profile matches the tasks in the proposal;
\item
a curriculum vitae or description of the profile of the persons,
including their gender, who will be primarily responsible for carrying
out the proposed research and/or innovation activities;
%
this includes a description of the profile of the to-be-recruited personnel
\item
a list of up to 5 relevant publications, and/or products, services
(including widely-used datasets or software), or other achievements
relevant to the call content;
\item
a list of up to 5 relevant previous projects or activities, connected
to the subject of this proposal;
\item
a description of any significant infrastructure and/or any major items
of technical equipment, relevant to the proposed work;
\item
any other supporting documents specified in the work programme for this call.
\end{compactitem}}

\subsection*{Université Paris Saclay}

University Paris-Sud is among the 40 top universities worldwide in the
2013 Shanghai ranking, and is one of the two best French research
universities. With about 27000 students, 1800 permanent teaching staff
and 1300 permanent research scientists from national research
organisations (CNRS, Inserm, INRA), it is the largest campus in
France. Since 2006, scientists from the University were awarded two
Fields medals, one Nobel Prize and a number of other international
(European Inventor Award 2013, Wolf Prize 2010, Holweck Prize 2009,
Japan prize 2007) and national prizes.  The Université Paris-Sud has a
complete array of competences, ranging from the purest of exact
sciences to clinical practices in medicine, covering life and health
sciences, legal sciences and economics. Research at the Université
Paris-Sud, an essential part of academic understanding, is
complemented by research activities with a high valorisation
potential. Research contracts and partnership with companies make the
Université Paris-Sud a key actor and a major player in French
research.  The University is located close to the Plateau de Saclay,
the largest cluster of public and private R\&D institutions in France
(with ca. 16000 research staff), and is one of the core members of the
University Paris Saclay – a world class university and a
world-renowned research and innovation hub.

In the context of this project, the Université Paris Saclay is the
home of one of the largest group of Sage developers worldwide. It's a
member of the Open Source Thematic Group of the Systematic Paris
Region Systems and ICT Cluster. The University also hosts a major
research group working on proof assistants (Coq), which naturally
opens the door for reaching toward this neighbor community.

% The main participants have accumulated 15 years of experience of
% collaborative open source software development for mathematics
% leadership, and community animation.

\subsubsection*{Curriculum vitae of the investigators}

%\input{CVs/Nathann.Cohen}
\paragraph{Florent Hivert}

% months=6
%
% Fair evaluation of the number of months you will be spending on this
% specific project along the four years

% salary=9000 Euro / Month
%
% Approximate monthly gross salary (in term of total cost for the
% employer). If you are uncomfortable having this information in a
% public file, you can alternatively send the information to Nicolas,
% or to your institution leader if the latter will be willing to fill
% in himself the budget forms on the eu portal.

% The above information will be used to evaluate the cost of the
% project for the institutions. You may remove the above comments once
% you have filled in the months= and salary= lines.

% About half a page of free text; for whatever it's worth, you may see
% Nicolas.Thiery.tex for an example.

Professor at the Laboratoire de Recherche en Informatique, Florent
Hivert is a senior researcher in Algebraic Combinatorics with 29
papers in international journals and 15 communications in
international conferences.
% including an invited lecture at FPSAC'10

With 100 tickets (co)authored and as many refereed, Hivert is himself a core
Sage developer, with contributions including key components of the Sage
infrastructure (documentation, automated test, combinatorics infrastructure,
paralellism, ...), specialized research libraries.

%\begin{participant}[type=R,PM=6,gender=male]{Samuel Lelièvre}
Maître de conférences since 2006 at
Laboratoire de mathématique d'Orsay, Université Paris-Sud,
% PhD Rennes 2004 under Anton Zorich,
% post-doc Warwick with Vladimir Markovic,
Samuel Lelièvre is an established researcher in Dynamics and Geometry,
with 10 papers published in international journals including
Annales scientifiques de l'École normale supérieure,
Crelle, GAFA, Geometry and Topology.
He participated in three ANR projects, and has collaborators
in France, Israel, the UK, the USA.
His research in Dynamics and Geometry
often involves explicit and experimental approaches,
for which he writes code in order to explore
combinatorial objects such as square-tiled surfaces,
translation surfaces, group actions, group presentations.

He uses and actively promotes \Sage since 2010.
He is in the top 15 contributors of the AskSage
questions and answers forum.
He co-organised six international meetings including two \Sage Days,
presented \Sage at PyCon-FR-2011 in Rennes,
supervised \Sage tutorials twice at the GDR-IM yearly school
for French PhD students at the interface of Mathematics and
Computer Science, and at the CIMPA/ICPAM school Bobo2012
on Discrete Mathematics (Bobo Dioulasso, Burkina Faso, 2012).
\end{participant}
%%% Local Variables:
%%% mode: latex
%%% TeX-master: "../proposal"
%%% End:

%\begin{participant}[type=leadPI,PM=6,salary=4200]{Viviane Pons}
  Maître de Conférences at the Laboratoire de Recherche en Informatique, Viviane Pons is a
  young researcher in Algebraic Combinatorics. She defended her thesis in 2013 and has 3
  papers in international journals and 3 communications in international
  conferences including a talk at PyCon US 2015. Before committing herself to research, 
  she spent two years working in industry as a Java and web developer.

  She discovered Sage during her first sage-days in 2010 and has since been an active user
  and contributor with 10 (co)authored tickets improving the support of combinatorial
  objects into Sage. She is very involved in the promotion of Sage, participating in
  Sage-Days and proposing Sage introduction tutorials or Sage presentations in various
  conferences. She is also one of the main developers of the project FindStat dedicated
  to databases in combinatorics.
\end{participant}
%%% Local Variables:
%%% mode: latex
%%% TeX-master: "../proposal"
%%% End:

%\paragraph{Nicolas M. Thiéry}


\subsubsection*{Publications, achievements}

\TODO{Il faut être plus formel dans la description des projets
  antérieurs : Acronyme, titre, agence de financement, durée.  Pareil
  pour les publi - auteurs, titre exact, année etc.}

\begin{enumerate}
\item Lead of the Sage-Combinat software project.
\item Coauthoring of the open source book ``Calcul Mathématique avec
  Sage'', the first of its kind comprehensive introduction to
  computational mathematics in Sage for education.
\item XXX tickets contributed to Sage.
\end{enumerate}


\subsubsection*{Previous projects or activities}

\begin{enumerate}
\item Home of six one week-long Sage Days workshops.
\item Co-Organizer of \TODO{XXX} Sage Days.
\item Founder and regular organizer of a bimonthly Sage User Group
  meeting in the greater Paris area.
\item Expertise exchanges with Logilab
\item \TODO{XXX}
\end{enumerate}

\subsubsection*{Significant infrastructure}

The Université Paris Sud hosts the lead developers of the open source
cloud infrastructure \texttt{Stratuslab} and its reference
infrastructure (\TODO{XXX cores}). The participants are regular users
of this infrastructure, and in close contact with the developers.

\clearpage
\begin{sitedescription}{UB}
% PIC: 
% see: http://ec.europa.eu/research/participants/portal/desktop/en/orga

% See ../proposal.tex, section Members of the Consortium for a
% complete description of what should go there

\begin{itemize}
\item INRIA, LaBRI, IMB
\item journal de th\'eorie des nombres
\item Plafrim and a mesocentre Avakas
\item Several softwares developped in Bordeaux: pari/GP, tulip, etc
\end{itemize}


\subsubsection*{Curriculum vitae}

% Curriculum of the personnel at this institution

\begin{participant}[type=leadPI,PM=12,salary=2500,gender=male]{Vincent Delecroix}
  \TOWRITE{VD}{Finalize CV}
  CNRS researcher at the Laboratoire Bordelais de recherche en informatique, Vincent
  Delecroix is a junior researcher in Number theory, Combinatorics and Dynamical systems.

  X publications.  He has several international collaboration (England, Mexico,
  United-States).

  \begin{itemize}
\item  Bobo
\item  cours Sage Bordeaux
\end{itemize}

Delecroix is a major developer of Sage in various components:
\begin{itemize}
\item  integers
\item  combinatorics
\item dynamical systems
\end{itemize}
\end{participant}
%%% Local Variables:
%%% mode: latex
%%% TeX-master: "../proposal"
%%% End:


%\begin{participant}[type=R,PM=6,salary=5700,gender=male]{Adrien Boussicault}
  Maître de Conférences at the LaBRI (Laboratoire Bordealais de Recherche en 
  informatique), Adrien Boussicault is a young researcher in Algebraic and 
  Enumerative Combinatorics. He has 8 papers in international journals.
  
  He is a young contributor in Sage. He wrote 3 tickets to implement 
  combinatorial objects. He co-organized Sage-Combinat Days 57.
\end{participant}
%%% Local Variables:
%%% mode: latex
%%% TeX-master: "../proposal"
%%% End:


\begin{participant}[type=PI,PM=12,salary=9800,gender=male]{Karim Belabas}
is a Professor of Mathematics in Bordeaux (France) since 2005. He is
a senior researcher in Number Theory, with particular interest in
computational and effective aspects. Karim has published about 25 articles
in international journals, including papers in Duke and Compositio (one of
which co-authored with Manjul Bhargava, 2014 Fields Medalist), and
edited the book ``Explicit methods in number theory''.

Karim was head of the Pure Math teaching department in Bordeaux from 2010 to
2014 and is vice-head of the Institut de Math\'ematiques de Bordeaux since
2015. He has held a grant from the French ANR worth \euro$200$k (ALGOL
project, 2007--2011) and was part of a \euro{2.5}m Marie-Curie Research
Training Network (GTEM project 2006--2010); he was responsible for three
deliverables and supervision of an early stage researcher during her PhD in
the Work Package ``Constructive Galois Theory''. He has (co-)organised 8
international conferences, including a special Trimester at IHP in 2004 and
an influential recurrent workshop on ``Explicit methods'' in Oberwolfach
(every two years since 2007) and 5 \PariGP Ateliers. He has (co-)supervised
11 PhD students and about 15 masters students.

Karim is a leading computational number theorist in France. He
is the project leader for the \PariGP free computer algebra system since
1995, which has had a major impact on hundreds of publications. He is one of
the system's main developers (about 60000 lines of code written, most of the
documentation, and 1300 bug-tracking tickets authored).
\end{participant}
%%% Local Variables:
%%% mode: latex
%%% TeX-master: "../proposal"
%%% End:

\paragraph{Bill Allombert}

% months=6
% salary=YYY

CNRS Ing\'enieur de Recherche. One of the main pari developer.


%?? others ??


\subsubsection*{Publications, products, achievements}

Some recent Publications :
\begin{enumerate}
\item 
Belabas, Karim; Friedman, Eduardo; Computing the residue of the Dedekind
zeta function.  Math. Comp. 84 (2015), no. 291, 357–369. 

\item
The PARI Group; PARI/GP version 2.7.0, Bordeaux, 2014,
http://pari.math.u-bordeaux.fr/.

\item
Belabas, Karim et al. Explicit methods in number theory. Rational points and
Diophantine equations, 179 pages, Panoramas et Synthèses 36, 179p., 2012.

\item
Allombert, Bill; Bilu, Yuri and Pizarro-Madariaga, Amalia;) CM-Points on
Straight Lines , to appear in "Analytic Number Theory" (dedicated do H. Maier),
Springer. 
\end{enumerate}



\subsubsection*{Previous projects or activities}

Current grants:
\begin{enumerate}
\item
 ANR PEACE (2012-2015)
    Goal: The discrete logarithm problem on algebraic curves is one of the rare
    contact points between deep theoretical questions in arithmetic geometry and
    every day applications. On the one side it involves a better understanding,
    from an effective point of view, of moduli space of curves, of abelian
    varieties, the maps that link these spaces and the objects they classify.
    On the other side, new and efficient algorithms to compute the discrete
    logarithm problem would have dramatic consequences on the security and
    efficiency of already deployed cryptographic devices. 

\item
ERC starting grant ANTICS (2011-2016) 
    Goal: "Rebuild algorithmic number theory on the firm grounds of theoretical
    computer science".
    Challenges: complexity (how fast can an algorithm be?), reliability
    (how correct should an algorithm be?), parallelisation.
\end{enumerate}

\subsubsection*{Significant infrastructure}
\TOWRITE{VD}{this still needs to be done}

Two center for computations: Plafrim and Avakas.
\end{sitedescription}

%%% Local Variables:
%%% mode: latex
%%% TeX-master: "../proposal"
%%% End:

%  LocalWords:  sitedescription th eorie des nombres Plafrim mesocentre Avakas developped
%  LocalWords:  subsubsection Belabas Synthèses Allombert Bilu Pizarro-Madariaga Maier
%  LocalWords:  parallelisation TOWRITE

\clearpage
\begin{sitedescription}{JU}
Jacobs University is a private Anglo-Saxon style research university.  It opened in 2001
and has an international student body (1320 students from 115 nations as of 2011).  The
KWARC (KnoWledge Adaptation and Reasoning for Content~\cite{KWARC:online}) Group headed by
{\emph{Prof.\ Dr.\ Michael Kohlhase}} specialises in knowledge management for STEM.
Formal logic, natural language semantics, and semantic web technology provide the
foundations for the research of the group.

\subsubsection*{Curriculum vitae}
% Curriculum of the personnel at this institution
\paragraph{Ursula Martin}

% months=3
%
% Fair evaluation of the number of months you will be spending on this
% specific project along the four years

% salary=XXX
%
% Approximate monthly gross salary (in term of total cost for the
% employer). If you are uncomfortable having this information in a
% public file, you can alternatively send the information to Nicolas,
% or to your institution leader if the latter will be willing to fill
% in himself the budget forms on the eu portal.

% The above information will be used to evaluate the cost of the
% project for the institutions. You may remove the above comments once
% you have filled in the months= and salary= lines.

Dr. Michael Kohlhase is full professor for Computer Science at Jacobs University Bremen
and an associate adjunct professor at Carnegie Mellon University.

He studied pure mathematics at the Universities of T"ubingen and Bonn (1983 - 1989) and
continued with computer science, in particular higher-order unification and automated
theorem proving (Ph.D. 1994, Saarland University). 

His current research interests include knowledge representation for mathematics,
inference-based techniques for natural language processing, and computer-supported
education. He has pursued these interests during extended visits to Carnegie Mellon
University, SRI International, and the Universities of Amsterdam, Edinburgh, and Auckland.

Michael Kohlhase is recipient of the dissertation award of the Association of German
Artificial Intelligence Institutes (AKI; 1995) and of a Heisenberg stipend of the German
Research Council (DFG 2000-2003). He was a member of the Special Research Action 378
(Resource-Adaptive Cognitive Processes), leading projects on both automated theorem
proving and computational linguistics. Michael Kohlhase was trustee of the Conference on
Automated Deduction (CADE), Mathematical Knowledge Management (MKM), and the CALCULEMUS
conference, he is a member of the W3C Math working group, president of the OpenMath
Society, and the general secretary of the Conference on Intelligence Computer Mathematics
(CICM).

%%% Local Variables:
%%% mode: latex
%%% TeX-master: "../../proposal"
%%% End:


 



\begin{picv}[PM=6]{Florian Rabe}
  is a post-doctoral research fellow at Jacobs University Bremen.  He completed his PhD in
  2008 and his habilitation in 2014 and holds the venia legendi.

  He has worked on the formal representation and management of mathematical knowledge for
  10 years.  He was a lead researcher in the LATIN project (2009-2012), which produced a
  highly modular and integrated library of formal languages for knowledge representation.
  He is currently a principal investigator in the OAF project, which builds on LATIN to
  produce an archive of libraries of formal mathematical knowledge.

  He is the creator and main developer of the MMT language and system, which are the
  backbone of both LATIN and OAF.  MMT has been developed for 8 years with contributions
  from $>10$ people and currently consists of $>30,000$ lines of Scala code.

  He served in the organization committee of 2 and the program committee of 6
  international conferences (2 as track chair) on intelligent computer mathematics, and
  has organized 4 international workshops on module systems and libraries for mathematical
  knowledge.  He has authored 65 research papers (11 in international journals) and has
  supervised 17 undergraduate and graduate theses.
\end{picv}

%%% Local Variables:
%%% mode: latex
%%% TeX-master: "../proposal"
%%% End:

\begin{participant}[PM=36]{Christian Maeder}
  Dr. Christian Maeder is a research developer at Jacobs University. He has extensive
  experience in designing and implementing logic-based software system. He is the lead
  implementor of the HETS system.
\end{participant}

%%% Local Variables:
%%% mode: latex
%%% TeX-master: "../proposal"
%%% End:


 



\begin{participant}[type=JRes,PM=24,salary=5900,gender=male]{Mihnea Iancu}
  Mihnea Iancu is a third-year doctoral student at the KWARC group. He is the lead
  implementor of the MathHub.info system. He has worked extensively on the representation
  for formal and informal mathematical knowledge in the MMT syste.m
\end{participant}

%%% Local Variables:
%%% mode: latex
%%% TeX-master: "../proposal"
%%% End:


 




\subsubsection*{Relevant previous experience:}

The KWARC group is the lead implementor of the OMDoc (Open Mathematical Document) format
for representing mathematical knowledge \cite{Kohlhase:OMDoc1.2} and redeveloped its
formal core in the OMDoc/MMT format~\cite{RabKoh:WSMSML13}. The latter has been
implemented in the MMT system~\cite{MMTSVN:on,RabKoh:WSMSML13} which provides efficient
implementations of the computational primitives such as type checking, flattening, and
presentation at a logic/foundation-independent level.  The group has developed services
powered by such semantically rich representations, different paths to obtaining them, as
well as platforms that integrate both aspects.  \emph{Services} include the adaptive
context-sensitive presentation framework provided by the MMT API and the semantic search
engine MathWebSearch\cite{KohSuc:asemf06,ProKoh:mwssofse12}. 

Semantic services can be integrated into the documents generated from OMDoc/MMT
representations, making them into ``active documents'', i.e. documents that are
interactive and adaptive to the user and situation.  For \emph{obtaining} rich content,
the group investigates assisted manual editing \cite{JucKoh:sidesc10:biblatex} as well as
automatic annotation using linguistic techniques \cite{GinJucAnc:alsaacl09}.  Finally,
KWARC has developed the \textsf{MathHub.info} portal a community-based library and
knowledge management system for flexiformal libraries, which can be used for semantic
publishing and eLearning~ \cite{KohDavGin:psewads11,MathHub:on,IanJucKoh:sdm14}.

The \textsf{OMDoc/MMT} knowledge representation format and the \textsf{MathHub.info}
system will an important basis for the developments Work Packages 4 and 6.

Michael Kohlhase has initiated and led the CALCULEMUS! IHP-Research and Training Network
and participated in the FP6 IST MoWGLI (Mathematics on the Web: Get it by Logic and
Interfaces) project, the FP6 CSA Once-CS (Open Network of Centres of Excellence in Complex
Systems), The FP7 EDC project WebALT (Web Advanced Learning Technologies).

\subsubsection*{Specific expertise:}
\begin{compactitem}
\item Modelling formal structures of mathematical knowledge in a web-scalable way.
\item Transforming large collections of legacy scientific publications to semantically
  structured markup.
\item Designing user interfaces for authoring and interacting with mathematical knowledge.
\end{compactitem}

\site{JU} lead \WPtref{dksbases} and tasks related to active and structured documents as
user interfaces (\WPtref{UI}) to the math VRE to be developed in \TheProject. It will run
an infrastructure for authoring and interacting with such documents and a search engine for
the \DKS base. 
\end{sitedescription}

%%% Local Variables: 
%%% mode: latex
%%% TeX-master: "../proposal"
%%% End: 

% LocalWords:  site-jacu.tex clange sitedescription emph compactitem pn semmath RabKoh
% LocalWords:  prosuming-flexiformal KohSuc asemf06 GinJucAnc alsaacl09 StaKoh ProKoh
% LocalWords:  tlcspx10 KohDavGin psewads11 ednote Radboud Bia ystok CALCULEMUS textsf
% LocalWords:  textbf keypubs OntoLangMathSemWeb uwb Deyan Ginev Stamerjohanns mwssofse12
% LocalWords:  searchability krasm Iacob KohKoh skmfe08 HilKohSta copmem06 ako IanJucKoh
%  LocalWords:  subsubsection sdm14 Centres vspace TOWRITE Modelling WPtref dksbases

\clearpage
\begin{sitedescription}{UJF}

% PIC: UJF == 999907429
% see: http://ec.europa.eu/research/participants/portal/desktop/en/organisation

% See ../proposal.tex, section Members of the Consortium for a
% complete description of what should go there

The Université Joseph Fourier (UJF)  is a research intensive university in an
international and high tech environment with $16\,600$ full time students
including $1\,200$ in a doctoral program, $1\,500$ lecturers and researchers and
$1\,500$ administrative and technical staff. Ranked 150th at the latest Shanga\"i
ranking, 5th french university, and top 75 natural sciences and mathematics.
UJF laboratories have successfully contributed to the FP7 european with 
105 projects, 26 of them being coordinated by UJF. Under H2020, UJF continues to
actively participate with already 4 projects.

The UJF partner gathers two teams from the Laboratoire Jean Kuntzmann
(CASYS team, with Jean-Guillaume Dumas) and the Laboratoire d’Informatique de
Grenoble (MOAIS, Clément Pernet). The 
CASYS team is specialised in Algebraic computations, cryptology, codes and hybrid symbolic-numeric dynamical systems. The MOAIS team is specialised in programming and scheduling
design on distributed resources for applications based on interactive simulation. The software
developed by this partner is significant. It includes the exact arithmetic
library \Givaro, the exact linear algebra libraries \fflas and \Linbox, and the
parallel programming runtime system \software{XKAAPI}.
Relying on this expertise, the UJF site will be coordinating the tasks related
to High Performance Mathematical Computing, forming Work Package~5:
\taskref{hpc}{hpc-pari}, 
\taskref{hpc}{hpc-gap}, 
\taskref{hpc}{hpc-linbox}, 
\taskref{hpc}{hpc-singular},
\taskref{hpc}{hpc-mpir}, 
\taskref{hpc}{hpc-combi},
\taskref{hpc}{pythran},
\taskref{hpc}{hpc-jupyter}.
It will also be in charge of the task~\taskref{hpc}{hpc-linbox} on the
development of high performance exact linear algebra with the \Linbox library.
\subsubsection*{Curriculum vitae}

% Curriculum of the personnel at this institution

\begin{participant}[type=leadPI,PM=15,gender=male]{Cl\'ement Pernet}
  Associate Professor at the joint Inria-LIG research group MOAIS, Cl\'ement Pernet is a
  junior researcher in Computer Algebra, parallel computing and coding theory with 16
  papers published in international journals or refereed international conferences. He is
  associate editor of the ACM transactions on Mathematical Software and has co-organised
  10 conferences, including 2 sage-days and the 2012 edition of ISSAC, the leading
  conference in computer algebra.

  Since he was a post-doc at University of Washington, under the supervision of William
  Stein, head of the \Sage project, he has had many contributions to \Sage on the exact
  linear algebra and the symbolic computation tools. He co-authored the book ``Calcul
  Mathématique avec \Sage'' with the chapter on Linear algebra.  Cl\'ement Pernet is the
  founder and lead developper of the \software{fflas-ffpack} library, kernel for dense linear algebra
  over a finite field, delivering high performance computation to \Linbox and \Sage. He is a
  core contributor to the \Linbox library and contributed to the \software{m4ri} library.

% \begin{description}
%   \item[Personal Data]\ 

% \begin{tabular}{ll}
% Gender:& male \\
% Nationality:  & French  \\
% Address:        & LIP-AriC, ENS de Lyon\\
%                 & 46, All\'ee d'Italie\\
%                 & F-69364 Lyon CEDEX 07\\
%                 & France \\
% Phone:          & +33 437 28 74 75 \\
% Email:          & clement.pernet@imag.fr \\
% URL:            & http://lig-membres.imag.fr/pernet\\ \\
% Status:         & Associate Professor\\
% \end{tabular}

% \item[Scientific Qualification]\ 

% \begin{tabular}{ll}
% PhD 2006& Universit\'e J. Fourier (Grenoble 1) \\
% Habilitation 2014& Universit\'e J. Fourier (Grenoble 1) \\
% \end{tabular}

% \item[Academic Career]\ 

% \begin{tabular}{ll}
%  2008 & Postdoc, University of Washington \\
%  2009 -- present& Associate Professor, Universit\'e J. Fourier (Grenoble 1)\\
%  2013 -- 2014 & CNRS Research leave at LIP, \'ENS de Lyon\\
%  2014 -- 2015 & Inria Research leave at LIP, \'ENS de Lyon\\
% \end{tabular}

% \item[Scientific Service]\ 

% \begin{tabular}{ll}
%  2008 & Pacific Institute for Math. Sciences (PIMS) fellowship award\\
%  2010--2011 & Coordinator of the CNRS PEPS grant \textit{Parallel Computer Algebra}\\
%  2012-2014 & Member of the Inria associate team QOLAPS (with NCSU, USA) \\
%  2012--2015 & Member (70\%) of the ANR HPAC (High Performance Algebraic
%  Computing) project\\ 
% \end{tabular}





% \item[Selected Mathematical Software]\ 

% \begin{tabular}{ll}
%  1997--present & Coauthor of {\sc{Singular}} libraries for adjoint ideals, absolute factorization, \\
%                           & integral bases, invariant theory, parametrization of rational curves, \\
%                           & primary decomposition, normalization, and sheaf cohomology\\

%  2009--present& Head of the {\sc{Singular}} developers group\\
%\end{tabular}
%\end{description}
\end{participant}
%%% Local Variables:
%%% mode: latex
%%% TeX-master: "../proposal"
%%% End:

\begin{participant}[PM=6,salary=6500]{Jean-Guillaume Dumas}
  Professor at the Laboratoire Jean Kuntzmann, Jean-Guillaume Dumas is a senior
  researcher in Computer Algebra with 40 papers published in international
  journals or refereed international conferences.  Among other things, he is
  vice-president of ACM Special interest group on symbolic and algebraic
  manipulations (SIGSAM), department chair within his Laboratoire (6 research
  teams, 130 members) and has collaborators in USA, Canada, Ireland, Germany and
  Luxembourg; he has also co-organized fifteen international conferences.

  Computer Algebra is a field at the frontier between mathematics and computer
  science, with heavy needs for computer exploration.  Jean-Guillaume Dumas is
  the main developer of the LinBox and Givaro C++ libraries (libgivaro1,
  libgivaro-dev, libgivaro-doc, liblinbox0, liblinbox-dev in Debian) used, e.g.,
  by Sage respectively as its exact linear algebra and its finite fields.

  Along the way, he coauthored part of the proposal for NSF-INRIA grant QOLAPS
  on Quantfier elimination, Optimization, Linear Algebra and Polynomial Systems
  and he is the director of the French ANR program on High-Performance Algebraic
  Computations.
\end{participant}

%%%%%%%%%%%%%%%%%%%%%%%%%%%%%%%%%%%%%%%%%%%%%%%%%%%%%%%%%%%%
%%% Local Variables:
%%% mode: latex
%%% mode: flyspell
%%% ispell-local-dictionary: "american"
%%% TeX-master: "../proposal"
%%% fill-column: 80
%%% End:

\paragraph{Pierrick Brunet}

% months=12

Junior Research and Development Engineer at INRIA Grenoble, Pierrick Brunet
is working on compilation of C/C++ OpenMP program to C/C++ programm with calls
to specific OpenMP runtimes.

With about 25\% of commits in the Pythran project, Pierrick is one of the core
devs of this project which compile a subset of the Python language to native 
Python modules.

%\input{CVs/First.Last.tex}

\begin{participant}[type=res,PM=24,salary=4200,gender=]{Unknown}
Full time developer on \Linbox, and its integration into \Sage in the framework
of Task~\taskref{hpc}{hpc-linbox}. The recruited
person should master \software{C++} programming and library development in general. A good
knowledge on scientific computing and in particular linear algebra is also required. 
\end{participant}

\subsubsection*{Publications, products, achievements}

% \begin{compactenum}
% \item Coauthoring of the open source book ``Calcul Mathématique avec
%   Sage'', the first of its kind comprehensive introduction to
%   computational mathematics in Sage for education.
% \end{compactenum}
\begin{description}
\item[Software projects]\
  \begin{description}
 \item[\software{fflas-ffpack}:] An open-source \software{C++} library offering dense
    linear algebra kernels over a finite field. In the same  spirit as the
    numerical \software{BLAS} (Basic Linear Algebra Subroutines), and
    \software{LAPACK} libraries, it delivers high performance for the most
    commonly used routines of scientific computing: matrix multiplication,
    solving linear systems, computing echelon forms, determinants,
    characteristic polynomials, etc. This library has set the standard
    approach for high performance exact dense linear algebra. It is currently
    used in \Sage, and has inspired the design of similar routines in
    most commercial computer algebra systems: \Maple, \Magma, etc.
  \item[\Linbox:] An open-source \software{C++} middleware library for
    exact linear algebra. It uses \software{fflas-ffpack} for its dense finite
    field linear algebra part and extends its functionalities to other
    computation domains (integers, rationals, polynomial rings) and type
    matrices (sparse and structures matrices, black-box
    matrices). \Linbox is integrated in \Sage. 
  \item[\Pythran:] \label{pythran-descr}An open-source \Python-to-\texttt{C++} optimizing compiler
    offering an high performance runtime for Scientific Python kernels. Dynamicity of
    the \Python language is not compliant with static compilation. That's why only
    a subset of the \Python language is supported by \Pythran. Thanks to these
    restrictions, \Pythran generate code up to 3000 faster than original module.
    \Pythran takes advantage of multi-cores and SIMD instruction units and,
    thanks to type inference, it requires little annotations. Its runtime
    supports a growing subset of the \Numpy package.
  \end{description}
\item[Selected Publications]\ 
\medskip\noindent
\begin{compactenum}[1.]
\item Coauthoring of the open source book ``Calcul Mathématique avec
  Sage'', the first of its kind comprehensive introduction to
  computational mathematics in Sage for education.

\item Parallel computation of echelon forms (with J-G. Dumas, T. Gautier and Z. Sultan). 
\emph{In Proc. Euro-Par'14}  (2014),  LNCS 499--510. DOI: 10.1007/978-3-319-09873-9\_42.

\item Pythran: Enabling static optimization of scientific python programs
  (Serge Guelton, Pierrick Brunet, Alan Raynaud, Adrien Merlini, and Mehdi Amini.)
\emph{Proceedings of the \Python for Scientific Computing Conference (SciPy)} June 2013.

\item Fast Computation of Hermite Normal forms of random integer matrices (with
W. Stein).
\emph{J. of Number Theory} {\bf{130.7}} (2010), 1675--16833. DOI: 10.1016/j.jnt.2010.01.017

\item Dense Linear Algebra over Word-size Prime Fields (with J.-G. Dumas and P. Giorgi). 
\emph{Trans. on Math. Software} {\bf{35.3}} (2008), 1--42. DOI: 10.1145/1391989.1391992.
\end{compactenum}

\end{description}

\subsubsection*{Previous projects or activities}

\begin{compactenum}
\item Direction of the ANR program on High-Performance Algebraic
  Computations 2012-2015.
\item Participation to the NSF-Inria associate teams QOLAPS (with NCSU, USA)
\item Coordination of a CNRS PEPS grant (parallel computer algebra)
\item Organisation of the ISSAC'12 conference, the main
  international conference in computer algebra, and of PASCO'15, 
  the ISSAC'15 satellite conference on parallel computer algebra.
\end{compactenum}


\end{sitedescription}



\begin{draft}
\vspace{1cm}

\begin{verbatim}
- [X] checked that sum of person months put into finance request is
  the same as sum of person months associated with the Work Packages
  (in proposal.tex, as defined as part of the \begin{workpackage}"
  command.
  
  Take into account person months associated with work package 1, time
  of all staff to be hired and work on the project (including
  investigators). Figure 5 helps with a quick check of the sums over
  different work packages.

- [X] completed site specific resource summary in resources.tex,
  including table of non-staff costs. This is compulsory (EU
  regulations) if the non-staff cost exceed 15% of the total cost, and
  is likely to be the case for most of the partners. We ask everybody
  to do it, to be consistent and show transparently how we have
  planned our total budget.

- [X] Have all our tasks a designated lead institution? Check in the
  Work Packages that all the tasks you are involved in have a
  dedicated lead party. If the lead party is "USO", then use:
  \begin{task}[lead=USO]

- [X] Have all our deliverables a designated lead institution [using
  the 'lead=' key]?

- [X] In the "Members of the consortium section", have we addressed "a
  description of the legal entity and its main tasks, with an
  explanation of how its profile matches the tasks in the
  proposal"? See Entry for Paris Sud and Southampton as examples.

- [X] In the Members of the consortium section, have we given
  descriptions of all the people we intend to hire (even if we don't
  know who that is yet). 
  
- [X] Do all our tasks include us in the list of sites involved?
\end{verbatim}
\end{draft}

%KEY-MORE-TODOS



%%% Local Variables: 
%%% TeX-master: "../proposal.tex"
%%% End: 

%  LocalWords:  sitedescription Laboratoire Kuntzmann Fousse d’Infor matique Pernet Roch
%  LocalWords:  Galet Taktuk ployment subsubsection Calcul Mathématique avec texttt emph
%  LocalWords:  fflas-ffpack texttt LinBox Pythran Dynamicity medskip noindent Guelton
%  LocalWords:  Pierrick Amini SciPy Hermite Storjohann NSF-Inria TOWRITE

\clearpage
\begin{sitedescription}{UK}
\subsubsection{University of Kaiserslautern}

% See proposal.tex, Members of the Consortium for a complete description of what should go there

\paragraph{Principal investigator Prof. Dr. Wolfram Decker}

\begin{description}
  \item[Personal Data]\ 

\begin{tabular}{ll}
Gender:& male \\
Nationality:  & German  \\
Address:        & Department of Mathematics\\
                & Technical University of Kaiserslautern\\
                & P.O. Box 3049\\
                & Germany \\
Phone:          & +49 631 205 2253 \\
Email:          & decker@mathematik.uni-kl.de \\
URL:            & www.mathematik.uni-kl.de/agag/mitglieder/professoren/\\ & prof-dr-wolfram-decker/\\
Status:         & Professor\\
\end{tabular}

\item[Scientific Qualification]\ 

\begin{tabular}{ll}
PhD 1984& University of Kaiserslautern \\
Habilitation 1989& University of Kaiserslautern \\
\end{tabular}

\item[Academic Career]\ 

\begin{tabular}{ll}
 1984 -- 1990& Assistant Professor, University of Kaiserslautern\cr
 1990 -- 2009& C3 Professor, University of the Saarland\cr
 2009 -- present& W3 Professor, Technical University (TU) Kaiserslautern\cr
\end{tabular}

\item[Synergistic Activities]\ 
\begin{description}
\item[PhD students]\ 

  Hirotachi Abo, Holger Cr\"ni, Hiep Dang, Hanieh Keneshlou, Dereje
  Kifle, Michael Messollen, Ngoc Anh Pham, Sorin Popescu, Andreas
  Steenpa\ss, Isabel Stenger, I. Made Sulandra, Shrawan Tiwari

\item[University Service]\ 

\begin{tabular}{ll}
 1994--1996 & Dean, Dep. of Math., University of the Saarland\\
  2014--present & Dean, Dep. of Math., TU Kaiserslautern\\
\phantom{A}
 \end{tabular}

\item[Scientific Service]\ 

\begin{tabular}{ll}
 1996--1999 & Coordinator of EuroProj (a European algebraic geometry network)\\
 2000--2004 & Chair of the programme management committee of EAGER\\ 
                      &  (a European algebraic geometry network)\\
 2010--present & Coordinator of the DFG Priority Programme SPP1489\\
                      & `Algorithmic and experimental methods\\
                      & in algebra, geometry, and number theory'\\
\end{tabular}

\item[Conferences (co)organized]\ 

\begin{tabular}{ll}
 1997-- 2004 & About 30 conferences, summer schools, and workshops\\ 
                           & in the framework of EuroProj and EAGER.\\
 1992--present & More than 20 coding sprints, conferences, summer schools,\\
                          & and workshops outside EuroProj and EAGER,\\
                          & including 3 conferences at Dagstuhl and 1 at Banff.\\
 2000                 & Chair of the Minisymposium on computer algebra, third ECM.\\
\end{tabular}
\end{description}

\item[Selected Grants]\ 

\begin{tabular}{ll}
1986-1987 & NATO-Grant of the DAAD (visit UC Berkeley)\\
1987--1994& In: DFG Priority Programme `Complex manifolds'\\
\phantom{1992--}1993 & Grant: Japanese Society for the Promotion of Science (visit Kyoto)\\
1992--1997& In: DFG Priority Programme `Algorithmic number theory and algebra'\\
2002--2006& In: DFG Priority Programme `Global methods in complex geometry'\\
2010--present & In: DFG Priority Programme SPP1489 (two grants)\\
1997--2004 & Seven grants for EU Highlevel Scientific Conferences
\end{tabular}

\item[Selected Publications]\ 
\medskip\noindent
\begin{enumerate}[1.]

\item On the uniqueness of the Horrocks-Mumford-bundle (with F.-O. Schreyer).
\emph{Math. Ann.} {\bf{273}} (1986), 415--443.  (MR0824431)

\item Stable rank 2 vector bundles with Chern-classes $c_1=-1$, $c_2=4$. 
\emph{Math. Ann.}  {\bf{275}} (1986), 481–-500. (MR0858291)

\item Construction of Surfaces in $\mathbb{P}_4$ (with L. Ein, F.-O. Schreyer). 
\emph{J. Algebraic Geometry} {\bf{2}} (1993), 185--237. (MR0858291)

\item Computational algebraic geometry today. In: \emph{Applications of algebraic geometry 
to coding theory, physics and computation (Eilat, 2001)}, 65--119, 
NATO Sci. Ser. II Math. Phys. Chem., 36, Kluwer Acad. Publ., Dordrecht, 2001. (MR1866896)

\item 
Computing in Algebraic Geometry. A Quick Start using SINGULAR (with C. Lossen).
\emph{Algorithms and Computation in Mathematics, 16}.  Springer, Berlin, 2006. xvi+327 pp. (MR2220403)

\item Parallel algorithms for normalization (with J.B\"ohm, S. Laplagne, G. Pfister, A. Steenpa\ss, S. Steidel).
\emph{J. Symbolic Comput.} {\bf{51}} (2013), 99--114.  (MR3005784)

\item A first course in computational algebraic geometry (with  G. Pfister). 
\emph{African Institute of Mathematics (AIMS) Library Series}. Cambridge 
University Press, Cambridge, 2013. viii+118 pp. (MR3052757)

\item
Local analysis of Grauert-Remmert-type normalization algorithms (with J.B\"ohm, M. Schulze). 
\emph{Internat. J. Algebra Comput.} {\bf{24}} (2014), 69--94. (MR3189667)
\end{enumerate}

\item[Selected Mathematical Software]\ 

\begin{tabular}{ll}
 1997--present & Coauthor of {\sc{Singular}} libraries for adjoint ideals, absolute factorization, \\
                          & integral bases, invariant theory, parametrization of rational curves, \\
                          & primary decomposition, normalization, and sheaf cohomology\\

 2009--present& Head of the {\sc{Singular}} developers group\\
\end{tabular}
\end{description}

%%% Local Variables: 
%%% mode: latex
%%% TeX-master: "../proposal.tex"
%%% End: 


\begin{participant}[type=PI,gender=male]{Dr. William Hart}

William Hart is a postdoctoral researcher at the University of
Kaiserslautern. He is the lead developer of the Flint and MPIR projects
as well as the main author and maintainer of the BSDNT bignum library, the
Nemo and ANTIC libraries and a contributor to various other software packages.

Before coming to Kaiserslautern, held a prestigious five year Career
Acceleration Fellowship ``Algorithms in Algebraic Number Theory'' at Warwick
University in the UK. He has been involved in a number of high performance
computing records, including computation of congruent numbers (subject to the
BSD conjecture) up to a trillion ($10^{12}$).

William is the main author of the FFT code for multiplication of large integers
and polynomials in both MPIR and Flint, which are used extensively by the Sage,
Singular and Macaulay 2 computer algebra systems.

The main focus of William's research has been in algorithms for fast arithmetic, fast integer and polynomial factorisation and to algebraic number theory,
including computation of modular equations and class invariants.
\end{participant}
%%% Local Variables:
%%% mode: latex
%%% TeX-master: "../proposal"
%%% End:


\paragraph{Publications, products, achievements}

\begin{enumerate}
\item {\sc{Singular}} computer Algebra system.
\item Wolfram Decker is coordinator of the DFG Priority Project SPP1489 \emph{Algorithmic and Experimental Methods in Algebra, Geometry, and
Number Theory'}.
\item {\sc{Flint and MPIR}} C libraries for number theory and bignum arithmetic.
\item William Hart held an EPSRC Career Acceleration Fellowship EP/G004870/1 
from 2008-2013, \emph{Algorithms in Algebraic Number Theory}
\end{enumerate}

\paragraph{Previous projects or activities}

\begin{enumerate}
\item Member of the DFG Priority Project \emph{Algorithmic Number Theory and Algebra}.
\end{enumerate}

\paragraph{Significant infrastructure}

Excellent computing infrastructure (high end servers), access to 
different types of compute clusters through the IT-Center of the 
TU Kaiserslautern.
\end{sitedescription}


%%% Local Variables:
%%% mode: latex
%%% TeX-master: "../proposal"
%%% End:

\clearpage
\subsection*{University of Oxford}

% PIC: 999984350 THE CHANCELLOR, MASTERS AND SCHOLARS OF THE UNIVERSITY OF OXFORD OXFORD UK GB12550673

% See proposal.tex, Members of the Consortium for a complete description of what should go there

\TOWRITE{DP/UM}{Description of the university of Oxford}

\subsubsection*{Curriculum vitae}

%\paragraph{Ursula Martin.}

% months=2.4
%
% Fair evaluation of the number of months you will be spending on this
% specific project along the four years

% salary=XXX
%
Professor Ursula Martin has  recently joined the University of Oxford, where
she holds a Professorship, in conjunction with a Senior EPSRC Fellowship, on a
joint arrangement between  the Department of Computer Science and the
Mathematical Institute. Her current  research concerns social and computational
techniques for creating mathematics, building on a significant track record at
the interface of mathematics and computing. Prior to this she worked at  Queen
Mary University of London, where as Vice Principal for Science and Engineering
she led strategic change, and was active in knowledge transfer
activities and developing young staff. 

Her work is characterized by strongly interdisciplinary collaboration in new
problem domains at the interface of mathematics and computer science,
identifying novel interactions between theory and practice, with real-world
problems inspiring scientific advance. Major achievements include results
linking randomness and symmetry, new unifying explanations of the power of
computational logic, and new practical techniques for using computational logic
and algebra in industry.

The work to be undertaken in the Work Package 5 (Social Aspects) fits very well
into her current project, which concerns crowdsourced mathematics: the overarching goal is
to understand and extend the human and computer creation of mathematics. 
It is mostly funded by her
2014 EPSRC Advanced Fellowship (EPSRC awards only one or two of these annually
in Computer Science) is a partnership of industry, government and international
academia.  

\TOWRITE{UM}{CV Ursula Martin}
%\begin{participant}[type=leadPI,PM=24,salary=7000,gender=male]{Dmitrii Pasechnik}
  is a Senior Research Fellow at the Department of Computer Science of the University of
  Oxford, where he also holds a Lectureship at Pembroke College. Before moving to Oxford
  in 2013, he taught mathematics for 8 years in Nanyang Technological University
  (Singapore). While there, he was successful in receiving individual grant funding
  totalling over \euro 500K, graduated 2 PhD students, supervised post-doctoral
  researchers, and co-organized a 2-months research program at Singapore Institute for
  Mathematical Sciences on a range of topics in computational mathematics, involving over
  100 participants.

  He works on a wide area of interconnected topics, related to computational algebra and
  optimization, combinatorics, algorithm, symbolic computing, and game theory, and
  authored over 70 papers on these topics, several of them using Sage and/or its
  components, such as GAP.

  He is an active Sage developer, and regularly contributes, himself or together with his
  undergraduate or graduate students, new or improved Sage interfaces to various
  mathematical packages and databases. He taught a Sage-based undegraduate course while at
  Nanyang, and has good understanding of the overall Sage development process, as well as
  of development of other open-source software and databases, including their
  social/community aspects.
\end{participant}
%%% Local Variables:
%%% mode: latex
%%% TeX-master: "../proposal"
%%% End:

\TOWRITE{DP}{CV Dima Pasechnik}

\subsubsection*{Publications, products, achievements}

\begin{enumerate}
\item \TOWRITE{DP/UM}{Publications Oxford}
\end{enumerate}

\subsubsection*{Previous projects or activities}

\begin{enumerate}
\item \TOWRITE{DP/UM}{Projects and activities in Oxford}
\end{enumerate}

\subsubsection*{Significant infrastructure}

\TOWRITE{DP/UM}{Significant infrastructure in Oxford}

\clearpage
\begin{sitedescription}{US}

The University of Silesia in Katowice was established in 1968. Now,
with 12 faculties and several interdisciplinary schools and centers,
over 30000 students and over 2000 academic staff the University is one
of the largest in Poland. Students are educated at three educational
levels: Bachelor, Master and Doctoral and their achievement are
accumulated using European Credit Transfer and Accumulation System
(ECTS). Located in the heart of Upper Silesia, Poland’s old industrial
region with distinct history and cultural identity, the university
attracts many scientists and students.

The origins of the {\em Faculty of Mathematics, Fhysics and Chemistry} date
back to the academic year 1968/1969 and coincide with the
establishment of the University of Silesia. One of the largest
university units, the faculty incorporates, as its name indicates,
three separate departments: mathematics, physics and chemistry, each
with several divisions and subdivisions carrying out the research and
educational activities. There are over 1900 students, both full-time
and part-time, educated at three educational levels: Bachelor`s,
Master`s and Doctoral. The Faculty is entitled to grant doctoral
degrees in the natural sciences. The Faculty staff consists of 243
academics who are both teachers and researchers.

\subsubsection*{Curriculum vitae}
% AK - rules say "CV or description of the profile of the persons"
% - can we name it "Track record" instead of CV?

% Curriculum of the personnel at this institution

\begin{participant}[PM=12,salary=1500]{Marcin Kostur}

is an assistant Professor at the Institute of Physics, author of  over 40
publication cited over 1000 times in the field of statistical physics,
solid state physics (Josephson Junction dynamics), microfluidics and
biophysics. He is experienced in application of GPU architecture to
numerical simulations of stochastic processed in physics. His recent
computiational interests are focused at the Open Source project
``Sailfish'' - HPC implementation of Lattice Boltzmann Method on GPU.

He is leader of two projects incorporating computations to the science education:

\begin{itemize}
\item Computing in high school science education - iCSE4schools,
  project funded by Erasmus+, Key Action 2 - Strategic Partnerships'',
  (budget: EUR 263.320, 2014-2017)
\item ``Computers in Science Education: iCSE'' http://icse.us.edu.pl
  (budget: c.a. EUR 1 milion, funded by EFS, 2011-2014 )
\end{itemize}

He is also co-Author and a task coordinator of PAAD (Platforma Analiz i
Archiwizacji Danych) founded by POIG program for 2014-2015 with a total budget
c.a. of EUR 4 million. The task ``Interactive HPC services for science''
main goal is to provide interactive interface to HPC infrastructure
(heterogenous cluster of 48 nodes, including 24 GPU and 24 Xeon Phi)
using innovative technology of ``web notebook'' interface.  




\end{participant}
%%% Local Variables:
%%% mode: latex
%%% TeX-master: "../proposal"
%%% End:


\begin{participant}[type=PI,PM=12,gender=male]{Jerzy Łuczka}
Prof. Dr. Jerzy Łuczka (\url{http://zft.us.edu.pl/luczka}) is
a full professor of physics at the University of Silesia (Katowice,
Poland) and the Head of the the Department of Theoretical Physics.

He published more than 150 papers in journals (all on ISI Master
Journal List) which have been cited almost 2000 times.

He is an Editor of European Physical Journal B, Chairman of the
Statistical and Nonlinear Physics Division (European Physical
Society), Fellow of the Institute of Physics (United Kingdom) and
Outstanding Referee (American Physical Society). He was Co-director of
the NATO Advanced Research Workshop ``Stochastic Systems. From
randomness to complexity'', 2002, Erice (Italy) and Member of the
Steering Committee of the program : ``Stochastic Dynamics: Fundamentals
and applications'' (European Science Foundation), 2003-2008.  He
received the DAAD research fellowship (Forschungsaufenthalte für
Hochschullehrer und Wissenschaftler) 1995, 2009 and 20012. He was a
leader of several Polish and two German-Polish grants. He has
collaborators in Germany, Italy and Spain. He has also co-organised
international conferences.

Łuczka’s research interests lie in areas of stochastic processes in
physics, quantum open systems, transport phenomena, physical
fundamentals of quantum information. He has teaching experience with
\Sage in physics, biophysics and econophysics.


\end{participant}
%%% Local Variables:
%%% mode: latex
%%% TeX-master: "../proposal"
%%% End:



\begin{participant}[type=PI,PM=12,salary=1400,gender=male]{Jan Aksamit}
got his PhD in 1982 and worked at University of Silesia as research
assistant, lecturer and senior lecturer. His skills combine forty
40-years of experience in teaching algebra, classical and quantum
mechanics, quantum information, statistical physics, and mathematical
methods of physics with proficiency in computing. He has actively
participated in the project iCSE (innovative Computing in Science
Education), where he created Sage enhanced textbook of Linear Algebra
(polish only: http://visual.icse.us.edu.pl/LA) using modern
interactive technologies. In this project his main task will be to
pursue to exploit the vast lecturing experience for creation of
interactive demonstrators of VRE.

\end{participant}
%%% Local Variables:
%%% mode: latex
%%% TeX-master: "../proposal"
%%% End:


%\input{CVs/First.Last.tex}
%\input{CVs/First.Last.tex}

\subsubsection*{Publications, products, achievements}


\begin{enumerate}

\item Sailfish: A flexible multi-GPU implementation of the lattice Boltzmann method
 Computer Physics Communications Vol.181(9), 2350-2368;2014. Web: http://sailfish.us.edu.pl 

\item  M.Januszewski and M.Kostur. ``Accelerating numerical solution of stochastic differential equations
with CUDA'',  Computer Physics Communications, 181(1):183-188, 2010. Web: https://github.com/marcinofulus/CUDASDE.git

\item iCSE - course materials  materials - http://visual.icse.us.edu.pl/iCSE\_main/

\end{enumerate}

\subsubsection*{Previous projects or activities}

Internationalization of research and education is one of the priority
directions of development of the University of Silesia. The University
scientists are actively engaged in research at the international
level, actively participates in European Commission initiatives
focused both on educational and scientific development, and implements
projects within the LLP/Erasmus+ programme, the Research Fund for Coal
and Steel, Framework Programmes, as well as the EU Structural Funds.

The institution has been involved in more than 40 FP7 proposals, of
which 15 have been funded.

The Faculty of Mathematics, Physics and Chemistry was engaged in the
implementation of several FP6 and FP7 projects, i.e.:

\begin{enumerate}
\item HadronPhysics (RII3/CT/2003/506078)
\item FlaviaNet (MRTN-CT-2006-035482)
\item LAGUNA (212343)
\item LHCPhenoNet (612536). 
\end{enumerate}

There are following projects which are directly connected to
infrastructures for virtual research environments:

\begin{enumerate}
\item 2011-2014 - iCSE (innovative Computing in Science Education) -
  grant from European Social Fund, EUR 1 Million, incorporating
  computational perspective in teaching of mathematics, physics and
  chemistry using cloud based Sage system and Python language.
\item 2014-30.11.2015 PAAD (Platform for data analysis and archiving) - EUR
  3.8Million, funded is mostly HPC center for research with
  interactive access based on web based notebook UI.
\item 2014-30.11.2015 CNS: Center of Applied Science - 2nd stage,
  Infrastructure grant includes EUR 500.000 funding for small HPC and
  cloud infrastructure for education. Technically this will be
  extension of research HPC center for educational purposes.
\end{enumerate}



\subsubsection*{Significant infrastructure}

The University of Silesia has finished currently executes grants from
ESF in total c.a. Million 120 Euro for infrastructure, laboratories and
computing centers. New HPC centers are under construction (PAAD and
CNS projects) which will provide necessary hardware for development
and implementation of virtual research environments.
\end{sitedescription}





\begin{draft}
\vspace{1cm}\TOWRITE{MK}{Complete check list below -- delete completed items if you wish}

\begin{verbatim}
- [X] ;-) checked that sum of person months put into finance request is
  the same as sum of person months associated with the Work Packages
  (in proposal.tex, as defined as part of the \begin{workpackage}"
  command.
  
  Take into account person months associated with work package 1, time
  of all staff to be hired and work on the project (including
  investigators). Figure 5 helps with a quick check of the sums over
  different work packages.

- [ ] completed site specific resource summary in resources.tex,
  including table of non-staff costs. This is compulsory (EU
  regulations) if the non-staff cost exceed 15% of the total cost, and
  is likely to be the case for most of the partners. We ask everybody
  to do it, to be consistent and show transparently how we have
  planned our total budget.

- [ ] Have all our tasks a designated lead institution? Check in the
  Work Packages that all the tasks you are involved in have a
  dedicated lead party. If the lead party is "USO", then use:
  \begin{task}[lead=USO]

- [ ] Have all our deliverables a designated lead institution [using
  the 'lead=' key]?

- [ ] In the "Members of the consortium section", have we addressed "a
  description of the legal entity and its main tasks, with an
  explanation of how its profile matches the tasks in the
  proposal"? See Entry for Paris Sud and Southampton as examples.

- [ ] In the Members of the consortium section, have we given
  descriptions of all the people we intend to hire (even if we don't
  know who that is yet). 
  
- [ ] Do all our tasks include us in the list of sites involved?
\end{verbatim}
\end{draft}

%KEY-MORE-TODOS


%%% Local Variables:
%%% mode: latex
%%% TeX-master: "../proposal"
%%% End:

%  LocalWords:  sitedescription Fhysics subsubsection M.Januszewski M.Kostur programme
%  LocalWords:  Programmes FlaviaNet LHCPhenoNet

\clearpage
\subsection*{University of Sheffield}

% PIC: 999976881
% see: http://ec.europa.eu/research/participants/portal/desktop/en/orga

% See ../proposal.tex, section Members of the Consortium for a
% complete description of what should go there

\subsubsection*{Curriculum vitae}

% Curriculum of the personnel at this institution

\begin{participant}[type=leadPI,PM=6,salary=10000,gender=male]{Neil Lawrence}
  Neil Lawrence received his bachelor's degree in Mechanical
  Engineering from the University of Southampton in 1994. Following a
  period as an field engineer on oil rigs in the North Sea he returned
  to academia to complete his PhD in 2000 at the Computer Lab in
  Cambridge University. He spent a year at Microsoft Research in
  Cambridge before leaving to take up a Lectureship at the University
  of Sheffield, where he was subsequently appointed Senior Lecturer in
  2005. In January 2007 he took up a post as a Senior Research Fellow
  at the School of Computer Science in the University of Manchester
  where he worked in the Machine Learning and Optimisation research
  group. In August 2010 he returned to Sheffield to take up a
  collaborative Chair in Neuroscience and Computer Science.

  Neil's main research interest is machine learning through
  probabilistic models. He focuses on both the algorithmic side of
  these models and their application. He has a particular focus on
  applications in personalized health and computational biology, but
  happily dabbles in other areas such as speech, vision and graphics.

  Neil was Associate Editor in Chief for IEEE Transactions on Pattern
  Analysis and Machine Intelligence (from 2011-2013) and is an Action
  Editor for the Journal of Machine Learning Research. He was the
  founding editor of the JMLR Workshop and Conference Proceedings
  (2006) and is currently series editor. He is Programme Chair for
  AISTATS 2012 and has served on the programme committee of several
  international conferences. He was an area chair for the NIPS
  conference in 2005, 2006, 2012 and 2013, Workshops Chair in 2010 and
  Tutorials Chair in 2013. He was general chair of AISTATS in 2010 and
  AISTATS Programme Chair in 2012. He is Program Chair of NIPS in
  2014.

  Neil is a strong advocate of open source software in machine
  learning and has given many invited talks on the subject. Since 2004
  his research group has made all their implementaitons available,
  most recently using Python and IPython as the main medium for
  communicating their work. Their Gaussian process python software
  framework\footnote{\url{https://github.com/SheffieldML/GPy}} is
  becoming a standard platform for research in these methods and
  underpins a series of Summer Schools and four day road shows that
  Neil has led in the
  area.\footnote{\url{http://ml.dcs.shef.ac.uk/gpss/}}
\end{participant}
%%% Local Variables:
%%% mode: latex
%%% TeX-master: "../proposal"
%%% End:

%\input{CVs/First.Last.tex}
%\input{CVs/First.Last.tex}

\subsubsection*{Publications, products, achievements}

\begin{enumerate}
\item \TOWRITE{XXX}{...}
\end{enumerate}

\subsubsection*{Previous projects or activities}

\begin{enumerate}
\item Organisers of the Gaussian Process Summer Schools (three 3-day workshops on Gaussian process models in python and the IPython notebook).
\item Organisers of five Gaussian Process and Data Science Road Shows (educating on data science and Gaussian process models in Uganda, Colombia, Italy, Australia and Kenya) Each workshop is 3-4 days long. 
\end{enumerate}

\subsubsection*{Significant infrastructure}

\TOWRITE{XXX}{...}

\clearpage
\begin{sitedescription}{USO}

% PIC: 
% see: http://ec.europa.eu/research/participants/portal/desktop/en/orga

% See ../proposal.tex, section Members of the Consortium for a
% complete description of what should go there


  The University of Southampton (USO) is one of the leading
  universities in the United Kingdom, was founded in 1952 and is a
  member of prestigious Russell Group of UK Universities. The
  University of Southampton has more than 19,000 undergraduate
  students and 4,000 postgraduates and is an excellent venue for
  conducting cutting-edge research and for providing high quality
  education. The university is truly international, drawing students
  from over 130 different countries and benefiting from a wide and
  varied culture. It is ranked in the top 1\% of universities
  worldwide (QS world university rankings 2014-15) and in the top 15
  of research led universities in the UK, and is participating in a
  high number of collaborative research projects and related
  initiatives. To ensure the impact of its research projects,
  University of Southampton’s Research \& Innovation Services (R\&IS)
  is responsible for professional protection of IP and supporting
  commercial development with industry. R\&IS has had considerable
  success, licensing annual revenue in excess of \EUR{1}million and
  launching twelve successful spin-out companies since 2000.  The
  university has a strong track record of working in European
  projects, especially within the Framework Programme. The EC 6th FP7
  Monitoring Report ranked USO 17th out of all higher and secondary
  education organisations for number of FP7 participations during
  2007-2012. Throughout the FP7 USO has received \EUR{132M} in
  research grants and has been involved in 319 projects, including 63
  ICT and 8 INFRASTRUCTURES Collaborative Projects. 

  The Faculty of Engineering and the Environment (FEE) consists of 370
  research postgraduate students and 340 academic and research
  staff, and is one of the lead engineering faculties in Europe, educating
  a range of professionals and generating research of the highest
  quality. Southampton's world-leading engineering ranking is
  confirmed by being ranked first in the UK for the volume and quality
  of the research in Electronic Engineering, Electrical Engineering
  and General Engineering in the latest Research Excellence Framework
  (REF) 2014. The faculty also hosts the University Technology Centres
  and Research Framework Agreements with key partners including:
  Airbus, Rolls-Royce, Lloyd’s Register, Microsoft and Network
  Rail. FEE has a strong background in working on international
  research projects, including 84 EU FP7 projects worth over \euro
  28M.


\medskip In the context of this proposal, Southampton has long
standing experience in high performance computer simulation to advance
science and engineering, with significant hardware infrastructure, a
critical mass of several hundred researchers working in the field, and
a dedicated doctoral training center in computational modelling.
Southampton's main tasks in this project are the extension and application of the
\Jupyter{} Notebook technology in computational materials research to
provide a virtual research environment demonstrator for a large research community.

\subsubsection*{Curriculum vitae}

% Curriculum of the personnel at this institution

\begin{participant}{Hans Fangohr}
  Hans Fangohr is Professor of Computational Modelling at the University of
  Southampton. He has studied Physics with specialisation in Computer Science and Applied
  Mathematics, gained his PhD in High Performance Computing (2002) and has since work in
  an Engineering department, carrying out interdisciplinary work based on computational
  science.

  He heads the University's interdisciplinary Computational Modelling Group
  (\url{http://cmg.soton.ac.uk}), and has more than 100 publications on applied computer
  simulation in magnetism, superconductivity, biochemistry, astrophysics and aircraft
  design, as well as development of computational methods.


  In 2013, he has attracted \EUR{5}m from the UK's Engineering and Physical Sciences
  Research Council (EPSRC) together with additional moneys from industry and his
  University to fund the \EUR{12}m Centre for Doctoral Training in Next Generation
  Computational Modelling (ngcm.soton.ac.uk) in the UK. This flagship activity will train
  about 75 PhD students (10 to 15 starting every year, beginning in September 2014) in the
  state-of-the-art and best-practice in computational modelling, the programming of
  existing and emerging parallel hardware and to apply these skills and tools to PhD
  research projects across a range of topics from Science and Engineering. The centre has
  chosen IPython as a key tool to deliver this teaching, document and communicate
  computational exploration and drive reproducible computation to push for excellent
  computational science.

  Hans Fangohr has led the development of the Open Source Nmag software
  (http://nmag.soton.ac.uk), which prodives a finite-element micromagnetic simulation
  suite to a community of material scientists, engineers and physicists who research
  magnetic nanostructures in academia and industry. He has designed the package in 2005 so
  that it has an IPython-compatible Python interface, to make the workflow of using the
  simulation package as accessible as possible to scientists without substantial
  computational background. He has extensive experience in micromagnetic simulation use
  and development.

  Hans has deep interested in excellence and innovation in learning and teaching. He has
  been awarded the prestigious Vice Chancellor’s teaching award ($\pounds 1000$) three
  times (in 2006, 2010, 2013) for initiating and realising three separate innovations in
  the university's teaching delivery, and has been voted ``best lecturer'' and ``funniest
  lecturer'' of the year by the students. Other Universities in the UK and elsewhere have
  adopted his teaching methods and materials. He has attracted grants to further develop
  learning and teaching activities, and given invited talks at international meetings on
  efficient learning and teaching of computational methods.
 
  Hans Fangohr is chairing the UK's national Scientific Advisory Committee for High
  Performance Computing.
\end{participant}

%%% Local Variables:
%%% mode: latex
%%% TeX-master: "../proposal"
%%% End:


\begin{participant}[PM=2,type=PI]{Ian Hawke}
Ian Hawke is a lecturer in Applied Mathematics at the University of
Southampton and a co-director of the $\pounds$10$\,$m EPSRC Centre for Doctoral
Training in Next Generation Computational Modelling. An expert in
nonlinear simulations of relativistic matter and numerical techniques,
he has taught numerical methods in many contexts for ten years. The
initial author of the “Whisky” relativistic hydrodynamics code, he has
been a contributor to and maintainer of a range of projects used
across the numerical relativity community, including the Einstein
Toolkit, the Cactus infrastructure and the Carpet mesh refinement
code. His recent research has concentrated on numerical methods for
relativistic matter beyond ideal fluids, including modelling sharp
transitions and surfaces, relativistic elasticity, and the first
nonlinear simulations of relativistic multifluids.

XXX Add Jupyter Notebook based MOOC experience XXX.
\end{participant}

%%% Local Variables:
%%% mode: latex
%%% TeX-master: "../proposal"
%%% End:


\begin{participant}[type=res, PM=32]{NN}
  We will hire a post-doctoral research fellow to carry out the work
  at Southampton, under the leadership of and together with Hans
  Fangohr. The fellow will have a background in computational science,
  ideally in micromagnetics, combined with solid IPython and
  \Jupyter{} experience. We further require good communication and
  team working skills, and in particular interest and skill in the
  development of education materials to best support this part of the
  project.
\end{participant}
%\input{CVs/First.Last.tex}

\subsubsection*{Publications, products, achievements}

\begin{enumerate}
\item Open Source micromagnetic simulation framework Nmag,
  \href{http://nmag.soton.ac.uk}{http://nmag.soton.ac.uk}, Thomas
  Fischbacher, Matteo Franchin, Giuliano Bordignon, Hans Fangohr: \emph{
A Systematic Approach to Multiphysics Extensions of Finite-Element-Based Micromagnetic Simulations: Nmag 
IEEE Transactions on Magnetics \textbf{43}, 6, 2896-2898 (2007)}
\item Other open source contributions to the micromagnetic simulation
  community: OVF2VTK, higher order anisotropy extensions to OOMMF,
  OVF2MFM, summarised at
  \href{http://www.southampton.ac.uk/~fangohr/software/index.html}{http://www.southampton.ac.uk/~fangohr/software/index.html} 
\item H. Fangohr.
\emph{A Comparison of C, Matlab and Python as Teaching Languages in Engineering}
Lecture Notes on Computational Science \textbf{3039}, 1210-1217 (2004)
\end{enumerate}

\subsubsection*{Previous projects or activities}

\begin{enumerate}
\item EPSRC Doctoral Training Centre in Complex Systems Simulations
  (\href{http://icss.soton.ac.uk}{http://icss.soton.ac.uk}), jointly
funded by EPSRC and the University of Southampton, \EUR{14m}, (2009--2018)
\end{enumerate}

\subsubsection*{Significant infrastructure}
\begin{enumerate}
\item The University of Southampton hosts the largest university owned
  Supercomputer "Iridis 4" in the UK (12,300 cores, 250 TFlops),
  the hardware (\EUR{3.75}m) is refreshed every 3 years.
\item A community of 200 academics and over 500 researchers and
  doctoral students are users of this facility and provide a wide
  network pushing forward excellent computational science in the
  context of solving real world problems.
\item EPSRC Centre for Doctoral Training in Computational Modelling in
  the United Kingdom,
  \EUR{12}m. (\href{http://ngcm.soton.ac.uk}{http://ngcm.soton.ac.uk}),
  (2013--2022)
\end{enumerate}
\end{sitedescription}




%KEY-MORE-TODOS



%%% Local Variables:
%%% mode: latex
%%% TeX-master: "../proposal"
%%% End:

%  LocalWords:  sitedescription Programme organisations programmes Centres subsubsection
%  LocalWords:  micromagnetic Nmag Fischbacher Franchin Bordignon Fangohr emph textbf
%  LocalWords:  Multiphysics summarised Iridis TFlops Modelling

\clearpage
\begin{sitedescription}{SA}

% PIC: 
% see: http://ec.europa.eu/research/participants/portal/desktop/en/orga
% 

% See ../proposal.tex, section Members of the Consortium for a
% complete description of what should go there

The Centre for Interdisciplinary Research in Computational Algebra (CIRCA)
fosters research at the interface of Mathematics and Computer Science including
abstract and algorithmic algebra and combinatorics, formal languages and
automata, mathematical software and constraint programming. Our success is 
founded on the close integration of theoretical and algorithmic research and 
the development and use of state-of-the-art software.

In 1997, CIRCA became the centre of the development of \GAP after the 
retirement of under~Prof.~J. Neub\"user who initiated the system in 
mid-1980s in Aachen. This move was supported by EPSRC, EU and Leverhulme 
grants. Now GAP and its packages comprise over 1 million lines of C and 
1.25 million lines of GAP code. The system is distributed freely under the GPL2,
and our records show that it has been installed in at least 3000 sites and 
cited in several thousands publications 
(see \url{http://bit.ly/gap_citations} and \url{http://www.gap-system.org/Doc/Bib/bib.html} ). 
GAP was used in landmark computations such as the ``Millennium Project'' 
to classify all groups of order up to 2000 
and the classification of the over $10^{19}$ semigroups of order 10. It is 
designed to be natural to use for mathematicians; to be powerful and flexible 
for experts and to be freely extensible so that it can encompass new mathematics. 
These objectives have been met and GAP was awarded the ACM/SIGSAM Richard D. Jenks
Memorial Prize for Excellence in Software Engineering applied to Computer Algebra in 2008.

Nowadays, as one of the centres of \GAP\ development, CIRCA has
excellent contacts with developers and users worldwide. Particularly 
relevant to this proposal are the Singular and homalg groups at 
Kaiserslautern, Soicher at QMUL, Praeger and others at UWA, Cooperman 
and his students at NEU and the other GAP centres in Aachen, Fort Collins 
and Braunschweig.

CIRCA's output includes first class results in Pure Mathematics and in
Computer Science, recognised by our highly-cited publications in top
international venues in both disciplines and our widely used research
software. Beyond the individual international connections of the 
investigators and research staff, CIRCA as a centre has national 
importance and international standing. CIRCA has been selected to host 
major conferences such as CP 2010, BCC 2009, PP 2007, and the 
``Groups St Andrews'' series in 2005 and 2013.


\subsubsection*{Curriculum vitae}

% Curriculum of the personnel at this institution

\paragraph{Steve Linton}

% months=XXX
%
% Fair evaluation of the number of months you will be spending on this
% specific project along the four years

% salary=YYY
%
% Approximate monthly gross salary (in term of total cost for the
% employer). If you are uncomfortable having this information in a
% public file, you can alternatively send the information to Nicolas,
% or to your institution leader if the latter will be willing to fill
% in himself the budget forms on the eu portal.

% The above information will be used to evaluate the cost of the
% project for the institutions. You may remove the above comments once
% you have filled in the months= and salary= lines.

% About half a page of free text; for whatever it's worth, you may see
% Nicolas.Thiery.tex for an example.

is a Professor of Computer Science
at St~Andrews. He has worked in computational algebra since 1986 
and has helped coordinate the development of \GAP\ %\cite{end}{gap} 
since its move from Aachen in 1997. He personally wrote key
features of \GAP, such as workspaces and exception handling,
and has overseen the development and releases of the whole system.
He directed CIRCA from~2000--2013. He is an editor of 
AAECC\footnote{Applicable Algebra and Error Correcting Codes}. 
He has been PI of four major EPSRC grants and coordinated the EU 
project~SCIEnce. He is the general chair of ISSAC 2015, the main 
conference in computer algebra.

\paragraph{Alexander Konovalov}

% months=XXX
%
% Fair evaluation of the number of months you will be spending on this
% specific project along the four years

% salary=YYY
%
% Approximate monthly gross salary (in term of total cost for the
% employer). If you are uncomfortable having this information in a
% public file, you can alternatively send the information to Nicolas,
% or to your institution leader if the latter will be willing to fill
% in himself the budget forms on the eu portal.

% The above information will be used to evaluate the cost of the
% project for the institutions. You may remove the above comments once
% you have filled in the months= and salary= lines.

% About half a page of free text; for whatever it's worth, you may see
% Nicolas.Thiery.tex for an example.

is a Senior Research Fellow in CIRCA and has worked on \GAP\ for more than 10 years. 
After holding the fellowship at the Vrije Universiteit
Brussel in 2006, researching computational group ring theory, he moved
to St~Andrews in 2007 to join EU project~SCIEnce. He leads many
aspects of the \GAP project, including release preparation, regression
testing and liaison with package authors. He has authored 38 papers
and 8~\GAP packages, and co-organised a number of events, most recently 
the LMS/EPSRC Short Instructional Course in Computational Group Theory,
%\footnote{\url{http://www-circa.mcs.st-and.ac.uk/cgt2013/}},
the HPC-GAP workshop (2013), and
%\footnote{\url{http://www.gap-system.org/hpcgap2013/}}
the Summer School on Experimental Methodology in Computational Science Research
(2014).
%\footnote{\url{https://blogs.cs.st-andrews.ac.uk/emcsr2014/}}. 
He is an editor of Journal of Software for Algebra and Geometry 
and a Fellow of the Software Sustainability Institute.
\begin{participant}[type=R, gender=male]{Markus Pfeiffer}
  \TOWRITE{AK}{Markus Pfeiffer's PM engagement}
is a Research Fellow at the University of St Andrews active in the 
School of Computer Science and the School of Mathematics and Statistics. 
His unusual breadth of knowledge encompasses expertise in formal language 
theory, decidability, and algebra, as well as practical computation and
programming languages.
Since receiving his PhD in 2013 he has been an active researcher in semigroup 
theory as well as contributing to the \GAP system both as a package author, 
and as a core system developer. 
\end{participant}
%%% Local Variables:
%%% mode: latex
%%% TeX-master: "../proposal"
%%% End:

%\input{CVs/First.Last.tex}

% TODO(AK): select and add more items below

\subsubsection*{Publications, products, achievements}

\begin{enumerate}
\item 
S.~Linton, K.~Hammond, A.~Konovalov, C.~Brown, P.W.~Trinder, H.-W.~Loidl, 
P.~Horn and D.~Roozemond, Easy Composition of Symbolic Computation Software using 
SCSCP: A New Lingua Franca for Symbolic Computation.
J. Symbolic Computation, 49 (2013), 95--119.
% http://dx.doi.org/10.1016/j.jsc.2011.12.019
\item
V.~Janjic, C.M.~Brown, M.~Neunh{\"o}ffer, K.~Hammond, S.~Linton, H-W.~Loidl. 
Space exploration using parallel orbits: a study in parallel symbolic computing.
in Parallel Computing: Accelerating Computational Science and Engineering (CSE). 
vol. 25, Advances in Parallel Computing, IOS Press, 2013, p. 225--232.
\item 
A.~Konovalov and S.~Linton. 
SCSCP --- Symbolic Computation Software Composability Protocol, 
Version 2.1.4; 2013. GAP package (\url{http://alexk.host.cs.st-andrews.ac.uk/scscp/}).
\item
V.~Slavici, D.~Kunkle, G.~Cooperman, S.~Linton. 
An efficient programming model for memory-intensive recursive 
algorithms using parallel disks. In ISSAC 2012: Proceedings of 
the 37th International Symposium on Symbolic and Algebraic Computation. 
New York, ACM Press, 2012, p. 327--334.
\item 
R.~Behrends, A.~Konovalov, S.~Linton, F.~L{\"u}beck, M.~Neunh{\"o}ffer.
Towards high-performance computational algebra with GAP.
Proceedings of the Third International Congress on Mathematical Software: 
Kobe, Japan, September 13-17, 2010. LNCS 6327, Springer, 2010, p. 58--61.
% http://dx.doi.org/10.1007/978-3-642-15582-6_12 

\end{enumerate}

\subsubsection*{Previous projects or activities}

\begin{enumerate}
\item
\textbf{Multidisciplinary Critical Mass in Computational
Algebra and Applications} EP/C523229 2005--2010, \pounds 1.1m.
Through a range of subprojects this grant developed CIRCA as 
a broad centre of excellence in computational algebra. We 
extended \GAP, developed new algorithms, and opened up new 
applications in AI, combinatorics and physics. The project 
produced over 200 refereed publications in a huge range of 
venues, and delivered, as intended, a sustainable step change 
in the scale and breadth of CIRCA's research.
\item 
\textbf{SCIEnce: Symbolic Computation Infrastructure for Europe}
(FP6 eRII3-CT-026133) 2006--2011, \euro 3.2m, coordinator.
This project tackled the fragmentation of the European community 
of researchers in, and users of, symbolic computation. Among the 
nine partners were four major system developers (of \GAP, \Maple, 
\MuPAD and \KANT), an international research institute (RISC-Linz) 
and other groups with specialist expertise. Project activities 
included symbolic web services, system composability, symbolic 
grid and cloud computing and a program of visits, workshops and 
summer schools. One important outcome was a new protocol \SCSCP, 
now used well beyond the original project.
\item
\textbf{HPCGAP: High Performance Computational Algebra and Discrete Mathematics} 
EP/G055181 2009--2014, \pounds 1.5m, four sites. This project aimed
at reengineering GAP to support simple, safe and efficient parallel 
programming on a range of systems from multicore laptops and desktops, 
through departmental and university clusters to HPC systems. By the 
end of the project, we had adapted a complex system including a 
language runtime and over 400 000 lines of interpreted code to enable 
safe and efficient parallel programs. The proposed project is very timely as
the multi-threaded version of GAP is becoming mainstream, and users and 
package developers need training and support to parallelise their code.
\end{enumerate}

\subsubsection*{Significant infrastructure}

CIRCA provides hosting for the GAP website and ftp-servers, runs the 
automated system to check, fetch and test updates of GAP packages, uses 
the Jenkins continuous integration system to run regression tests for the 
development version and release candidates on its computer infrastructure, 
and manages GAP releases preparation.
\end{sitedescription}


\begin{draft}
  \vspace{1cm}\TOWRITE{SL}{Complete check list below -- delete completed items if you wish}

\begin{verbatim}
- [X] checked that sum of person months put into finance request is
  the same as sum of person months associated with the Work Packages
  (in proposal.tex, as defined as part of the \begin{workpackage}"
  command.
  
  Take into account person months associated with work package 1, time
  of all staff to be hired and work on the project (including
  investigators). Figure 5 helps with a quick check of the sums over
  different work packages.

- [X] completed site specific resource summary in resources.tex,
  including table of non-staff costs. This is compulsory (EU
  regulations) if the non-staff cost exceed 15% of the total cost, and
  is likely to be the case for most of the partners. We ask everybody
  to do it, to be consistent and show transparently how we have
  planned our total budget.

- [ ] Have all our tasks a designated lead institution? Check in the
  Work Packages that all the tasks you are involved in have a
  dedicated lead party. If the lead party is "USO", then use:
  \begin{task}[lead=USO]

     SL: all done excepot possibly for Social Aspects


- [X] Have all our deliverables a designated lead institution [using
  the 'lead=' key]?

- [ ] In the "Members of the consortium section", have we addressed "a
  description of the legal entity and its main tasks, with an
  explanation of how its profile matches the tasks in the
  proposal"? See Entry for Paris Sud and Southampton as examples.
  
- [X] In the Members of the consortium section, have we given
  descriptions of all the people we intend to hire (even if we don't
  know who that is yet). 
  
- [ ] Do all our tasks include us in the list of sites involved?
\end{verbatim}
\end{draft}

%KEY-MORE-TODOS



%%% Local Variables:
%%% mode: latex
%%% TeX-master: "../proposal"
%%% End:

%  LocalWords:  sitedescription Neub centres recognised subsubsection Konovalov Roozemond
%  LocalWords:  textbf

\clearpage
% PIC: 999837104

\subsection*{Université de Versailles -- Saint-Quentin-en-Yvelines}

{\bf PRiSM Laboratory.} The research teams of the PRiSM laboratory
(Parall\'elisme, Réseaux, Syst\`emes et Mod\'elisation) are involved
in two main scientific themes of UVSQ: Mathematics and Computer
science on one hand, ``Design, Modelization and Implementation of
Systems'' on the other hand. These two directions are not separated
from each other, as shown by many collaborations with other labs, and
the participation of many PRiSM teams to both directions. Within the
``Mathematics and Computer Science'' theme, the PRiSM teams study
cryptology and security, models for algorithms and operational
research. All the teams also participate to the ``Design, Modelization
and Implementation of Systems'' theme, with a particular focus on
communication systems (networks and telecommunication), embedded
systems, mobile systems, high speed networks, and database systems.

PRiSM is home to the ``Cryptology and Information Security''. In its
research activities, the cryptography team aims at widely covering the
various themes of academic research in cryptology, public key and
secret key cryptography, cryptanalysis, security of implementations,
number theory, multivariate cryptography, hash functions, etc. The
cryptology team brings its specificity in the computer science courses
at UVSQ and, since several years, the university offers several
teaching programs with a part devoted to cryptology and information
security. In particular, the research graduate program ``Applied
Algebra'' offers a full course in cryptology. It has been complemented
by a professional graduate program, called SeCReTS (Security of
Contents, Networks, Telecommunications and Systems). Many activities
of the team, require the use of advanced computer algebra. For this,
the team has a long history of using computer algebra systems (GAP,
Pari, Maple, Magma, ...). In recent years, with the arrival of young
researchers, and with the affirmation of Sage in research and
teaching, the team has moved from a pure user perspective to a
contributor one, taking active part in the development of computer
algebra software.

%%%%%%

\subsubsection*{Curriculum vitae of the investigators}

\begin{participant}[PM=12,type=leadPI,gender=male]{Luca De Feo}
  got his PhD in 2010 at Ecole Polytechnique. He was appointed Maître de Conférences at
  Versailles-St-Quentin-en-Yvelines University in 2011. His research interests cover
  Algorithmic Number Theory, Computer Algebra, Cryptology and Automated deduction, and he
  has already published 8 papers in international journals or refereed international
  conferences.

  He is an active \Sage contributor, with a dozen of tickets co-authored and about as much
  reviewed. He is also active in promoting the use of \Sage for research and for teaching:
  most of his papers feature a publicly available \Sage implementation, he teaches \Sage to
  undergraduate and graduate students, he participates and organises various events for
  the introduction of \Sage to beginners and young researchers.
\end{participant}
%%% Local Variables:
%%% mode: latex
%%% TeX-master: "../proposal"
%%% End:

%\begin{participant}[PM=2,type=R,gender=male]{Nicolas Gama}
  got his PhD in 2009 at École Normale Supérieure. He was appointed Maître de
  Conférences at University of Versailles-St-Quentin-en-Yvelines in 2010.  His research
  interests cover Lattice reduction algorithms, Theory of computer sciences, Algorithmic
  Number Theory, and Cryptology. He has already published 12 papers in international
  journals or refereed international conferences.

  He developed a fork of the NTL library to ease the development of parallel lattice
  algorithms, and added various blockwise lattice primitives, tools like high dimensional
  gaussian sampling over lattices and modulo lattices which can be directly used to
  implement the most recent lattice-based schemes. This fork is scheduled to be merged
  with the main branch of NTL, and the wrapper library for Sage should then be updated
  accordingly.
\end{participant}
%%% Local Variables:
%%% mode: latex
%%% TeX-master: "../proposal"
%%% End:


\subsubsection*{Publications, achievements}

Recent publications:

\begin{enumerate}
\item L. De Feo, J. Doliskani, É. Schost; Fast arithmetic for the
  algebraic closure of finite fields. ISSAC '14. ACM, 2014. pp
  122-129.
\item L. De Feo, É. Schost; Transalpyne: a language for automatic
  transposition. ACM SIGSAM Bulletin, 2010, 44 (1/2), pp. 59-71.
\end{enumerate}

Software:

\begin{enumerate}
\item FAAST, a C++ library for Fast Arithmetic in Artin-Schreier
  Towers. \url{http://github.com/defeo/FAAST}.
\end{enumerate}

\subsubsection*{Previous projects or activities}

Current grants:

\begin{enumerate}
\item ANR CLE (2013-2017): Cryptography from Learning with Errors.
\item DIGITEO project ARGC (2013-2016): ``Fast arithmetic for geometry
  and cryptology''. The project explores fast algorithms and
  implementations for algebraic geometry and curve-based cryptography.
\item DIGITEO project IdealCodes (2014-2016): IdealCodes
  <\url{http://idealcodes.github.io/}> spans the three research areas
  of algebraic coding theory, cryptography, and computer algebra, by
  investigating the problem of lattice reduction.
\end{enumerate}



%%% Local Variables: 
%%% mode: latex
%%% TeX-master: "../proposal"
%%% End: 

\clearpage
\begin{sitedescription}{UW}
% PIC: 999976784
% see: http://ec.europa.eu/research/participants/portal/desktop/en/orga

The Mathematics Institute at the University of Warwick was ranked 23rd
worldwide in the 2013 QS world university subject rankings.  Five
members of the Department are Fellows of the Royal Society, and one,
Regius Professor Martin Hairer, was awarded a Fields Medal in 2014.
Mathematics and Statistics at Warwick currently hold £35.8M in
research grants from EPSRC (the next highest in the UK being Cambridge
at £22.8M and Oxford at £24.2M).  Nine members of the department
currently hold ERC grants.


% See ../proposal.tex, section Members of the Consortium for a
% complete description of what should go there

\subsubsection*{Curriculum vitae}

% Curriculum of the personnel at this institution

\begin{participant}[type=leadPI,PM=3,gender=male]{John E. Cremona}
  Professor of Mathematics.  DPhil (Oxford, 1981) under Birch.
  Previous posts: Michigan, Dartmouth (US), Exeter, and Nottingham (as
  chair and Head of Pure Mathematics). Cremona has around 50
  publications, including a book and papers in Compositio and Crelle.
  He has held grants from EPSRC and other UK sources worth \pounds2.5M
  as well as \euro2.5m from the EU for Marie-Curie Research Training
  Networks in 2000-2004 and 2006-2010.  He was a Scientist in Charge
  of one of twelve teams in both of these networks, and leader of the
  work package ``Effective Cohomology Computations'' in the second,
  responsible for several deliverables.  He has been on the Scientific
  Committee of 30 international conferences (including several Sage
  Days), and given many invited lecture series.  He co-organised
  semester-long research programmes at IHP Paris (2004) and MSRI
  (2011).  He has been an editor for five journals including the LMS
  Journal of Computation and Mathematics and the Journal of the
  Foundations of Computational Mathematics (FoCM).  He has supervised
  16 PhD students, a dozen Masters students, two EU-funded Marie-Curie
  fellows and currently has three EPSRC-funded postdoctoral research
  assistants working for the LMFDB project.  Cremona has given over 30
  invited conference addresses and seminars in 9 countries in the last
  10 years; most recently he was a Plenary Speaker at the 2014 FoCM
  meeting in Montevideo, where he spoke about the LMFDB project to a
  wide international audience of computational mathematicians.

  Cremona's research includes areas of particular relevance to the
  current project.  His methods for systematically enumerating
  elliptic curves, which are the subject of a book and numerous
  papers, have been used to compile a definitive database of elliptic
  curves which is very widely cited, and now forms part of the LMFDB.
  Cremona's experience in managing such computations and the
  management, publication and electronic dissemination of the
  resulting large datasets set a standard which large-scale
  number-theoretical database projects such as the LMFDB now seek to
  match.  Cremona's experience and reputation in this field have been
  important for the LMFDB project.

  Cremona has been a leading computational number theorist in the UK
  since his PhD thesis in 1981, following in the tradition of Birch
  and Swinnerton-Dyer.  He has written thousands of lines of code in
  his C++ library eclib (one of the standard packages included in Sage
  since its inception) which includes his widely-used program {\tt
    mwrank} for computing ranks of elliptic curves.  As well as
  writing thousands of lines of new python code for Sage, he has also
  contributed to the active number-theoretical packages Pari/GP and
  Magma.
\end{participant}
%%% Local Variables:
%%% mode: latex
%%% TeX-master: "../proposal"
%%% End:


%\input{CVs/First.Last.tex}
%\input{CVs/First.Last.tex}
%\input{CVs/First.Last.tex}

\subsubsection*{Publications, products, achievements}
\begin{enumerate}
\item
The Number Theory research group at Warwick was started only in 2006,
but has rapidly risen to international status and one of the largest
and most vibrant groups in Europe, comprising 25 members (professors,
lecturers, postdoctoral researchers and early stage researchers).  Of
the group's members, two (Loeffler and Dokchitser) hold Royal Society
Research Fellowships and one (Bartel) a Royal Commission 1851
Fellowship.  Loeffler won a Leverhulme Foundation Prize jointly with
Zerbes.
\item
Several members of the Number Theory group at Warwick are \Sage
developers, including John Cremona, who has contributed thousands of
lines of code to \Sage\ since 2006 both through his {\tt eclib} C++
library and through original \Python code which forms part of the
\Sage\ library; David Loeffler, who has contributed substantially to
the modular forms module in Sage; and postdoc Marc Masdeu, who has
worked on the \Sage-Flint interface.
\end{enumerate}

\subsubsection*{Previous projects or activities}

\begin{enumerate}
\item
In 2013 Professors John Cremona and Samir Siksek, together with
co-investigators at Bristol, were awarded a six-year major grant of
£2.2M from the UK Engineering and Physical Sciences Research Council
(EPSRC) to support the L-functions and Modular Forms Database (LMFDB)
project.  This grant funds three postdoctoral researchers at Warwick,
computer equipment to host its database and website, and regular LMFDB
workshops.
\item
Each year Warwick hosts a year-long Warwick EPSRC Symposium focussing
on one area of mathematical research.  The 2012-13 Number Theory
Symposium included six research workshops and a summer school ``Number
Theory for Cryptography'' and raised the international profile of the
number theory group substantially.
\end{enumerate}

\subsubsection*{Significant infrastructure}

Computing infrastructure available to the group is excellent, with
seven dedicated machines (over 300 cores) as well as access through
Warwick's Centre for Scientific Computing which hosts a 6000-core
linux cluster and a 3500-core cluster of workstations.
\end{sitedescription}
%%% Local Variables:
%%% mode: latex
%%% TeX-master: "../proposal"
%%% End:


\clearpage
\begin{sitedescription}{ZH}

% PIC: 
% see: http://ec.europa.eu/research/participants/portal/desktop/en/orga

% See ../proposal.tex, section Members of the Consortium for a
% complete description of what should go there

The University of Zurich consistently ranks among the top 15 research institutions in Europe. It is the largest university in Switzerland, with over 26000 students, and offers the  most comprehensive academic program of the country.  It has close to 600 professors and over 5000 academic staff. 

Switzerland ranks high in innovation, competitiveness and research spending, and much of this is enthusiasm for research is concentrated around Zurich. UZH also benefits from synergies with the ETH Zurich. 

The Mathematics Institute has 17 professors and around 60 PhD students, part of a graduate school run jointly with ETH Zurich. Also joint is a Computational Science program uniting 47 researchers, mostly in the sciences, who make use of computational methods. 


\subsubsection*{Curriculum vitae}

% Curriculum of the personnel at this institution

\paragraph{Paul-Olivier Dehaye}

% months = 6
% 
%
% Fair evaluation of the number of months you will be spending on this
% specific project along the four years

% salary=YYY
%
% Approximate monthly gross salary (in term of total cost for the
% employer). If you are uncomfortable having this information in a
% public file, you can alternatively send the information to Nicolas,
% or to your institution leader if the latter will be willing to fill
% in himself the budget forms on the eu portal.

% The above information will be used to evaluate the cost of the
% project for the institutions. You may remove the above comments once
% you have filled in the months= and salary= lines.

Paul-Olivier Dehaye is a Swiss National Science Foundation Assistant Professor 
at the University of Zurich. After his Phd at Stanford (2006), he has also worked in Oxford, 
at the Institut des Hautes Etudes Scientifiques and at ETH Zurich. He currently has 13 
papers published in international peer-reviewed journals. He is currently supervising three PhD students and one post-doc.

His main research is at the intersection of Number Theory and Combinatorics, and in 
particular in Random Matrix Theory conjectures. He has additional interests in FLOSS, 
semantic tools, massive online education and crowdsourcing, all with the view of 
enabling larger scale mathematical and scientific collaborations. He is also member of the program committee of CICM 2015 (Conference on Intelligent Computer Mathematics).

He is a contributor to the \Sage, \LMFDB and \OpenEdX projects, and has organised two 
conferences relating to these projects. The first was held in 2013 in Edinburgh, and organised
jointly with Nicolas Thiery. Its official title was  \emph{Online databases: from L-functions to combinatorics}, 
and it served as a precursor to some aspects of this grant, by bringing the \SageCombinat and \LMFDB communities together. 
The second was held in June 2014 in Zurich and organised jointly with Stanford. It aimed at building a community around the open 
source python-based MOOC platform \OpenEdX, and opened a series of conferences now held twice annually. 

Dehaye has also taught for two years now a python course using \OpenEdX, which aims to bring 
first year students to the level of potential contributor to \Sage. This course also has a 
project-based component. It is now run locally for a small audience, but could be scaled up 
in various ways. 



\subsubsection*{Publications, products, achievements}
\begin{compactenum}
\item Dehaye is editor for the LMFDB, and has contributed to the project since its inception (2007). His students are also contributors. 
\item For several of his papers, Dehaye used extensive computer-assisted experimentation (using mostly the combinatorial components of \Sage) to inform the formulation of the eventual theorems, including for instance:
\begin{compactitem}
\item \emph{Combinatorics of lower order terms in the moments conjecture for the Riemann zeta function}, \textsf{arXiv:1201.4478}
\item \emph{Integrality of hook ratios}, \textsf{arXiv:1111.5959}, in \emph{Proceedings of the Formal Power Series and Algebraic Combinatorics 2012 (Nagoya) conference}.
\item \emph{A multiset hook length formula and some applications}, with Guoniu Han, in \emph{Discrete Mathematics}, (311) 23--24, pp.~2690--2702, 2011.
\item \emph{A note on moments of derivatives of characteristic polynomials}, in \emph{DMTCS Proc. Formal Power Series and Algebraic Combinatorics 2010}, vol.~12.
\item \emph{Joint moments of derivatives of characteristic polynomials}, in \emph{Algebra and Number Theory Journal} {\bf 2} (2008), no.~1,  pp.~31--68.
\end{compactitem}
\item Dehaye has been extensively involved in teaching \Python and \Sage at UZH, through an online platform called \OpenEdX. This has led him to organise the first community-driven conference around this (open-source) software, and to develop (together with students) additional tools, such as \texttt{edx-presenter}. 
\end{compactenum}

\subsubsection*{Previous projects or activities}
\begin{description}
\item[Swiss National Foundation PP00P2/138906: Combinatorics of partitions and number theoretic aspects] This grant covers research at the intersection of number theory and combinatorics. Some of its aims are to uncover combinatorial structures that lurk in complicated formulas for moments of $L$-functions (such as the Riemann zeta function). As such, it is simultaneously a heavy user of numerical methods from analytic number theory and of combinatorial techniques implemented in \Sage.
\end{description}

\subsubsection*{Significant infrastructure}
\begin{compactenum}
\item The Faculty of Sciences of the UZH benefits from very strong specialised IT support in the form of the S3IT group. They operate for instance a research cloud and a local supercomputer,  and provide further assistance for the design of hardware and software systems to further research. They have a pool of software engineers that can be hired on projects such as this one for shorter periods.
\item UZH has a stake in Piz Daint, which is, at the time of the submission, 
the sixth largest (and most energy-efficient) supercomputer in the world
(and currently being upgraded).
\end{compactenum}
\end{sitedescription}


\begin{draft}
\vspace{1cm}

\begin{verbatim}
- [X] checked that sum of person months put into finance request is
  the same as sum of person months associated with the Work Packages
  (in proposal.tex, as defined as part of the \begin{workpackage}"
  command.
  
  Take into account person months associated with work package 1, time
  of all staff to be hired and work on the project (including
  investigators). Figure 5 helps with a quick check of the sums over
  different work packages.
  
  Only 1 for Zurich in management!

- [X] completed site specific resource summary in resources.tex,
  including table of non-staff costs. This is compulsory (EU
  regulations) if the non-staff cost exceed 15% of the total cost, and
  is likely to be the case for most of the partners. We ask everybody
  to do it, to be consistent and show transparently how we have
  planned our total budget.

- [X] Have all our tasks a designated lead institution? Check in the
  Work Packages that all the tasks you are involved in have a
  dedicated lead party. If the lead party is "USO", then use:
  \begin{task}[lead=USO]

- [X] Have all our deliverables a designated lead institution [using
  the 'lead=' key]?

- [X] In the "Members of the consortium section", have we addressed "a
  description of the legal entity and its main tasks, with an
  explanation of how its profile matches the tasks in the
  proposal"? See Entry for Paris Sud and Southampton as examples.

- [X] In the Members of the consortium section, have we given
  descriptions of all the people we intend to hire (even if we don't
  know who that is yet). 
  
- [X] Do all our tasks include us in the list of sites involved?
\end{verbatim}
\end{draft}

%KEY-MORE-TODOS


%%% Local Variables:
%%% mode: latex
%%% TeX-master: "../proposal"
%%% End:

%  LocalWords:  sitedescription subsubsection Dehaye emph textsf Guoniu organise texttt
%  LocalWords:  edx-presenter Piz Daint

\clearpage
\begin{sitedescription}{LL}

Logilab (\url{http://www.logilab.fr/}) is a french SME focused on using the web and free software to help
scientists. It has been in business since 2000 and counts over 20 engineers and
PhDs proficient in software engineering, knowledge representation, design and
management of IT infrastructure, etc.

Logilab invests 15\% of its turnover in research and development and has been
part of several R\&D projects at the national and european levels, always to
provide technical expertise and support to the other partners.

In the context of this project, Logilab will innovate to support the partners
with tools and infrastructure, including open databases to flexibly store
mathematical objects, user interfaces to visualize complex mathematical
properties, fluid workflow tools to ease large-scale collaboration, etc.

%Logilab's PIC number is 948455525.

% See ../proposal.tex, section Members of the Consortium for a
% complete description of what should go there

\subsubsection*{Curriculum vitae}

% Curriculum of the personnel at this institution

\begin{participant}[type=PI,PM=2]{Florent Cayré}
  Engineer with a Master Degree from École Centrale de Paris (top
  French engineering school), Florent Cayré spent six years in SNECMA
  as an engineer designing the numeric tools for the turbines design
  and then was head of the group ``Méthodes de conception de turbines
  ; aérothermique et combustion'' performing collaborative R\&D programs
  and development and integration of various numeric tools (\Python, \software{C++},
  \software{C}, \software{Fortran}). He co-funded the SecondWeb company tat carries 
  out the development of complex Web applications based on \software{CubicWeb} platform.

  Being the Head of the scientific computing department of Logilab since 2012,
  Florent is responsible for team management, strategic vision,
  projects monitoring and technical expertise. He developed several
  tools for defining and managing computations, and producing enhanced
  result reports through \IPython notebooks.
\end{participant}

\begin{participant}[type=lead PI,PM=4]{Olivier Cayrol}
  Engineer with a Master Degree from École Centrale de Lyon (top
  French engineering school), Olivier Cayrol spent three years at the
  R\&D department of PSA-Peugeot Citroën, as developer and project
  manager on the modelling and simulation of electronic embedded car
  control devices.

  Co-founder and deputy-CEO of Logilab, he designed and developed the
  system that generates the Logilab's documents from ReST data
  sources. This system is based on several free software products such as
  \software{Sphinx} or \software{reportlab}, and defines numerous extensions for answering
  the specific needs.
\end{participant}

\begin{participant}[type=PI,PM=6,gender=male]{David Douard}
  holds a Master Degree from the Ecole Nationale Sup\'{e}rieure de Physique de
  Strasbourg and a PhD in computer science from Universit\'{e} Paris VI.

  As a PhD studend, he developed a graphical user interface for the
  computational system he used for his research as well as lots of processing
  and visualisation tools.

  He worked two years at EDF where he was the lead developer of the software in
  charge of evaluating financial risks on the energy market with a specific,
  intensive work on the graphical user interface and on the code driving the
  simulations.

  He has been working at Logilab since 2006, building complex scientific
  applications involving the management and visualisation of large amounts of
  data. He has trained tens of engineers and researchers to \software{C/C++},
  \Python, GUI libraries (\software{Tk}, \software{PyQt}, \software{wxPython}),
  \software{Fortran} and scientific computing.
\end{participant}

\begin{participant}[type=PI,PM=10]{Julien Cristau}
  Master Degree and PhD from University Paris VII. Research on 
  automata and linear games. As a Debian Developer since 2007 he
  maintained key components such a the X11 windowing system and
  acted as a Debian Release Manager since 2011.
  
  Working as a software engineer in the R\&D department of Logilab since
  2011, he developed software using many different languages and systems,
  helped release and distribute software on many different platforms,
  maintained parts of the infrastructure and trained other people.
\end{participant}

\begin{participant}[type=R,PM=12,gender=male]{Serge Guelton}
  \TOWRITE{Serge Guelton}{CV}
  Core dev of Pythran ...
\end{participant}
%%% Local Variables:
%%% mode: latex
%%% TeX-master: "../proposal"
%%% End:


\subsubsection*{Publications, products, achievements}

\begin{enumerate}
 \item CubicWeb is a semantic web framework that is available under the LGPL
   license and received a DataConnexion prize from Etalab (the french government
   team dedicated to Open Data)

\item Logilab has been contributing to free software since its creation in 2000
  and is known for it in France and several other countries. It authored PyLint,
  the static Python code checker used worldwide, and has always had at least one
  Debian Developer on staff, thus supporting the largest free software
  distribution used by millions of people.

 \item At OBHM 2013, the 19th Annual Meeting of the Organization for Human Brain
   Mapping, Logilab presented a poster which explains the work done using
   CubicWeb on brain imaging and genetics data in collaboration with INRIA,
   INSERM and the CEA during the Brainomics project co-financed by Agence
   nationale de la Rercherche.

\end{enumerate}

\subsubsection*{Previous projects or activities}

\begin{enumerate}
\item Logilab was a member of the consortium for the ASWAD EU project, with a
  role of software developer. The project demonstrated the interest of free
  software for setting up workflows in public administrations.
\item Logilab was a member of the consortium for the KIDDANET EU project, with a
  role of software developer. The project implemented a proxy to protect kids
  browsing the internet.
\item Logilab was a member of the consortium for the PYPY EU project, with a
  role of software developer. The project implemented a Python interpreter in
  Python, to explore new ways to compile and optimize the execution of Python
  code.
\item Logilab was a member of the consortium for the OpenHPC french project,
  with a role of software developer. The project advanced the state of the art
  of free software for high performance simulation.
\item Logilab was a member of the consortium for the BRAINOMICS french ANR
  project, with a role of software developer. The project advanced the state of
  the art of shared databases for the brain imaging and genetics data.
\item Logilab was a member of the consortium for the THALER french ANR project,
  with a role of software developer. The project advanced the state of the art
  of free software for high performance simulation of molecular dynamics.
\end{enumerate}

\subsubsection*{Significant infrastructure}

Logilab is maintaining its own infrastructure, using virtualisation techniques
and tools such as OpenStack, SaltStack, Docker, etc.

\end{sitedescription}


%KEY-MORE-TODOS


%%% Local Variables:
%%% mode: latex
%%% TeX-master: "../proposal"
%%% End:

%  LocalWords:  sitedescription Logilab TOWRITE subsubsection publi Brainomics KIDDANET
%  LocalWords:  fr

\clearpage
\begin{sitedescription}{SR}

Dedicated to tackling scientific challenges with long-term impact and of genuine importance to real life, Simula Research Laboratory (Simula) offers an environment that emphasizes and promotes basic research. At the same time, we are deeply involved in research education and application-driven innovation and commercialisation. 

Simula was established as a non-profit, limited company in 2001, and is fully owned by the Norwegian Ministry of Education and Research. Its research is funded through competitive grants from national funding agencies and the EC, research contracts with industry, and a basic allowance from the state.  Simula?s operations are conducted in a seamless integration with the two subsidiaries Simula School of Research and Innovation and Simula Innovation.

At its outset, the laboratory was given the mandate of becoming an internationally leading research institution within select fields in information and communications technology. These fields are (i) communication systems, including cyber-security; (ii) scientific computing, aiming at fast and reliable solutions of mathematical models in biomedicine, geoscience, and renewable energy; and (iii) software engineering, focusing on testing and verification of mission-critical software systems, and on planning and cost estimation of large software development projects. Recent evaluations state that Simula has met its challenge and is an acknowledged contributor to top-level research in its focus areas. Specifically, in the 2012 national evaluation of ICT research organized by the Research Council of Norway and conducted by an international expert panel, Simula received the highest average score (4.67) on a 1-5 scale among all evaluated institutions.  In comparison, the national average was 3.38. Only five of the 62 research groups evaluated were awarded the top grade (5), and two of these five groups are located at Simula.

Simula is currently hosting one Norwegian Centre of Excellence, Center for Biomedical Computing (2007-2017), and one Norwegian Centre for Research-based Innovation, Certus (2011-2018). In addition, we participate as research partner in another Centre for Research-based Innovation, Centre for Cardiological Innovation (2011-2018), hosted by Oslo University Hospital. These two center-oriented schemes are the most prestigious funding instruments offered by the Research Council of Norway. 

\subsubsection*{Curriculum vitae}

% Curriculum of the personnel at this institution

\begin{participant}[type=leadPI, gender=male, PM = 8]{\bf Hans Petter Langtangen}
 is director of Center for Biomedical Computing at Simula Research Laboratory, a Norwegian Center of Excellence doing inter-disciplinary research in the intersection of mathematics, physics, computer science, geoscience and medicine. Langtangen is on 80\% leave from a position as professor at the Department of Informatics, University of Oslo.

Langtangen received his PhD from the Department of Mathematics, University of Oslo, in 1989, and then worked at SINTEF before being hired as assistant professor at the University of Oslo in 1991. After being promoted to full professor of mechanics at the Department of Mathematics in 1998, he moved in 1999 to a professorship in computer science. In the period 1999-2002 he also held an adjunct professor position at the Department of Scientific Computing at Uppsala University in Sweden. The Simula Research Laboratory was formed in 2001, and Langtangen has since then worked with research and management at this laboratory. The scientific computing activity at Simula has been awarded the highest grade, Excellent, by five panels of top-ranked international scientists in the period 2001-2012.

Langtangen's research is inter-disciplinary and involves continuum mechanical modeling, applied mathematics, stochastic uncertainty quantification, and scientific computing, with applications to biomedicine and geoscience in particular. He has also been occupied with developing and distributing scientific software to make the research results more widely accessible and help accelerate research elsewhere. For over three decades he has been very active with teaching and supervision.

The scientific production consists of 4 authored books, 3 edited books, about 60 papers in international journals, about 60 peer-reviewed book chapters and conference papers, and over 130 scientific presentations. The publications cover fluid flow, elasticity, wave propagation, heat transfer, finite element methods, uncertainty quantification, and implementation techniques for scientific software. Langtangen is on the editorial board of 7 journals and serves as Editor-in-Chief of the leading SIAM Journal on Scientific Computing. He is also a member of the Norwegian Academy of Science and Letters.

\end{participant}

%\input{CVs/First.Last.tex}
%\input{CVs/First.Last.tex}

\subsubsection*{Publications, products, achievements}

\begin{enumerate}
\item  A. Logg, K.-A. Mardal, G. N. Wells et al. Automated Solution of Differential Equations by the Finite Element Method, Springer (2012). [doi:10.1007/978-3-642-23099-8]
\item P. E. Farrell, D. A. Ham, S. W. Funke, and M. E. Rognes.	Automated Derivation of the Adjoint of High-Level Transient Finite Element Programs. SIAM J. Sci. Comput. 35-4 (2013), pp. C369-C393
\item H.P. Langtangen. A Primer on Scientific Programming with Python. Texts in Computational Science and Engineering, Springer (2014), 792 pp..
\end{enumerate}

\subsubsection*{Previous projects or activities}

\begin{enumerate}
\item The Centre for Biomedical Computing, a Norwegian Centre of Excellence, awarded by the Research Council of Norway. Duration: 2007-2017. Budget: 75 MNOK (10 MEUR). 
\item The FEniCS Project (www.fenicsproject.org) Duration: 2007--on-going. 
\end{enumerate}

\subsubsection*{Significant infrastructure}

The fully owned Simula subsidiary Simula Innovation handles pre-commercial innovation projects, creation and follow-up of company spin-offs, and general support for entrepreneurs.
\end{sitedescription}
%%% Local Variables:
%%% mode: latex
%%% TeX-master: "../proposal"
%%% End:

\clearpage
\begin{sitedescription}{UG}

Today Ghent University attracts over 41000 students, with a foreign student population of about 4100.
In 2015, the university invested over 242 million Euros in research projects
on behalf of public and private partners, and employed around 6600 academic staff members.
Ghent University is ranked 71st in the Shanghai and 118th in the Times ranking.
The University has participated in more than 200 research projects in the EU's Sixth Framework Programme (2002--2006)
and in 260 projects in the Seventh Framework Programme, of which 27 ERC grants and 26 Marie Curie Fellowships.
Ghent University coordinated 42 collaborative projects in FP7.
Up till now, Ghent University is under H2020 (the new Framework Programme (2014--2020)) involved in 100 projects, and coordinates 9 of them.
Among these newly acquired projects, Ghent University hosts 15 ERC grant holders and supervises 9 Marie Sk{\l}odowska Curie fellowships.
The university provides excellent training opportunities to both young and experienced researchers,
and awarded 660 PhD degrees in 2015 of which over 30\% went to international young researchers.
The university is one of the fastest growing European universities in terms of research capacity and productivity,
and its commitment to European research excellence is reflected by the recent extension of the `European Office'
in its Research Office, i.e.~the department overseeing, guiding and administering research projects.

\subsubsection*{Curriculum vitae}

\begin{participant}[type=leadPI,PM=32.5,gender=male]{Jeroen Demeyer}
is a post-doctoral assistant at Ghent University (Belgium).
He has broad mathematical interests, mainly in number theory and its
connections with logic.
He received his PhD in mathematics in Ghent in 2007.
Since then, he has been working as a post-doc in Ghent,
except for a stay of 15 months at the Scuola Normale Superiore in Pisa (Italy).
At Ghent University, he introduced \Sage{} in the mathematics education
and teaches two courses with it.

He is a very active contributor to \Sage{} and he has been its
release manager for a period of 3 years, from 2011 to 2013.
He is the main author of the \software{cysignals} package
for interrupt and signal support in \Cython{}.
He also made various contributions to other projects,
such as \PariGP{}, \Cython{} and \Jupyter{}.
\end{participant}

%%% Local Variables:
%%% mode: latex
%%% TeX-master: "../proposal"
%%% End:


\subsubsection*{Publications, products, achievements}

At the Department of Mathematics of Ghent University,
researchers have always been interested in computational aspects:
Jeroen Demeyer has contributed to \Sage{} and
Jan De Beule and Michel Lavrauw are co-authors
of the \GAP{} package FinInG for finite incidence geometry.

% \subsubsection*{Previous projects or activities}
%
% Nothing significant that I can think of...

\subsubsection*{Significant infrastructure}

Ghent University has a large High Performance Cluster infrastructure,
consisting of 5 distinct clusters for various purposes, totalling 11328 CPU cores.
On top of this, it also hosts the Tier-1 Flemish Supercomputer,
a single cluster with 8448 cores, to be used for all research institutions in Flanders.

Ghent University is using SageMath for teaching in the mathematics education.
To support this, there are two 12-core servers running a Sage Notebook server.
There is also a VM running Jupyter Hub.
Eventually, this should also be used for teaching purposes,
but this is currently in an experimental phase.
\end{sitedescription}
%%% Local Variables:
%%% mode: latex
%%% TeX-master: "../proposal"
%%% End:

\clearpage
\begin{sitedescription}{XFEL}
  \label{sitedescription:xfel}

% PIC:
% see: http://ec.europa.eu/research/participants/portal/desktop/en/orga

% See ../proposal.tex, section Members of the Consortium for a
% complete description of what should go there

  The European X-Ray Free-Electron Laser Facility GmbH is a limited
  liability company under German law. At present, 11 countries are
  supporting European XFEL through cash and in-kind contributions:
  Denmark, France, Germany, Hungary, Italy, Poland, Russia, Slovakia,
  Spain, Sweden, and Switzerland. The company is in charge of the
  construction and operation of the European XFEL, a 3.4$\,$km long X-ray
  free-electron laser facility extending from Hamburg to the
  neighbouring town of Schenefeld in the German federal state of
  Schleswig-Holstein. Civil construction started in early 2009; the
  beginning of user operation is planned for 2017. With its repetition
  rate of 27,000 pulses per second and a peak brilliance a billion
  times higher than that of the best synchrotron X-ray radiation
  sources, it is expected that the European XFEL will enable the
  investigation of still open scientific problems in a variety of
  disciplines (physics, structural biology, chemistry, planetary
  science, study of matter under extreme conditions and many others).

  European XFEL, a landmark on the ESFRI Roadmap, is a single site
  X-ray research infrastructure. When operational, 3~beamlines and
  6~experiments will be available for scientific users. The SASE1
  beamline comprises the instruments Single Particles, clusters, and
  Biomolecules and Serial Femtosecond Crystallography (SPB/SFX) and
  Femtosecond X-ray Experiments (FXE), SASE 2 includes Materials
  Imaging and Dynamics (MID) and High Energy Density Science (HED) and
  SASE3 Small Quantum Systems (SQS), and Spectroscopy \& Coherent
  Scattering (SCS).


\medskip In the context of this proposal, Hans Fangohr has long
standing experience in high performance computer simulation to advance
science and engineering, and the education of researchers in the
most effective pursuit of computational science.

European XFEL is using IPython and the Jupyter
Notebook as core utilities in their large scale experiment control,
data capture and data analysis.

\subsubsection*{Curriculum vitae}

% Curriculum of the personnel at this institution
%
\input{CVs/Hans.Fangohr.XFEL.tex}
% \begin{participant}[PM=2,type=PI]{Ian Hawke}
Ian Hawke is a lecturer in Applied Mathematics at the University of
Southampton and a co-director of the $\pounds$10$\,$m EPSRC Centre for Doctoral
Training in Next Generation Computational Modelling. An expert in
nonlinear simulations of relativistic matter and numerical techniques,
he has taught numerical methods in many contexts for ten years. The
initial author of the “Whisky” relativistic hydrodynamics code, he has
been a contributor to and maintainer of a range of projects used
across the numerical relativity community, including the Einstein
Toolkit, the Cactus infrastructure and the Carpet mesh refinement
code. His recent research has concentrated on numerical methods for
relativistic matter beyond ideal fluids, including modelling sharp
transitions and surfaces, relativistic elasticity, and the first
nonlinear simulations of relativistic multifluids.

XXX Add Jupyter Notebook based MOOC experience XXX.
\end{participant}

%%% Local Variables:
%%% mode: latex
%%% TeX-master: "../proposal"
%%% End:


\input{CVs/Marijan.Beg.XFEL.tex}
% %\input{CVs/First.Last.tex}
%
\subsubsection*{Publications, products, achievements}

\begin{compactenum}
\item Open Source micromagnetic simulation framework Nmag,
  \href{http://nmag.soton.ac.uk}{http://nmag.soton.ac.uk}, Thomas
  Fischbacher, Matteo Franchin, Giuliano Bordignon, Hans Fangohr: \emph{
A Systematic Approach to Multiphysics Extensions of Finite-Element-Based Micromagnetic Simulations: Nmag
IEEE Transactions on Magnetics \textbf{43}, 6, 2896-2898 (2007)}
\item Other open source contributions to the micromagnetic simulation
  community: OVF2VTK, higher order anisotropy extensions to OOMMF,
  OVF2MFM, summarised at
  \href{http://www.southampton.ac.uk/~fangohr/software/index.html}{http://www.southampton.ac.uk/~fangohr/software/index.html}
\item H. Fangohr.
\emph{A Comparison of \software{C}, \Matlab and \Python as Teaching Languages in Engineering}
Lecture Notes on Computational Science \textbf{3039}, 1210-1217 (2004)
\end{compactenum}

\end{sitedescription}



%KEY-MORE-TODOS



%%% Local Variables:
%%% mode: latex
%%% TeX-master: "../proposal"
%%% End:

%  LocalWords:  sitedescription Programme organisations programmes Centres subsubsection
%  LocalWords:  micromagnetic Nmag Fischbacher Franchin Bordignon Fangohr emph textbf
%  LocalWords:  Multiphysics summarised Iridis TFlops Modelling

\clearpage
\begin{sitedescription}{FAU}
  Friedrich Alexander Universit\"at Erlangen/N\"urnberg (FAU) is a public research
  university in the cities of Erlangen and Nuremberg, Germany. FAU is the second largest
  state university in the state Bavaria. It has 5 faculties, 23 departments/schools, 30
  clinical departments, 19 autonomous departments, 656 professors, and ca 40\,000
  students.

  Prof. Dr. Michael Kohlhase has moved to \site{FAU} from \site{JU} 1. September 2016 and
  continues his research there. 

  The KWARC (KnoWledge Adaptation and Reasoning for Content~\cite{KWARC:online}) Group
  headed by {\emph{Prof.\ Dr.\ Michael Kohlhase}} specialises in knowledge management for
  STEM.  Formal logic, natural language semantics, and semantic web technology provide the
  foundations for the research of the group. The KWARC group moves to FAU over fall 2016. 
  

\subsubsection*{Curriculum vitae}
% Curriculum of the personnel at this institution
\paragraph{Ursula Martin}

% months=3
%
% Fair evaluation of the number of months you will be spending on this
% specific project along the four years

% salary=XXX
%
% Approximate monthly gross salary (in term of total cost for the
% employer). If you are uncomfortable having this information in a
% public file, you can alternatively send the information to Nicolas,
% or to your institution leader if the latter will be willing to fill
% in himself the budget forms on the eu portal.

% The above information will be used to evaluate the cost of the
% project for the institutions. You may remove the above comments once
% you have filled in the months= and salary= lines.

Dr. Michael Kohlhase is full professor for Computer Science at Jacobs University Bremen
and an associate adjunct professor at Carnegie Mellon University.

He studied pure mathematics at the Universities of T"ubingen and Bonn (1983 - 1989) and
continued with computer science, in particular higher-order unification and automated
theorem proving (Ph.D. 1994, Saarland University). 

His current research interests include knowledge representation for mathematics,
inference-based techniques for natural language processing, and computer-supported
education. He has pursued these interests during extended visits to Carnegie Mellon
University, SRI International, and the Universities of Amsterdam, Edinburgh, and Auckland.

Michael Kohlhase is recipient of the dissertation award of the Association of German
Artificial Intelligence Institutes (AKI; 1995) and of a Heisenberg stipend of the German
Research Council (DFG 2000-2003). He was a member of the Special Research Action 378
(Resource-Adaptive Cognitive Processes), leading projects on both automated theorem
proving and computational linguistics. Michael Kohlhase was trustee of the Conference on
Automated Deduction (CADE), Mathematical Knowledge Management (MKM), and the CALCULEMUS
conference, he is a member of the W3C Math working group, president of the OpenMath
Society, and the general secretary of the Conference on Intelligence Computer Mathematics
(CICM).

%%% Local Variables:
%%% mode: latex
%%% TeX-master: "../../proposal"
%%% End:


 



\begin{picv}[PM=6]{Florian Rabe}
  is a post-doctoral research fellow at Jacobs University Bremen.  He completed his PhD in
  2008 and his habilitation in 2014 and holds the venia legendi.

  He has worked on the formal representation and management of mathematical knowledge for
  10 years.  He was a lead researcher in the LATIN project (2009-2012), which produced a
  highly modular and integrated library of formal languages for knowledge representation.
  He is currently a principal investigator in the OAF project, which builds on LATIN to
  produce an archive of libraries of formal mathematical knowledge.

  He is the creator and main developer of the MMT language and system, which are the
  backbone of both LATIN and OAF.  MMT has been developed for 8 years with contributions
  from $>10$ people and currently consists of $>30,000$ lines of Scala code.

  He served in the organization committee of 2 and the program committee of 6
  international conferences (2 as track chair) on intelligent computer mathematics, and
  has organized 4 international workshops on module systems and libraries for mathematical
  knowledge.  He has authored 65 research papers (11 in international journals) and has
  supervised 17 undergraduate and graduate theses.
\end{picv}

%%% Local Variables:
%%% mode: latex
%%% TeX-master: "../proposal"
%%% End:

%\begin{participant}[PM=36]{Christian Maeder}
  Dr. Christian Maeder is a research developer at Jacobs University. He has extensive
  experience in designing and implementing logic-based software system. He is the lead
  implementor of the HETS system.
\end{participant}

%%% Local Variables:
%%% mode: latex
%%% TeX-master: "../proposal"
%%% End:


 



\begin{participant}[type=JRes,PM=24,salary=5900,gender=male]{Mihnea Iancu}
  Mihnea Iancu is a third-year doctoral student at the KWARC group. He is the lead
  implementor of the MathHub.info system. He has worked extensively on the representation
  for formal and informal mathematical knowledge in the MMT syste.m
\end{participant}

%%% Local Variables:
%%% mode: latex
%%% TeX-master: "../proposal"
%%% End:


 




\subsubsection*{Relevant previous experience:}

The KWARC group is the lead implementor of the OMDoc (Open Mathematical Document) format
for representing mathematical knowledge \cite{Kohlhase:OMDoc1.2} and redeveloped its
formal core in the OMDoc/MMT format~\cite{RabKoh:WSMSML13}. The latter has been
implemented in the MMT system~\cite{MMTSVN:on,RabKoh:WSMSML13} which provides efficient
implementations of the computational primitives such as type checking, flattening, and
presentation at a logic/foundation-independent level.  The group has developed services
powered by such semantically rich representations, different paths to obtaining them, as
well as platforms that integrate both aspects.  \emph{Services} include the adaptive
context-sensitive presentation framework provided by the MMT API and the semantic search
engine MathWebSearch\cite{KohSuc:asemf06,ProKoh:mwssofse12}. 

Semantic services can be integrated into the documents generated from OMDoc/MMT
representations, making them into ``active documents'', i.e. documents that are
interactive and adaptive to the user and situation.  For \emph{obtaining} rich content,
the group investigates assisted manual editing \cite{JucKoh:sidesc10:biblatex} as well as
automatic annotation using linguistic techniques \cite{GinJucAnc:alsaacl09}.  Finally,
KWARC has developed the \textsf{MathHub.info} portal a community-based library and
knowledge management system for flexiformal libraries, which can be used for semantic
publishing and eLearning~ \cite{KohDavGin:psewads11,MathHub:on,IanJucKoh:sdm14}.

The \textsf{OMDoc/MMT} knowledge representation format and the \textsf{MathHub.info}
system will an important basis for the developments Work Packages 4 and 6.

Michael Kohlhase has initiated and led the CALCULEMUS! IHP-Research and Training Network
and participated in the FP6 IST MoWGLI (Mathematics on the Web: Get it by Logic and
Interfaces) project, the FP6 CSA Once-CS (Open Network of Centres of Excellence in Complex
Systems), The FP7 EDC project WebALT (Web Advanced Learning Technologies).

\subsubsection*{Specific expertise:}
\begin{compactitem}
\item Modelling formal structures of mathematical knowledge in a web-scalable way.
\item Transforming large collections of legacy scientific publications to semantically
  structured markup.
\item Designing user interfaces for authoring and interacting with mathematical knowledge.
\end{compactitem}

\site{JU} lead \WPtref{dksbases} and tasks related to active and structured documents as
user interfaces (\WPtref{UI}) to the math VRE to be developed in \TheProject. It will run
an infrastructure for authoring and interacting with such documents and a search engine for
the \DKS base. 
\end{sitedescription}

%%% Local Variables: 
%%% mode: latex
%%% TeX-master: "../proposal"
%%% End: 

% LocalWords:  site-jacu.tex clange sitedescription emph compactitem pn semmath RabKoh
% LocalWords:  prosuming-flexiformal KohSuc asemf06 GinJucAnc alsaacl09 StaKoh ProKoh
% LocalWords:  tlcspx10 KohDavGin psewads11 ednote Radboud Bia ystok CALCULEMUS textsf
% LocalWords:  textbf keypubs OntoLangMathSemWeb uwb Deyan Ginev Stamerjohanns mwssofse12
% LocalWords:  searchability krasm Iacob KohKoh skmfe08 HilKohSta copmem06 ako IanJucKoh
%  LocalWords:  subsubsection sdm14 Centres vspace TOWRITE Modelling WPtref dksbases

\clearpage
\begin{sitedescription}{LEEDS}

% Our person months should be 54 
% Dissemination: 24, HPC: 12, Social: 18

% PIC: 999976881
% see: http://ec.europa.eu/research/participants/portal/desktop/en/orga

% See ../proposal.tex, section Members of the Consortium for a
% complete description of what should go there
The University of Leeds is a leading Research University in the
United Kingdom that was ranked 101st in the World in the most recent
2018 QS World University Rankings. In the 2014 UK Research Excellence Framework (REF),
Leeds was ranked in 10th place for research power, a measure of the number of people producing the highest quality research.
The assessment identified that over 80\% of research at Leeds has a top quality rating of either “world leading” or “internationally excellent”.

Dr Mike Croucher is a leading expert in Research Software Engineering (RSE) based in the Departments of Computer Science and IT Services.
He has an EPSRC RSE fellowship and is Head of Research Computing at The University of Leeds which includes High Performance Computing, Research Software Engineering and Data Analytics.
Mike is a Fellow of the Software Sustainability Institute and has over 13 years experience in Research Software.

The University of Leeds will focus on tasks on HPC (\taskref{hpc}{hpc-jupyter}),
dissemination (\taskref{dissem}{project-intro}).
\TheProject outputs. Most of the research tasks will be accomplished by the two researchers hired on the project, under the supervision of Mike Croucher who has replaced Neil lawrence as leadPI in the course of the project. The total amount ot PM goes up to 72.

\subsubsection*{Curriculum vitae}

% Curriculum of the personnel at this institution

%\begin{participant}[type=leadPI,PM=6,salary=10000,gender=male]{Neil Lawrence}
  Neil Lawrence received his bachelor's degree in Mechanical
  Engineering from the University of Southampton in 1994. Following a
  period as an field engineer on oil rigs in the North Sea he returned
  to academia to complete his PhD in 2000 at the Computer Lab in
  Cambridge University. He spent a year at Microsoft Research in
  Cambridge before leaving to take up a Lectureship at the University
  of Sheffield, where he was subsequently appointed Senior Lecturer in
  2005. In January 2007 he took up a post as a Senior Research Fellow
  at the School of Computer Science in the University of Manchester
  where he worked in the Machine Learning and Optimisation research
  group. In August 2010 he returned to Sheffield to take up a
  collaborative Chair in Neuroscience and Computer Science.

  Neil's main research interest is machine learning through
  probabilistic models. He focuses on both the algorithmic side of
  these models and their application. He has a particular focus on
  applications in personalized health and computational biology, but
  happily dabbles in other areas such as speech, vision and graphics.

  Neil was Associate Editor in Chief for IEEE Transactions on Pattern
  Analysis and Machine Intelligence (from 2011-2013) and is an Action
  Editor for the Journal of Machine Learning Research. He was the
  founding editor of the JMLR Workshop and Conference Proceedings
  (2006) and is currently series editor. He is Programme Chair for
  AISTATS 2012 and has served on the programme committee of several
  international conferences. He was an area chair for the NIPS
  conference in 2005, 2006, 2012 and 2013, Workshops Chair in 2010 and
  Tutorials Chair in 2013. He was general chair of AISTATS in 2010 and
  AISTATS Programme Chair in 2012. He is Program Chair of NIPS in
  2014.

  Neil is a strong advocate of open source software in machine
  learning and has given many invited talks on the subject. Since 2004
  his research group has made all their implementaitons available,
  most recently using Python and IPython as the main medium for
  communicating their work. Their Gaussian process python software
  framework\footnote{\url{https://github.com/SheffieldML/GPy}} is
  becoming a standard platform for research in these methods and
  underpins a series of Summer Schools and four day road shows that
  Neil has led in the
  area.\footnote{\url{http://ml.dcs.shef.ac.uk/gpss/}}
\end{participant}
%%% Local Variables:
%%% mode: latex
%%% TeX-master: "../proposal"
%%% End:

\begin{participant}[type=R, PM=6, salary=7500,gender=male]{Michael Croucher} is a Research Software Engineer at The University of Sheffield. He received his bachelor's degree in Theoretical Physics from The University of Sheffield in 1999 and completed his Theoretical Physics PhD there in 2005. He subsequently took up a research-support post in The University of Manchester's IT Services department before being appointed as Head of Application Software Support for the Faculty of Engineering and Physical sciences.
 
Michael is passionate about improving the quality of research
software. He enables researchers to ask larger and more complex
research questions by improving the software they develop. By teaching
and demonstrating fundamental software engineering principles, he
assists academic colleagues in producing robust, reproducible, fast
and correct software.
 
He achieves these aims via a number of means:
 
Consultancy: He works directly on research code written in various
languages. For a sample of recent client testimonials, see
\href{http://www.walkingrandomly.com/?page_id=5122}{http://www.walkingrandomly.com/?page\_id=5122}
 
Outreach: He is the author of \url{http://www.walkingrandomly.com/} -
a blog focused on mathematics and scientific computing with over
400,000 unique visitors a year. The associated twitter account,
@walkingrandomly, has almost 3000 followers.  He is a fellow of the
Software Sustainability Institute, an organisation that promotes and
supports research software engineering.
 
Education: He has taught programming and Software Carpentry classes to
hundreds of researchers and uses his contacts with education and
industry to arrange specialised teaching events relating to research
software.
 
Mentoring: He acts as a 'code-coach' to new researchers, providing
code reviews and private tutorials.
 
Vendor liaison: He has strong relationships with several vendors of
scientific software including Mathworks, Wolfram Research, Maplesoft
and NAG. These relationships have led to many fruitful collaborations
between them and academic colleagues.
 
High Performance Computing: He has been involved with teams that
develop and support Institutional HPC systems such as the Manchester
University Condor Pool - a 3000+ CPU core facility made by utilising
spare time on hundreds of desktop PCs. In this team, his primary role
was to assist researchers in transitioning their workflow from the
desktop to the Condor system.
\end{participant}

\begin{participant}[type=res,PM=1]{NN}
  We will support one month of an administrators time for assistance
  with workshop organisation and project administration across the
  whole duration of \TheProject.
\end{participant}


%\input{CVs/First.Last.tex}

\end{sitedescription}


\begin{draft}
\vspace{1cm}
\begin{verbatim}
- [X] checked that sum of person months put into finance request is
  the same as sum of person months associated with the Work Packages
  (in proposal.tex, as defined as part of the \begin{workpackage}"
  command.
  
  Take into account person months associated with work package 1, time
  of all staff to be hired and work on the project (including
  investigators). Figure 5 helps with a quick check of the sums over
  different work packages.

- [X] completed site specific resource summary in resources.tex,
  including table of non-staff costs. This is compulsory (EU
  regulations) if the non-staff cost exceed 15% of the total cost, and
  is likely to be the case for most of the partners. We ask everybody
  to do it, to be consistent and show transparently how we have
  planned our total budget.

- [X] Have all our tasks a designated lead institution? Check in the
  Work Packages that all the tasks you are involved in have a
  dedicated lead party. If the lead party is "USO", then use:
  \begin{task}[lead=USO]

- [X] Have all our deliverables a designated lead institution [using
  the 'lead=' key]?

- [X] In the "Members of the consortium section", have we addressed "a
  description of the legal entity and its main tasks, with an
  explanation of how its profile matches the tasks in the
  proposal"? See Entry for Paris Sud and Southampton as examples.

- [X] In the Members of the consortium section, have we given
  descriptions of all the people we intend to hire (even if we don't
  know who that is yet). 
  
- [X] Do all our tasks include us in the list of sites involved?
\end{verbatim}
\end{draft}

%KEY-MORE-TODOS

%  LocalWords:  sitedescription SITraN Hallamshire taskref hpc hpc-jupyter dissem Fusi pp
%  LocalWords:  subsubsection organisation Lippert Stegle Hensman Luengo modelling Biol
%  LocalWords:  Computat Organisers IPython Bioconductor tigre gprege Jupyter vspace

%\begin{sitedescription}{UWS}

\TOWRITE{WS}{Expand description of University of Washington}

% PIC:
% see: http://ec.europa.eu/research/participants/portal/desktop/en/orga
% AK: the link above is broken
%
% See ../proposal.tex, section Members of the Consortium for a
% complete description of what should go there

The University of Washington is a large research university,
which receives over a billion dollars a year in federal funding.

\subsubsection*{Curriculum vitae}
% AK - rules say "CV or description of the profile of the persons"
% - can we name it "Track record" instead of CV?

% Curriculum of the personnel at this institution

\begin{participant}[type=R,PM=0,salary=9000,gender=male]{William Stein}
% I put PM=0, since I will be completely unpaid by this grant.

William Stein is a number theorist at the University of Washington
who does research on elliptic curves and modular forms.
He founded the Sage project in 2004, wrote a large
amount of the code of Sage, organized dozens of Sage Days workshops,
and won the {\em 2013 Richard Dimick Jenks
Memorial Prize for Excellence in
Software Engineering applied to Computer Algebra} for his work on Sage.
He has written dozens of research papers and
also published two mathematics textbooks, and has two more books
under contract to publish within the next year.

Stein's main current software development project is SageMathCloud,
which is an open source web application he launched in 2013
that lets people use Sage and much more collaboratively online.
In addition to growing the user base and implementing
new functionality, he is also creating a company to make
SageMathCloud better supported and
more widely available and scalable.

\end{participant}
%%% Local Variables:
%%% mode: latex
%%% TeX-master: "../proposal"
%%% End:

%\input{CVs/First.Last.tex}
%\input{CVs/First.Last.tex}

\subsubsection*{Publications, products, achievements}

\subsubsection*{Previous projects or activities}

\begin{enumerate}
\item William Stein has received many grants from the
National Science Foundation for work on \Sage and number theory.
\item  There have been an average of 4 Sage Days workshops at UW
every year since 2006.
\end{enumerate}

\subsubsection*{Significant infrastructure}

University of Washington hosts a farm of 30 computers (well over 500
cores) dedicated to the \Sage project.

\end{sitedescription}



\begin{draft}
\vspace{1cm}\TOWRITE{WS}{Complete check list below -- delete completed items if you wish}

\begin{verbatim}
- [ ] checked that sum of person months put into finance request is
  the same as sum of person months associated with the Work Packages
  (in proposal.tex, as defined as part of the \begin{workpackage}"
  command.
  
  Take into account person months associated with work package 1, time
  of all staff to be hired and work on the project (including
  investigators). Figure 5 helps with a quick check of the sums over
  different work packages.

- [ ] completed site specific resource summary in resources.tex,
  including table of non-staff costs. This is compulsory (EU
  regulations) if the non-staff cost exceed 15% of the total cost, and
  is likely to be the case for most of the partners. We ask everybody
  to do it, to be consistent and show transparently how we have
  planned our total budget.

- [ ] Have all our tasks a designated lead institution? Check in the
  Work Packages that all the tasks you are involved in have a
  dedicated lead party. If the lead party is "USO", then use:
  \begin{task}[lead=USO]

- [ ] Have all our deliverables a designated lead institution [using
  the 'lead=' key]?

- [ ] In the "Members of the consortium section", have we addressed "a
  description of the legal entity and its main tasks, with an
  explanation of how its profile matches the tasks in the
  proposal"? See Entry for Paris Sud and Southampton as examples.

- [ ] In the Members of the consortium section, have we given
  descriptions of all the people we intend to hire (even if we don't
  know who that is yet). 
  
- [ ] Do all our tasks include us in the list of sites involved?
\end{verbatim}
\end{draft}

%KEY-MORE-TODOS



%%% Local Variables:
%%% mode: latex
%%% TeX-master: "../proposal"
%%% End:

%  LocalWords:  sitedescription TOWRITE subsubsection


%\cite{science-project}

\subsection{Third Parties Involved in the Project (including use of third party resources)}
\label{section:ThirdParties}

\paragraph{CNRS}\ 

\eucommentary{Please complete, for each participant, the table
(see page 27 of "VRETemplate.PDF"),
or simply state "No third parties involved", if applicable.}

The University of Bordeaux will be involved as a Third Party Linked to CNRS.
Indeed, both the LaBRI and the IMB are Joint Research Units for which both the
University of Bordeaux and CNRS are supervisory authorities ; Karim Belabas and
Adrien Boussicault are staff of the University of Bordeaux and actively
participate in the project (see~\ref{desc:Bordeaux}).
The University of
Bordeaux was founded in the 15th century and nowadays the city hosts two
universities, dozens of schools and 100 research laboratories with partner
institutions such as CNRS, Inserm, INRA and INRIA. It is an important centre of
studies and research in France with approximately 50,000 students, 2,000 PhD
students and 5,000 researchers.

\smallskip

\bgroup
\def\arraystretch{1.5}  % 1 is the default
\noindent \begin{tabular}{|p{0.6\textwidth}|c|}
\hline
Does the participant plan to subcontract certain
tasks & No \\
\hline
\multicolumn{2}{|l|}{} \\
\hline
Does the participant envisage that part of its work
is performed by linked third parties & Yes \\
\hline
\multicolumn{2}{|l|}{\textit{Two participants of the project are staff of the University of Bordeaux: Karim Belabas and Adrien Boussicault (see above).}} \\
\hline
Does the participant envisage the use of
contributions in kind provided by
third parties & No \\
\hline
\multicolumn{2}{|l|}{} \\
\hline
\end{tabular}
\egroup

\bigskip

\paragraph{University of Silesia}\ 

\bgroup
\def\arraystretch{1.5}  % 1 is the default
\noindent \begin{tabular}{|p{0.6\textwidth}|c|}
\hline
Does the participant plan to subcontract certain
tasks & Yes \\
\hline
\multicolumn{2}{|l|}{See~\ref{resources.summary} for justification} \\
\hline
Does the participant envisage that part of its work
is performed by linked third parties & No \\
\hline
\multicolumn{2}{|l|}{} \\
\hline
Does the participant envisage the use of
contributions in kind provided by
third parties & No \\
\hline
\multicolumn{2}{|l|}{} \\
\hline
\end{tabular}
\egroup

\paragraph{Logilab}\ 

\bgroup
\def\arraystretch{1.5}  % 1 is the default
\noindent \begin{tabular}{|p{0.6\textwidth}|c|}
\hline
Does the participant plan to subcontract certain
tasks & Yes \\
\hline
\multicolumn{2}{|l|}{See~\ref{resources.summary} for justification} \\
\hline
Does the participant envisage that part of its work
is performed by linked third parties & No \\
\hline
\multicolumn{2}{|l|}{} \\
\hline
Does the participant envisage the use of
contributions in kind provided by
third parties & No \\
\hline
\multicolumn{2}{|l|}{} \\
\hline
\end{tabular}
\egroup

\paragraph{FAU}\ 

\bgroup
\def\arraystretch{1.5}  % 1 is the default
\noindent \begin{tabular}{|p{0.6\textwidth}|c|}
\hline
Does the participant plan to subcontract certain
tasks & Yes \\
\hline
The justification for subcontracting is as follows: 
The Isabelle system is so complex that it is impossible for anybody but a few experts to 
implement the export. Even then it takes considerable effort as the low-level data structures have to 
improved in order to even make the export feasible. Makarius Wenzel is the main developer of Isabelle 
and has developed an API to access the relevant data structures. He is the only candidate who can implement 
the export. FAU will subcontract the Isabelle Case study to Makarius Wenzel's company for 30 000€.& \\
\hline
Does the participant envisage that part of its work
is performed by linked third parties & No \\
\hline
\multicolumn{2}{|l|}{} \\
\hline
Does the participant envisage the use of
contributions in kind provided by
third parties & No \\
\hline
\multicolumn{2}{|l|}{} \\
\hline
\end{tabular}
\egroup


\paragraph{Other participants}\ 

There is no subcontracting costs for other participants.

%No third parties involved.

%\TODO{Or Seattle? See https://www.iprhelpdesk.eu/node/2549}

%%% Local Variables:
%%% mode: latex
%%% TeX-master: "proposal"
%%% End:
