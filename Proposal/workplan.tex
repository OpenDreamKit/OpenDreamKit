
\TOWRITE{ALL}{Proofread 3.1 work plan (except for the work packages themselves) pass 2}

\eucommentary{Please provide the following:\\
\begin{compactitem}
\item
brief presentation of the overall structure of the work plan;
\item
timing of the different work packages and their components (Gantt chart or similar);
\item
detailed work description, i.e.:
\begin{compactitem}
\item
a description of each work package (table 3.1a);
\item
a list of work packages (table 3.1b);
\item
a list of major deliverables (table 3.1c);
\end{compactitem}
\item
graphical presentation of the components showing how they inter-relate (Pert chart or similar).
\end{compactitem}
}

\subsubsection{Overall Structure of the Work Plan}\label{sec:workplan-structure}
\ifgrantagreement
The
\else
As shown in Figure~\ref{fig:wplist}, the
\fi
work plan is broken down into
seven work packages: \WPref{component-architecture} about components,
\WPref{UI} for user interfaces, \WPref{hpc} for parallelisation of the
components, \WPref{dksbases} for databases and finally
\WPref{social-aspects} for social aspects. This is complemented by the
the usual management and dissemination work packages
(\WPref{management}) and (\WPref{dissem}). The Gantt chart on
Page~\pageref{fig:gantt} illustrates the timeline for the various
tasks for these work packages%., including inter-task dependencies.

\ifgrantagreement\else
%\makeatletter\wp@total@RM{management}\makeatother
\wpfigstyle{\footnotesize\def\tabcolsep{3.5pt}}
%\wpfig[pages,type,start,end]
{\wpfig}
\fi
%\newpage
\subsubsection{How the Work Packages will Achieve the Project Objectives}
\label{sssec:how_the_work_packages_will_achieve}

% (Section~\ref{sect:objectives},page~\pageref{sect:objectives})

The following table recalls the objectives of \TheProject and lists
the work packages that contribute to achieving each of them.

\begin{center}
\begin{tabular}{|l|l|l|}\hline
\textbf{Objective} & \textbf{Purpose} & \textbf{WPs} \\\hline \hline
Objective 1
 & Develop and standardise math soft and data for VRE
 & \WPref{component-architecture},  \WPref{UI}, \WPref{hpc}, \WPref{dksbases} \\\hline
Objective 2
 & Develop core VRE components
 & \WPref{component-architecture}, \WPref{UI}, \WPref{hpc}, \WPref{dksbases} \\\hline
Objective 3
 & Bring together communities
 & \WPref{dissem}, \WPref{component-architecture} \\\hline
Objective 4
 & Update a range of softwares
 & \WPref{component-architecture}, \WPref{hpc} \\\hline
Objective 5
 & Foster a sustainable ecosystem
 & \WPref{component-architecture}, \WPref{UI}, \WPref{hpc}, \WPref{dksbases} \\\hline
Objective 6
 & Explore social aspects
 & \WPref{social-aspects} \\\hline
Objective 7
 & Identify and extend ontologies
 & \WPref{dksbases} \\\hline
Objective 8
 & Effectiveness of the VRE
 & \WPref{dissem}, \WPref{social-aspects} \\\hline
Objective 9
 & Effective dissemination
 & \WPref{dissem}, \WPref{social-aspects} \\\hline
\end{tabular}
\end{center}

\TOWRITE{ALL This next section is freshly rewritten to be more
  detailed. It doesn't show or explain dependencies between WPs or
  anything like that, which would be nice, but would take too
  long. Anyway please check}

\paragraph{Work Programme for Objective 1: }

\taskref{component-architecture}{interface-systems} (Interfaces
between Systems) directly addresses the core of objective 1, making
existing systems compatible with one another in mathematically sound
ways. Other tasks in \WPref{component-architecture} (component
architecture) support this, by making components more portable and
easier to deploy (\taskref{component-architecture}{mod-packaging}:
Modularisation and Packaging;
\taskref{component-architecture}{portability}). \taskref{component-architecture}{extract-smc}
will bring us the benefit of lessons learned and components built for
\SMC. \taskref{dksbases}{data-design} deals with the data-centric
aspects of the interfaces. Additionally elements of \WPref{UI} (user interface) and \WPref{hpc} (HPC)
will also contribute to the framework with user interface pluggability
and interfaces optimised for HPC.

\paragraph{Work Programme for Objective 2: }

We have identified a need for a number of new core components for
\TheProject and planned their construction at appropriate stages of
various workpackages. A new adapter infrastructure is part of
\taskref{component-architecture}{interface-systems}; new virtual
appliances will be built in
\taskref{component-architecture}{portability}; new components will be
extracted from \SMC in  \taskref{component-architecture}{extract-smc};
new documentation components will be developed in
\taskref{UI}{sage-sphinx} and \taskref{UI}{dynamic-inspect}; new
mathematical software will be developed in \taskref{hpc}{hpc-combi}
and new database tools in \taskref{dksbases}{data-memo} and \taskref{dksbases}{mws}.


\paragraph{Work Programme for Objective 3: }

Representatives of a number of communities have already come together
simply to prepare this proposal, and the whole project will work to
bring them together. Specifically developers of many systems  will be brought together to complete
work package~\WPref{component-architecture},
especially~\taskref{component-architecture}{interface-systems}.
Bringing broader communities together is the core purpose of
work package~\WPref{dissem}, which includes workshops, web sites,
demonstrator packages and outreach activities.

\paragraph{Work Programme for Objective 4: }

The concept of this project is centred on leveraging the communities
vast investment in existign open source software systems, and wherever
possible we will proceed by extending and updating existing software components.
In work package \WPref{component-architecture} we will address
portability (\taskref{component-architecture}{portability} and
modularity (\taskref{component-architecture}{extract-smc}) and also
adapt the components to use the new interfaces being designed in
\taskref{component-architecture}{interface-systems}. Work package
\WPref{hpc} is largely about updating software for performance, while
workpackage \WPref{UI} deals with adaptation of UI components and of
other systems to work with them.

\paragraph{Work Programme for Objective 5: }

A number of tasks relate to developing promoting and supporting
sustainable models for collaborative software development. On a
practical level \taskref{component-architecture}{workflow} will adress
processes and technologies, \taskref{dksbases}{data-memo} concerns
collaborative accumulation of data. On a personal level, much of
\WPref{dissem} deals with ensuring a wide and committed user/developer
community. Finally in~\taskref{social-aspects}{isocial-decisionmaking}
we will actually conduct research into the social mechanisms of
collaborative software development, and lessons from this research
will be embedded into the structures we leave behind.

\paragraph{Work Programme for Objective 6: }

Objective 6 is covered by a dedicated work package \WPref{social-aspects} on social aspects.
It ranges from analysis of the needs with~\taskref{social-aspects}{social-input} to
evaluation with~\taskref{social-aspects}{oommf-nb-evaluation}.

\paragraph{Work Programme for Objective 7: }

Objective 7 is addressed directly by \WPref{dksbases}, which deals with data
and its meaning.

\paragraph{Work Programme for Objective 8: }

Producing and evaluating systems that demonstrate our achievements is
a key feature of this project, and this work in embedded throughout
the project. The integration and publicisation of these demonstrators
is key to \WPref{dissem} (dissemination) especially later in the
project, while their formal evaluation is found in \WPref{social-aspects}.

\paragraph{Work Programme for Objective 9: }

Dissemination is the heart of~\WPref{dissem}.
Members of \TheProject will organise workshops within \taskref{dissem}{dissemination}
or \taskref{dissem}{project-intro} as well as less formal meetings
with interested groups. In addition, we will follow open software development
processes throughout the project, so that our work is immediately
available to any interested party. We will announce important
developments or releases through our own web pages and the
established channels of the component systems. Our scientific findings
will be published in the open scientific literature and announced at
scientific meetings and conferences in the usual way and reported in
annual project reports.


\subsubsection{Work Plan Timing: GANTT Chart showing Task Dependencies and Information
  Flows}

Since \TheProject consists mainly in improving independent tools and
integrating them into a VRE, its tasks are fairly independent from each
other, which is reflected by the GANTT chart in Figure~\ref{fig:gantt}

\gantttaskchart[draft,xscale=.33,yscale=.33,milestones]

\ifgrantagreement\else
\newpage
\subsubsection{Deliverables}\label{sec:deliverables}
\inputdelivs{9.3cm}
\fi

\newpage
\subsubsection{Milestones}\label{sec:milestones}
\eucommentary{Milestones means control points in the project that help to chart progress. Milestones may
correspond to the completion of a key deliverable, allowing the next phase of the work to begin.
They may also be needed at intermediary points so that, if problems have arisen, corrective
measures can be taken. A milestone may be a critical decision point in the project where, for
example, the consortium must decide which of several technologies to adopt for further
development.}

The work in the \TheProject project is structured by four milestones, which could be
briefly characterised as: starting up and building prototypes; moving from prototypes to
fully functional implementations; further engagement with the community and producing
research outputs; evaluation and final releases. They coincide with the project meetings
held at the end of each year of the project (four other meetings will be held in the
middle of each year).  Given the nature of the project, with a large number of essentially
independent tasks, there is no need for milestones attached to specific collections of
tasks or deliverables.  Given that the meetings are the main face-to-face interaction
points in the project, we have chosen to schedule the milestones for these events, where
they can be discussed in detail, tracking the progress in each work package through status
reports on the tasks and deliverables and take corrective measures, where necessary, and
critical decisions regarding further plans.  We envisage that this setup will give the
project the vital coherence in spite of the broad interdisciplinary mix of various
backgrounds of the participants.

\paragraph{General Milestones}

\begin{milestones}
  \milestone[id=startup,month=12,
  verif={Completed all corresponding deliverables and reported the progress in the 2nd Project meeting report.}]
  {Startup}
  {By Milestone 1 we will have carried out the requirements study, design and prototype implementations and started community building activities.}

  \milestone[id=proto1,month=24,
  verif={Completed all corresponding deliverables and reported the progress in the 4th Project meeting report.}]
  {Implementations}
  {By Milestone 2 we will have constructed first fully functional interface implementations and released enhanced versions of \TheProject components, and train early adopters of \TheProject.}

  \milestone[id=community,month=36,
  verif={Completed all corresponding deliverables and reported the progress in the 6th Project meeting report.}]
  {Community/ Experiments}
  {By Milestone 3 we will have gathered and evaluated feedback on \TheProject software and established the portfolio of experiments produced with \TheProject through further engaging with the community.}

  \milestone[id=eval,month=48,
  verif={Completed all corresponding deliverables and reported the progress in the 8th Project meeting report.}]
  {Evaluation}
  {By Milestone 4 we will have released final versions of all \TheProject components and completed the project evaluation.}
\end{milestones}

\paragraph{Milestone for WP 3}
We propose 1 milestone:

\begin{milestones}
%original delivery date proposal is M36 but milestone is linked to D3.10 which is planned for M48...
	\milestone[id=WP3availability,month=42,
	 verif={Have \ODK's components available on major platforms}]
	 {Work Package 3 aims at deploying all computational components
	 developed by \ODK available on the three major platforms (i.e.
	 Windows, Mac, Linux) via their standard distribution channels.}
\end{milestones}

\paragraph{Milestones for WP 4}
We propose two milestones:

\begin{milestones}
  \milestone[id=WP4prototype,month=36,
    verif={Prototype VRE for mathematical researchers}]
  {Prototype VRE for mathematical researchers}
  {
  % note: delivref doesn't work here
  User story: A group of mathematical researchers with access to
  common computational resources, such as a shared lab computer or
  cloud servers, shall be able to deploy a prototype VRE with
  \JupyterHub, integrating \ODK components.
  The Jupyter kernels for mathematical software developed as part of \ODK
  make computational mathematical components accessible in a \Jupyter
  environment, enabling a Jupyter-based deployment of the relevant
  tools for the researchers.
  The process of working on notebooks is greatly improved by review tools
  developed as part of WP4,
  enabling researchers to collaborate to some degree
  in a shared computational environment.
  }
  \milestone[id=WP4collaborative,month=48,
  verif={Collaborative VRE for mathematical researchers}]
  {Collaborative VRE for mathematical researchers}
  {
  The prototype VRE shall be extended with improved ease of deployment, new
  functionality such as interactive 3D visualization and real-time
  collaboration, enabling researchers to collaborate productively in a shared
  computational environment. Finally, integrating notebooks and semantic
  knowledge into a publication / knowledge system enable a continuous process
  of leveraging \ODK components from research to publication.
  }
\end{milestones}

\paragraph{Milestones for WP 6}

\begin{milestones}
  \milestone[id=WP6interop1,month=36,
  verif={Demonstrator Online Public, works on selected case study examples}]
  {First MitM-based interoperability prototype (GAP, SageMath, LMFDB)}
  {We intend to present a fully functional prototype of the integration of at least the
    systems GAP, SageMath, and LMFDB via the SCSCP Protocol at the second review 
    meeting. This prototype will be the basis for additional integration work for 
    additional systems and the use interface from WP4.}
\milestone[id=WP6interop2,month=42,   verif={Demonstrator Online Public, works on selected case study examples}]
  {Second MitM-based interoperability prototype}
  {The goal of this milestone is to take into account all the operational 
    experiences with the first prototype and add more systems and integrate some
    of the UI components from The experiences with the preparation of 
    this prototype will allow us to estimate the joining costs of adding a system 
    to the OpenDreamKit VRE toolkit, which is an important measure of the 
    flexibility of the MitM approach.}
\end{milestones}

%%% Local Variables:
%%% mode: latex
%%% TeX-master: "proposal"
%%% End:

%  LocalWords:  verif ldots


% ---------------------------------------------------------------------------
% Include Work package descriptions
% ---------------------------------------------------------------------------

\subsection{Work Package Descriptions}\label{sec:workpackages}
%% WP titles and order are defined in deliverables.tex
%%% workpackage style may be broken -- fix this!!

%% Local WP number counter - should possibly be global and hidden?
\begin{workplan}
\begin{workpackage}[id=management,type=MGT,wphases=0-48!.2,
  title=Project Management,short=Management,
  lead=PS,
  PSRM=28,SARM=2,  
  USORM=2,LLRM=2,UVRM=2,UJFRM=2,UBRM=2,UORM=2, USHRM=2, USORM=2, UWRM=2, JURM=2, UKRM=2, USRM=2, ZHRM=2, SRRM=2, UWSRM=2]

\begin{wpobjectives}
  The objectives of this work package are to undertake all project management activities,
  including:
  \begin{compactitem}
  \item monitoring the overall progress of the project and the use of
    resources;
  \item ensuring the timely production of deliverables and other
    project outputs;
  \item reporting to the European Commission on financial matters;
  \item preparing for and attending the annual project review
    meetings; and
  \item managing the project Advisory Board.
  \end{compactitem}

  % The objective of  is to undertake all project management
  % activities, including setting up joint infrastructure, organizing
  % meetings, and producing overview reports.
\end{wpobjectives}

\begin{wpdescription}
  This workpackage will perform all the activities related to monitoring of progress
  towards the project milestones shown on Page~\pageref{sec:milestones} and the
  deliverables listed on Page~\pageref{sec:deliverables}, assuring the quality of the
  deliverables, ensuring the collation and distribution of the required reports,
  questionnaires and deliverables including the annual reports to the European Commission,
  arranging project management meetings, tracking the project budget in terms of
  expenditure and person-months, obtaining financial certificates as required, convening
  project management meetings, ensuring that important project documents such as the
  project contract and the consortium agreement are properly maintained and amended as
  necessary, ensuring that contractual details are complied with, monitoring compliance
  with the grant agreement, preparing for the annual review meetings, and reviewing
  research results against the aims and objectives of the project. It also involves
  managing and supporting the project Advisory Board, including supporting attendance at
  project meetings, convening Advisory Board meetings, and obtaining feedback on the
  project direction and results.
\end{wpdescription}

\TODO{MK: I would combine the first three into one ``basic project infrastructure''}
\begin{wpdelivs}
\begin{wpdeliv}[due=1,id=tickets,dissem=PU,nature=DEC]{Create tickets for all relevant tasks / deliverables}
\end{wpdeliv}
\begin{wpdeliv}[due=1,id=mailinglists,dissem=PU,nature=DEC]{Internal and external mailing lists}
\end{wpdeliv}
\begin{wpdeliv}[due=1,id=swrepository,dissem=PU,nature=DEC]{Internal software repository}
\end{wpdeliv}
\begin{wpdeliv}[due=12,id=periodic-rep-1,dissem=PU,nature=R]{Project Periodic Report (first year)}
 \end{wpdeliv}
\begin{wpdeliv}[due=24,id=periodic-rep-2,dissem=PU,nature=R]{Project Periodic Report (second year)}
 \end{wpdeliv}
\begin{wpdeliv}[due=36,id=periodic-rep-3,dissem=PU,nature=R]{Project Periodic Report (third year)}
 \end{wpdeliv}
\begin{wpdeliv}[due=48,id=periodic-rep-4,dissem=PU,nature=R]{Project Periodic Report (fourth year)}
 \end{wpdeliv}
\begin{wpdeliv}[due=48,id=final-mgt-rep,dissem=PU,nature=R]{Project Final Report}
 \end{wpdeliv}
\end{wpdelivs}
\end{workpackage}
%%% Local Variables: 
%%% mode: latex
%%% TeX-master: "../proposal"
%%% End: 

%  LocalWords:  workpackage wphases wpobjectives wpdescription pageref wpdelivs wpdeliv
%  LocalWords:  dissem mailinglists swrepository final-mgt-rep compactitem

\begin{workpackage}[id=community,wphases=5-36!.7,
%<<<<<<< HEAD
title=Community Building and Engagement,
SARM=1,USHRM=8]
%=======
%  title=Community Building and Engagement,
%  lead=PS,
%  PSRM=12,SARM=1,USHRM=8]
%>>>>>>> e490abbfa8a91427570f1a7695a6a95cd4610713

\begin{wpobjectives}
  The objective of this work package is to further develop the community at the
  European scale, foster cross teams collaborations, spread the
  expertise, and engage the greater community to participate to the
  definition of the needs, and the implementation and use of the
  produced solutions.
% \begin{itemize}
% \item
% \item
% \item
% \item
% \item
% \end{itemize}
\end{wpobjectives}

\begin{wpdescription}
  We will organize regular open workshops (e.g. Sage Days, Pari Days,
  summer schools, etc.); some of them will be focused on development
  and coding sprints, and others on training.

\TODO{Neil: I have a series of Gaussian process summer schools and road shows that I'rm organizing. These will also shift to more of a focus on data science across this year, I'd be happy to include these here if that's appropriate.}

  This work package will also provide general travel budget to fund
  short to long term visits between the participants, to collaborate
  on specific features. A typical such visit would bring together an
  IPython developer with a GAP developer for a couple of days to
  implement a first prototype of notebook interface to GAP.

  This work package will complement and lean on a parallel COST
  network whose role is to build and animate the greater community.


\end{wpdescription}

\begin{wpdelivs}
  \begin{wpdeliv}[due=6,id=ws1,dissem=PU,nature=O]{Workshop 1}
  \end{wpdeliv}
  \begin{wpdeliv}[due=12,id=needs,dissem=PU,nature=R]{Report on community needs}
  \end{wpdeliv}
  \begin{wpdeliv}[due=18,id=ws2,dissem=PU,nature=O]{Workshop 2}
  \end{wpdeliv}
  \begin{wpdeliv}[due=30,id=ws3,dissem=PU,nature=O]{Workshop 3}
  \end{wpdeliv}
  \begin{wpdeliv}[due=42,id=ws4,dissem=PU,nature=O]{Workshop 4}
  \end{wpdeliv}
\end{wpdelivs}
\end{workpackage}
%%% Local Variables:
%%% mode: latex
%%% TeX-master: "../proposal"
%%% End:

\addtocounter{wpno}{1}
\begin{Workpackage}{\thewpno}
\wplabel{wp:x}
\WPTitle{\wpname{\thewpno}}
\WPStart{Month 1}
\WPParticipant{SA}{1}

\begin{WPObjectives}
  The objective of this work package is to develop and demonstrate a
  set of API's enabling components such as database interfaces,
  computational modules, separate systems such as GAP or Sage to be
  flexibly combined and run smoothly across a wide range of
  environments (cloud, local, server, ...).
\end{WPObjectives}

\begin{WPDescription}
  This work package includes work on:
  \begin{itemize}
  \item Portability:
    \begin{itemize}
    % Jean-Pierre:
    % Should we mention port to non-x86_64 archs and non-Linuces?
    %
    % For CPUs:
    % - I guess at least ARM and ppc64 (IBM POWER*) really make sense.
    % - Sparc is less convincing though the latest sparc CPUs
    % are muche more interesting for math computation as the
    % previous ones, e.g. the GMP folk specifically added assembly
    % for them in their latest release.
    % - Itanium is dead, but it can help discovering bugs as any non
    % standard archs.
    % - Supporting any of these would mean buying (potentially very
    % expensive) hardware.
    %
    % For OSes?
    % - Should we mention OS X which is a pain at each new release?
    % - A BSD variant would be interesting, let's say FreeBSD which
    % is basically (almost) already supported
    % - Solaris? and/or OpenIndiana? Interesting if we mention sparc...
    % - Windows is already included below, my opinion is:
    %  * provide live USB, VMs and Cygwin32 first as these three are
    %  basically already working solutions
    %  * go Cygwin64 as it is still POSIX
    %  * explorate a MinGW solution, at least GAP and PARI should be
    %  problematic
    %  * try to use MSVC
    \item Sharing experience and best practices.
    \item Port to Windows (GAP, Sage, Singular).
    \item Shared multiplatform test infrastructure.
    \end{itemize}

  \item Interfaces between systems:
    \begin{itemize}
    \item Self adaptation to the environment, better schemes for
      automatically selecting appropriate algorithms / components for
      a given task.
    \item Semantic-enabled handles to objects stored in other systems (NT):\\

      Handles are a popular design pattern for interfaces between two
      systems A and B; instead of exchanging objects back and forth,
      only handles to those objects are exchanged, letting e.g. A
      manipulate an object which actually resides in B. Typical
      features include remote method calls, introspection, or
      documentation queries. The next step would be for A to be aware
      of the semantic of the object, using an adapter infrastructure
      to propagate category/ontologies information. For example, we
      would want GAP's categories to be mapped to Sage's categories,
      so that a handle to a GAP group would automatically appear
      within Sage like a native Sage group.
    \end{itemize}

  \item Modularization
    \begin{itemize}
    \item common architecture for module maintenance and
      distribution (related to point 1 above)
    \item Sharing experience and best practices
    \item Modularization of Sage
    \item Refactorization of GAP's package mechanism; namespaces?
    \end{itemize}

  \item Deployment and distribution

  \item High Performance Computing and Parallelism:\\
    As in all other areas of science, properly supporting of massively
    parallel architecture is a major challenge.

    Many of the computational components have already gone a long way
    in this direction. For example, grant \TODO{...} founded the
    GAP-HPC project which adapted the GAP kernel to support HPC. The
    expertise gained there was then transferred to the ongoing
    Singular-HPC project.

    Building on this, this project will:
    \begin{itemize}
    \item Foster further sharing of HPC expertise and best practices
      between computational components.
    \item Develop novel infrastructure for HPC in the context of
      combinatorics.
    \item Investigate and implement HPC-friendly ways of combining
      components together, so that calling components can benefit from
      the HPC features of called components, with self-adaptation to
      the environment and cooperative sharing of resources.
    \item Support work on HPC-enabling more components (Linbox)
    \item Investigate 
    \end{itemize}
  \end{itemize}
\end{WPDescription}

\begin{WPDeliverables}
\begin{itemize}

%%%%%%%%%%%%%%%%%%%%%%%%%%%%%%%%%%%%%%%%%%%%%%%%%%%%%%%%%%%%%%%%%%%%%%%%%%%%%%
% Deliverables: portability and distribution

\item \ref{del:distribution} Make sure that Sage and therefore all the
  components it depends on (including GAP, Linbox, Pari, Singular,
  ...)  have standard packages in the main Linux distributions:
  Debian/Ubuntu, Redhat, Gentoo, ...

  \TODO{Get feedback from our experts, and make this precise; what can
    we actually promise to achieve? how much work is this? Do we have
    personnel for this?  There is strong expertise in Logilab with a
    Debian developer working there; he could advise someone on
    this. Logilab is interested in this because it's meeting similar
    issues with some of its clients software like Salomé.}

x\item \ref{del:virtual_machines} (Month 12): Creation, deployment, and
  distribution of preconfigured virtual machines for Pari, Sage,
  ... as a cloud service, in particular within the StratusLab
  infrastructure. This includes build bots and test bots for
  continuous integration over a variety of operating systems.
  % Requires: licenses

\item \ref{del:portability_cygwin} (Month 12, Month 24): Fully
  functional one-click install Sage distribution for Windows using a
  32bits version of Cygwin.
  % JPF: this should take a few months of work

  This 32bits version would work right away on Windows 64 bits with
  Cygwin 32 bits; more work would be required for a version working on
  a 64 bits of Cygwin.
  % JPF: I agree.

  In both cases, this includes complete port of Singular, GAP, Pari,
  ...  to cygwin.

  % Participants involved: Paris Sud, Kaiserslautern, Saint Andrews, Bordeaux


  % Comments on this by Bill Hart
  % The big problems you will have on Windows 64 on Cygwin include:
  %
  % * anything with assembly language -- the ABI is different on Windows, so
  % it'll need rewriting, or you can incur a performance penalty by using
  % generic C fallback code
  % * the memory allocator on Windows is not so great
  % * bugs exposed due to being on a different platform, e.g. segfaults due to
  % off-by-one errors that were masked by the granularity of malloc on Linux
  % * build issues, due to identifying Cygwin and using the correct header
  % files, which are often different on Cygwin than linux
  % * issues with PATH vs LD_LIBRARY_PATH
  % * Windows has a case insensitive file system
  % * EOL issues
  % * Windows is not able to rapidly create and delete files, which some
  % libraries (esp. test code) calls for
  % * memory limitations (many people using Windows are using laptops with
  % limited memory, only a portion of which is realistically available to
  % Cygwin)
  % * autotools versions that don't support Windows (usually autotools has a
  % release that is used in all the distributions, which doesn't work correctly
  % on Windows, and this is followed up by a version which has all the Windows
  % patches)
  % * building takes forever on Windows. Mingw2 has now gotten parallel build
  % working on Windows and the speed is within a factor of 5 of Linux. But I'm
  % not sure the improvements have propagated to Cygwin yet.
  % * Cygwin 64 is new, contains quite a few bugs still, and things keep
  % changing with every version as they try to get things right.
  % * Although projects will likely accept patches for Windows, they are less
  % likely to maintain support themselves. I would like to think Singular would
  % be an exception to this. And obviously flint and MPIR work on Windows (even
  % with MSVC as of the next version of flint -- or now if you use our bleeding
  % edge repo version).
  %
  % Comments by Jean-Pierre on some of the above and mor:
  % * first things first: I already completely built Sage on Cygwin64, though it
  % was surely not completely functional.
  % * assembly: that's right, note that as far as Sage and it's dependencies are
  % concerned, only a few of them actually use assembler, and yes all of them
  % provide fallback generic C code IIRC
  % * PATH vs LD_...: basically the same problem as for Cygwin32, so it's already
  % been taken care of for the Cygwin32 port
  % case issue: not a problem IIRC
  % * EOL issues: I don't thing so, Cygwin is POSIX like
  % * autotools issues: most of Sage dependencies are now updated, I used to track
  % the few problematic ones in 2013
  % * time to build: not so long, sure longer than on a POWER7 machine, but I do
  % it on a usual x86_64 laptop running Debian within a Windows VM in a few hours!
  % what we actually really need is patch/build bots to test on Cygwin 32/64!
  % * upstream cooperation: I agree Windows is often a low priority issue, but
  % most teams have welcomed my Cygwin patches

\item \ref{del:modularization} Modularization of the Sage distribution

  Separation of the different components of Sage (communication with
  third-party softwares, build system, Sage native code). This is a
  prerequisite for easier packaging and integration in standard Linux
  distributions and lmonade, native integration within the IPython
  notebook and other interfaces (larchenv, Spyder, ...) and
  collaboration with sister projects.

%\TODO{lmonade has very similar objectives but uses the gentoo prefix whereas Linux distributions use very different packaging systems:
%\begin{itemize}
%\item gentoo prefix (gentoo)
%\item pacman (arch),
%\item yum (redhat),
%\item apt (debian),
%\item easy\_install
%\item Python index packaging (pip)
%\end{itemize}}

%%%%%%%%%%%%%%%%%%%%%%%%%%%%%%%%%%%%%%%%%%%%%%%%%%%%%%%%%%%%%%%%%%%%%%%%%%%%%%
% Deliverables: Interfaces

\item \ref{del:scscp_sage} Add support for the
  \href{http://www.symbolic-computing.org/}{SCSCP} interface protocol
  to all relevant components (e.g. Sage, ...).
  \TOWRITE{SL/AK}{Brief description of what SCSCP is, reference to
    previous grant, relevance to the goals of this grant; maybe this
    should go in the work package description}

\item Some IPython/Jupyter deliverables here.
  \TODO{review what it can already do in term of choice of
    computational resource and storage back-end.}

\item Contribution by Kaiserslautern: libSingular, pySingular?,
  GAP-Singular, Singular-Sage.

  Moving code from Sage into Singular when relevant

%%%%%%%%%%%%%%%%%%%%%%%%%%%%%%%%%%%%%%%%%%%%%%%%%%%%%%%%%%%%%%%%%%%%%%%%%%%%%%
% Deliverables: HPC

\item \ref{del:hpc_configure} (Month ...) Configure the components of
  Sage's distribution (e.g. Atlas, Linbox, GAP, Singular, ...) to be
  systematically HPC-enabled, and make sure that Sage's calls to such
  components indeed enable HPC.

\item \ref{del:hpc_components}
  Develop HPC features in components
  \begin{itemize}
  \item \TOWRITE{JGD}{Linbox}
  \item \TOWRITE{WD}{Singular}
  \end{itemize}

%%%%%%%%%%%%%%%%%%%%%%%%%%%%%%%%%%%%%%%%%%%%%%%%%%%%%%%%%%%%%%%%%%%%%%%%%%%%%%
% Deliverables: to be sorted ...

\item Transparent integration of Ipython capabilities for cluster computing.
\item Implementation of a transparent abstraction over mpi.
\item Develop or integrate existing solutions for MapReduce operations
  over big data.

\item FLINT development (key component for several systems)?

\item Some demonstrators of cross-disciplinary/cross-software calculations

\end{itemize}
\end{WPDeliverables}
\begin{verbatim}
Raw material:

Component Architecture
----------------------

Recomputation connection belongs here?

Collaboration with unreliable (or restricted!) networking connections
(peer-to-peer, opportunistic syncing, 3rd world). This is technically
interesting, and gets in support for non-networked working. Not sure
if it belongs here or not.

- Security concerns

Goal: Fostering collaborations/integration between components in an open source ecosystem
=============================================================================

- How to make systems "cooperate" rather than "predate each other".
- E.g. reduce the version issues

- Foster collaboration with upstream libraries by sharing the
  development and maintenance of the interfaces, typically as
  standalone upstream Python bindings (e.g. py-Singular).

- How to make it easy to develop simultaneously two interdependent
  components (e.g. Sage+Singular)

- Foster communication

- Social aspect:
  Credit, Citations, Recognition, Funding

Documentation system
====================

In which package?

Improvements to Sphinx

Sage heavily customizes the Sphinx documentation system, hacking deep
in it in some cases, with quite some duplication in some cases.
Refactor the whole thing, generalizing and contributing back upstream
as much as possible (e.g. parallel compilation).
\end{verbatim}

\end{Workpackage}

\begin{workpackage}[id=UI,wphases=24-48,
  title=User Interfaces,
  PSRM=1,
  JURM=12, % active documents
  USHRM=12, % Supporting reproducible data science and sharing of models
  LLRM=1, % Jupyter
  SARM=1, % GAP
  UKRM=1, % Singular
  UBRM=1, % Pari
  USORM=21] % Southampton, \OOMMFNB

\begin{wpobjectives}
  The objective of this work package is to provide modern, robust,
  and flexible user interfaces for computation, supporting real-time
  sharing, integration with collaborative problem-solving,
  multilingual documents, paper writing and publication, links to
  databases, etc.
\end{wpobjectives}

\begin{wpdescription}
  Project Jupyter is a set of open-source software projects for interactive and exploratory
  computing. These software projects help make scientific computing and data science reproducible
  and multi-language (Python, Julia, R, Haskell, etc.). The main application offered by Jupyter is
  the Jupyter notebook, a web-based interactive computing platform that allows users to create
  data- and code-driven narratives that combine live code, equations, narrative text, interactive
  dashboards and other rich media. These documents provide a complete record of a computation that
  can be shared with others.


  \TOWRITE{Jupyter/...}{Add references of Jupyter's use in Europe}
  The Jupyter notebook is being used in all areas of academic (University of California, Berkeley,
  Stanford, University of Washington, New York University, Cal Poly, MIT, Harvard, Columbia, etc.)
  and government (NASA JPL, LBL, KBase, White House Hackathon) research as well as industry
  (Google, IBM, Facebook, Oracle, Otto Group, Microsoft, Bloomberg, JP Morgan, WhatsApp, O’Reilly,
  Quantopian, Logilab, GraphLab, Enthought, Continuum, Authorea, BuzzFeed, etc.) and journalism (538, New
  York Times, etc.). Because the architecture and building blocks of Jupyter are open, they are
  being used to build numerous other commercial and non-profit products and services. The Jupyter
  Notebook has between 500,000 and 1.5 million individual users worldwide.

  \TOWRITE{Jupyter}{One paragraph overview description of the Jupyter
    related tasks in the User Interface work package}

  \TOWRITE{Marcin/Hans/...}{Generalize the next paragraph to cover all
    User Interface demonstrators}

  The last tasks in this work package is focused on an
  \emph{application of the Jupyter notebook technology to a simulation
    package} that is actively used in materials research by a wide
  range of scientists and engineers, in academia and industry. We will
  develop a state-of-the art Jupyter notebook interface and frontend
  and demonstrate the power that this approach delivers to accelerate
  computational science, deliver better value for money and make
  computational science more robust.  We have chosen the Object
  Oriented MicroMagnetic Framework (OOMMF) simulation package
  \cite{OOMMF-url} as the target tool which is used to simulate
  magnetic nanostructures in over 1800 publications
  \cite{OOMMF-citations-url}. We use \OOMMFNB{} (for OOMMF NoteBook) as
  an identifier for this case study.
\end{wpdescription}

\begin{tasklist}
\begin{task}[title=Uniform notebook interface for all interactive components,id=ipython-kernels]
  In this task, we will implement Jupyter interfaces for the
  interactive computation components of \TheProject, including GAP,
  Pari, Sage, and Singular. A first release
  \delivref{UI}{ipython-kernels-basic} will focus on basic functionality,
  and a second release \delivref{UI}{ipython-kernels} will cover advanced
  features like 3D graphics or transparent documentation browsing (as
  live worksheets whenever relevant).

  % Note from William: my student Andrew Ohana just mostly did
  % something like that for IPython, but then stopped.  Anyway, it's
  % very do-able based on a summer project from another student and a
  % bunch of work I did with THREE.js for SMC.

  Sage itself will require a specific treatment as it already has a
  notebook interface. Its development started about at the same time
  as the Jupyter notebook, with similar target features but a
  different agenda: the Sage notebook had to be available very quickly
  to solve pressing needs of the Sage community; instead the Jupyter
  notebook was to take its time and build robust foundations from the
  ground up. The two projects have exchanged a lot, and the Jupyter
  notebook, which benefits from a much larger user base and thus
  developer pool, has mostly caught up with the Sage notebook in terms
  of functionality. It's thus time for the Sage community to outsource
  this key but non disciplinary component and phase out the Sage
  notebook in favor of the Jupyter notebook.

  % In charge: Jupyter dev + dev in Orsay + community?
  The Sage and Jupyter convergence \delivref{UI}{ipython-kernel-sage} will
  require:
  \begin{itemize}
  \item Robust migration path and tools for Sage worksheets,
  \item Support for math, 2D, and interactive 3D output.,
    % \item Bundling of the Jupyter notebook and its dependencies within
    %   the Sage distribution. DONE
  \item Import and export of ReST documents, with full support for
    Sage's specific roles (math, ...),
  \item Support for remote Sage kernel, typically on the cloud, or
    running with a different Python version (Sage as a library),
  \item A migration path for interactive widgets implemented with
    Sage's \texttt{@interact} functionality.
  \end{itemize}

  Joint meetings and visits between the developers of Jupyter and of
  the computing components will be a key asset for this task.

\end{task}

\begin{task}[id=notebook-collab,title=Notebook improvements for collaboration]
  In this task, we will further improve tools for collaboration between
  authors of shared Jupyter notebooks.
  
  Version control tools, such as Git and Mercurial, have become an integral part of open and
  collaborative science and software. Version control tools allow reviewing proposed changes via
  diffing tools, and resolving conflicting changes with merge tools. Jupyter notebook documents,
  being text files, are relatively well suited to tracking in version control. However, being
  structured JSON data, diffing and merging are difficult. Tools shall be developed to provide
  better support for visual diffing and merging of Notebook documents, and integrated into existing
  version control workflows \delivref{UI}{jupyter-collab}.
  
  Given the interactive nature of Jupyter notebooks, live collaboration, where multiple authors
  work on the document simultaneously, as in Google Docs, is particularly desirable. The addition
  of potentially shared execution adds both value and challenge to live collaboration. Various
  projects have added some amount of live sessions from outside (SageMathCloud, Colaboratory), but
  outside the core project. There are many different aspects of collaboration to explore,
  including shared or separate execution for authors collaborating on a single notebook,
  UI to indicate not only authorship,
  but which author triggered which execution, and other challenges.
  Various avenues for live session collaboration will be explored for integration into Jupyter itself
  \delivref{UI}{jupyter-live-collab}.
\end{task}




\begin{task}[id=notebook-verification,title=Reproducible Notebooks]
  In this task, we will develop tools that allow to re-execute
  notebook documents with automated regression testing. The computed
  output will be compared against the stored output, and deviations
  reported as assertion errors.

  Notebooks are used in a variety of contexts, like training and
  teaching material (tutorials, documentation, books) or computer
  experimentation logbooks, where reproducibility is critical: the
  notebooks shall remain functional and correct in the long run, even
  when the underlying computational software or infrastructure changes
  over time or across platforms.

  This task is a critical step toward reproducibility, allowing the
  notebook author to get an immediate notice when, e.g., a backward
  incompatible change occurs. It becomes even possible to anticipate
  such situations upstream by including important notebooks directly
  in the automated test suite of the computational software, giving an
  easy way for casual users to contribute regression tests.

  Technically speaking, Jupyter notebooks store outputs as rich mime-type structures,
  with JSON metadata. Using this metadata, it will be possible to express
  expectations of output, allowing more flexible and powerful tests
  than direct text comparison \delivref{UI}{jupyter-test}.
  Prior work has been done in Sage for ReST files, e.g. \lstinline{sage -t notebook.rst},
  and this model will be extended to notebooks.
\end{task}

\begin{task}[id=dynamic-inspect,title=Dynamic documentation and exploration system]

  Introspection has become a critical tool in interactive computation,
  allowing user to explore on the fly the properties and capabilities
  of the objects under manipulation. This becomes particularly acute
  in systems like Sage where large parts of the class hierarchy is
  built dynamically, and static documentation builders like Sphinx
  cannot anymore render all the available information.

  In this task, we will investigate how to further enhance the user
  experience. This will include:

  \begin{itemize}
  \item On the fly generation of Javadoc style documentation, through
    introspection, allowing e.g. the exploration of the class
    hierarchy, available methods, etc.
  \item \TOWRITE{Logilab}{Inclusion of database queries and views}
  \item \delivref{UI}{ipython-advanced-interacts} (Month 36) Exploratory
    support for semantic-aware interactive widgets providing views on
    objects represented and or in databases

    Preliminary steps are demonstrated in the \texttt{Larch
      Environment} project (see demo video on
    \url{http://www.larchenvironment.com/}) and sage-explorer.

    Ultimate goal: automatically generated LMFDB-style interfaces.
    Mention Knowls, as dynamic context-free items of knowledge
  \end{itemize}


  Whenever possible, those features will be implemented generically
  for any computation kernel by extending the Jupyter protocol with
  introspection and documentation queries.

  % In charge: Jupyter dev + dev in Orsay + NT?
\end{task}

\begin{task}[title=Structured documents,id=structdocs]
  % \item \delivref{ipython-structured-documents} (Month ???)
  Support for writing interactive structured documents, and in
  particular papers, books, experimentation log books and reports,
  presentations, course notes, etc, with the following features:
  \begin{itemize}
  \item Static printed/PDF/HTML version and interactive version.\\
    Achieved by either importing or exporting document files in some
    standard format (LaTeX, ReST, Markdown, ...).
  \item Tests (\localtaskref{notebook-verification}).
  \item Collaborative editing.
  \item Version control.
  \end{itemize}
\end{task}

\TODO{include here everything about this topic in Needs.rst}

\TODO{Wherever relevant, create tickets with details, and refer to
  them here.}

\begin{task}[id=oommf-python-interface,title=OOMMF case study: Create Python interface to OOMMF code]
  % 6 person months

  First, we will identify best option for interfacing from Python to OOMMF
  core (C++) routines. The technical options include CTypes, Cython, Swig,
  and Boost-Python, all with particular
  advantages/disadvantages. Following analysis of the current OOMMF
  code layout and compilation model, we will use the most suitable
  tool, bearing in mind our ambition not to modify the OOMMF code so
  that the python interface we create remains functional and
  maintainable with minimal effort while the OOMMF core code is
  developed further by the OOMMF authors.

  The interface will expose the raw C++ objects in Python, and for
  clarity we will refer to this interface as \texttt{OOMMF-py-raw}, to
  annotate that this gives access to OOMMF from Python but in a RAW
  way. Creation of this \texttt{OOMMF-py-raw} is technically doable as
  OOMMF had been written allowing to do this from Tcl. The
  \texttt{OOMMF-py-raw} library for Python provides access to the
  OOMMF functionality but requires some care when being used.

  Secondly, we will create a user friendly Python library that
  combines the OOMMF-RAW capabilities we expect to become the main
  user interface to OOMMF in the medium term future. This will make
  use of object orientation to assist users in efficient and safe
  exploitation of the available facilities, following the design of
  the well-received high level Nmag simulation package
  \cite{Fischbacher2007a} interface \cite{Nmag-url}.

  Once this is completed, several new features will be available to
  OOMMF users: (i) ability to drive OOMMF from Python, (ii)
  computational steering, and (iii) combination of OOMMF simulation
  with the existing Python eco-system of computational tools.

  %Can remove the next paragraph if we are pushed for space.

  We illustrate the advantage of (iii) through an example: to solve
  the micromagnetic standard problem 3
  \cite{Micromagnetic-Standardproblem-3}, traditionally multiple OOMMF
  simulation runs would have to be conducted, and for each of those a
  new configuration file as to be written. Between these the size of
  the simulated geometry needs to be modified until two particular
  values of energy are the same. Given the new interface developed in
  this work package, this whole process can be replaced by one Python
  script that creates multiple OOMMF simulations, combined with a root
  finding method for the automatic iterative determination of the
  required simulation geometry.

  Parallel in developing this, a set of unit tests is created that can
  be run periodically as regression tests. For all tasks relating to
  \OOMMFNB, documentation and tests are created simultaneously with
  the code. All codes, tests and documentation will be made available as open source.

  We anticipate to start this task \localtaskref{oommf-python-interface}
  in month 4.
\end{task}

\begin{task}[title=OOMMF case study: Extend \texttt{OOMMF-py} with Jupyter
    notebook attributes and GUI templates,id=oommf-py-ipython-attributes]
  % 6 person months

  The web server based Notebook environment (Jupyter) allows to host,
  execute and document the Python-based OOMMF simulation in an
  executable document. In this interactive environment, the
  representation of objects can be overloaded, and can include
  representation of objects as text, as bitmaps or svg files. We will
  create this functionality so that magnetisation vector field objects
  can be presented as a rendered 3d and 2d-view of the magnetisation
  field, and similar features for scalar fields such as field
  components and energies. This allows the interactive exploration and
  computational steering of the behaviour of magnetic
  nanostructures. Depending on the development of 3d packages such as
  vispy, it may be possible to provide interactive data objects in the
  notebook.

  Beyond that, the Jupyter Widgets allow the creation of graphical
  user interface (GUI) like elements, and we will generate code to
  display these widgets on demand to (i) set up micromagnetic
  simulation using a GUI, and (ii) assist in post-processing
  simulation results. Not all OOMMF users are keen on using GUIs for
  simulation set up or post processing, but in particular new or
  infrequent users benefit significantly from this. Recent pilot work
  has shown that it is possible to make Jupyter Widgets interact with
  the python interpreter session and this allows to activate a
  GUI-like (widget based) interface when desired but to quickly return
  to the interpreter prompt, taking forward the results (data) from
  the GUI session \cite{IPython-widget-GUI-demo-youtube-2014} and
  providing a continuous path from scripting to GUI. We
  believe that having the ability to mix and match GUI-based and
  command driven analysis combines the best of both approaches and
  provides significant additional value.
\end{task}

\begin{task}[title=OOMMF case study: \OOMMFNB{} tutorial and
  documentation, id=oommf-tutorial-and-documentation]
  % 6 person months

  We will create documentation and a new tutorial on usage of OOMMF
  that introduces micromagnetic modelling in the new framework of
  \OOMMFNB{}, combined with complete documentation of the
  \texttt{OOMMF-Py} library. The documentation will be provided in
  form of executable Jupyter notebooks.

  The tutorial, in terms of content, will take guidance from the
  tutorial provided for Nmag \cite{Nmag-tutorial-url} but tailored for the
  special simulation capabilities of OOMMF, and will introduce the
  special capabilities of the new IPython interface for OOMMF.

  The output of this activity will deliver multiple benefits:
  providing a systematic introduction to \texttt{OOMMF-py} suitable for both
  those users (i) new to micromagnetic modelling and those (ii) new to
  the \texttt{OOMMF-py} interface. Because the documentation is developed in an
  Jupyter notebook, the documents are executable. For new learners
  this is a great simplification because they can skip through the
  given document and execute the given examples there and then: at the
  moment, this is a process of manually writing a script, or locating
  it in the directory structure of files, then executing this,
  subsequently opening and processing the data files, etc. In the new
  model, this end-to-end simulation will start from specifying the
  material parameters in the notebook (all of this is given in the
  tutorial), to running the simulation in the notebook to processing
  of computed data while the simulation runs (or subsequently) in the
  notebook; thus providing one virtual research environment, with all
  the associated benefits of making best use of the scientist's time
  using the tool and environment.

  The documentation and tutorial will include a number of typical
  micromagnetic case studies that (i) demonstrate the correctness of
  the code by executing some of the micromagnetic standard problems
  and (ii) demonstrate the additional power gained by the
  IPython-based OOMMF interface. We expect this substantial, executable
  documentation to become the standard resources that introduces
  researchers to computational micromagnetics.
\end{task}

\begin{task}[id=oommf-nb-ve,title=OOMMF case study: \OOMMFNB{} virtual environments]
  % 3 person months
  Recently, a TeMPorary Jupyter NoteBook has been made available at
  \href{http://tmpnb.org}{http://tmpnb.org}, that allows anybody to go
  to this URL and use their very own Jupyter notebook for quick
  calculations and tests online. We will provide similar functionality
  but for a server that provides the \OOMMFNB{} software and \OOMMFNB{}
  documentation and tutorials so that the tutorial can be executed
  immediately on that web server; thus removing the barrier of having
  to install (the OOMMF and Jupyter notebook) code before being able to interactively drive and test a
  simulation system.

  We will further provide as open source the scripts that allow
  creation of virtual environments (such as vagrant scripts to
  generate VirtualBox \cite{Virtualbox} images, and Docker
  \cite{Docker} containers). These virtual environments underpin the
  web hosted temporary \OOMMFNB{} service (we anticipate to use Docker
  on the web hosted service) but are also of use to those users who
  want to download a complete virtual machine (such as a virtualbox
  image) to run their simulations within that machine. The same
  virtual machine images can also be used for Cloudhosted computing services.

  % XXX HF Financial details should probably go elsewhere.
  We request 3100 EUR (ex VAT) to purchase a machine to provide these
  services (shared memory, 16 cores, 64GB RAM, small solid state drive
  to make the system more responsive). This machine will also support
  the regression testing and continuous integration (see task
  \taskref{dissem}{dissemination-of-oommf-nb-virtual-environment}). Setup and
  maintenance of the machine is part of this work task.
\end{task}

\end{tasklist}

\begin{wpdelivs}
  \begin{wpdeliv}[due=12,id=ipython-kernels-basic,dissem=PU,nature=O]
      {Basic Jupyter interface for GAP, Pari, Sage, Singular}
  \end{wpdeliv}
  
  \begin{wpdeliv}[due=12,id=ipython-kernels,dissem=PU,nature=O]
      {Full featured Jupyter interface for GAP, Pari, Singular}
  \end{wpdeliv}
  
  \begin{wpdeliv}[due=12,id=ipython-kernel-sage,dissem=PU,nature=DEM]
      {Sage notebook / Jupyter notebook convergence}
  \end{wpdeliv}

  \begin{wpdeliv}[due=18,id=jupyter-test,dissem=PU,nature=O]
      {Using notebooks for verification tests}
  \end{wpdeliv}
  
  \begin{wpdeliv}[due=12,id=jupyter-collab,dissem=PU,nature=O]
      {Improvements to notebook collaboration}
  \end{wpdeliv}

  \begin{wpdeliv}[due=36,id=jupyter-live-collab,dissem=PU,nature=O]
      {Explore live notebook collaboration}
  \end{wpdeliv}
  
  
  \begin{wpdeliv}[due=36,id=ipython-advanced-interacts,dissem=PU,nature=DEM]
      {Exploratory support for semantic-aware interactive widgets providing views on objects
      represented and or in databases}
  \end{wpdeliv}

  % Shared Jupyter sessions embedded in voice-over-IP or
  % teleconference calls or reciprocally.
  %
  % NOTE: This task is probably outdated by appear.in which makes
  % video-conferencing in the browser trivial
  %
  % \delivref{ipython-collaborative}
  % Eugen Dedu:
  % I think such a module can be thought of as a screen-capturing
  % module, i.e. allow Ekiga to capture the screen of a Sage user (this
  % is currently not possible).  This is not a difficult task to do.
  % Julien Puydt: ekiga can do that since something like 2008 with my
  % experimental gstreamer plugin, and I shall be able to present
  % interesting sample code to the ekiga-devel mailing-list in something
  % like two-three weeks (after I'm done with my students), which will
  % hopefully be part of the next version.
  % 
  % But as Nicolas noted in his answer, some kind of interative session
  % where people can share a sage session would be better.
  % 
  % I think the feature decomposes in the following pieces:
  % - IPython should have a way to share sessions between several
  % participants using an open and standard protocol ;
  % - ekiga should implement it.
  % 
  % In my opinion ekiga, because of its dependency on ptlib and opal
  % libraries and the use of complex protocols like SIP and H323, needs
  % highly technical people.  Students cannot help much, but engineers
  % are appropriate.
  \begin{wpdeliv}[due=9,id=oommf-py-raw,dissem=PU,nature=O]
      {Python Interface to OOMMF}
\end{wpdeliv}
  \begin{wpdeliv}[due=15,id=oommf-nb,dissem=PU,nature=DEM]
      {Jupyter notebook Interface for OOMMF (\OOMMFNB{})}
\end{wpdeliv}
  \begin{wpdeliv}[due=21,id=oommf-nb-documentation,dissem=PU,nature=DEC]
      {\OOMMFNB{}    executable tutorial and documentation}
\end{wpdeliv}
  \begin{wpdeliv}[due=24,id=oommf-nb-tmp,dissem=PU,nature=DEC]
      {\OOMMFNB{} dynamic web service available}
\end{wpdeliv}
  \begin{wpdeliv}[due=24,id=oommf-nb-virtual,dissem=PU,nature=O]
      {\OOMMFNB{} virtual machine images available for download}
  \end{wpdeliv}
\end{wpdelivs}
\end{workpackage}

\begin{verbatim}

About the availability of people to hire, I have a full-time,
experienced developer whose contract runs out in fall 2015, he would be
ideal for the project. I also have a doctoral student who needs
employment after the MathSearch project (until 10/2015) runs out. So I
do have people who would directly be available.

Michael

===================8<---------------------------------

Task 4.10. Structured Documents (12 PM total, 3 PM per deliverable)
   -> This existing task we could just take over based on our MathHub.info
         system, which would need to be adapted to the task.
Deliverables:
   D1: Active Documents based on sTeX
   D2: Distributed, Collaborative, Versioned Editing of Active Documents
in MathHub.info
   D3: Notebook Import into MathHub.info (interactive display)
   D4: in-place computation in active documents (context/computation).
Comments:
  MathHub.info is a portal for reading and interacting with "active
documents"
  (i.e. documents that have an additional semantic layer that supports
semantic services like
   - definition lookup, type-inference, unit conversion, ...)
  Notebooks are essentially "programs with documentation", whereas
active documents are
  documents with a semantic knowledge layer. Regular publications are an
important
  boundary case: Active Documents look like papers, but are
web-standards compatible
  and interactive.
  sTeX is a semantic variant of LaTeX that we can transform into OMDoc/MMT,
  which is the native knowledge representation format for active documents
  and machine-actionable knowledge about math and symbolic programs.

===================8<---------------------------------

Task K-4.11 Math Search Engine (10 PM total; 2 each for D1/2, 3 each for
D3/4)
   D1: Full-text search (formulae + Keywords) over LaTeX-based documents
         (e.g. arXiv subset)
   D2: Full-text search (F+K) over Notebooks (in the format determined
in task 4.7)
   D3: Formula search in CAS programs and Software Modules
   D4: Search from Notebooks/Active Documents (for local context to
inform search)
Comments:
   We already have a search engine, therefore we only need to build
harvesters for D1/2;
   D3/4 are more speculative.

\end{verbatim}

%%% Local Variables: 
%%% mode: latex
%%% TeX-master: "../proposal.tex"
%%% End: 

\addtocounter{wpno}{1}
\begin{Workpackage}{\thewpno}
\label{wp:hpc}
\wplabel{wp:x}
\WPTitle{\wpname{\thewpno}}
\WPStart{Month 1}
\WPParticipant{PS}{1}
\WPParticipant{LL}{1} % Pythran
\WPParticipant{SA}{1} % GAP
\WPParticipant{UK}{1} % Singular
\WPParticipant{UB}{1} % Pari
\WPParticipant{UG}{1} % Pari

\begin{WPObjectives}
  The objective of this work package is to improve the performance of
  the computational components of \TheProject, in particular on
  massively parallel architectures. This includes notably:
  \begin{itemize}
  \item Fine grained High Performance Computing on many-cores
    architectures.
  \item Coarse grained or embarrassingly parallel computing on grids
    or on the cloud.
  \item Compilation of high level interpreted code to optimized
    parallel C/C++.
  \item Develop novel HPC infrastructure in the context of
    combinatorics.
  \end{itemize}
  A key aspect will be to foster further sharing expertise and best
  practices between computational components.
\end{WPObjectives}

\begin{WPDescription}
  As in all other areas of science, properly supporting massively
  parallel architecture is a major challenge. Many of the
  computational components in \TheProject have already gone a long way
  in this direction. For example, an adaptation of the \GAP kernel for
  HPC was developed during the 2009-2013 EPSRC project. The expertise
  gained there was then transferred to the ongoing \Singular-HPC
  project, in particular through the rehiring of one of the developers
  of \GAP-HPC.

  In this work package, we will build on this momentum to further
  implement HPC support in the components Tasks~\ref{task:hpc_pari},
  \ref{task:hpc_linbox}, and \ref{task:hpc_singular}.

  \TODO{transition}

  Many of the computational components of \TheProject use a high level
  interpreted language for their library. This is notably the case of
  \Sage. Performance is achieved by compiling critical sections using
  the \Cython \Python-to-C compiler. In
  Tasks~\ref{task:pythran_cython} and~\ref{task:pythran_sage}, we will also
  boost performance by further developing and applying such
  compilation tools.
\end{WPDescription}

\begin{task}{Pari}
  \label{task:hpc_pari}
  \TOWRITE{KB}{Task around HPC/parallelism in Pari?}

  \TODO{deliverable}
\end{task}

\begin{task}{Linbox}
  \label{task:hpc_linbox}
  \TOWRITE{JGD/CP}{Task around HPC/parallelism in Linbox}

  \TODO{deliverable}
\end{task}

\begin{task}{Singular}
  \label{task:hpc_singular}
  \TOWRITE{WD}{Task around HPC/parallelism in Singular}

  \TODO{deliverable}
\end{task}


\begin{task}{Pythran-Cython convergence}
  \label{task:pythran_cython}
  \TOWRITE{SG}{Expand task: Pythran-Cython convergence}

  \Pythran is a \Python to C++ compiler for a subset of the \Python
  language. It is meant to efficiently compile scientific programs,
  and takes advantage of multi-cores and SIMD instruction units.
  Thanks to type inference, it requires little annotations.

  \Cython is a \Python to C compiler that was originally developed for
  \Sage and is now a thriving project of its own. It can handle
  essentially any \Python code, and in particular classes, but relies
  heavily on annotations for producing optimized code.

  Therefore, \Pythran and \Cython are similar in spirit but have
  complementary feature sets. In this task, we will investigate the
  opportunity and feasibility of a convergence between \Cython and
  \Pythran: depending on the code at hand, one strategy or the other
  would be automatically selected. Also, it should be able to integrate an
  improved compiler-runtime cooperation in the \Cython compiler thanks to
  part of the \Pythran-runtime and the extra typing information provided by
  \Cython. An effort will be made to improve more and more the parallelism in
  the Pythran-runtime.

  This work will be achieved by a close collaboration between the
  \Pythran developers hired for \TheProject and \Cython developers
  involved in the \Sage project. It should quicken \Sage execution time at least
  on numpy centric codes.

\end{task}

\begin{task}{\Pythran for \Sage}
  \label{task:pythran_sage}
  \TOWRITE{SG}{Expand task: Pythran for Sage}
  Currently, \Sage doesn't provide facilities to improve user written Python
  code improvement as the Cython compiler required heavily annotated code. As
  Pythran doesn't need these informations, a notebook interface to compile Pythran
  compliant code will he added in Sage to improve user kernels.

  Internally, \Sage uses \Cython for compiling the critical sections of
  its libraries. In this task, we will explore opportunities to
  benefit from Pythran compilation within the Sage library, in
  particular toward better support for parallelism. A specific
  challenge is that the \Sage library uses quite heavily
  object-oriented programming.

  This task will strongly benefit from Task~\ref{task:pythran_cython},
  while providing in return a real life large-scale use case for it.

  A first step to support object-oriented programming will be to make Pythran
  typing better which will improve error information provided for the user.
\end{task}


\begin{WPDeliverables}
\item
\ref{task:pythran_cython}
(Month XX):
Add Pythran-runtime support in Cython toward a unified interface.
\item
(Month XX):
Improve Pythran-runtime support for parallelism.
\item
\ref{task:pythran_sage}
(Month XX):
Facility to compile Pythran compliant user kernels.
\item
(Month XX):
Make Pythran typing better to improve error information.
\item
Explorative task:
Add support for classes in Pythran.
\end{WPDeliverables}
\end{Workpackage}

%%% Local Variables: 
%%% mode: latex
%%% TeX-master: "../proposal.tex"
%%% End: 

\begin{workpackage}[id=dksbases,wphases=1-48!.5,
  title=Data/Knowledge/Software-Bases,lead=JU,
  ZHRM=1,JURM=36,USHRM=12,UWRM=3]

\begin{wpobjectives}
The objectives of this work package are: to design and implement interfaces that can be used for a wide range of mathematical data and to standardise metadata allowing for interoperability, searching, documentation, tracability, versioning and visualisation.
\end{wpobjectives}


\begin{wpdescription}
Mathematics is the only science that has not yet benefitted greatly from the systematic interchange of data. At the same time, mathematics has a richer notion of data than other disciplines.
Indeed, "mathematical data" consists of three kinds of objects:
\begin{itemize}
\item[] [D]: proper (numeric/symbolic) data
\item[] [K]:  the knowledge about the mathematical objects given as statements (definitions, theorems or proofs; either formal or rigorously informal)
\item[] [S] : software that computes (with) the mathematical objects
\end{itemize}

All three kinds of "data" are equally important for mathematics and are tightly interlinked:
\begin{itemize}
\item[] [D] serves as examples for [K] or as counterexamples for conjectures in [K];
\item[] [S] computes [D] and establishes properties of [D] (given as [K]);
\item[] [D] tests [S], [S] is verified with respect to [K];
\item[] theorems and proofs in [K] induce and justify algorithms for [S];
\item[] [D] induces conjectures and guides proofs in [K].
\end{itemize}

Many mathematical databases now exist, but their internal structure does not reveal this richness. This weakness prevents the formulation of new conjectures, the testing of new hypotheses, and generally an exploratory approach to mathematical data. The past has shown that such an approach can be fruitful: 
\begin{itemize}
\item both the Riemann Hypothesis and the Birch and Swinnerton-Dyer conjectures resulted from exploratory $L$-function computations, and now stand among the seven Clay Millenium Problems;
\item the Monstrous Moonshine conjecture finds its origin in a numerical co\"incidence between dimensions of representations of the Monster group and coefficients of the $j$-function, and its conclusion eventually led to Borcherds' Fields medal.
\end{itemize}

Therefore to facilitate future advances, we need ways to represent DKS in the same systems, and -- since current computational/experimental mathematics involve extensive DKS -- we need a new kind of "database", which we will call Mathematical Data/Knowledge/Software-bases.

This complexity is on vivid display in the \emph{L-functions and Modular Forms database} project (\LMFDB): while the general shape of the functional equation of an $L$-function is dependent on a lot of theoretical knowledge, it also requires parameter data and the coefficients of the associated Dirichlet series. Once this is obtained, highly optimised (and heavily parallelizable) algorithms can be run to compute values of this function. 

We propose in this work package to design and build an infrastructure that would make it
easy for either individual mathematicians or a distributed collaboration to manage and use
such interlinked mathematical data. This work would provide part of the backend to Work
Packages \TODO{work package on interfaces, and???}, and would draw on previous work with
the \LMFDB and FindStat (which will be treated as prototypes for our purposes, to serve as
exemplars to other projects) and in return will substantially enhance their
capabilities. Prerequisites should be kept to a minimum (depending on contributors' and
users' needs and goals), and in particular would not require any background in databases
to contribute new data or perform queries.
\end{wpdescription}
\begin{tasklist}
\begin{task}[title=Survey of existing databases,id=data-assessment]
All the systems considered in this proposal (\GAP, \Sage, \Pari, \Singular) include data as part of their regular distribution. In this task, we will survey existing databases, the technology used to implement them, how they were linked to the rest of the existing infrastructure and the functionalities offered. We will also select additional external data and projects to add to this effort, aiming to maximise the impact of our work. 
\end{task}

\begin{task}[title={Design of new infrastructure, formulation of requirements}, id=data-design]
Ontologies are the canonical method used to implement databases that require significant data interchange. However, because of extreme reification in mathematics, this is not entirely suitable for our goals. We will design a new infrastructure for \TheProject, drawing on existing emerging standards. 

We will organise a workshop associated to this task.
\end{task}

\begin{task}[title=Triform Theories in OMDoc/MMT,id=data-triform]
OMDoc/MMT is a representation language for mathematical knowledge and documents. Carette and Farmer have developed the notion of biform theories (K/S) in a uniform representational approach; our work here would extend this along the data axis, which will require a specialised but integrated treatment.
\end{task}

\begin{task}[title=Computational Foundation for Python/Sage (or some CAS),id=data-foundationCAS]
In the OMDoc/MMT world a foundation is a logical base language that gives the formal meaning to all objects represented/formalized in it. We have created a very initial computational foundation for Scala and implemented it in the MMT API. This can be used to execute (or verify) computations directly in OMDoc/MMT and thus forms the basis for various integration tasks for OMDoc/MMT biform theories that integrate Scala computations. Here we propose to develop a somewhat more complete computational foundation for Python and/or parts of Sage (coverage to be determined). Bi/Triform theories come in three parts:
\begin{itemize}
\item syntax: what operators/types are there, how do they nest, 
\item computation:  what does the computation relation look like (sometimes called operational semantics). The declarative semantics of a computational foundation can be given as an OMDoc/MMT theory morphism into another foundation (e.g. a set theory);
\item ??? three parts
\end{itemize}
\end{task}

\begin{task}[title=OEIS Case Study (Coverage and automated import),id=data-OEIS]
  In this case study we test the practical coverage of the trifunctional modules, by
  transforming an existing, high-profile database (the Online Sequence of Integer
  Sequences http://www.oeis.org) into OMDoc/MMT. The OEIS has about 250 thousand
  sequences, with formulae, descriptions, definitions, references, software, etc. in a
  structured text file (but no standardized format for formulae and references), so we
  expect to get 250 k theories. Having the OEIS in OMDoc/MMT form allows to do Knowledge
  Management services (presentation, definition lookup, formula search, ...) in
  MathHub.info (see WP4.?). The OEIS is a good case study, since the DKM are licensed
  under a CC license which allows derived works. The large size will allow statistically
  significant semantic cross-validation of the heuristic transformation process and thus
  achieve a significant DKS community resource.
\end{task}

% Michael, I think triformal theories would be easier to start with findstat.org
% There are many reasons: more consistent structure in the mathematical data, more established research patterns, more consistent database storage, tighter integration of the code with sage code (in fact copy paste), etc

\begin{task}[title=FindStat Case Study (triformal theories),id=data-findstat]
  In this task we would develop triformal theories for the FindStat project to test the
  design from \localtaskref{data-foundationCAS}.  Similarly to the previous task, in this
  case study, we first develop a thorough OMDoc/MMT model, which should only involve a
  handful of MMT theories (combinatorial collections, maps, statistics,...), each with a
  few hundred realisations. Together with \TOWRITE{POD}{WP4}, this will again allow for
  easier knowledge management services, and in particular improved search services.

  This Task will be co-developed with \localtaskref{data-foundationCAS}, it will validate
  the design of triformal theories and be iterated to test the design changes.
\end{task}

\begin{task}[title=\LMFDB Case study (triformal theories),id=data-LMFDB]
  In this task we would develop triformal theories for an exemplary part of the \LMFDB
  project to test the design from \localtaskref{data-foundationCAS}.  We will identify a
  fragment of the \LMFDB that we want to model and design the model. Then we will perform
  cross-validation of the three model parts against each other (essentially model-based
  testing of software and inference). Finally, we will pick an algorithm from the LFMDB
  and verify it against its specification and the computational foundation developed in
  \localtaskref{data-foundationCAS}. (decrease importance of verification as opposed to
  interoperability)
\end{task}



\begin{task}[title=Memoization and production of new data,id=data-memo]
Many CAS users run large and intensive computations, for which they want to collect the results while simultaneously working on software improvements. \Sage currently has a limited \lstinline{cached_method}, that is not persistent across sessions and does not enable to publish the result or share it with a smaller group of collaborators. We propose to use, extend and contribute back to some established persistent memoization infrastructure, such as \texttt{python-joblib}, \texttt{redis-simple-cache} or \texttt{dogpile.cache}. The caching should apply recursively to lower level functions, and should be trivial to setup and configure for the end user: in a single line, the user only needs to select an existing function and maybe provide some additional semantic information, and has the option to change the defaults for a few parameters, such as the backend (shared dropbox folder, remote directory, database, git repository, ...). The interface could be through a \Python decorator. 
Additionally, it should be easy to launch a data-bot to populate the database, all the versioning and provenance tracking should be handled (user, algorithm, software version, ...), and the system should have useful data properties (atomicity, merging, and error detection). 
%Mock code:
%    \begin{lstlisting}
%       mycloud = storage("ssh:xxx@yyy/zzz")
%       memoize(sage.combinat...., storage=mycloud, input=ZZ, output=Posets(), key="catalan")
%    \end{lstlisting}
\end{task}
\end{tasklist}

\begin{wpdelivs}
  \begin{wpdeliv}[due=12,id=conv,dissem=PU,nature=DEC]
        {Conversion of existing and new databases to unified interoperable system}
     \begin{itemize}
     \item Polytopes in Polymake
     \item graphs, graph properties
     \item Finite groups (Max)
     \item Lattices
     \end{itemize}
   \end{wpdeliv}
  \begin{wpdeliv}[due=24,id=persistent-memoization,dissem=PU,nature=O]
    {Shared persistent memoization library for Python/Sage} 
    Recomputation?,  Ease of publishing, importing, ...
  \end{wpdeliv}
  \begin{wpdeliv}[due=9,id=wsrep,dissem=PU,nature=R]{Workshop Report}
  \end{wpdeliv}
  \begin{wpdeliv}[due=12,id=dkstheories,dissem=PU,nature=R]
        {Design of Triform (DKS) Theories (Specification/RNC Schema/Examples)}
  \end{wpdeliv}
  \begin{wpdeliv}[due=24,id=dksimp,dissem=PU,nature=O]
        {Implementation of Triform Theories in the MMT API.}
  \end{wpdeliv}
  \begin{wpdeliv}[due=12,id=pssyntax,dissem=PU,nature=DEC]
        {Python/Sage Syntax Foundation Module in OMDoc/MMT}
  \end{wpdeliv}
  \begin{wpdeliv}[due=24,id=psfoundation,dissem=PU,nature=O]
        {Python/Sage Computational Foundation Module in OMDoc/MMT}
  \end{wpdeliv}
  \begin{wpdeliv}[due=36,id=pssem,dissem=PU,nature=O]
      {Python/Sage Declarative Semantics in OMDoc/MMT}
  \end{wpdeliv}
  \begin{wpdeliv}[due=12,id=lfmmod,dissem=PU,nature=R]
      {\LMFDB deep modelling: Fragment Identification \& Initial Model Design}
  \end{wpdeliv}
  \begin{wpdeliv}[due=18,id=lfmval,dissem=PU,nature=R]
      {\LMFDB Data vs. Knowledge vs. Software Validation}
  \end{wpdeliv}
  \begin{wpdeliv}[due=36,id=lfmverif,dissem=PU,nature=O]
      {\LMFDB Algorithm verification wrt. a Triformal theory}
  \end{wpdeliv}
  \begin{wpdeliv}[due=46,id=lfmint,dissem=PU,nature=R]
      {\LMFDB full integration of algorithms, data and presentation (not so much verification)}
  \end{wpdeliv}
  \begin{wpdeliv}[due=9,id=oeisparser,dissem=PU,nature=O]
      {Heuristic Parser for the OEIS}
  \end{wpdeliv}
  \begin{wpdeliv}[due=18,id=oeisvalidation,dissem=PU,nature=R]
      {Cross-Validation for OEIS DKS-Theories}
  \end{wpdeliv}
\end{wpdelivs}


Another connection: on several occasions, we found that software was the best way to
represent certain databases of mathematical knowledge. E.g. in Algebraic Combinatorics we
have a whole zoo of Hopf algebras. Many of them are implemented in MuPAD/Sage by
specifying the objects that index the basis together with computation rules for the
product and coproduct. When we want to retrieve information about such algebras, it's
usually much more convenient to look at the code than to search through the
literature. Especially since the code is usually more correct than the literature because
it's *tested*.


We may also think of providing an interface to \LMFDB via SCSCP
protocol (http://www.symbolic-computing.org/scscp) so it may
be accessed by a variety of other systems (see their current
list at http://www.symbolic-computing.org/scscp)


database access to \LMFDB as a python library
\end{workpackage}
%%% Local Variables:
%%% mode: latex
%%% TeX-master: "../proposal"
%%% End:

%  LocalWords:  workpackage dksbases wphases wpobjectives standardise visualisation emph
%  LocalWords:  wpdescription Swinnerton-Dyer Millenium Borcherds optimised tasklist conv
%  LocalWords:  parallelizable maximise organise biform specialised trifunctional TOWRITE
%  LocalWords:  triformal findstat.org data-findstat localtaskref realisations texttt wrt
%  LocalWords:  Memoization python-joblib texttt redis-simple-cache texttt dogpile.cache
%  LocalWords:  lstlisting mycloud memoize sage.combinat wpdelivs wpdeliv dissem Polymake
%  LocalWords:  Recomputation wsrep dkstheories dksimp pssyntax psfoundation pssem lfmmod
%  LocalWords:  modelling lfmval lfmverif lfmint oeisparser oeisvalidation Hopf coproduct

%\input{WorkPackages/DevelopmentModelsForAnAcademicFreeSoftwareEcosystem}
%\addtocounter{wpno}{1}
\begin{Workpackage}{\thewpno}
\wplabel{wp:x}
\WPTitle{\wpname{\thewpno}}
\WPStart{Month 1}
\WPParticipant{SA}{1}

\begin{WPObjectives}
The objectives of \theWP{} are to:
\begin{itemize}
\item
\item
\item
\item
\item
\end{itemize}
\end{WPObjectives}

\begin{WPDescription}
This workpackage  ...
\end{WPDescription}

\begin{WPDeliverables}
\begin{itemize}
\item
\ref{del:x}
(Month X): 
X.
\end{itemize}
\end{WPDeliverables}
\end{Workpackage}

\begin{workpackage}[id=social-aspects,wphases=12-24!.5,
  title=Social Aspects,
  UORM=1]

\TOWRITE{DP/UM}{workpackage Social Aspects}

\begin{wpobjectives}
The objectives of this work package are to:
\begin{itemize}
\item
\item
\item
\item Development models for an academic free software ecosystem
\item Supporting the Mathematical Process
\end{itemize}
\end{wpobjectives}

\begin{wpdescription}
This workpackage  ...
\end{wpdescription}

% Things to investigate?
% - User surveys. Cf. https://groups.google.com/d/msg/sage-devel/v8Kfky4p6D4/_xRM0bggCo8J
% - The discussion about Code of Conducts and the like

\begin{wpdelivs}
  \begin{wpdeliv}[due=12,id=social_...,dissem=??,nature=??]
      {...}
\end{wpdeliv}
\end{wpdelivs}
\end{workpackage}
%%% Local Variables:
%%% mode: latex
%%% TeX-master: "../proposal"
%%% End:

\addtocounter{wpno}{1}
\begin{Workpackage}{\thewpno}
\wplabel{wp:x}
\WPTitle{\wpname{\thewpno}}
\WPStart{Month 1}
\WPParticipant{SA}{1}

\begin{WPObjectives}
  The objective of this work package is to organize and optimize the
  communication with the larger community. This includes:
  \begin{itemize}
  \item Reviewing emerging technologies
  \item Advertising the project (press, web, ...).
  \item Disseminating results and deliverables.
  \end{itemize}
\end{WPObjectives}

\begin{WPDescription}
  The first task of \theWP{} is to produce periodic reviews of
  emerging technologies and relevant developments elsewhere, and
  implications for our plans. This include the review of standard
  components and service for storage and sharing, computational
  resources, authentication, package management, etc.  This may
  further include negotiating access or shared development when
  appropriate. This information will be fed to the other work
  packages, in particular\TODO{ref: Component Architecture}.

  Dissemination: software, APIs, technologies, research results, ...
\end{WPDescription}

\begin{WPDeliverables}
\begin{itemize}
% Or make those into a single deliverable?
\item \ref{del:periodic-rep-1}
  (Month 12): Year 1 report.
\item \ref{del:periodic-rep-2}
  (Month 24): Year 2 report.
\item \ref{del:periodic-rep-3}
  (Month 36): Year 3 report.
\item \ref{del:periodic-rep-4}
  (Month 48): Year 4 report.
% \deliverable{del:pressrelease} % Press release.
% \deliverable{del:website} % Project presentation (web site). 
% \deliverable{del:workshop1}  % Report on first project workshop, year 1. 
% \deliverable{del:dissemplan1} % Final plan for using and disseminating knowledge.
% \deliverable{del:workshop2}  % Report on second project workshop, year 2
% \deliverable{del:workshop3}  % Report on third project workshop, year 3
% \deliverable{del:dissemplan2} % Final plan for using and disseminating knowledge.
\end{itemize}
\end{WPDeliverables}

Raw material:
\begin{itemize}
\item Documentation improvements: overview, cross links, overview of
  recent improvements
\item Thematic tutorials
\item Collections of pedagogical documents\\
  E.g. a complete collection of interactive class notes with computer
  lab projects for the ``Algèbre et Calcul formel'' option of the
  French math aggregation (starting from 2014-2015, only open-source
  systems will be supported, and Sage is a major player).
  % See http://nicolas.thiery.name/Enseignement/Agregation/ as a starter
  % Math labs with Sage for first year students in France (L1): http://math.univ-lyon1.fr/~omarguin/
\item Localization of the Sage user interface and key documents in
  various European languages.
\item Distribution of the documents either in the main distribution of
  Sage or through the online repository (see collaborative tools).
\item Massive online introduction course to Sage, drawing on the sage tutorial/notebooks.
Could be "First year Sage course in a box".
\item Taking the opportunity of Python courses to propose Sage as a natural extension
for mathematics; an example is French's 
% TODO: The url macro eats the accented letters.
``Classes pr\'eparatoires''\footnote{
\url{http://en.wikipedia.org/wiki/Classe_préparatoire_aux_grandes_écoles}}, 
where Python has been recently selected as the language to learn programming\footnote{See 
the ``Annexe'' at 
\url{http://www.education.gouv.fr/pid25535/bulletin_officiel.html?cid_bo=71586}}.
%\item \TODO{please expand!}
\end{itemize}

% Jeroen: About teaching: in Gent, Sage is already integrated in the
% courses (maybe you can add this, don't know if it's relevant)
% starting in the first year. It's good for the students because it
% helps in 2 ways: it helps them to understand the mathematics better
% and it helps them to learn basic down-to-earth programming (they
% also have a programming course in Java but that contains a lot of
% theory about complicated class structures)
% Same thing in Orsay
% More python centered but same in UZH

\end{Workpackage}

\end{workplan}

%%% Local Variables:
%%% mode: latex
%%% TeX-master: "../proposal"
%%% End:



%%% Local Variables:
%%% mode: latex
%%% TeX-master: "proposal"
%%% End:
