\section{Scientific/Technical Methodology and Work Plan}\label{sec:methodology}
\begin{todo}{from the proposal template}
  A detailed work plan should be presented, broken down into work packages\footnote{A work
    package is a major sub-division of the proposed project with a verifiable end-point –
    normally a deliverable or an important milestone in the overall project.} (WPs) which
  should follow the logical phases of the implementation of the project, and include
  consortium management and assessment of progress and results. (Note that your overall
  approach to management will be described later, in Section 2).

Notes: The number of work packages used must be appropriate to the complexity of the work
and the overall value of the proposed project. The planning should be sufficiently
detailed to justify the proposed effort and allow progress monitoring by the Commission.

Any significant risks should be identified, and contingency plans described
\end{todo}
\newpage\section{Objectives and Work Schedule \deu{(Ziele und Arbeitsprogramm)}}

\subsection{Objectives}

\begin{objective}[id=firstobj,title=Supporting Authors]
  This is the first objective, after all we have to write proposals all the time, and we
  would rather spend time on research. 
\end{objective}

\begin{objective}[id=secondobj,title=Supporting Reviewers]
  They are only human too, so let's have a heart for them as well. 
\end{objective}

\subsection{Work Schedule} 
\begin{todo}{from the proposal template}
  Give a short high-level introduction to how the work in the project should proceed,
  explain Tables~\ref{fig:wplist} and~\ref{fig:gantt}.
\end{todo}
The project is organized around \pdatacount{all}{wp} work packages, which we summarize in
Figure~\ref{fig:wplist}. 

\wpfig

We ensures the dissemination and creation of the periodic integrative reports containing
the periodic Project Management Report, the Project Management Handbook, an Knowledge
Dissemination Plan ({\WPref{management}}), the Proceedings of the Annual {\pn} Summer
School as well as non-public Dissemination and Exploitation plans ({\WPref{dissem}}), as
well as a report of the {\pn} project milestones.

\begin{workplan}   
\begin{workpackage}[id=management,title=Project Management,wphases=1-24!.3,
 RM=2,RAM=8]
  Based on the ``Bewilligungsbescheid'' of the DFG, and based on the financial and
  administrative data agreed, the project manager will carry out the overall project
  management, including administrative management.  A project quality handbook will be
  defined, and a {\pn} help-desk for answering questions about the format (first
  project-internal, and after month 12 public) will be established. The project management
  will consist of the following tasks
\begin{tasklist} 
\begin{task}[id=foo,wphases=0-3]%,requires=\taskin{t1}{dissem}]
  To perform the administrative, scientific/technical, and financial management of the
  project 
\end{task}
\begin{task}[id=bar,wphases=13-17!.5]
  To co-ordinate the contacts with the DFG and other funding bodies, building on the
  results in \taskref{management}{foo}
\end{task}
\begin{task}
  To control quality and timing of project results and to resolve conflicts
\end{task}
\begin{task}
  To set up inter-project communication rules and mechanisms
\end{task}
\end{tasklist}

\end{workpackage}
 
\begin{workpackage}[id=dissem,title=Dissemination and Exploitation,
RM=8]
Much of the activity of a project involves small groups of nodes in joint work. This work
 package is set up to ensure their best wide-scale integration, communication, and
 synergetic presentation of the results. Clearly identified means of dissemination of
 work-in-progress as well as final results will serve the effectiveness of work within the
 project and steadily improve the visibility and usage of the emerging semantic services.


 The work package members set up events for dissemination of the research and
 work-in-progress results for researchers (workshops and summer schools), and for industry
 (trade fairs). An in-depth evaluation will be undertaken of the response of test-users.
 
 \begin{tasklist}
  \begin{task}[id=t1,wphases=6-7]
    sdfkj
  \end{task}
  \begin{task}[id=t2,wphases=12-13]
    sdflkjsdf
  \end{task}
  \begin{task}[id=t3,wphases=18-19]
    sdflkjsdf
  \end{task}
 \begin{task}[id=t4,wphases=22-24] 
 \end{task}
\end{tasklist}

Within two months of the start of the project, a project website will go live. This
website will have two areas: a members' area and a public area.\ldots
\end{workpackage}
 

\begin{workpackage}[id=class,
   title=A LaTeX class for EU Proposals,short=Class,
  RM=12,RAM=8]
We plan to develop a {\LaTeX} class for marking up EU Proposals

We will follow strict software design principles, first comes a
requirements analys, then \ldots
\begin{tasklist}
  \begin{task}[id=c1,wphases=0-2]
    sdfsdf
  \end{task}
  \begin{task}[id=c2,wphases=4-8]
    sdfsdf
  \end{task}
  \begin{task}[id=c3,wphases=10-14]
    sdfsdf
  \end{task}
  \begin{task}[id=c4,wphases=20-24]
    sdfsdfd
  \end{task}
\end{tasklist}
\end{workpackage} 

\begin{workpackage}[id=temple,title= Proposal Template,
  short=Template,RM=12]

We plan to develop a template file for {\pn} proposals

We abstract an example from existing proposals
\begin{tasklist}
  \begin{task}[id=temple1,wphases=6-12]
    sdfdsf 
  \end{task}
  \begin{task}[id=temple2,wphases=18-24]%,requires=\taskin{c3}{class}]
    sdfsdf
  \end{task} 
\end{tasklist}
\end{workpackage}

\begin{workpackage}[id=workphase,title=A work package without tasks,
  wphases=0-4!.5]
  And finally, a work package without tasks, so we can see the effect on the gantt chart
  in fig~\ref{fig:gantt}.
\end{workpackage}
\end{workplan} 

\ganttchart[draft,xscale=.45] 

%%% Local Variables: 
%%% mode: LaTeX
%%% TeX-master: "proposal"
%%% End: 

% LocalWords:  workplan.tex wplist dfgcount wa mansubsus duratio ipower wpfig
% LocalWords:  ganttchart xscale workplan workarea pdataref dissem workpackage foo
% LocalWords:  tasklist taskin taskref sdfkj sdflkjsdf sdfsdf sdfsdfd sdfdsf pn
% LocalWords:  firstobj secondobj pdatacount WAref ednote OBJref wphases
% LocalWords:  ldots OBJtref workphase gantttaskchart


\newpage
\subsection{Work Package List}\label{sec:wplist}

\begin{todo}{from the proposal template}
Please indicate one activity per work package:
RTD = Research and technological development; DEM = Demonstration; MGT = Management of the consortium
\end{todo}

%\makeatletter\wp@total@RM{management}\makeatother
\wpfigstyle{\footnotesize}
\wpfig[pages,type,start,end]

\newpage%% Deliverables list.
%% Deliverables ordered by Workpackage
%% Workpackages are numbered automatically in sequence - the WP number has no effect

\workpackage{1}{Project Management}
\deliverable{mgt:mailinglists}
\deliverable{mgt:projectwebsite}
\deliverable{mgt:swrepository}
\deliverable{mgt:periodic-rep-1}
\deliverable{mgt:periodic-rep-2}
\deliverable{mgt:periodic-rep-3}
\deliverable{mgt:periodic-rep-4}
\deliverable{mgt:final-mgt-rep}
% Metrics: in PM and a bit in each work package

\workpackage{2}{Community Building and Engagement}
\deliverable{del:xx}

\workpackage{3}{Component Architecture}
\deliverable{del:xx}

%\workpackage{XXX}{Standardization} % => Component architecture + advertisement in the dissemination
%\deliverable{del:xx}

\workpackage{4}{User Interfaces}
\deliverable{del:xx}

\workpackage{5}{HPC and massively parallel components}
\deliverable{del:xx}

\workpackage{6}{Next generation mathematical databases}

\deliverable{del:xx}
%SL

\workpackage{7}{Social Aspects}
\deliverable{del:xx}
%UM
%\workpackage{XXX}{Development Models for an Academic Free Software Ecosystem}
%\workpackage{XXX}{Supporting the Mathematical Process} % => A chunk of Social Aspects


\workpackage{8}{Dissemination, Exploitation and Communication}
\deliverable{del:pressrelease} % Press release.
\deliverable{del:website} % Project presentation (web site). 
\deliverable{del:workshop1}  % Report on first project workshop, year 1. 
\deliverable{del:dissemplan1} % Final plan for using and disseminating knowledge.
\deliverable{del:workshop2}  % Report on second project workshop, year 2
\deliverable{del:workshop3}  % Report on third project workshop, year 3
\deliverable{del:dissemplan2} % Final plan for using and disseminating knowledge.

\newpage\eucommentary{Milestones means control points in the project that help to chart progress. Milestones may
correspond to the completion of a key deliverable, allowing the next phase of the work to begin.
They may also be needed at intermediary points so that, if problems have arisen, corrective
measures can be taken. A milestone may be a critical decision point in the project where, for
example, the consortium must decide which of several technologies to adopt for further
development.}

The work in the \TheProject project is structured by four milestones, which could be
briefly characterised as: starting up and building prototypes; moving from prototypes to
fully functional implementations; further engagement with the community and producing
research outputs; evaluation and final releases. They coincide with the project meetings
held at the end of each year of the project (four other meetings will be held in the
middle of each year).  Given the nature of the project, with a large number of essentially
independent tasks, there is no need for milestones attached to specific collections of
tasks or deliverables.  Given that the meetings are the main face-to-face interaction
points in the project, we have chosen to schedule the milestones for these events, where
they can be discussed in detail, tracking the progress in each work package through status
reports on the tasks and deliverables and take corrective measures, where necessary, and
critical decisions regarding further plans.  We envisage that this setup will give the
project the vital coherence in spite of the broad interdisciplinary mix of various
backgrounds of the participants.

\paragraph{General Milestones}

\begin{milestones}
  \milestone[id=startup,month=12,
  verif={Completed all corresponding deliverables and reported the progress in the 2nd Project meeting report.}]
  {Startup}
  {By Milestone 1 we will have carried out the requirements study, design and prototype implementations and started community building activities.}

  \milestone[id=proto1,month=24,
  verif={Completed all corresponding deliverables and reported the progress in the 4th Project meeting report.}]
  {Implementations}
  {By Milestone 2 we will have constructed first fully functional interface implementations and released enhanced versions of \TheProject components, and train early adopters of \TheProject.}

  \milestone[id=community,month=36,
  verif={Completed all corresponding deliverables and reported the progress in the 6th Project meeting report.}]
  {Community/ Experiments}
  {By Milestone 3 we will have gathered and evaluated feedback on \TheProject software and established the portfolio of experiments produced with \TheProject through further engaging with the community.}

  \milestone[id=eval,month=48,
  verif={Completed all corresponding deliverables and reported the progress in the 8th Project meeting report.}]
  {Evaluation}
  {By Milestone 4 we will have released final versions of all \TheProject components and completed the project evaluation.}
\end{milestones}

\paragraph{Milestone for WP 3}
We propose 1 milestone:

\begin{milestones}
%original delivery date proposal is M36 but milestone is linked to D3.10 which is planned for M48...
	\milestone[id=WP3availability,month=42,
	 verif={Have \ODK's components available on major platforms}]
	 {Work Package 3 aims at deploying all computational components
	 developed by \ODK available on the three major platforms (i.e.
	 Windows, Mac, Linux) via their standard distribution channels.}
\end{milestones}

\paragraph{Milestones for WP 4}
We propose two milestones:

\begin{milestones}
  \milestone[id=WP4prototype,month=36,
    verif={Prototype VRE for mathematical researchers}]
  {Prototype VRE for mathematical researchers}
  {
  % note: delivref doesn't work here
  User story: A group of mathematical researchers with access to
  common computational resources, such as a shared lab computer or
  cloud servers, shall be able to deploy a prototype VRE with
  \JupyterHub, integrating \ODK components.
  The Jupyter kernels for mathematical software developed as part of \ODK
  make computational mathematical components accessible in a \Jupyter
  environment, enabling a Jupyter-based deployment of the relevant
  tools for the researchers.
  The process of working on notebooks is greatly improved by review tools
  developed as part of WP4,
  enabling researchers to collaborate to some degree
  in a shared computational environment.
  }
  \milestone[id=WP4collaborative,month=48,
  verif={Collaborative VRE for mathematical researchers}]
  {Collaborative VRE for mathematical researchers}
  {
  The prototype VRE shall be extended with improved ease of deployment, new
  functionality such as interactive 3D visualization and real-time
  collaboration, enabling researchers to collaborate productively in a shared
  computational environment. Finally, integrating notebooks and semantic
  knowledge into a publication / knowledge system enable a continuous process
  of leveraging \ODK components from research to publication.
  }
\end{milestones}

\paragraph{Milestones for WP 6}

\begin{milestones}
  \milestone[id=WP6interop1,month=36,
  verif={Demonstrator Online Public, works on selected case study examples}]
  {First MitM-based interoperability prototype (GAP, SageMath, LMFDB)}
  {We intend to present a fully functional prototype of the integration of at least the
    systems GAP, SageMath, and LMFDB via the SCSCP Protocol at the second review 
    meeting. This prototype will be the basis for additional integration work for 
    additional systems and the use interface from WP4.}
\milestone[id=WP6interop2,month=42,   verif={Demonstrator Online Public, works on selected case study examples}]
  {Second MitM-based interoperability prototype}
  {The goal of this milestone is to take into account all the operational 
    experiences with the first prototype and add more systems and integrate some
    of the UI components from The experiences with the preparation of 
    this prototype will allow us to estimate the joining costs of adding a system 
    to the OpenDreamKit VRE toolkit, which is an important measure of the 
    flexibility of the MitM approach.}
\end{milestones}

%%% Local Variables:
%%% mode: latex
%%% TeX-master: "proposal"
%%% End:

%  LocalWords:  verif ldots


\subsection{Work Package Descriptions}\label{sec:workpackages}
\begin{workplan}
\begin{workpackage}[id=management,type=MGT,wphases=0-24!.2,
  title=Project Management,short=Management,
  jacuRM=2,barRM=2,efoRM=2,bazRM=2]
We can state the state of the art and similar things before the summary in the boxes
here. 
\wpheadertable
\begin{wpobjectives}
  \begin{itemize}
    \item To perform the administrative, scientific/technical, and financial
      management of the project
    \item To co-ordinate the contacts with the EU
    \item To control quality and timing of project results and to resolve conflicts
    \item To set up inter-project communication rules and mechanisms
  \end{itemize}
\end{wpobjectives}

\begin{wpdescription}
  Based on the Consortium Agreement, i.e. the contract with the European Commission, and
  based on the financial and administrative data agreed, the project manager will carry
  out the overall project management, including administrative management.  A project
  quality handbook will be defined, and a {\pn} help-desk for answering questions about
  the format (first project-internal, and after month 12 public) will be established. The
  project management will\ldots we can even reference deliverables:
  \delivref{management}{report2} and even the variant with a title:
  \delivtref{management}{report2}
\end{wpdescription}

\begin{wpdelivs}
  \begin{wpdeliv}[due=1,id=mailing,nature=O,dissem=PP,miles=kickoff]
    {Project-internal mailing lists}
  \end{wpdeliv}
  \begin{wpdeliv}[due=3,id=handbook,nature=R,dissem=PU,miles=consensus]
    {Project management handbook}
  \end{wpdeliv}
\begin{wpdeliv}[due={6,12,18,24,30,36,42,48},id=report2,nature=R,dissem=public,miles={consensus,final}]
  {Periodic activity report} 
  Partly compiled from activity reports of the work package
  coordinators; to be approved by the work package coordinators before delivery to the
  Commission.  Financial reporting is mainly done in months 18 and 36.\Ednote{how about
    these numbers?}
  \end{wpdeliv}
 \begin{wpdeliv}[due=6,id=helpdesk,dissem=PU,nature=O,miles=kickoff]
    {{\pn} Helpdesk}
  \end{wpdeliv}
  \begin{wpdeliv}[due=36,id=report6,nature=R,dissem=PU,miles=final]
    {Final plan for using and disseminating the knowledge}
  \end{wpdeliv}
  \begin{wpdeliv}[due=48,id=report7,nature=R,dissem=PU,miles=final]
    {Final management report}
  \end{wpdeliv}
\end{wpdelivs}
\end{workpackage}

%%% Local Variables: 
%%% mode: LaTeX
%%% TeX-master: "propB"
%%% End: 

% LocalWords:  wp-management.tex workpackage efoRM bazRM wpheadertable pn ldots
% LocalWords:  wpobjectives wpdescription delivref delivtref wpdelivs wpdeliv
% LocalWords:  dissem Ednote pdataRef deliv mansubsusintReport wphases
\newpage
\begin{workpackage}%
[id=dissem,type=RTD,lead=efo,
 wphases=10-24!1,
 title=Dissemination and Exploitation,short=Dissemination,
 efoRM=8,jacuRM=2,barRM=2,bazRM=2]
We can state the state of the art and similar things before the summary in the boxes
here. 
\wpheadertable

\begin{wpobjectives}
  Much of the activity of a project involves small groups of nodes in joint work. This
  work package is set up to ensure their best wide-scale integration, communication, and
  synergetic presentation of the results. Clearly identified means of dissemination of
  work-in-progress as well as final results will serve the effectiveness of work within
  the project and steadily improve the visibility and usage of the emerging semantic
  services.
\end{wpobjectives}

\begin{wpdescription}
  The work package members set up events for dissemination of the research and
  work-in-progress results for researchers (workshops and summer schools), and for
  industry (trade fairs). An in-depth evaluation will be undertaken of the response of
  test-users.

  Within two months of the start of the project, a project website will go live. This
  website will have two areas: a members' area and a public area.\ldots
\end{wpdescription}

\begin{wpdelivs}
  \begin{wpdeliv}[due=2,id=website,nature=O,dissem=PU,miles=kickoff]
     {Set-up of the Project web server}
   \end{wpdeliv}
   \begin{wpdeliv}[due=8,id=ws1proc,nature=R,dissem=PU,miles={kickoff}]
     {Proceedings of the first {\pn} Summer School.}
   \end{wpdeliv}
   \begin{wpdeliv}[due=9,id=dissem,nature=R,dissem=PP]
     {Dissemination Plan}
   \end{wpdeliv}
   \begin{wpdeliv}[due=9,id=exploitplan,nature=R,dissem=PP,miles=exploitation]
     {Scientific and Commercial Exploitation Plan}
   \end{wpdeliv}
   \begin{wpdeliv}[due=20,id=ws2proc,nature=R,dissem=PU,miles={exploitation}]
     {Proceedings of the second {\pn} Summer School.}
   \end{wpdeliv}
   \begin{wpdeliv}[due=32,id=ss1proc,nature=R,dissem=PU,miles={exploitation}]
     {Proceedings of the third {\pn} Summer School.}
   \end{wpdeliv}
   \begin{wpdeliv}[due=44,id=ws3proc,nature=R,dissem=PU,miles=exploitation]
     {Proceedings of the fourth {\pn} Summer School.}
   \end{wpdeliv}
 \end{wpdelivs}
\end{workpackage}

%%% Local Variables: 
%%% mode: LaTeX
%%% TeX-master: "propB"
%%% End: 

% LocalWords:  wp-dissem.tex workpackage dissem efo fromto bazRM wpheadertable
% LocalWords:  wpobjectives wpdescription ldots wpdelivs wpdeliv ws1proc pn
% LocalWords:  exploitplan ws2proc ss1proc ws3proc pdataRef deliv
% LocalWords:  mansubsusintReport
\newpage
\begin{workpackage}[id=class,type=RTD,lead=jacu,
                    wphases=3-9!1,
                    title=A {\LaTeX} class for EU Proposals,short=Class,
                    jacuRM=12,barRM=12]
We can state the state of the art and similar things before the summary in the boxes
here. 
\wpheadertable
\begin{wpobjectives}
\LaTeX is the best document markup language, it can even be used for literate
programming~\cite{DK:LP,Lamport:ladps94,Knuth:ttb84}

  To develop a {\LaTeX} class for marking up EU Proposals
\end{wpobjectives}

\begin{wpdescription}
  We will follow strict software design principles, first comes a requirements analys,
  then \ldots
\end{wpdescription}

\begin{wpdelivs}
  \begin{wpdeliv}[due=6,id=req,nature=R,dissem=PP,miles=kickoff]
     {Requirements analysis}
   \end{wpdeliv}
   \begin{wpdeliv}[due=12,id=spec,nature=R,dissem=PU,miles=consensus]
     {{\pn} Specification }
   \end{wpdeliv}
   \begin{wpdeliv}[due=18,id=demonstrator,nature=P,dissem=PU,miles={consensus,final}]
     {First demonstrator ({\tt{article.cls}} really)}
   \end{wpdeliv}
   \begin{wpdeliv}[due=24,id=proto,nature=P,dissem=PU,miles=final]
     {First prototype}
   \end{wpdeliv}
    \begin{wpdeliv}[due=36,id=release,nature=P,dissem=PU,miles=final]
      {Final {\LaTeX} class, ready for release}
    \end{wpdeliv}
  \end{wpdelivs}
Furthermore, this work package contributes to {\pdataRef{deliv}{managementreport2}{label}} and
{\pdataRef{deliv}{managementreport7}{label}}.
\end{workpackage}

%%% Local Variables: 
%%% mode: LaTeX
%%% TeX-master: "propB"
%%% End: 
\newpage
\begin{workpackage}[id=temple,type=DEM,lead=bar,
  wphases=6-12!1,
  title={\pn} Proposal Template,short=Template,barRM=6,bazRM=6]
We can state the state of the art and similar things before the summary in the boxes
here. 
\wpheadertable

\begin{wpobjectives}
  To develop a template file for {\pn} proposals
\end{wpobjectives}

\begin{wpdescription}
  We abstract an example from existing proposals
\end{wpdescription}

\begin{wpdelivs}
  \begin{wpdeliv}[due=6,id=req,nature=R,dissem=PP,miles=kickoff]
    {Requirements analysis}
  \end{wpdeliv}
  \begin{wpdeliv}[due=12,id=spec,nature=R,dissem=PU,miles=consensus]
    {{\pn} Specification }
  \end{wpdeliv}
  \begin{wpdeliv}[due=18,id=demonstrator,nature=D,dissem=PU,miles={consensus,final}]
    {First demonstrator ({\tt{article.cls}} really)}
  \end{wpdeliv}
  \begin{wpdeliv}[due=24,id=proto,nature=P,dissem=PU,miles=final]
    {First prototype}
  \end{wpdeliv}
  \begin{wpdeliv}[due=36,id=release,nature=P,dissem=PU,miles=final]
    {Final Template, ready for release}
  \end{wpdeliv}
\end{wpdelivs}
Furthermore, this work package contributes to {\pdataRef{deliv}{managementreport2}{label}} and
{\pdataRef{deliv}{managementreport7}{label}}.
\end{workpackage}

%%% Local Variables: 
%%% mode: LaTeX
%%% TeX-master: "propB"
%%% End: 

% LocalWords:  wp-temple.tex workpackage fromto pn bazRM wpheadertable wpdelivs
% LocalWords:  wpobjectives wpdescription wpdeliv req dissem tt article.cls
% LocalWords:  pdataRef deliv systemsintReport
\newpage
\end{workplan}
\newpage\subsection{Significant Risks and Associated Contingency Plans}\label{sec:risks}
\begin{todo}{from the proposal template}
  Describe any significant risks, and associated contingency plans
\end{todo}
\begin{oldpart}{need to integrate this somewhere. CL: I will check other proposals to see how they did it; the Guide does not really prescribe anything.}
\paragraph{Global Risk Management}
The crucial problem of \pn (and similar endeavors that offer a new basis for communication
and interaction) is that of community uptake: Unless we can convince scientists and
knowledge workers industry to use the new tools and interactions, we will
never be able to assemble the large repositories of flexiformal mathematical knowledge we
envision. We will consider uptake to be the main ongoing evaluation criterion for the network.
\end{oldpart}

%%% Local Variables: 
%%% mode: latex
%%% TeX-master: "propB"
%%% End: 



%%% Local Variables: 
%%% mode: latex
%%% TeX-master: "propB"
%%% End: 

% LocalWords:  workplan newpage wplist makeatletter makeatother wpfig
% LocalWords:  workpackages wp-dissem wp-class wp-temple
