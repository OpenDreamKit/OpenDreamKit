\begin{sitedescription}{XFEL}
  \label{sitedescription:xfel}

% PIC:
% see: http://ec.europa.eu/research/participants/portal/desktop/en/orga

% See ../proposal.tex, section Members of the Consortium for a
% complete description of what should go there

  The European X-Ray Free-Electron Laser Facility GmbH is a limited
  liability company under German law. At present, 11 countries are
  supporting European XFEL through cash and in-kind contributions:
  Denmark, France, Germany, Hungary, Italy, Poland, Russia, Slovakia,
  Spain, Sweden, and Switzerland. The company is in charge of the
  construction and operation of the European XFEL, a 3.4$\,$km long X-ray
  free-electron laser facility extending from Hamburg to the
  neighbouring town of Schenefeld in the German federal state of
  Schleswig-Holstein. Civil construction started in early 2009; the
  beginning of user operation is planned for 2017. With its repetition
  rate of 27,000 pulses per second and a peak brilliance a billion
  times higher than that of the best synchrotron X-ray radiation
  sources, it is expected that the European XFEL will enable the
  investigation of still open scientific problems in a variety of
  disciplines (physics, structural biology, chemistry, planetary
  science, study of matter under extreme conditions and many others).

  European XFEL, a landmark on the ESFRI Roadmap, is a single site
  X-ray research infrastructure. When operational, 3~beamlines and
  6~experiments will be available for scientific users. The SASE1
  beamline comprises the instruments Single Particles, clusters, and
  Biomolecules and Serial Femtosecond Crystallography (SPB/SFX) and
  Femtosecond X-ray Experiments (FXE), SASE 2 includes Materials
  Imaging and Dynamics (MID) and High Energy Density Science (HED) and
  SASE3 Small Quantum Systems (SQS), and Spectroscopy \& Coherent
  Scattering (SCS).


\medskip In the context of this proposal, Hans Fangohr has long
standing experience in high performance computer simulation to advance
science and engineering, and the education of researchers in the
most effective pursuit of computational science.

European XFEL is using IPython and the Jupyter
Notebook as core utilities in their large scale experiment control,
data capture and data analysis.

\subsubsection*{Curriculum vitae}

% Curriculum of the personnel at this institution
%
\input{CVs/Hans.Fangohr.XFEL.tex}
% \begin{participant}[PM=2,type=PI]{Ian Hawke}
Ian Hawke is a lecturer in Applied Mathematics at the University of
Southampton and a co-director of the $\pounds$10$\,$m EPSRC Centre for Doctoral
Training in Next Generation Computational Modelling. An expert in
nonlinear simulations of relativistic matter and numerical techniques,
he has taught numerical methods in many contexts for ten years. The
initial author of the “Whisky” relativistic hydrodynamics code, he has
been a contributor to and maintainer of a range of projects used
across the numerical relativity community, including the Einstein
Toolkit, the Cactus infrastructure and the Carpet mesh refinement
code. His recent research has concentrated on numerical methods for
relativistic matter beyond ideal fluids, including modelling sharp
transitions and surfaces, relativistic elasticity, and the first
nonlinear simulations of relativistic multifluids.

XXX Add Jupyter Notebook based MOOC experience XXX.
\end{participant}

%%% Local Variables:
%%% mode: latex
%%% TeX-master: "../proposal"
%%% End:


\input{CVs/Marijan.Beg.XFEL.tex}
% %\input{CVs/First.Last.tex}
%
\subsubsection*{Publications, products, achievements}

\begin{compactenum}
\item Open Source micromagnetic simulation framework Nmag,
  \href{http://nmag.soton.ac.uk}{http://nmag.soton.ac.uk}, Thomas
  Fischbacher, Matteo Franchin, Giuliano Bordignon, Hans Fangohr: \emph{
A Systematic Approach to Multiphysics Extensions of Finite-Element-Based Micromagnetic Simulations: Nmag
IEEE Transactions on Magnetics \textbf{43}, 6, 2896-2898 (2007)}
\item Other open source contributions to the micromagnetic simulation
  community: OVF2VTK, higher order anisotropy extensions to OOMMF,
  OVF2MFM, summarised at
  \href{http://www.southampton.ac.uk/~fangohr/software/index.html}{http://www.southampton.ac.uk/~fangohr/software/index.html}
\item H. Fangohr.
\emph{A Comparison of \software{C}, \Matlab and \Python as Teaching Languages in Engineering}
Lecture Notes on Computational Science \textbf{3039}, 1210-1217 (2004)
\end{compactenum}

\end{sitedescription}



%KEY-MORE-TODOS



%%% Local Variables:
%%% mode: latex
%%% TeX-master: "../proposal"
%%% End:

%  LocalWords:  sitedescription Programme organisations programmes Centres subsubsection
%  LocalWords:  micromagnetic Nmag Fischbacher Franchin Bordignon Fangohr emph textbf
%  LocalWords:  Multiphysics summarised Iridis TFlops Modelling
