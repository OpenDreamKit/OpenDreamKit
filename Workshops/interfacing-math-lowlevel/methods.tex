Consider a high-level language H and a low-level language L.
Our goal is to allow users to write H-programs in a way makes use of fast code in L.
The latter may be a pre-existing library or co-developed by the same user.

We develop a set of criteria that allow classifying the various solutions.

% % % % % % % % % % % % % % % % % % % % %
\subsection{Type: Relationship between High- and Low-Level}

This criterion describes the relationship between H and L.
Here we name these relationships from the perspective of the high-level language.

\paragraph{Extension: Calling L-Code from H}
Like every programming language, H must provide a set of built-in primitive types and functions.
The simplest examples are the type of machine integers and the associated operations.

Usually, the implementation language of H already provides a systematic way of defining such primitive operations.
If L is this implementation language, it is straightforward to expose this to H-users.
This is often called \emph{foreign function interface}.

Instantiating the foreign function interface may be as easy as supplying as declaring a fresh H-identifier $h$ and annotating it with an existing L-identifier $l$.
This is often called an H-\emph{binding} for $l$.
Given an L-library, it is then straightforward to write such H-declarations for all L-primitives that are to be exposed.

\ednote{drawbacks: writing the bindings can be cumbersome; difficult to instantiate polymorphic L-types with H-types}

\paragraph{Reification: Operating on H-Code in L}
If an implementation (compiler or interpreter) of H in L is available, one can treat H-code as L-data.
H-declarations and objects $h$ can be built and manipulated as L-objects for the syntax tree of $h$.
Moreover, a convenience function can be provided that takes $h$ as an H-string and parses it into its syntax tree.
A dual function, often called \emph{evaluation} takes such a syntax tree and returns the result of running the H-code.
These functions are often already part of an L-library that can be easily integrated into L-projects.

This can be used in two ways.
Most obviously, one can call H-code from L:
this just requires building the expression to evaluate and then calling the evaluation function.

But more importantly for our interest, it can also be used to use L-code in H:
Here we build the L-object $l$ representing the desired H-object and then writing H-code that links against $l$.
Crucially, we are free to use objects $l$ that are not the semantics of any object expressible in L-syntax.

\paragraph{Translation: Compiling H-Code to L}
If H-code is compiled to a target into which we can also compile L-code, it is possible allow L-snippets in H-code.
When compiling the H-code, the L-snippets are compiled via different chain and spliced into the results.
In the simplest case, H is compiled into L itself, in which case the L-snippets can be retained without changed.

This may require the user to be familiar with the way in which H is compiled in order to correctly refer to H-identifiers in the L-snippets.

% % % % % % % % % % % % % % % % % % % % %
\subsection{Directionality of Function Calls}

This criterion describes the direction of in which data flows between H and L.

\paragraph{Unidirectional}
In the simplest interfaces, H calls L-functions and L returns results.
This requires
\begin{compactitem}
 \item mapping H-functions to L-functions
 \item translating H-objects (the function arguments) to L
 \item translating L-objects (the result) to H
\end{compactitem}

The maps of H-function to L-functions can often be restricted to a fixed set of mappings of named H-functions to L-functions.
Often the corresponding L-function arises from a named L-function through a simple operation, whose overlap in constant in the size in the arguments.
Typical examples are reordering or wrapping arguments.

\paragraph{Bidirectional}
Often a more complex kind of interface is are needed, where H calls L-functions, which may call back to H.
Seemingly, this is already possible using the requirements of unidirectional interfaces: all we have to do is map the callback to an L-function.
However, while unidirectional calls often need only a fixed set of named L-functions, the callback is typically a new function (anonymous or user-defined) that is not covered by the fixed set of mappings.

The options for mapping callbacks depend crucially on the interface type.
For example, if H is translated to L anyway, the callback function is already available available as an L-function.
Thus, the callback can be called directly from L without having to translate the arguments to H and the result back to L.


% % % % % % % % % % % % % % % % % % % % %
\subsection{Level: Set of Types and Objects that can be shared}

Efficient interfacing requires that translations between H- and L-types and objects can be done efficiently at the memory level.
Ideally, the necessary translation is the identity function.
In many situations a constant-time translation, e.g., by wrapping, is also acceptable.
However, a constant-time translation of basic objects may cause linear-time overhead in the translation of complex objects, e.g., when every element of a list has to be wrapped in order to translate a list.

\paragraph{Machine Types}
At the very least the types and functions provided by the underlying hardware machine (which are usually part of all programming languages anyway) can be shared between L and H.

\paragraph{Certain Named Types and Functions}
It is desirable to share also complex types and operations.
Often a fixed set of named types and functions is shared by individually creating bindings for them.

\paragraph{Arbitrary Expressions}
Ideally, arbitrary expressions can be shared, e.g., by using a shallow embedding of H and into L.

Here a \emph{shallow} embedding of A in B translates A-entities to B-entities of the same kind, e.g., A-types are represented as B-types and A-exceptions as B-exceptions.
That distinguishes shallow embeddings from deep ones such as the ones induced by reification, which are much easier obtain.