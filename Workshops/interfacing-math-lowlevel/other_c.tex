\subsection{Pari: CyPari}

Strategy: The code of Pari comes endowed with a description of each function:

TODO: insert example here

From this code, cypari autogenerates the following Cython binding which is then compiled to C:

TODO: insert example here

TODO: blurb about types: Pari has a dozen types of objects; conversions between Pari types and Sage types is written by hand.

\subsection{GAP: libsemigroups}

libsemigroups is a C++11 library for computational semigroups. It was written with three goals in mind: accelerate core code from GAP's semigroup package (low level language, parallelism, ...), include a modernized existing C library Semigroupe (parallelism, design issues), and make the result available to a larger community (GAP, )
Distribution: conda and github but no wheel (supposedly not well suited)
    
\paragraph{GAP bindings}

TODO: summary, pros, cons

Original version: handwritten
    
Conversion overhead but still faster than original computation (unlike numpy approach)

Second attempt: mimic pybind11 using compile time introspection (presumably available in both clang and g++).

\paragraph{Python bindings}

Handwritten using Cython; totally incomplete and painful to maintain

Potential plan for writing a C API, because many languages (clojure, ...) can directly call C code 

\paragraph{Difficulties}
- "old" compilers don't quite support C++ 11 (old but still current in conda, GAP tests, and other systems, at that point)
- code duplication
- *not* tested on windows; would presumably work on cygwin and windows subsystem but not natively (relies on automake/autoconf)

\subsection{Julia}

(Sebastian and Thomas)

Features:
- Implemented as a GAP package JuliaInterface
- works on top of vanilla GAP and Julia (some improvements will be submitted upstream GAP for tighter integration)
- conversions between GAP and Julia objects
- Manipulate Julia objects from GAP via handles
- Manipulate GAP objects from Julia via handles
- Run arbitrary Julia functions from GAP by name
- Evaluate arbitrary Julia commands as strings
- Application: load Julia code that calls back to GAP; this can lead to 20x speed improvements,
  presumably due to compilation + use of native Julia data structures (lists, ...) that are faster
  than GAP's.

Implementation:
- GAP and Julia cohabit in the same memory space, and interact through their C-API.
- Currently each system uses its own garbage collector, with GAP keeping a list
  of Julia objects; this leads to loops, and therefore memory leaks.
- The next step is to add a compilation option to GAP to make it optionally use
  Julia's garbage collector instead of it's own. Same for HPC-GAP?
- Same strategy as cypari, ...: stay lazy and convert objects only if really needed

Demo of calling functionalities of Nemo, Arb, ...

Difficulties:
- Cost of conversions; for example for big integers: the representation in gap isn't the same as in Julia => costly translation (sometimes carried through strings!), question is when to do it?
- Julia currently recompiles all loaded code (e.g. the interface to Singular) on the fly. This is slow and a big
  burden in practice. The Julia folks are working on caching.

- Would it make sense to 
- How comes that Singular is so slow to start (compared to launching a plain Singular):
- Semantic handles: thinking about it; not clear best way to proceed

\subsection{PolyMake (Perl)}

PolyMake uses Perl as H and C++ as L.

Perl is extended with entities defined in C++ by statically generating bindings.
Polymorphic C-classes are handled by dynamically generating bindings for ground instances.

Extension with is not only used for efficiency and expressivity.
Users also write parts of the library in L that benefit from the OO-style type hierarchies available in C++.

C++ and the Perl interpreter share the same memory space.
The interaction is mostly Perl calling C++.
But its also possible to call PolyMake from C++ (including all the rule-based Perl mechanism).\ednote{How?}

The rule-based mechanism is essentially a method selection mechanism, filling a similar purpose to GAP's method selection mechanism.
Like in GAP, when determining a property of an object, one of multiple available methods is chosen based on conditions about the object.
However, unlike in GAP, if these conditions are properties that have not been determined yet, they are determined through back-chaining.
Instead of a simple priority-based choice between multiple applicable methods, PolyMake must choose between all applicable branches of the resulting search tree.
