\documentclass[a4paper,11pt]{article}

\newcommand{\XX}{\textbf{XX}\xspace}
\newcommand{\TheProject}{\XX}

\usepackage{lscape} % for landscape
\usepackage{comments}
% %\usepackage[final]{comments}
\usepackage{verbatim}
\usepackage{listings}
\usepackage{supertabular,array}
\makeatletter
\newcommand\arraybslash{\let\\\@arraycr}
\makeatother
% \setlength\tabcolsep{1mm}
% \renewcommand\arraystretch{1.3}
%% Related Projects
\newcommand{\scienceproject}{\mbox{\textsc{SCIEnce}}}
\newcommand{\OOMMFNB}{OOMMF-NB}

\newcommand{\software}[1]{\texttt{#1}\xspace}
\newcommand{\GAP}{\software{GAP}}
\newcommand{\libGAP}{\software{libGAP}}
\newcommand{\Singular}{\software{Singular}}
\newcommand{\Sage}{\software{Sage}}
\newcommand{\SageCombinat}{\software{Sage-Combinat}}
\newcommand{\Python}{\software{Python}}
\newcommand{\IPython}{\software{IPython}}
\newcommand{\Jupyter}{\software{Jupyter}}
\newcommand{\Cython}{\software{Cython}}
\newcommand{\Pythran}{\software{Pythran}}
\newcommand{\Numpy}{\software{Numpy}}
\newcommand{\Pari}{\software{PARI}}
\newcommand{\PariGP}{\software{PARI/GP}}
\newcommand{\Linbox}{\software{LinBox}}
\newcommand{\LMFDB}{\software{LMFDB}}
\newcommand{\OpenEdX}{\software{OpenEdX}}
\newcommand{\Linux}{\software{Linux}}
\newcommand{\LATEX}{\software{\LaTeX}}
\newcommand{\SMC}{\software{SageMathCloud}}
\newcommand{\Simulagora}{\software{Simulagora}}
\newcommand{\Magma}{\software{Magma}}
\newcommand{\Mathematica}{\software{Mathematica}}
\newcommand{\Maple}{\software{Maple}}
\newcommand{\Matlab}{\software{Matlab}}
\newcommand{\Arxiv}{\software{arXiv}}

%%% Local Variables: 
%%% mode: latex
%%% TeX-master: "proposal"
%%% End: 

% Partners
\newparticipant{PS}{Université Paris Sud}{UPS}{FR}
\newparticipant{SA}{University of St Andrews}{USTAN}{UK}
\newparticipant{LL}{Logilab}{Logilab}{FR}
\newparticipant{UB}{Université Bordeaux}{UB}{FR}
\newparticipant{UK}{University of Kaiserslautern}{UK}{DE}
\newparticipant{UO}{University of Oxford}{UO}{UK}
\newparticipant{UW}{University of Warwick}{UW}{UK}
\newparticipant{US}{University of Silesia}{US}{PL}
\newparticipant{UV}{Université de Versailles}{UVSQ}{FR}
% Participant or third party?
\newparticipant{UWS}{University of Washington at Seattle}{UWS}{US}
% Jean-Pierre Flori
% Paul-Olivier Dehaye
% Luca DeFeo
% Jean-Guillaume Dumas
% Clément Pernet

% Personalised comments for each author
\newcommand{\slcomment}[1]{\comment{SL}{#1}}
\newcommand{\akcomment}[1]{\comment{AK}{#1}}

%% Related Projects
\newcommand{\scienceproject}{\mbox{\textsc{SCIEnce}}}



\begin{document}

\begin{titlepage}

\begin{center}
{\Large \textbf{COVER PAGE}}
\end{center}

\begin{tabular}{llr}
\textbf{Title of Proposal:} & \textbf{\TheProject{}: Collaborative ecosystems for mathematical research and software development} & \\[2ex] % \includeimage[scale=0.5]{logo} \\
\textbf{Date of preparation:} & \textbf{\today} & \comment{}{$
$Revision: 0.0$ $}\\[2ex]
\textbf{List of participants} && \\[2ex]


\end{tabular}

\begin{center}
\begin{tabular}{|l|p{3in}|l|l|}\hline
Participant no & Participant organisation name & Country\\

\hline 
1 (Coordinator) & \longparticipant{1} & \country{1}  \\ \hline
& & \\ \hline
& & \\ \hline
\end{tabular}
\end{center}

\tableofcontents

\end{titlepage}
\newpage

\begin{draft}
\section*{Outline of Project (for Proposers)}

\TODO{This is the place for various READMEs not included in the final submission}

\subsection*{Vision}

An internal attempt at specifying our vision through short
(unsubstantiated) answers.

\begin{verbatim}
> 1) Who are we?

Lead or core developers of some of the major open source components
for pure mathematics and applications:

- Computational components: GAP, Linbox, MPIR, Pari, Sage, Singular
- Databases: LMFDB (findstat as well)
- Knowledge management: MathHub

Together with, in a larger scientific domain, lead developers for:

- Collaborative user interfaces (IPython, SageMathCloud)
- Database and Scientific Computing for the industry (Logilab)
- Numerical code optimization/parallelisation (Pythran)

> 2) What is our goal?

Building blocks with a sustainable development model that can be
seamlessly combined together to build versatile high performance
VRE's, each tailored to a specific need in pure mathematics and
application.

> 2.5) What is our strategy?

Maximize sustainability and impact by reusing and improving existing
building blocks, and reaching toward larger communities whenever possible.
E.g. factoring out our common user interface needs at the level
of IPython/Jupyter will save us time (sustainability), and impact
the larger scientific computing community.
The improvements to the building blocks will impact all their users,
whether they use the VRE or not.

> 3) From where do we start?

- Building blocks with a sustainable development model
- Proof-of-concept prototypes of VRE (SMC, Simulagora)
- Experience on combining together some of the building blocks

> 4) How do we connect or differ from other projects?

The other projects focus on either one or a few of the building
blocks, or on a specific VRE.

We articulate our work with each of them.

> 5) Why are we excellent?

The consortium puts together recognized experts in all
areas and most building blocks that are relevant to the goal. There is
simultaneously a variety of point of views and a record of past
experiences collaborating together at smaller scale
(e.g. GAP-Singular). The approach is bottom up.  Most joint tasks
consist in bringing together people with a common need. There is
experience in community building.  Most participants are
simultaneously users and developers of their tools.

All of this makes me confident that we will indeed be able to
productively collaborate. And do stuff that is first class and useful.

On Sat, Dec 13, 2014 at 11:18:10PM +0100, Wolfram Decker wrote:
> 0) What precisely is our starting point and why are we the right people to
> achieve what we promise to do? Are we leaders in the area touched
> by the proposal? How do we connect? Is there some past
> collaborative success?
> 1) You still do not say what we actually will provide. What precisely will
> the VRE offer to its users?

I more or less answered those points above. Let me know if I should
elaborate.

> Who will be its users? Will those already familiar with the involved
> CAS use it? Will it make the CAS more attractive for a much larger
> community?

One objective is definitely to make CAS and others more attractive by
lowering a lot the entry barrier to access the soft (and db, ...). A
typical situation that most of us ran into is, when collaborating with
other less tech-savvy mathematicians, to have trouble sharing code,
data, and in-the-writing papers with them. SMC was launched with this
idea in mind, and the success proves the concept.

At the same time, the improvements in the building blocks will also
impact CAS users that are happy with their current user interface /
work-flow.

Improvements to IPython will impact a much larger community.

> 2) You motivate what we wish to do by the success of SageMathCloud.
> But why do we than need another VRE? How do we differ from
> SageMathCloud?

There is no one-size-fits-all VRE. One might want to run a VRE on
one's own computer resources for a variety of reason (speed of access,
specific resources, privacy, independence, ...). One might want a
different combination of software (e.g. a lightweight VRE with only
Singular).  One might want to focus on data with LMFDB-style database
searches, or on interactive computing, or on document writing, or some
combination thereof.

> Do we have a chance to compete? Or will we rather join forces? In
> which way?

We join forces (the plan is to have William/UW in the consortium, as
non funded participant). SMC focuses on one specific cloud based
VRE. We focus on the building blocks and the glue. Both project are
mutually beneficial. See the language p. 14 of the proposal.

> 3) You motivate what we wish to do by the success of LMFDB. But what
> are our connections to this database? Will we enhance it? Will we connect
> it to other stuff we do? Will we create other databases?

LMFDB is a prototype of large scale database. We want to make it
easier for other groups of mathematicians to setup similar databases
in their area. Reciprocally, like SMC, the LMFDB with benefit back
from the improved building blocks.

> 4) Why is Europe in the lead if there is already SageMathCloud?
> Where precisely is Europe in the lead?

Europe is the lead in many of the building blocks.
\end{verbatim}

% \subsection*{Mission statement for the grant}

% Our mission is to promote the next generation of community-developed
% open source software, databases, and services adapted to the needs of
% collaborative research in pure mathematics and applications.

% Our research will cover a wide variety of aspects, ranging from
% software development models, user interfaces \TODO{virtual
%   environments?}, deployment frameworks and novel collaborative tools,
% component architecture, design, and standardization of software
% \TODO{system?} and databases, to links to publication, data archival
% and reproducibility of experiments, development models and tools, and
% social aspects.

% It will consolidate Europe's leading position in computational
% mathematics and build on the remarkable success of the ecosystem of
% projects GAP, Python/Sage, Pari, Singular, LMFDB.

\subsection*{Description of the call}

\verbatiminput{call_description}

% \TODO{What do we mean by ``new generation''}.

\renewcommand{\thepage}{\arabic{page}}
\setcounter{page}{1}
\black
\cleardoublepage
\end{draft}

%%% Local Variables: 
%%% mode: latex
%%% TeX-master: "proposal"
%%% End: 

%  LocalWords:  verbatiminput renewcommand thepage setcounter cleardoublepage


% ---------------------------------------------------------------------------
%  Section 1: Excellence
% ---------------------------------------------------------------------------

\section{Excellence}

\subsection{Objectives}
\label{sect:objectives}


From their early days, computers have been used in pure mathematics,
either to prove theorems or, like the telescope for astronomers, to
explore new theories. Major achievements include the proof of the four
color theorem or ... Usage has grown to the point that certain areas
of mathematics now completely depend on experimental methods.



European mathematicians have been the early pioneers in the area, and
have grown a long tradition of collaborative open source software
development with systems like GAP, Singular, or Pari/GP playing a
major role for decades.

This project gathers developers of the leading mathematical software
in Europe and of key components, together with researchers in social
sciences, with mission to promote a new generation of
community-developed open source software, databases, and services
adapted to the needs of collaborative research in pure mathematics and
applications.

Our research will cover a wide variety of aspects, ranging from
software development models, user interfaces \TODO{virtual
  environments?}  deployment frameworks and novel collaborative tools,
component architecture, design, and standardization of software
\TODO{system?} and databases, to links to publication, data archival
and reproducibility of experiments, development models and tools, and
social aspects.

It will consolidate Europe's leading position in computational
mathematics and build on the remarkable success of the dynamic
Python/Sage ecosystem and its sister European projects (GAP, Pari,
Singular, LMFDB, ...).


\eucommentary{\emph{Describe the specific objectives for the project, 
which should be clear, measurable, realistic and achievable within the 
duration of the project. Objectives should be consistent with the expected 
exploitation and impact of the project (see section 2).}}

\TOWRITE{ALL}{This is an example of using TOWRITE command}

\TODO{This is an example of using TODO command}



\draftpage

\subsection{Relation to the Work Programme}

\eucommentary{
Indicate the work programme topic to which your proposal relates, and 
explain how your proposal addresses the specific challenge and scope 
of that topic, as set out in the work programme.}

\draftpage

\subsection{Concept and Approach}

\eucommentary{
-- Describe and explain the overall concept underpinning the project. 
Describe the main ideas, models or assumptions involved. Identify 
any trans-disciplinary considerations;\\
-- Describe any national or international research and innovation activities which will be
linked with the project, especially where the outputs from these will feed into the
project;\\
-- Describe and explain the overall approach and methodology, distinguishing, as
appropriate, activities indicated in the relevant section of the work programme, e.g.
Networking Activities, Service Activities and Joint Research Activities, as detailed in
the Part E of the Specific features for Research Infrastructures of the Horizon 2020
European Research Infrastructures (including e-Infrastructures) Work Programme 2014-
2015;\\
-- Describe how the Networking Activities will foster a culture of co-operation between the
participants and other relevant stakeholders.\\
-- Describe how the Service activities will offer access to state-of-the-art infrastructures,
high quality services, and will enable users to conduct excellent research.\\
-- Describe how the Joint Research Activities will contribute to quantitative and qualitative
improvements of the services provided by the infrastructures.\\
-- As per Part E of the Work Programme, where relevant, describe how the project will
share and use existing basic operations services (e.g. authorisation and accounting
systems, service registry, etc.) with other e-infrastructure providers and justify why such
services should be (re)developed if they already exist in other e-infrastructures. Describe
how the developed services will be discoverable on-line.\\
-- Where relevant, describe how sex and/or gender analysis is taken into account in the
project�s content.}

\draftpage

\subsection{Ambition}

\eucommentary{-- Describe the advance your proposal would provide beyond the 
state-of-the-art, and the extent the proposed work is ambitious. Your answer 
could refer to the ground-breaking nature of the objectives, concepts 
involved, issues and problems to be addressed, and approaches and methods to be used.\\
-- Describe the innovation potential which the proposal represents. Where relevant, refer to
products and services already available, e.g. in existing e-Infrastructures.}

\draftpage

% ---------------------------------------------------------------------------
%  Section 2: Impact
% ---------------------------------------------------------------------------

\section{Impact}
\label{sec:impact}

\subsection{Expected Impacts}

\eucommentary{Please be specific, and provide only information that applies 
to the proposal and its objectives. Wherever possible, use quantified 
indicators and targets.\\
Describe how your project will contribute to:\\
-- the expected impacts set out in the work programme, under the relevant topic
(including key performance indicators/metrics for monitoring results and impacts);\\
-- improving innovation capacity and the integration of new knowledge 
(strengthening the competitiveness and growth of companies by developing 
innovations meeting the needs of European and global markets; and, where 
relevant, by delivering such innovations to the markets;\\
-- any other environmental and socially important impacts (if not already 
covered above).\\
Describe any barriers/obstacles, and any framework conditions (such as 
regulation and standards), that may determine whether and to what extent 
the expected impacts will be achieved. (This should not include any risk 
factors concerning implementation, as covered in section 3.2.)}

\draftpage

\subsection{Measures to Maximise Impact}

\subsubsection{Dissemination and Exploitation of Results}
\label{subsubsect:dissemination}

\eucommentary{-- Provide a draft 'plan for the dissemination and exploitation 
of the project's results'. The plan, which should be proportionate to the 
scale of the project, should contain measures to be implemented both during 
and after the project.\\
Dissemination and exploitation measures should address the full range 
of potential users and uses including research, commercial, investment, 
social, environmental, policy making, setting standards, skills and 
educational training.\\
The approach to innovation should be as comprehensive as possible, 
and must be tailored to the specific technical, market and organisational 
issues to be addressed\\
-- Explain how the proposed measures will help to achieve the expected impact of the
project . Provide a draft business plan for financial sustainability as stated in the Part
E of the Specific features for Research Infrastructures of the Horizon 2020 European
Research Infrastructures (including e-Infrastructures) Work Programme 2014-2015.\\
-- Where relevant, include information on how the participants will 
manage the research data generated and/or collected during the 
project, in particular addressing the following issues: 
What types of data will the project generate/collect? What 
standards will be used? How will this data be exploited and/or 
shared/made accessible for verification and re-use (If data cannot 
be made available, explain why)? How will this data be curated and preserved?\\ \\
-- Include information about any open source software used or developed by the
project.\\
You will need an appropriate consortium agreement to manage (amongst other things) 
the ownership and access to key knowledge (IPR, data etc.). Where relevant, 
these will allow you, collectively and individually, to pursue market opportunities
arising from the project's results.\\
The appropriate structure of the consortium to support exploitation is addressed 
in section 3.3. \\ \\
-- Outline the strategy for knowledge management and protection. Include measures to
provide open access (free on-line access, such as the �green� or �gold� model) to
peer-reviewed scientific publications which might result from the project.\\
Open access publishing (also called 'gold' open access) means that an article is
immediately provided in open access mode by the scientific publisher. The associated costs
are usually shifted away from readers, and instead (for example) to the university or
research institute to which the researcher is affiliated, or to the funding agency supporting
the research.\\
Self-archiving (also called 'green' open access) means that the published article or the
final peer-reviewed manuscript is archived by the researcher - or a representative - in an
online repository before, after or alongside its publication. Access to this article is often -
but not necessarily - delayed (�embargo period�), as some scientific publishers may wish to
recoup their investment by selling subscriptions and charging pay-per-download/view fees
during an exclusivity period.}

\draftpage

\subsubsection{Communication activities}
\label{subsubsect:communication}

\eucommentary{Describe the proposed communication measures for promoting the 
project and its findings during the period of the grant. Where appropriate 
these measures should include social media and public events with user 
participation. Measures should be proportionate to the scale of the project, 
with clear objectives. They should be tailored to the needs of various audiences, 
including groups beyond the project's own community. Where relevant, include 
measures for public/societal engagement on issues related to the project.}

\clearpage

% ---------------------------------------------------------------------------
%  Section 3: Implementation
% ---------------------------------------------------------------------------



\section{Implementation}

\subsection{Work Plan --- Work packages, deliverables and milestones}
\label{sect:workplan}

\eucommentary{Please provide the following:\\
\begin{itemize}
\item
brief presentation of the overall structure of the work plan;
\item
timing of the different work packages and their components (Gantt chart or similar);
\item
detailed work description, i.e.:
\begin{itemize}
\item
a description of each work package (table 3.1a);
\item
a list of work packages (table 3.1b);
\item
a list of major deliverables (table 3.1c);
\end{itemize}
\item
graphical presentation of the components showing how they inter-relate (Pert chart or similar).
\end{itemize}
}

\subsubsection*{Overall Structure of the Work Plan}

The work plan is broken down into XX workpackages as shown
in Figure~\ref{}: WP2 deals with  ...
In addition, there is one management work package (WP1) and one
general dissemination work package (\ref{m}). The Gantt chart on
Page~\pageref{fig:gantt} illustrates the timeline for the
various tasks for these work packages, including inter-task
dependencies.

%\newpage
\subsubsection*{How the Work Packages will Achieve the Project Objectives}
\label{sssec:how_the_work_packages_will_achieve}

\TOWRITE{ALL}{This needs to explain that we're actually going to meet the 
objectives.  Needs to be done after objectives and WPs.}

The project objectives (Section~\ref{sect:objectives},
page~\pageref{sect:objectives}) and the corresponding work
packages that contribute to achieving those objectives are:

\begin{center}
\begin{tabular}{|l|l|l|}\hline
\textbf{Objective} & \textbf{Purpose} & \textbf{WPs} \\\hline \hline
Objective 1 & XX & \textbf{WPX} \\\hline
\end{tabular}
\end{center}

\paragraph*{Work Programme for Objective 1: }

Objective 1 is covered by WPX, which will ...

\landscape

\subsubsection*{Work Plan Timing: GANTT Chart showing Task Dependencies and Information Flows}


\vspace{-0.7in} \centerline{\hbox to \columnwidth{\hss%\includeimage[scale=0.85,angle=270]{ParaPhrase-Gantt2.pdf}
\hss}}
\label{fig:gantt}
\vspace{-1in} % Fool LaTeX into avoiding unnecessary page break
\endlandscape

\newpage

%\input{deliverables-dates}
%% Deliverables list.
%% Deliverables ordered by Workpackage
%% Workpackages are numbered automatically in sequence - the WP number has no effect

\workpackage{1}{Project Management}
\deliverable{mgt:mailinglists}
\deliverable{mgt:projectwebsite}
\deliverable{mgt:swrepository}
\deliverable{mgt:periodic-rep-1}
\deliverable{mgt:periodic-rep-2}
\deliverable{mgt:periodic-rep-3}
\deliverable{mgt:periodic-rep-4}
\deliverable{mgt:final-mgt-rep}
% Metrics: in PM and a bit in each work package

\workpackage{2}{Community Building and Engagement}
\deliverable{del:xx}

\workpackage{3}{Component Architecture}
\deliverable{del:xx}

%\workpackage{XXX}{Standardization} % => Component architecture + advertisement in the dissemination
%\deliverable{del:xx}

\workpackage{4}{User Interfaces}
\deliverable{del:xx}

\workpackage{5}{HPC and massively parallel components}
\deliverable{del:xx}

\workpackage{6}{Next generation mathematical databases}

\deliverable{del:xx}
%SL

\workpackage{7}{Social Aspects}
\deliverable{del:xx}
%UM
%\workpackage{XXX}{Development Models for an Academic Free Software Ecosystem}
%\workpackage{XXX}{Supporting the Mathematical Process} % => A chunk of Social Aspects


\workpackage{8}{Dissemination, Exploitation and Communication}
\deliverable{del:pressrelease} % Press release.
\deliverable{del:website} % Project presentation (web site). 
\deliverable{del:workshop1}  % Report on first project workshop, year 1. 
\deliverable{del:dissemplan1} % Final plan for using and disseminating knowledge.
\deliverable{del:workshop2}  % Report on second project workshop, year 2
\deliverable{del:workshop3}  % Report on third project workshop, year 3
\deliverable{del:dissemplan2} % Final plan for using and disseminating knowledge.


\addtocounter{subsubsection}{1}
\addcontentsline{toc}{subsubsection}{\protect\numberline{\thesubsubsection}Work
Package List}
\fbox{\begin{minipage}{\textwidth}\begin{center}{\Large\bf
        Work package list} % (full duration of project)}
  \end{center}
  \end{minipage}}

\bigskip\bigskip

\begin{tabular}{|p{1.2cm}|p{9.15cm}|p{0.8cm}|p{1.2cm}|p{1cm}|p{0.9cm}|p{0.9cm}|}
\hline
{\bf Work \mbox{package} No} & {\bf Work package title} &
{\bf Lead \mbox{partic.} no.} &
{\bf Lead short name} &
{\bf Person months} & {\bf Start month} & {\bf End month} \\\hline 

\newcounter{wp}

\addtocounter{wp}{1}
\workpackageentry{\thewp}{SA}{}{1}{60}

\addtocounter{wp}{1}
\workpackageentry{\thewp}{}{}{}{}

\addtocounter{wp}{1}
\workpackageentry{\thewp}{}{}{}{}

\addtocounter{wp}{1}
\workpackageentry{\thewp}{}{}{}{}

\addtocounter{wp}{1}
\workpackageentry{\thewp}{}{}{}{}

\addtocounter{wp}{1}
\workpackageentry{\thewp}{}{}{}{}

\addtocounter{wp}{1}
\workpackageentry{\thewp}{}{}{}{}

\addtocounter{wp}{1}
\workpackageentry{\thewp}{}{}{}{}

\addtocounter{wp}{1}
\workpackageentry{\thewp}{SA}{}{}{}

{\textbf{Total}} & & & &
\textbf{\large XXX}&
&
\\\hline
\end{tabular}


% \textbf{Summary:}\\[1ex]

\newpage

\fbox{\begin{minipage}{\textwidth}\begin{center}\Large\bf List of Deliverables
  \end{center}
  \end{minipage}}

\label{sect:deliverables}

\bigskip\bigskip\bigskip

\begin{minipage}{\textwidth}
\begin{center}
\begin{tabular}{|p{0.8cm}|p{8.75cm}|p{0.8cm}|p{1.2cm}|p{1.2cm}|p{1.2cm}|p{1.2cm}|}  \hline
\textbf{Del. no.}              & \textbf{Deliverable name}        & \textbf{WP no.} & \textbf{Lead}
& \textbf{Type}              & \textbf{Dissemi- nation level}   & \textbf{Delivery date}
\\ \hline

%% Year 1

\ref{del:xx}  & Requirements Analysis                            
& WP? & & R & CO &  ?? \\
\hline
\end{tabular}
\end{center}
\end{minipage}


\newpage

%% Set up the milestone numbers.
\eucommentary{Milestones means control points in the project that help to chart progress. Milestones may
correspond to the completion of a key deliverable, allowing the next phase of the work to begin.
They may also be needed at intermediary points so that, if problems have arisen, corrective
measures can be taken. A milestone may be a critical decision point in the project where, for
example, the consortium must decide which of several technologies to adopt for further
development.}

The work in the \TheProject project is structured by four milestones, which could be
briefly characterised as: starting up and building prototypes; moving from prototypes to
fully functional implementations; further engagement with the community and producing
research outputs; evaluation and final releases. They coincide with the project meetings
held at the end of each year of the project (four other meetings will be held in the
middle of each year).  Given the nature of the project, with a large number of essentially
independent tasks, there is no need for milestones attached to specific collections of
tasks or deliverables.  Given that the meetings are the main face-to-face interaction
points in the project, we have chosen to schedule the milestones for these events, where
they can be discussed in detail, tracking the progress in each work package through status
reports on the tasks and deliverables and take corrective measures, where necessary, and
critical decisions regarding further plans.  We envisage that this setup will give the
project the vital coherence in spite of the broad interdisciplinary mix of various
backgrounds of the participants.

\paragraph{General Milestones}

\begin{milestones}
  \milestone[id=startup,month=12,
  verif={Completed all corresponding deliverables and reported the progress in the 2nd Project meeting report.}]
  {Startup}
  {By Milestone 1 we will have carried out the requirements study, design and prototype implementations and started community building activities.}

  \milestone[id=proto1,month=24,
  verif={Completed all corresponding deliverables and reported the progress in the 4th Project meeting report.}]
  {Implementations}
  {By Milestone 2 we will have constructed first fully functional interface implementations and released enhanced versions of \TheProject components, and train early adopters of \TheProject.}

  \milestone[id=community,month=36,
  verif={Completed all corresponding deliverables and reported the progress in the 6th Project meeting report.}]
  {Community/ Experiments}
  {By Milestone 3 we will have gathered and evaluated feedback on \TheProject software and established the portfolio of experiments produced with \TheProject through further engaging with the community.}

  \milestone[id=eval,month=48,
  verif={Completed all corresponding deliverables and reported the progress in the 8th Project meeting report.}]
  {Evaluation}
  {By Milestone 4 we will have released final versions of all \TheProject components and completed the project evaluation.}
\end{milestones}

\paragraph{Milestone for WP 3}
We propose 1 milestone:

\begin{milestones}
%original delivery date proposal is M36 but milestone is linked to D3.10 which is planned for M48...
	\milestone[id=WP3availability,month=42,
	 verif={Have \ODK's components available on major platforms}]
	 {Work Package 3 aims at deploying all computational components
	 developed by \ODK available on the three major platforms (i.e.
	 Windows, Mac, Linux) via their standard distribution channels.}
\end{milestones}

\paragraph{Milestones for WP 4}
We propose two milestones:

\begin{milestones}
  \milestone[id=WP4prototype,month=36,
    verif={Prototype VRE for mathematical researchers}]
  {Prototype VRE for mathematical researchers}
  {
  % note: delivref doesn't work here
  User story: A group of mathematical researchers with access to
  common computational resources, such as a shared lab computer or
  cloud servers, shall be able to deploy a prototype VRE with
  \JupyterHub, integrating \ODK components.
  The Jupyter kernels for mathematical software developed as part of \ODK
  make computational mathematical components accessible in a \Jupyter
  environment, enabling a Jupyter-based deployment of the relevant
  tools for the researchers.
  The process of working on notebooks is greatly improved by review tools
  developed as part of WP4,
  enabling researchers to collaborate to some degree
  in a shared computational environment.
  }
  \milestone[id=WP4collaborative,month=48,
  verif={Collaborative VRE for mathematical researchers}]
  {Collaborative VRE for mathematical researchers}
  {
  The prototype VRE shall be extended with improved ease of deployment, new
  functionality such as interactive 3D visualization and real-time
  collaboration, enabling researchers to collaborate productively in a shared
  computational environment. Finally, integrating notebooks and semantic
  knowledge into a publication / knowledge system enable a continuous process
  of leveraging \ODK components from research to publication.
  }
\end{milestones}

\paragraph{Milestones for WP 6}

\begin{milestones}
  \milestone[id=WP6interop1,month=36,
  verif={Demonstrator Online Public, works on selected case study examples}]
  {First MitM-based interoperability prototype (GAP, SageMath, LMFDB)}
  {We intend to present a fully functional prototype of the integration of at least the
    systems GAP, SageMath, and LMFDB via the SCSCP Protocol at the second review 
    meeting. This prototype will be the basis for additional integration work for 
    additional systems and the use interface from WP4.}
\milestone[id=WP6interop2,month=42,   verif={Demonstrator Online Public, works on selected case study examples}]
  {Second MitM-based interoperability prototype}
  {The goal of this milestone is to take into account all the operational 
    experiences with the first prototype and add more systems and integrate some
    of the UI components from The experiences with the preparation of 
    this prototype will allow us to estimate the joining costs of adding a system 
    to the OpenDreamKit VRE toolkit, which is an important measure of the 
    flexibility of the MitM approach.}
\end{milestones}

%%% Local Variables:
%%% mode: latex
%%% TeX-master: "proposal"
%%% End:

%  LocalWords:  verif ldots


\fbox{\begin{minipage}{\textwidth}\begin{center}\Large\bf List of milestones
  \end{center}
  \end{minipage}}
\label{sect:milestones}

\bigskip\bigskip\bigskip

\begin{minipage}{\textwidth}
\begin{center}
\begin{tabular*}{\textwidth}{|p{1.5cm}|p{6.7cm}|p{2.5cm}|p{1.5cm}|p{3.6cm}|}  \hline
\textbf{Milestone number} & \textbf{Milestone name} & \textbf{Related work
  package(s)} & \textbf{Estimated date} & \textbf{Means of
  verification} (deliverables shown here + success criteria below) \\
\hline
\ref{mil:initial} &
  Completed initial requirements analysis.  &
  WPX &
  1 &
\ref{del:requirements-analysis}.
\\
\ref{mil:final} &
&
WPX &
&
\\
\hline
\end{tabular*}
\end{center}
\end{minipage}

\vspace{10pt}
\begin{center}
\begin{tabular*}{\textwidth}{|p{1.5cm}|p{13.3cm}|p{1.9cm}|}\hline
\textbf{Milestone} & \textbf{Success Criteria} & \textbf{Contributes to
  Objective(s)} \\\hline
\ref{mil:initial} &
Completed requirements analysis (Deliverable~\ref{del:requirements-analysis}). &
 \textbf{1, 3.}
\\
\ref{mil:final} & 
XX
& \textbf{XX}
\\\hline
\end{tabular*}
\end{center}


% ---------------------------------------------------------------------------
% Include Workpackage descriptions
% ---------------------------------------------------------------------------

%% WP titles and order are defined in deliverables.tex
%%% Workpackage style may be broken -- fix this!!

%% Local WP number counter - should possibly be global and hidden?

\newcounter{wpno}

\addtocounter{wpno}{1}

\begin{Workpackage}{\thewpno}
\WPTitle{\wpname{\thewpno}}
\WPStart{Month 1}
\WPParticipant{SA}{48}
\WPParticipant{PS}{48}

\begin{WPObjectives}
  The objective of \theWP{} is to undertake all project management
  activities, including setting up joint infrastructure, organizing
  meetings, and producing overview reports.
\end{WPObjectives}

\begin{WPDescription}
This workpackage will perform ...
\end{WPDescription}

\begin{WPDeliverables}
\begin{itemize}
\item
\ref{mgt:mailinglists}
(Month 1): 
Internal and external mailing lists.
\item
\ref{mgt:swrepository}
(Month 1): 
Internal software repository.\TODO{Needed?}
\item
\ref{mgt:periodic-rep-1}
(Month 12): 
Project Periodic Report (first year).
\item
\ref{mgt:periodic-rep-2}
 (Month 24): 
Project Periodic Report (second year).
\item
\ref{mgt:periodic-rep-3}
(Month 36): 
Project Periodic Report (third year).
\item
\ref{mgt:periodic-rep-4}
(Month 48): 
Project Periodic Report (fourth year).
\item
\ref{mgt:final-mgt-rep}
(Month 48): 
Project Final Report
\end{itemize}
\end{WPDeliverables}
\end{Workpackage}

\addtocounter{wpno}{1}
\begin{Workpackage}{\thewpno}
\wplabel{wp:x}
\WPTitle{\wpname{\thewpno}}
\WPStart{Month 1}
\WPParticipant{SA}{1}

\begin{WPObjectives}
  The objective of \theWP{} is to further develop the community at the
  European scale, foster cross teams collaborations, spread the
  expertise, and engage the greater community to participate to the
  definition of the needs, and the implementation and use of the
  produced solutions.
% \begin{itemize}
% \item
% \item
% \item
% \item
% \item
% \end{itemize}
\end{WPObjectives}

\begin{WPDescription}
  We will organize regular open workshops (e.g. Sage Days, Pari Days,
  summer schools, etc.); some of them will be focused on development
  and coding sprints, and others on training.

  This work package will also provide general travel budget to fund
  short to long term visits between the participants, to collaborate
  on specific features. A typical such visit would bring together an
  IPython developer with a GAP developer for a coupe day to implement
  a first prototype of notebook interface to GAP.

  This work package will complement and lean on a parallel COST
  network whose role is to build and animate the greater community.
\end{WPDescription}

\begin{WPDeliverables}
\begin{itemize}
\item
\ref{del:x}
(Month 12): Report on community needs
\item Workshop 1 ...
\item Workshop 2 ...
\item Workshop 3 ...
\item \TODO{make a list}
\end{itemize}
\end{WPDeliverables}
\end{Workpackage}

\addtocounter{wpno}{1}
\begin{Workpackage}{\thewpno}
\wplabel{wp:x}
\WPTitle{\wpname{\thewpno}}
\WPStart{Month 1}
\WPParticipant{SA}{1}

\begin{WPObjectives}
  The objective of \theWP{} is to produce periodic reviews of relevant
  developments elsewhere and implications for our plans, including
  negotiating access or shared development when appropriate.

  It will feed this information to the other work packages, in
  particular Component Architecture
\end{WPObjectives}

\begin{WPDescription}
This workpackage  ...
\end{WPDescription}

\begin{WPDeliverables}
\begin{itemize}
\item
\ref{et:periodic-rep-1}
(Month 12): Year 1 report.
\item
\ref{et:periodic-rep-2}
(Month 12): Year 2 report.
\item
\ref{et:periodic-rep-3}
(Month 12): Year 3 report.
\item
\ref{et:periodic-rep-4}
(Month 12): Year 4 report.
\end{itemize}
\end{WPDeliverables}
\end{Workpackage}

%\addtocounter{wpno}{1}
\begin{Workpackage}{\thewpno}
\wplabel{wp:x}
\WPTitle{\wpname{\thewpno}}
\WPStart{Month 1}
\WPParticipant{SA}{1}

\begin{WPObjectives}
  The objective of this work package is to develop and demonstrate a
  set of API's enabling components such as database interfaces,
  computational modules, separate systems such as GAP or Sage to be
  flexibly combined and run smoothly across a wide range of
  environments (cloud, local, server, ...).
\end{WPObjectives}

\begin{WPDescription}
  This work package includes work on:
  \begin{itemize}
  \item Portability:
    \begin{itemize}
    % Jean-Pierre:
    % Should we mention port to non-x86_64 archs and non-Linuces?
    %
    % For CPUs:
    % - I guess at least ARM and ppc64 (IBM POWER*) really make sense.
    % - Sparc is less convincing though the latest sparc CPUs
    % are muche more interesting for math computation as the
    % previous ones, e.g. the GMP folk specifically added assembly
    % for them in their latest release.
    % - Itanium is dead, but it can help discovering bugs as any non
    % standard archs.
    % - Supporting any of these would mean buying (potentially very
    % expensive) hardware.
    %
    % For OSes?
    % - Should we mention OS X which is a pain at each new release?
    % - A BSD variant would be interesting, let's say FreeBSD which
    % is basically (almost) already supported
    % - Solaris? and/or OpenIndiana? Interesting if we mention sparc...
    % - Windows is already included below, my opinion is:
    %  * provide live USB, VMs and Cygwin32 first as these three are
    %  basically already working solutions
    %  * go Cygwin64 as it is still POSIX
    %  * explorate a MinGW solution, at least GAP and PARI should be
    %  problematic
    %  * try to use MSVC
    \item Sharing experience and best practices.
    \item Port to Windows (GAP, Sage, Singular).
    \item Shared multiplatform test infrastructure.
    \end{itemize}

  \item Interfaces between systems:
    \begin{itemize}
    \item Self adaptation to the environment, better schemes for
      automatically selecting appropriate algorithms / components for
      a given task.
    \item Semantic-enabled handles to objects stored in other systems (NT):\\

      Handles are a popular design pattern for interfaces between two
      systems A and B; instead of exchanging objects back and forth,
      only handles to those objects are exchanged, letting e.g. A
      manipulate an object which actually resides in B. Typical
      features include remote method calls, introspection, or
      documentation queries. The next step would be for A to be aware
      of the semantic of the object, using an adapter infrastructure
      to propagate category/ontologies information. For example, we
      would want GAP's categories to be mapped to Sage's categories,
      so that a handle to a GAP group would automatically appear
      within Sage like a native Sage group.
    \end{itemize}

  \item Modularization
    \begin{itemize}
    \item common architecture for module maintenance and
      distribution (related to point 1 above)
    \item Sharing experience and best practices
    \item Modularization of Sage
    \item Refactorization of GAP's package mechanism; namespaces?
    \end{itemize}

  \item Deployment and distribution

  \item High Performance Computing and Parallelism:\\
    As in all other areas of science, properly supporting of massively
    parallel architecture is a major challenge.

    Many of the computational components have already gone a long way
    in this direction. For example, grant \TODO{...} founded the
    GAP-HPC project which adapted the GAP kernel to support HPC. The
    expertise gained there was then transferred to the ongoing
    Singular-HPC project.

    Building on this, this project will:
    \begin{itemize}
    \item Foster further sharing of HPC expertise and best practices
      between computational components.
    \item Develop novel infrastructure for HPC in the context of
      combinatorics.
    \item Investigate and implement HPC-friendly ways of combining
      components together, so that calling components can benefit from
      the HPC features of called components, with self-adaptation to
      the environment and cooperative sharing of resources.
    \item Support work on HPC-enabling more components (Linbox)
    \item Investigate 
    \end{itemize}
  \end{itemize}
\end{WPDescription}

\begin{WPDeliverables}
\begin{itemize}

%%%%%%%%%%%%%%%%%%%%%%%%%%%%%%%%%%%%%%%%%%%%%%%%%%%%%%%%%%%%%%%%%%%%%%%%%%%%%%
% Deliverables: portability and distribution

\item \ref{del:distribution} Make sure that Sage and therefore all the
  components it depends on (including GAP, Linbox, Pari, Singular,
  ...)  have standard packages in the main Linux distributions:
  Debian/Ubuntu, Redhat, Gentoo, ...

  \TODO{Get feedback from our experts, and make this precise; what can
    we actually promise to achieve? how much work is this? Do we have
    personnel for this?  There is strong expertise in Logilab with a
    Debian developer working there; he could advise someone on
    this. Logilab is interested in this because it's meeting similar
    issues with some of its clients software like Salomé.}

x\item \ref{del:virtual_machines} (Month 12): Creation, deployment, and
  distribution of preconfigured virtual machines for Pari, Sage,
  ... as a cloud service, in particular within the StratusLab
  infrastructure. This includes build bots and test bots for
  continuous integration over a variety of operating systems.
  % Requires: licenses

\item \ref{del:portability_cygwin} (Month 12, Month 24): Fully
  functional one-click install Sage distribution for Windows using a
  32bits version of Cygwin.
  % JPF: this should take a few months of work

  This 32bits version would work right away on Windows 64 bits with
  Cygwin 32 bits; more work would be required for a version working on
  a 64 bits of Cygwin.
  % JPF: I agree.

  In both cases, this includes complete port of Singular, GAP, Pari,
  ...  to cygwin.

  % Participants involved: Paris Sud, Kaiserslautern, Saint Andrews, Bordeaux


  % Comments on this by Bill Hart
  % The big problems you will have on Windows 64 on Cygwin include:
  %
  % * anything with assembly language -- the ABI is different on Windows, so
  % it'll need rewriting, or you can incur a performance penalty by using
  % generic C fallback code
  % * the memory allocator on Windows is not so great
  % * bugs exposed due to being on a different platform, e.g. segfaults due to
  % off-by-one errors that were masked by the granularity of malloc on Linux
  % * build issues, due to identifying Cygwin and using the correct header
  % files, which are often different on Cygwin than linux
  % * issues with PATH vs LD_LIBRARY_PATH
  % * Windows has a case insensitive file system
  % * EOL issues
  % * Windows is not able to rapidly create and delete files, which some
  % libraries (esp. test code) calls for
  % * memory limitations (many people using Windows are using laptops with
  % limited memory, only a portion of which is realistically available to
  % Cygwin)
  % * autotools versions that don't support Windows (usually autotools has a
  % release that is used in all the distributions, which doesn't work correctly
  % on Windows, and this is followed up by a version which has all the Windows
  % patches)
  % * building takes forever on Windows. Mingw2 has now gotten parallel build
  % working on Windows and the speed is within a factor of 5 of Linux. But I'm
  % not sure the improvements have propagated to Cygwin yet.
  % * Cygwin 64 is new, contains quite a few bugs still, and things keep
  % changing with every version as they try to get things right.
  % * Although projects will likely accept patches for Windows, they are less
  % likely to maintain support themselves. I would like to think Singular would
  % be an exception to this. And obviously flint and MPIR work on Windows (even
  % with MSVC as of the next version of flint -- or now if you use our bleeding
  % edge repo version).
  %
  % Comments by Jean-Pierre on some of the above and mor:
  % * first things first: I already completely built Sage on Cygwin64, though it
  % was surely not completely functional.
  % * assembly: that's right, note that as far as Sage and it's dependencies are
  % concerned, only a few of them actually use assembler, and yes all of them
  % provide fallback generic C code IIRC
  % * PATH vs LD_...: basically the same problem as for Cygwin32, so it's already
  % been taken care of for the Cygwin32 port
  % case issue: not a problem IIRC
  % * EOL issues: I don't thing so, Cygwin is POSIX like
  % * autotools issues: most of Sage dependencies are now updated, I used to track
  % the few problematic ones in 2013
  % * time to build: not so long, sure longer than on a POWER7 machine, but I do
  % it on a usual x86_64 laptop running Debian within a Windows VM in a few hours!
  % what we actually really need is patch/build bots to test on Cygwin 32/64!
  % * upstream cooperation: I agree Windows is often a low priority issue, but
  % most teams have welcomed my Cygwin patches

\item \ref{del:modularization} Modularization of the Sage distribution

  Separation of the different components of Sage (communication with
  third-party softwares, build system, Sage native code). This is a
  prerequisite for easier packaging and integration in standard Linux
  distributions and lmonade, native integration within the IPython
  notebook and other interfaces (larchenv, Spyder, ...) and
  collaboration with sister projects.

%\TODO{lmonade has very similar objectives but uses the gentoo prefix whereas Linux distributions use very different packaging systems:
%\begin{itemize}
%\item gentoo prefix (gentoo)
%\item pacman (arch),
%\item yum (redhat),
%\item apt (debian),
%\item easy\_install
%\item Python index packaging (pip)
%\end{itemize}}

%%%%%%%%%%%%%%%%%%%%%%%%%%%%%%%%%%%%%%%%%%%%%%%%%%%%%%%%%%%%%%%%%%%%%%%%%%%%%%
% Deliverables: Interfaces

\item \ref{del:scscp_sage} Add support for the
  \href{http://www.symbolic-computing.org/}{SCSCP} interface protocol
  to all relevant components (e.g. Sage, ...).
  \TOWRITE{SL/AK}{Brief description of what SCSCP is, reference to
    previous grant, relevance to the goals of this grant; maybe this
    should go in the work package description}

\item Some IPython/Jupyter deliverables here.
  \TODO{review what it can already do in term of choice of
    computational resource and storage back-end.}

\item Contribution by Kaiserslautern: libSingular, pySingular?,
  GAP-Singular, Singular-Sage.

  Moving code from Sage into Singular when relevant

%%%%%%%%%%%%%%%%%%%%%%%%%%%%%%%%%%%%%%%%%%%%%%%%%%%%%%%%%%%%%%%%%%%%%%%%%%%%%%
% Deliverables: HPC

\item \ref{del:hpc_configure} (Month ...) Configure the components of
  Sage's distribution (e.g. Atlas, Linbox, GAP, Singular, ...) to be
  systematically HPC-enabled, and make sure that Sage's calls to such
  components indeed enable HPC.

\item \ref{del:hpc_components}
  Develop HPC features in components
  \begin{itemize}
  \item \TOWRITE{JGD}{Linbox}
  \item \TOWRITE{WD}{Singular}
  \end{itemize}

%%%%%%%%%%%%%%%%%%%%%%%%%%%%%%%%%%%%%%%%%%%%%%%%%%%%%%%%%%%%%%%%%%%%%%%%%%%%%%
% Deliverables: to be sorted ...

\item Transparent integration of Ipython capabilities for cluster computing.
\item Implementation of a transparent abstraction over mpi.
\item Develop or integrate existing solutions for MapReduce operations
  over big data.

\item FLINT development (key component for several systems)?

\item Some demonstrators of cross-disciplinary/cross-software calculations

\end{itemize}
\end{WPDeliverables}
\begin{verbatim}
Raw material:

Component Architecture
----------------------

Recomputation connection belongs here?

Collaboration with unreliable (or restricted!) networking connections
(peer-to-peer, opportunistic syncing, 3rd world). This is technically
interesting, and gets in support for non-networked working. Not sure
if it belongs here or not.

- Security concerns

Goal: Fostering collaborations/integration between components in an open source ecosystem
=============================================================================

- How to make systems "cooperate" rather than "predate each other".
- E.g. reduce the version issues

- Foster collaboration with upstream libraries by sharing the
  development and maintenance of the interfaces, typically as
  standalone upstream Python bindings (e.g. py-Singular).

- How to make it easy to develop simultaneously two interdependent
  components (e.g. Sage+Singular)

- Foster communication

- Social aspect:
  Credit, Citations, Recognition, Funding

Documentation system
====================

In which package?

Improvements to Sphinx

Sage heavily customizes the Sphinx documentation system, hacking deep
in it in some cases, with quite some duplication in some cases.
Refactor the whole thing, generalizing and contributing back upstream
as much as possible (e.g. parallel compilation).
\end{verbatim}

\end{Workpackage}

%\input{WPs/Standardization}
%\TOWRITE{ALL}{Proofread WP 4 User Interfaces pass 2}
\begin{draft}
%\begin{verbatim}
%- [ ] do all tasks list all sites involved in them?
%- [ ] does the table of sites and their PM efforts match lists of sites for each task?
%      (each site from the table is listed in all relevant tasks, and no site is listed only in the table or only at some task)
%\end{verbatim}
%fixed: \TODO{D4.14 and D4.15 are not referenced in any task}
\end{draft}

\begin{workpackage}[id=UI,wphases=0-48,
  title=User Interfaces,
  lead=SR,
  PSRM=12,  % Sage-Jupyter interface, sphinx documentation dynamic documentation and exploration system
  UVRM=2,   % Sage-Jupyter interface
  JURM=4,  % Jacobs: active documents
  FAURM=8, % active documents
  USHRM=7, % Supporting reproducible data science and sharing of models
  LLRM=12, % Help on several computer-centered tasks, dynamic SparQL in notebooks
  SARM=18, % GAP
  UKRM=2, % Singular
  UBRM=26,  % Pari
  USORM=11, % Southampton, micromagnetic VRE some contribution (% month) to 3d visualisation
  XFELRM=5, % taking over from Southampton
  SRRM=28,
  UGRM=14,
  USRM=4, % University of Silesia, 3d without subcontracting
  LEEDSRM=1, %real PM for Leeds is 0.64
  swsites]    % rotate partner logos so that table fits on page.

\begin{wpobjectives}
  The objective of this work package is to provide modern, robust,
  and flexible user interfaces for computation, supporting real-time
  sharing, integration with collaborative problem-solving,
  multilingual documents, paper writing and publication, links to
  databases, etc.
\end{wpobjectives}

\begin{wpdescription}
  Project \Jupyter (formerly \IPython notebook) provides a browser
  based approach to constructing executable documents which comprise
  of code (in multiple languages), mathematics, text, and diagrams (see
  Section~\ref{sec:jupyter}). The
  framework is an ideal portal through which \VREs can be operated. In
  this work package, we will add new functionality to the \Jupyter
  notebook that fosters excellence in computational science and
  research. In particular, we will push towards reproducible and
  effective science by allowing structured documents (such as reports,
  books, theses) from notebooks, and by allowing those notebooks to be
  re-executed as self-contained regression tests. We will unify the
  notebook infrastructure used in \Sage with \Jupyter, push forward
  dynamic documentation exploration capabilities, and work towards
  concurrent multi-user editing of notebooks. We will also develop
  exemplar \Jupyter notebooks for education and research
  (e.g. \taskref{dissem}{ibook}).

  To demonstrate the power of the \TheProject environment to
  accelerate computational science, deliver better value for money and
  make computational science more robust, we will put together a
  micromagnetic
  \VRE(\ref{sec:introduction-micromagnetic-vre-demonstrator}) as a
  demonstrator.

\end{wpdescription}

\begin{tasklist}
\begin{task}[title=Uniform notebook interface for all interactive
  components,id=ipython-kernels,lead=PS,
  partners={SR,UK,USH,USO,LL,SA,UV,UG,XFEL},
  PM=24, wphases=0-36,issue=69]
  In this task, we will implement \Jupyter interfaces for the
  interactive computation components of \TheProject, including \GAP,
  \PariGP, \Sage, and Singular. A first release
  \localdelivref{ipython-kernels-basic} will focus on basic functionality,
  and a second release \localdelivref{ipython-kernels} will cover advanced
  features like 3D graphics or transparent documentation browsing (as
  live worksheets whenever relevant).

  % Note from William: my student Andrew Ohana just mostly did
  % something like that for IPython, but then stopped.  Anyway, it's
  % very do-able based on a summer project from another student and a
  % bunch of work I did with THREE.js for SMC.

  One of our objectives is to ensure the sustainability of the project
  (Objective~\ref{objective:sustainable}). The current \Sage notebook
  interface was developed alongside that of \Jupyter, but with
  slightly different goals. A notebook interface for \Sage is a vital
  integrative component, and development was fast tracked to ensure
  its availability to allow the project to move forward. However,
  \Jupyter, whilst it initially proceeded more slowly, has a larger
  developer base and has now caught up with the \Sage notebook in
  terms of functionality. In line with
  Objective~\ref{objective:sustainable} \Sage will now phase out its
  own notebook and switch focus to the \Jupyter notebook, outsourcing
  this key but non disciplinary component.

  % In charge: Jupyter dev + dev in Orsay + community?
  The \Sage and \Jupyter convergence \localdelivref{ipython-kernel-sage} will
  require:
  \begin{compactitem}
  \item Robust migration path and tools for \Sage worksheets,
  \item Support for math, 2D, and interactive 3D scene visualisation,
    % \item Bundling of the \Jupyter notebook and its dependencies within
    %   the Sage distribution. DONE
  \item Import and export of ReST documents, with full support for
    \Sage's specific roles (math, ...),
  \item Support for remote \Sage kernel, typically on the cloud, or
    running with a different Python version (\Sage as a library),
  \item A migration path for interactive widgets implemented with
    \Sage's \texttt{@interact} functionality.
  \end{compactitem}

  Joint meetings and visits between the developers of \Jupyter and of
  the computing components will be a key component of this task.

\end{task}

\begin{task}[id=notebook-collab,title=Notebook improvements for collaboration,lead=SR, partners={PS,USH,JU,FAU,USO,LL}, PM=20, wphases=0-48, issue=70]
  In this task, we will further improve tools for collaboration
  between authors of shared \Jupyter notebooks and draw from the
  experience of collaboration as set in Simulagora, SageMathCloud,
  etc.

  Version control tools, such as Git and Mercurial, have become an
  integral part of open and collaborative science and
  software. Version control tools allow proposed changes to be
  reviewed (`diffing') and resolve conflicts through combination of
  changes (`merging'). \Jupyter notebook documents are stored in text
  files as JSON formatted data. This makes them well suited to
  tracking in version control, but the JSON structure can make diffing
  and merging difficult. We will deploy tools to provide better
  support for visual diffing and merging of Notebook documents. These
  tools will be integrated into existing version control workflows
  \localdelivref{jupyter-collab}. The MathHub.info system already has
  a distributed Git-based versioning system, which can serve as an
  entry point here.

  Given the interactive nature of \Jupyter notebooks, live
  collaboration, where multiple authors work on the document
  simultaneously (like in Google Docs), is particularly
  desirable. However, there are particular challenges for
  collaborative editing of \emph{executable} documents. The potential
  for \emph{shared execution} adds both value and challenge to the
  live collaboration. Some attempts have been made to deal with live
  collaborative sessions (e.g. \SMC, Colaboratory) but so far these
  have been outside the core \Jupyter project. In this task we will
  explore different models of single-notebook collaboration, including
  shared or separate execution \localdelivref{jupyter-collab}. We will
  consider not only indicating authorship, but which author
  triggered which execution, and explore other challenges.  Various
  avenues for live session collaboration will be explored for
  integration into \Jupyter itself
  \localdelivref{jupyter-live-collab}.
  
  These tools will continue to be developed throughout the project
  beyond the deliverable dates, as they are adopted by the community.
\end{task}

\begin{task}[id=notebook-verification,title=Reproducible Notebooks,lead=SR, partners={PS,USO}, PM=4, wphases=12-24, issue=71]
  In this task, we will develop tools that allow re-execution
  notebook documents with automated regression testing. The computed
  output will be compared against the stored output, and deviations
  reported as assertion errors.

  Notebooks are used in a variety of contexts, like training and
  teaching material (tutorials, documentation, books) or computer
  experimentation logbooks, where reproducibility is
  critical. Reproducibility dictates that the notebooks should remain
  functional and correct in the long run, even when the underlying
  computational software or infrastructure changes over time or across
  platforms.

  This task is a critical component of reproducibility, allowing the
  notebook author to get an immediate notice when, e.g., a backward
  incompatible change occurs. It becomes even possible to anticipate
  such situations upstream by including important notebooks directly
  in the automated test suite of the computational software, giving an
  easy way for casual users to contribute regression tests.

  Technically speaking, \Jupyter notebooks store outputs as rich
  mime-type structures, with JSON metadata. Using this metadata, it
  will be possible to express expectations of output, allowing more
  flexible and powerful tests than direct text comparison
  \localdelivref{jupyter-test}.  Prior work has been done in \Sage for
  ReST files, e.g. \lstinline{sage -t notebook.rst}, and this model
  will be extended to notebooks.
\end{task}

\begin{task}[id=sage-sphinx, title=Refactor \Sage's \Sphinx documentation system, lead=PS,PM=6, partners={SR,UV,UG}, wphases=0-36, issue=72]
  \Sage, like \Python and many other \Python based projects, uses the
  \Sphinx documentation system. Due to particularly stringent needs,
  many layers of customisation and adaptations have accumulated over
  the years, in particular for proper scaling to the sheer size of the
  Sage documentation (13k pages just for the reference manual).

  A deep refactorisation (\localdelivref{sage-sphinx}) is critically
  needed to get rid of multiple duplication, and foster sustainability
  by outsourcing back to \Sphinx all generic aspects (parallel
  compilation, index generation, ...).
  \TOWRITE{VP}{Viviane, this seems a little short, can we provide a little more detail of what the refactorisation will involve?}
  % In charge: dev in Orsay or Logilab + visit of Sphinx dev  + FH
\end{task}

\begin{task}[id=dynamic-inspect,title=Dynamic documentation and exploration system,lead=PS, partners={SR,USO,UV,LL,UG}, PM=6, wphases=0-36, issue=73]
  Introspection has become a critical tool in interactive computation,
  allowing user to explore, on the fly, the properties and
  capabilities of the objects under manipulation. This challenge
  becomes particularly acute in systems like \Sage where large parts
  of the class hierarchy is built dynamically, and static
  documentation builders like \Sphinx cannot anymore render all the
  available information.

  In this task, we will investigate how to further enhance the user
  experience. This will include:
  \begin{compactitem}
  \item On the fly generation of Javadoc style documentation, through
    introspection, allowing e.g. the exploration of the class
    hierarchy, available methods, etc.
  \item Widgets based on the HTML5 and web component standards to display
    graphical views of the results of SPARQL queries, as well as populating data
    structures with the results of such queries,
  \item \localdelivref{ipython-advanced-interacts} (Month 36)
    Exploratory support for semantic-aware interactive widgets
    providing views on objects of the underlying computational or
    database components. Preliminary steps are demonstrated in the
    \texttt{Larch Environment} project (see demo video on
    \url{http://www.larchenvironment.com/}) and
    \software{sage-explorer}
    (\url{https://github.com/jbandlow/sage-explorer}). The ultimate
    aim would be to automatically generate \LMFDB-style interfaces.
  \end{compactitem}
  Whenever possible, those features will be implemented generically
  for any computation kernel by extending the \Jupyter protocol with
  introspection and documentation queries.
  % In charge: \Jupyter dev + dev in Orsay + NT?
\end{task}

\begin{task}[title=Structured documents,id=structdocs,
  lead=JU,PM=22,partners={SR,USH,LL,FAU},wphases=0-24,issue=74]
  \Jupyter notebooks consist of a sequence of cells that contain
  either text or a program (see Section~\ref{sec:jupyter}). Complex
  documents, such as books, articles or reports, require a richer
  description that covers the the structure of the document and the
  semantics of its elements. This task will investigate this problem
  and try to find a way to write these documents exploiting the
  breakthroughs achieved in the other tasks to this workpackage.

  Several technical complementary options can be explored:
  \begin{compactitem}
  \item MathHub.info is a portal for reading and interacting with
    ``active documents'' (i.e. documents that have an additional
    semantic layer that supports semantic services like definition
    lookup, type-inference, unit conversion,\ldots)
  \item \Jupyter notebooks are essentially ``programs with documentation'' and lack the
    semantical structure needed by complex documents.
  \item sTeX is a semantic variant of LaTeX that can be transformed into OMDoc/MMT, which
    is the native knowledge representation format for active documents and
    machine-actionable knowledge about math and symbolic programs.
  \end{compactitem}

  After gathering the needs and the requirements for the writing of
  complex documents in the mathematical field, we will study these
  designs and build a solution that meets the expectations
  (\localdelivref{adstex}). The implementation will be achieved
  through an iterative process that incrementally improves existing
  software solutions, making them interoperable and synergistic.
  Results of this convergence will be reported
  in~\localdelivref{adcomp}, \localdelivref{ipython-kernel-sage} and
  \localdelivref{jupyter-import} and used in \taskref{dissem}{ibook}.
\end{task}

\begin{task}[id=mathhub,title=Active Documents Portal,lead=FAU,PM=12,partners={JU},
  wphases=12-36!.5,issue=75]
  We will extend the existing \url{http://mathhub.info} system to a
  portal for interacting with active/structured documents (see
  \localtaskref{structdocs}) and releasing the portal initially for
  internal use in the \TheProject and later for general
  use. \url{MathHub.info} already provides very basic sTeX editing and
  versioning. In \TheProject we will extend it on the computational
  side based on the integrated format from
  \localtaskref{structdocs}. The resulting portal will be made
  available to the consortium as~\localdelivref{mathhub-editing} and
  would be used for semantically enhanced code documentation and
  knowledge representation (see \WPref{dksbases}).
\end{task}

\begin{task}[title=Visualisation system for 3D data in web-notebook
,id=vis3d,lead=SR, partners={US,PS,USO}, PM=13, wphases=0-24, issue=76]
\TOWRITE{MRK,HPL}{wphases does not agree with PM. (13 vs 24}
%12 months from Simular,
% 1 month from Southampton for testing in the micromagnetic VRE demonstrator

The \Jupyter notebook provides an attractive environment for building
user interfaces for research. However, the current support for inline
visualisation is limited to curve plots and 2D scalar fields. Many
scientific simulations need visualisation of 3D scalar and vector
fields, as shown in Figure~\ref{fig:3d-plots}.  Experimentations in
low dimensional topology and differential geometry also relies on good
drawing capabilities
(e.g. \href{http://www.math.uic.edu/t3m/SnapPy/}{SnapPy} or
\href{http://sagemanifolds.obspm.fr/}{SageManifolds} based on \IPython
and \Sage). The amount of data can be tremendous, especially in
time-dependent problems computed in a distributed fashion over
large-scale computational clusters. Interactive inspection of such
simulations can be a valuable tool which accelerates
research. However, for inspection, one does not need to transfer and
gather the full dataset at each time step---getting selected computed
fields on user request or preprocessing certain quantities like cross
sections with some predefined frequency will mostly suffice.

In this task we will first investigate available technologies for fast
in-browser visualisation of the typical structures to be displayed
(isosurfaces, streamlines, vector fields, cross sections, etc.).
There are several existing solutions which could provide basis for
further development. One of the best known and most advanced is
\href{http://threejs.org/}{three.js} which provides a basis for 3D
visualisation in a web browser. Three.js is WebGL based, but also
provides canvas based rendering for system which do not support
WebGL. It has already been experimentally deployed in Sage Cell Server
and SMC projects. Other promising technologies include visualisation
libraries using exclusively OpenGL. They can be deployed in browser
based systems by using of the WebGL API (which is a restricted subset
of the regular OpenGL API). This can be accomplished by visualisation
executed purely on the GPU. The \href{http://vispy.org/}{VisPy} and
\href{http://glumpy.github.io/}{glumpy} projects have found GPU-only
solutions for common visualisation objects (lines, arrows, markers,
text, iso-lines, iso-surfaces, text, etc) where data does not exit the
GPU. The VisPy project already offers an experimental interface with
the \Jupyter notebook that could be extended to cope with our
specifications. Through this tight collaboration with the authors,
\TheProject could benefit from both dedicated and state-of-the art
visualisation techniques.

The \href{http://www.math.uic.edu/t3m/SnapPy/}{SnapPy} and
\href{http://sagemanifolds.obspm.fr/}{SageManifolds} projects will be
considered for deployment of tools we develop (see associated
deliverable \localdelivref{vis3d}).
\end{task}


\begin{task}[title=Visualisation of 3D fluid dynamics data in web-notebook
,id=cfd-vis,lead=SR, partners={US,PS,USO,XFEL},PM=5,wphases=12-36,issue=77]

We propose to let computational fluid dynamics (CFD) be a driving
application for the development of 3D visualisation in \Jupyter
notebooks (\taskref{UI}{vis3d}) since CFD is one of the most demanding
cases of scientific visualisation. The same time this task
(with deliverable \localdelivref{vis3d}) will be
a demonstrator for (\taskref{UI}{vis3d}).

Successfully handling CFD makes the tool immediately applicable to a
range of other fields such as heat transfer, electromagnetics,
material science, and 3D algebraic structures in
mathematics. Simulations would be initialised inside the notebook and
executed on HPC clusters. This approach will significantly lower the
threshold for using parallel computing codes that can be hard to
install correctly on local workstations (see also \WPref{hpc}). Such
use cases with 3D visualisation will greatly extend the potential
applications of the \Jupyter notebook concept throughout science and
engineering.

As an example code for a 3D live web notebook with fluid dynamics
simulations, we will use the Lattice Boltzmann solver which is under
development at the University of Silesia:
\href{http://sailfish.us.edu.pl/}{Sailfish}.  This code is an advanced
Lattice Boltzmann solver designed from the ground up for distributed
systems of GPU compute clusters. It is implemented predominantly in
Python, and it uses run-time code generation techniques to
automatically build optimised code for CUDA and OpenCL devices. Since
running Sailfish requires specialised hardware, it is reasonable to
use it on dedicated HPC installations.
\end{task}

\begin{task}[lead=UB,title=Common option system for various displays
  in Sage,id=Sage-display,PM=12,wphases=0-24,issue=78]
  \TOWRITE{CNRS}{There are no deliverables associated with this task
    that is listed at 12 person months. Perhaps some explanation of
    the challenges of the task would also help.}

  Given a mathematical object, it often has various possible
  representations on a computer. From raw text to \LaTeX, from simple
  2d picture to a complicated 3d animation.

  In this task, we will provide a uniform option system for displaying
  an object within \Sage (raw text, \LaTeX, tikz, matplotlib, jmol,
  tachyon, \ldots). We will implement some of the missing display and
  will benefit of the work done in \taskref{UI}{cfd-vis}.
\end{task}

\begin{task}[lead=USO,title=Case study: micromagnetic VRE built from
  \TheProject,id=oommf-py-ipython-attributes,PM=6,partners={SR,USH},wphases=13-19,issue=79]
  % 6 person months
  In this task, we use the \TheProject architecture to assemble a
  virtual research environment software tailored for the large
  micromagnetic research community
  (see Section \ref{sec:introduction-micromagnetic-vre-demonstrator}).

  The micromagnetic VRE will be based on the \Jupyter notebook, the
  Python interface to the micromagnetic simulation library OOMMF
  (\taskref{component-architecture}{oommf-python-interface}),
  and the additional features added to \Jupyter in this work
  package.

  The \Jupyter notebook environment allows to host, execute and
  document the Python-based OOMMF simulation in an executable
  document. In this interactive environment, objects can be displayed
  using various representations, including, for example, textual
  representation (i.e. strings), bitmap images and SVG (vector
  graphics) files. We will create functionality so that magnetisation
  vector field objects can be presented as a rendered 3d and 2d-view
  of the magnetisation field (Figure~\ref{fig:3d-plots}), and similar
  features for scalar fields such as field components and energies for
  static and time dependent data (linking to
  \localtaskref{cfd-vis}). This allows computational steering and the
  interactive exploration of the behavior of magnetic nanostructures.

  Beyond that, the \Jupyter Widgets allow the creation of graphical
  user interface (GUI) elements, and we will generate code to display
  these widgets on demand to (i) set up micromagnetic simulations
  using a GUI, and (ii) assist in common post-processing simulation
  results. Recent pilot work has shown that it is possible to make
  \Jupyter Widgets interact with the Python interpreter session and
  this allows to activate a GUI-like (widget based) interface when
  desired but to quickly return to the interpreter prompt, taking
  forward the results (data) from the GUI session
  \cite{IPython-widget-GUI-demo-youtube-2014} and providing a
  continuous path from scripting to GUI. Having the ability to mix and
  match GUI-based and command driven analysis combines the best of
  both approaches, caters for users' preferences, and provides
  significant additional value.
\end{task}

\begin{task}[lead=UB,title=Python/Cython bindings for \Pari,PM=16,id=pari-python,partners={UB,UG},wphases=0-48,issue=80]
  \Pari is a C-library and GP is its standalone interpreter. Partial
  Python/Cython bindings are provided by Sage. There is also a not
  more developed library \software{cypari}.

  The task aims to develop an independent Python/Cython library that
  would provide bindings for \PariGP and which would tightly be
  developed within the \PariGP team.

  Firstly, starting from \Sage and \software{cypari}, we will provide a standalone \Pari bindings in Python
  and integrate it in \Sage (\localdelivref{pari-python-lib1}). Secondly, different optimisation
  will be provided for a tighter communication between \Pari and \Python.
  \begin{compactitem}
  \item Use the declaration files of \software{GP} to automatise \Cython declarations.
  \item Replace copy from the \Pari stack by direct pointers.
  \item More tests and documentation.
  \item Integrate the parallelisation features from \taskref{hpc}{hpc-pari} within \Python.
  \item Implement a crossed bug report system between \Sage and \Pari (using
  results of \taskref{social-aspects}{isocial-decisionmaking}).
  \end{compactitem}
  The deliverable for this second step is \localdelivref{pari-python-lib2}. For this task we might
  require expertise of some \Sage, \Pari or \Cython developers (Jeroen Demeyer, Peter Bruin).
\end{task}

\begin{task}[lead=XFEL,title=Demonstrator: micromagnetic VRE notebooks,
  id=oommf-tutorial-and-documentation,PM=6,partners={SR,PS,USO},wphases=19-25,issue=81]
  % 5 person months + 1 month co-investigator [Ian Hawke's experience]
  The purpose of the micromagnetic \VRE
  (\localtaskref{oommf-py-ipython-attributes}) is to enable excellent
  computational research. To maximise the value of this grant's
  investment for the community, we will not carry out micromagnetic
  research but instead produce a set of executable notebooks using the
  micromagnetic \VRE to demonstrate its power and applicability.

  We will create executable notebook documents
  within the micromagnetic \VRE
  including (i) a new tutorial on computational micromagnetics with
  OOMMF, (ii) the complete documentation of the \texttt{OOMMF-Py}
  library (\taskref{component-architecture}{oommf-python-interface}),
  and (iii) a set of typical micromagnetic case studies. The tutorial,
  in terms of content, will take guidance from the tutorial provided
  for Nmag \cite{Nmag-tutorial-url} and will introduce the additional
  features of the \Jupyter-driven micromagnetic \VRE. We expect this
  substantial and executable documentation of the micromagnetic \VRE to
  become the standard resource that introduces researchers to
  computational micromagnetics, in particular through the online
  portal (\localtaskref{oommf-nb-ve}).

  %% This block is about the benefits of using the notebook. It should
  %% go somewhere else in more generic form:
  %The output of this activity will deliver multiple benefits:
  %providing a systematic introduction to \texttt{OOMMF-py} suitable for both
  %those users (i) new to micromagnetic modelling and those (ii) new to
  %the \texttt{OOMMF-py} interface. Because the documentation is developed in an
  %\Jupyter notebook, the documents are executable. For new learners
  %this is a great simplification because they can skip through the
  %given document and execute the given examples there and then: at the
  %moment, this is a process of manually writing a script, or locating
  %it in the directory structure of files, then executing this,
  %subsequently opening and processing the data files, etc. In the new
  %model, this end-to-end simulation will start from specifying the
  %material parameters in the notebook (all of this is given in the
  %tutorial), to running the simulation in the notebook to processing
  %of computed data while the simulation runs (or subsequently) in the
  %notebook; thus providing one virtual research environment, with all
  %the associated benefits of making best use of the scientist's time
  %using the tool and environment.
\end{task}

\begin{task}[lead=XFEL,id=oommf-nb-ve,title=Online portal for
  micromagnetic VRE demonstrator,PM=3,partners={SR,FAU},wphases=25-28,issue=82]
  % 3 person months
  Recently, a TeMPorary \Jupyter NoteBook (TMPNB) has been made
  available (at \href{http://tmpnb.org}{http://tmpnb.org}) that allows
  anybody to open this URL and use their very own \Jupyter notebook
  for quick calculations and tests online. We will provide such a
  portal which provides the
  micromagnetic \VRE for anonymous use. This service allows users to
  execute the demonstrator tutorial and documentation notebooks
  (\localtaskref{oommf-tutorial-and-documentation}) and run the
  calculations in real time on the web server, without having to
  install any software on their own machine.  This web service will be
  based on Docker \cite{Docker} virtualisation technology and we will
  make available the scripts to create VirtualBox \cite{Virtualbox}
  images, and Docker containers. The same virtual machine images can
  also be used for Cloud hosted computing services.

  %{HF}{Do we need the resource request here? Or should it
  %just be in resources.tex: Either works, in the resources file there
  %is only the total sum mentioned and a link to here. So no
  %duplication of information, and the particular machine is maybe
  %better explained here. I'll comment this out to 'resolve' it.}

  We request \euro{6000} to purchase a machine to provide these
  services (shared memory, 64 cores, 128GB RAM, Solid-state drive (SDD)
  to make the system more responsive).
  %This machine will also support
  %the regression testing and continuous integration (see task
  %\taskref{dissem}{dissemination-of-oommf-nb-virtual-environment}).
  %Setup and
  %maintenance of the machine is part of this work task.
\end{task}

\end{tasklist}

\begin{wpdelivs}
  \begin{wpdeliv}[due=18,id=pari-python-lib1,dissem=PU,nature=OTHER,lead=UB,issue=83, status=delivered]
	  {Python/Cython bindings for \Pari and its integration in Sage}
  \end{wpdeliv}
  \begin{wpdeliv}[id=adstex,due=9,miles=startup,nature=R,dissem=PU,lead=JU,issue=91, status=delivered]
    {Active/Structured Documents Requirements and existing Solutions} Presenting sTeX and
    \Jupyter to the consortium, comparing and evaluating as stepping stones.
  \end{wpdeliv}
    \begin{wpdeliv}[id=mathhub-editing,due=18,miles=startup,nature=DEM,dissem=PU,lead=FAU,issue=92, status=delivered]
      {Distributed, Collaborative, Versioned Editing of Active Documents in MathHub.info}
    \end{wpdeliv}
  \begin{wpdeliv}[due=12,miles=proto1,id=ipython-kernels-basic,dissem=PU,nature=OTHER,lead=PS,issue=93, status=delivered]
      {Basic \Jupyter interface for GAP, \PariGP, \Sage, Singular}
  \end{wpdeliv}
  \begin{wpdeliv}[due=12,id=ipython-kernel-sage,miles=startup,dissem=PU,nature=DEM,lead=PS,issue=94, status=delivered]
      {\Sage notebook / \Jupyter notebook convergence}
  \end{wpdeliv}
  \begin{wpdeliv}[due=12,id=jupyter-collab,miles=startup,dissem=PU,nature=OTHER,lead=SR,issue=95, status=delivered]
      {Tools for collaborating on notebooks via version-control}
  \end{wpdeliv}
  \begin{wpdeliv}[due=24,id=ipython-kernels,miles=startup,dissem=PU,nature=OTHER,lead=PS,issue=96,status=delivered]
      {Full featured \Jupyter interface for GAP, \PariGP, Singular}
  \end{wpdeliv}
  \begin{wpdeliv}[due=18,miles=proto1,id=jupyter-test,dissem=PU,nature=OTHER,lead=SR,issue=98, status=delivered]
      {Facilities for running notebooks as verification tests}
  \end{wpdeliv}
  \begin{wpdeliv}[id=adcomp,due=18,miles=proto1,nature=DEM,dissem=PU,lead=JU,issue=97, status=delivered]
    {In-place computation in active documents (context/computation)}
  \end{wpdeliv}
  \begin{wpdeliv}[due=36,miles=proto1,id=pari-python-lib2,dissem=PU,nature=OTHER,lead=UB,issue=84,status=delivered]
	  {Second version of the \Pari Python/Cython bindings}
  \end{wpdeliv}
    \begin{wpdeliv}[id=jupyter-import,due=36,miles=proto1,nature=DEM,dissem=PU,lead=FAU,issue=85,status=delivered]
      {Notebook Import into MathHub.info (interactive display)}
    \end{wpdeliv}

  \begin{wpdeliv}[due=36,id=vis3d,miles=UI-vre,dissem=PU,nature=OTHER,lead=SR,issue=86,status=delivered]
      {\Jupyter extension for 3D visualisation, demonstrated with computational fluid dynamics}
  \end{wpdeliv}
  \begin{wpdeliv}[due=24,miles=proto1,id=sage-sphinx,dissem=PU,nature=OTHER,lead=UG,issue=87,status=delivered]
      {Refactorisation of \Sage's \Sphinx documentation system}
  \end{wpdeliv}
  \begin{wpdeliv}[due=36,id=cfd-vis,dissem=PU,nature=OTHER,lead=SR,issue=88,status=canceled]
      {Computational Fluid dynamics visualisation in web notebook}
      This deliverable was merged into \localdelivref{vis3d}.
  \end{wpdeliv}
  \begin{wpdeliv}[due=36,miles=UI-vre-prototype,id=jupyter-live-collab,dissem=PU,nature=OTHER,lead=SR,issue=89,status=delivered]
      {Exploratory support for live notebook collaboration}
  \end{wpdeliv}
  \begin{wpdeliv}[due=36,id=ipython-advanced-interacts,miles=UI-vre-prototype,dissem=PU,nature=DEM,lead=PS,issue=90,status=delivered]
      {Exploratory support for semantic-aware interactive widgets providing views on objects
      represented and or in databases}
  \end{wpdeliv}
% communication with live computing process
% post simulation data analysis module
% visualization of vector and scalar fields
% editor for geometry and boundary conditions  on regular meshes


  % Shared \Jupyter sessions embedded in voice-over-IP or
  % teleconference calls or reciprocally.
  %
  % NOTE: This task is probably outdated by appear.in which makes
  % video-conferencing in the browser trivial
  %
  % \delivref{ipython-collaborative}
  % Eugen Dedu:
  % I think such a module can be thought of as a screen-capturing
  % module, i.e. allow Ekiga to capture the screen of a Sage user (this
  % is currently not possible).  This is not a difficult task to do.
  % Julien Puydt: ekiga can do that since something like 2008 with my
  % experimental gstreamer plugin, and I shall be able to present
  % interesting sample code to the ekiga-devel mailing-list in something
  % like two-three weeks (after I'm done with my students), which will
  % hopefully be part of the next version.
  %
  % But as Nicolas noted in his answer, some kind of interactive session
  % where people can share a sage session would be better.
  %
  % I think the feature decomposes in the following pieces:
  % - IPython should have a way to share sessions between several
  % participants using an open and standard protocol ;
  % - ekiga should implement it.
  %
  % In my opinion ekiga, because of its dependency on ptlib and opal
  % libraries and the use of complex protocols like SIP and H323, needs
  % highly technical people.  Students cannot help much, but engineers
  % are appropriate.
  \end{wpdelivs}
\end{workpackage}

%%% Local Variables:
%%% mode: latex
%%% TeX-master: "../proposal.tex"
%%% End:

%  LocalWords:  workpackage wphases Jupyter OOMMFNB wpobjectives wpdescription TOWRITE
%  LocalWords:  Paderborn IPython KBase Hackathon Quantopian Logilab Enthought Authorea
%  LocalWords:  emph Jupyther nanostructures tasklist delivref THREE.js texttt diffing
%  LocalWords:  notebook-collab jupyter-collab Colaboratory jupyter-live-collab Javadoc
%  LocalWords:  notebook.rst Knowls structdocs localtaskref Needs.rst CTypes Cython Nmag
%  LocalWords:  oommf-python-interface OOMMF-py-raw micromagnetic oommf-py magnetisation
%  LocalWords:  Micromagnetic-Standardproblem-3 oommf-py-ipython-attributes vispy taskref
%  LocalWords:  oommf-nb IPython-widget-GUI-demo-youtube-2014 Dedu
%  LocalWords:  oommf-tutorial-and-documentation modelling micromagnetics oommf-nb-ve mws
%  LocalWords:  TeMPorary oommf-nb-virtual Virtualbox Cloudhosted dissem
%  LocalWords:  dissemination-of-oommf-nb-virtual-environment wpdelivs wpdeliv Eugen tikz
%  LocalWords:  Ekiga Puydt gstreamer ekiga-devel ptlib compactitem refactorization numpy
%  LocalWords:  cfd-vis paraview ldots electromagnetics isosurfaces notebooksearch adstex
%  LocalWords:  cassearch simulagora Simulagora mathhub-editing adcomp nbad-search glumpy
%  LocalWords:  swsites visualisation introduction-micromagnetic-vre-demonstrator mathhub
%  LocalWords:  matplotlib jmol maximise virtualisation localdelivref refactorisation
%  LocalWords:  WPref dksbases Simular initialised

%\addtocounter{wpno}{1}
\begin{Workpackage}{\thewpno}
\wplabel{wp:x}
\WPTitle{\wpname{\thewpno}}
\WPStart{Month 1}
\WPParticipant{SA}{1}

\begin{WPObjectives}
The objectives of \theWP{} are to:
\begin{itemize}
\item
\item
\item
\item
\item
\end{itemize}
\end{WPObjectives}

\begin{WPDescription}
This workpackage  ...
\end{WPDescription}

\begin{WPDeliverables}
\begin{itemize}
\item
\ref{del:x}
(Month X): 
X.
\end{itemize}
\end{WPDeliverables}
\end{Workpackage}

%\input{WPs/DevelopmentModelsForAnAcademicFreeSoftwareEcosystem}
%\input{WPs/NextGenerationMathematicalDatabases}
%\begin{workpackage}[id=social-aspects,wphases=12-24!.5,
  title=Social Aspects,
  UORM=1]

\TOWRITE{DP/UM}{workpackage Social Aspects}

\begin{wpobjectives}
The objectives of this work package are to:
\begin{itemize}
\item
\item
\item
\item Development models for an academic free software ecosystem
\item Supporting the Mathematical Process
\end{itemize}
\end{wpobjectives}

\begin{wpdescription}
This workpackage  ...
\end{wpdescription}

% Things to investigate?
% - User surveys. Cf. https://groups.google.com/d/msg/sage-devel/v8Kfky4p6D4/_xRM0bggCo8J
% - The discussion about Code of Conducts and the like

\begin{wpdelivs}
  \begin{wpdeliv}[due=12,id=social_...,dissem=??,nature=??]
      {...}
\end{wpdeliv}
\end{wpdelivs}
\end{workpackage}
%%% Local Variables:
%%% mode: latex
%%% TeX-master: "../proposal"
%%% End:

%\addtocounter{wpno}{1}
\begin{Workpackage}{\thewpno}
\wplabel{wp:x}
\WPTitle{\wpname{\thewpno}}
\WPStart{Month 1}
\WPParticipant{SA}{1}

\begin{WPObjectives}
  The objective of this work package is to organize and optimize the
  communication with the larger community. This includes:
  \begin{itemize}
  \item Reviewing emerging technologies
  \item Advertising the project (press, web, ...).
  \item Disseminating results and deliverables.
  \end{itemize}
\end{WPObjectives}

\begin{WPDescription}
  The first task of \theWP{} is to produce periodic reviews of
  emerging technologies and relevant developments elsewhere, and
  implications for our plans. This include the review of standard
  components and service for storage and sharing, computational
  resources, authentication, package management, etc.  This may
  further include negotiating access or shared development when
  appropriate. This information will be fed to the other work
  packages, in particular\TODO{ref: Component Architecture}.

  Dissemination: software, APIs, technologies, research results, ...
\end{WPDescription}

\begin{WPDeliverables}
\begin{itemize}
% Or make those into a single deliverable?
\item \ref{del:periodic-rep-1}
  (Month 12): Year 1 report.
\item \ref{del:periodic-rep-2}
  (Month 24): Year 2 report.
\item \ref{del:periodic-rep-3}
  (Month 36): Year 3 report.
\item \ref{del:periodic-rep-4}
  (Month 48): Year 4 report.
% \deliverable{del:pressrelease} % Press release.
% \deliverable{del:website} % Project presentation (web site). 
% \deliverable{del:workshop1}  % Report on first project workshop, year 1. 
% \deliverable{del:dissemplan1} % Final plan for using and disseminating knowledge.
% \deliverable{del:workshop2}  % Report on second project workshop, year 2
% \deliverable{del:workshop3}  % Report on third project workshop, year 3
% \deliverable{del:dissemplan2} % Final plan for using and disseminating knowledge.
\end{itemize}
\end{WPDeliverables}

Raw material:
\begin{itemize}
\item Documentation improvements: overview, cross links, overview of
  recent improvements
\item Thematic tutorials
\item Collections of pedagogical documents\\
  E.g. a complete collection of interactive class notes with computer
  lab projects for the ``Algèbre et Calcul formel'' option of the
  French math aggregation (starting from 2014-2015, only open-source
  systems will be supported, and Sage is a major player).
  % See http://nicolas.thiery.name/Enseignement/Agregation/ as a starter
  % Math labs with Sage for first year students in France (L1): http://math.univ-lyon1.fr/~omarguin/
\item Localization of the Sage user interface and key documents in
  various European languages.
\item Distribution of the documents either in the main distribution of
  Sage or through the online repository (see collaborative tools).
\item Massive online introduction course to Sage, drawing on the sage tutorial/notebooks.
Could be "First year Sage course in a box".
\item Taking the opportunity of Python courses to propose Sage as a natural extension
for mathematics; an example is French's 
% TODO: The url macro eats the accented letters.
``Classes pr\'eparatoires''\footnote{
\url{http://en.wikipedia.org/wiki/Classe_préparatoire_aux_grandes_écoles}}, 
where Python has been recently selected as the language to learn programming\footnote{See 
the ``Annexe'' at 
\url{http://www.education.gouv.fr/pid25535/bulletin_officiel.html?cid_bo=71586}}.
%\item \TODO{please expand!}
\end{itemize}

% Jeroen: About teaching: in Gent, Sage is already integrated in the
% courses (maybe you can add this, don't know if it's relevant)
% starting in the first year. It's good for the students because it
% helps in 2 ways: it helps them to understand the mathematics better
% and it helps them to learn basic down-to-earth programming (they
% also have a programming course in Java but that contains a lot of
% theory about complicated class structures)
% Same thing in Orsay
% More python centered but same in UZH

\end{Workpackage}

\endinput

\subsubsection{WorkPackage 4: User Interfaces}
%Explain, task per task, the work carried out in WP during the reporting period giving details of the work carried out by each beneficiary involved.

%%%%%%%%%%%%%%%%%%%%%%%%%%%%%%%%%%%%%%%%%%%%%%%%%%%%%%%%%%%%%%%%%%%%%%%%%%%%%%
\paragraph{Overview}

The objective of WorkPackage 4 is to provide modern, robust, and flexible user interfaces for
computation, supporting real-time sharing, integration with collaborative problem-solving,
multilingual documents, paper writing and publication, links to databases, etc. This work is focused primarily around the \Jupyter project, in the form of:

\begin{itemize}
    \item Enhancing existing \Jupyter tools (\localtaskref{UI}{notebook-collab})
    \item Building new tools in the \Jupyter ecosystem (\localtaskref{UI}{notebook-verification}, \localtaskref{UI}{notebook-collab}, \localtaskref{UI}{vis3d})
    \item Improving the use of \ODK components in \Jupyter and \Sage environments (\localtaskref{UI}{ipython-kernels}, \localtaskref{UI}{sage-sphinx}, \localtaskref{UI}{dynamic-inspect}, \localtaskref{UI}{pari-python})
    \item Demonstrating effectiveness of WorkPackage 4 results in specific scientific applications (\localtaskref{UI}{cfd-vis}, \localtaskref{UI}{oommf-py-ipython-attributes}, \localtaskref{UI}{oommf-nb-ve}, \localtaskref{UI}{oommf-tutorial-and-documentation})
    \item Work on Active Documents, which have some goals in common with \Jupyter notebooks (\localtaskref{UI}{structdocs}, \localtaskref{UI}{mathhub})
\end{itemize}

All deliverables for WorkPackage 4 have been delivered and highly successful in previous reporting periods.
There are no new deliverables in Reporting Period 3.
However, the work of software is never really complete.
Work has continued on some tasks to further improve,
mature, and maintain the results of WorkPackage 4
toward sustainability and to best serve \ODK objectives
based on feedback from \ODK and the wider user community.

%%%%%%%%%%%%%%%%%%%%%%%%%%%%%%%%%%%%%%%%%%%%%%%%%%%%%%%%%%%%%%%%%%%%%%%%%%%%%%
\subparagraph{Milestones}

\subparagraph{\longmilestoneref{UI-vre}}

\emph{“The prototype VRE shall be extended with improved ease of deployment, new
  functionality such as interactive 3D visualization and real-time
  collaboration, enabling researchers to collaborate productively in a shared
  computational environment. Finally, integrating notebooks and semantic
  knowledge into a publication / knowledge system enable a continuous process
  of leveraging \ODK components from research to publication.”}


The \Jupyter-based prototype for this has been previously delivered in \longmilestoneref{UI-vre-prototype},
and is extended in \longtaskref{UI}{notebook-collab} to more mature functionality.

WorkPackage 4 has resulted in a number of useful pieces of software
for mathematical researchers,
sometimes creating new software,
improving existing software,
or establishing new or improved connections between two existing systems.

Combining the above, Milestone~\longmilestoneref{UI-vre} has
been reached:
from the obtained toolkit, we can produce a \Jupyter-based VRE,
integrating \ODK components.
The Jupyter kernels delivered in \localtaskref{UI}{ipython-kernels}
enable access to a broader collection of mathematical software.
The interactive utility of software such as \Pari is improved in \localtaskref{UI}{pari-python},
and general interactivity and exploration of mathematical objects in \Sage is improved in \localtaskref{UI}{dynamic-inspect}.
The scope of what classes of work can be made interactive is increased
by the development of interactive three-dimensional visualization tools in \localtaskref{UI}{vis3d}.
Further, the process of collaboration on notebook documents is improved by \localtaskref{UI}{notebook-collab}
and prototype support for live collaboration with \localtaskref{UI}{notebook-collab}.
By focusing on \Jupyter as our User Interface of choice,
all of these tools can be combined in a single VRE,
hosted in the cloud or and made accessible to any researcher,
building on the Docker images created in \longdelivref{component-architecture}{virtual-machines}.

The work in this final reporting period has focused on stabilising and maturing the software delivered in previous periods.

%%%%%%%%%%%%%%%%%%%%%%%%%%%%%%%%%%%%%%%%%%%%%%%%%%%%%%%%%%%%%%%%%%%%%%%%%%%%%%
\paragraph{Tasks}

\subparagraph{\longtaskref{UI}{ipython-kernels}}
\label{UI@ipython-kernels}

All deliverables for this task have been delivered in previous reporting periods.

Kernels for \ODK components \GAP, \Pari, \Sage, and \Singular,
had been delivered in the form of \delivref{UI}{ipython-kernels-basic}
in RP1 and \longdelivref{UI}{ipython-kernels} in RP2.
Work has continued to develop these kernels in this reporting period
to bring them to further maturity and sustainability.

\smallskip
\subparagraph{\longtaskref{UI}{notebook-collab}}
\label{UI@notebook-collab}

All deliverables for this task have been delivered in previous reporting periods.

Prototype components and plan for \delivref{UI}{jupyter-live-collab} had been delivered in RP2.
This has been developed to further complete prototypes of real-time collaboration in JupyterLab in collaboration with the \Jupyter community.
We are optimistic about its completion and adoption in JupyterLab in the near future.
Real-time collaboration has proven to be the largest and most challenging
effort in WP4,
both in terms of technical effort and in community engagement.
The reason being that real-time collaboration needs extensive work
in development in the core of JupyterLab itself,
which required collaboration and coordination with the JupyterLab community for assembling plans and implementation,
aligning with other goals of the JupyterLab project,
including development of new features in the phosphorjs framework on with JupyterLab is based,
and a complete refactor of the JupyterLab data model.
This work has involved participation in workshops and meetings,
as well as addition of \ODK team members to the core JupyterLab team.
As of August 2019, real-time collaboration has been implemented in JupyterLab in a \texttt{datastore} branch on the official jupyterlab repository on GitHub,
and is expected to arrive in a public release of JupyterLab soon.

In addition, further releases of \texttt{nbdime} from \delivref{UI}{jupyter-collab} have been made.

This work furthers \ODK objective 5 of promoting sustainable software in math and science.


\smallskip
\subparagraph{\longtaskref{UI}{notebook-verification}}
\label{UI@notebook-verification}

All deliverables for this task have been delivered in previous reporting periods.

\longdelivref{UI}{jupyter-test} was delivered in the form of a new Python package, \texttt{nbval},
which enables testing and verification of existing notebooks via a plugin to the Python testing
framework \textbf{pytest}.
In this reporting period, nbval has received further activity and contributions and new releases.
nbval integrates with nbdime from \delivref{UI}{jupyter-collab} to deliver
testable, reproducible notebooks via traditional software development testing practices.
This work furthers \ODK objective 5 of promoting sustainable software in math and science.

\smallskip
\subparagraph{\longtaskref{UI}{sage-sphinx}}
\label{UI@sage-sphinx}

%%% Updated for RP3 by Jeroen Demeyer %%%
Even though this reporting period contains no explicit deliverables
for this task, significant foundation work was carried out which we
now describe. Documentation tools such as Sphinx rely on introspection
to harvest the documentation out the sources. For performance, a large
fraction of the SageMath sources is however written in Cython
(compiled Python) which, until recently, had an incompatible and
limited introspection API. This forced SageMath and other projects to
maintain bespoke and fragile Sphinx extensions to harvest their
documentation.

Tackling this required to dig deep into the system and design,
implement, and get accepted a change to Python itself: PEP (Python
Enhancement Proposal) 590. PEP 590 makes available Python's fast
calling protocol to custom code, thereby enabling full support for
introspection and documentation to Python functions implemented in C
-- e.g. Cython functions --, with no performance loss. This has been
implemented in the upcoming Python~3.8 and Cython~3.0 releases. We
expect not only Cython and therefore SageMath to benefit from this,
but also other similar projects such as Pythran or Numba.

\smallskip
\subparagraph{\longtaskref{UI}{dynamic-inspect}} Due M36 (\delivref{UI}{ipython-advanced-interacts})
\label{UI@dynamic-inspect}

All deliverables for this task have been delivered in previous reporting periods.

As planned in \delivref{UI}{ipython-advanced-interacts}, \ODK
packages \emph{Sage-Combinat-Widgets} and \emph{Sage-Explorer} were
further developed during RP3.
%
%In versions 0.5.0 to 0.7.6,
\emph{Sage-Combinat-Widgets} has gained in
flexibility and has been applied to a range of new mathematical
objects. User interfaces features like feedback have been enhanced,
and documentation has been augmented and gained a tutorial.
%
%With version 0.5.0,
\emph{Sage-Explorer} has gone through a complete new design and reengineering process,
at the same time for better modularity in the code and for better ergonomics.
%
Finally, the \emph{Francy} Jupyter-based graph visualisation library
was generalized to support \Python -- and therefore \SageMath -- in
addition to \GAP.
%
All three benefited from feedback, if not contributions, from end-users.

% Both build on the robust
% foundation of Jupyter Widgets, and explore what it can bring to
% interactive mathematics. The former focuses on interactive
% visualization and edition of mathematical objects, taking
% combinatorics and discrete math as use case. The latter, which uses
% the former as building block, provides rich, detailed, and efficient
% interactive exploration of objects, their properties and
% interrelations. Both are
% \href{https://github.com/sagemath/sage-explorer}{demonstrated online}
% via the Binder service.


\smallskip
\subparagraph{\longtaskref{UI}{structdocs}}
\label{UI@structdocs}

All deliverables for this task have been delivered in previous reporting periods.

Active structured documents are a common need with many use cases, and has many potential
solutions.  Requirements and venues for collaborations were explored through discussions
between participants, in particular at the occasion of
\href{https://wiki.sagemath.org/days77/}{Sage Days 77} workshop (see the
\href{https://wiki.sagemath.org/days77/live-structured-documents}{notes}), and the ODK
meeting in Bremen. The findings were reported in \longdelivref{UI}{adstex}.

In \longdelivref{UI}{adcomp}, We have presented a general framework for in-situ computation in active documents. This is
a contribution towards using mathematical documents -- the traditional form mathematicians
interact with mathematical knowledge and computations -- as a user interface for a
mathematical virtual research environments. This is also a step towards integrating the
two main UI frameworks under investigation in the \ODK project: \Jupyter notebooks and
active documents -- see~\longdelivref{UI}{adstex} -- at a conceptual level. The system is
prototypical at the moment, but can already be embedded into active documents via a
Javascript framework and is ready for use in the \ODK project. The user interface and \SCSCP
connections are quite fresh and need substantial testing and optimizations.

\ODK hosted a workshop on live structured documents in October 2017,
which resulted in the development of \href{https://github.com/minrk/thebelab}{thebelab} software for interactive computing on any website,
enabling interactivity in traditional web-based documentation,
and further development of the \MathHub facilities for evaluation in structured documents.

We developed a JupyterLab extension dedicated to teaching computerscience languages,
such as Python or Sage. JupyterLabTraining (\href{https://gitlab.com/logilab/jupyterhub-training})
is an extension that provides an environment where learners can autonomously do a
series of exercises in order to learn a new programming language. Each exercise is
an independant Jupyter notebook containing the questions, a cell where the learner will
write her code, a hidden cell containing automated tests, and a button to run these tests
and check the code that has been written answers the questions. The left panel shows
the list of all the exercises; they can be sorted by topic (keyword), complexity or
learning track. Thanks to this environment, each learner can do the exercises at his
own pace and choose the exercises that focus on his own points of interest. The
learning process is thus much more efficient for each person.


\subparagraph{\longtaskref{UI}{mathhub}}
\label{UI@mathhub}

All deliverables for this task have been delivered in previous reporting periods.

One of the most prominent features of a virtual research environment (VRE) is a unified user interface. The \ODK approach is to create a mathematical VRE by integrating various pre-existing mathematical software systems. There are two approaches that can serve as a basis for the \ODK UI: computational notebooks and active documents. The former allows for mathematical text around the computation cells of a read-eval-print loop of a mathematical software system and the latter makes semantically annotated documents active.

\MathHub is a portal for active mathematical documents ranging from formal libraries of theorem provers to informal – but rigorous – mathematical documents lightly marked up by preserving LaTeX markup.

As the authoring, maintenance, and curation of theory-structured mathematical ontologies and the transfer of mathematical knowledge via active documents are an important part of the \ODK VRE toolkit, the editing facilities in \MathHub play a great role for the project,
as delivered in \longdelivref{UI}{mathhub-editing}.

\subparagraph{\longtaskref{UI}{vis3d}}
\label{UI@vis3d}

All deliverables for this task have been delivered in previous reporting periods.

The software developed for this task has been delivered in earlier reporting periods.
Packages such as ipyvolume and k3d-jupyter have received further development,
improved compatibility with JupyterLab,
and developed toward maturity and stability,
with growing community adoption.
Several contributions have been made to JupyterLab and
the \Jupyter ecosystem to further support similar work,
benefiting a wide user community.

\subparagraph{\longtaskref{UI}{cfd-vis}} % M12-36
\label{UI@cfd-vis}

No work to report in this period.


\subparagraph{\longtaskref{UI}{Sage-display}} % M24, no deliverables

No work to report in this period.

\subparagraph{\longtaskref{UI}{oommf-py-ipython-attributes}} % M13-19
\label{UI@oommf-py-ipython-attributes}

\ednote{@fangohr: proofread/update report on T4.11: micromagnetics VRE case study}

The micromagnetic virtual research environment is hosted in the
\Jupyter Notebook. The computational backend is the existing \OOMMF
(Object Oriented MicroMagnetic Framework) simulation tool, which is
accessible through the new Python interface that has been created as
part of \ODK
(\localtaskref{component-architecture}{oommf-python-interface}). The
\Jupyter Notebook allows us to integrate the micromagnetic model
specification, the execution of the simulation, and the postprocessing
and data representation within a single executable document; providing
a new computational research environment for micromagnetic simulation
that uses the most widely used simulation code. We have enhanced this
environment further by exploiting that the notebook allows objects to
represent themselves in different ways within the notebook. For
example, Python objects that represent mathematical equations in the
micromagnetic VRE appear rendered as \LaTeX{} in the notebook. It
allows users to interactively compose and explore computational
models, and to be able to inspect what they have put together in the
language of the scientist (i.e. through equations) rather than through
the language of the computer (i.e. code). The addition of this
representation options does not stop the code from being valid \Python
that can be run outside the notebook. We have also provided a
graphical representation of the mesh and discretisation cell as the
appropriate representation of a finite difference mesh to further
assist the effective communication between code and science user and
graphical representation of vector field objects.  We have used
dissemination workshops to seek feedback from users and to refine
interface.

\subparagraph{\longtaskref{UI}{pari-python}}
\label{UI@pari-python}

\ednote{@jdemeyer, @videlec: proofread/update report on T4.12: Pari bindings}

There has been a great deal of progress delivering improved \Pari.
This work has resulted in benefits to the wider Python and \Sage communities
via substantial contributions to the \Sage codebase,
the benefits of which go well beyond this deliverable,
being used by projects outside \ODK.

The end results of this first state of the work are the packages
\href{https://github.com/sagemath/cysignals}{cysignals} and
\href{https://github.com/defeo/cypari2}{CyPari2}, both installable
in a pure \Python environment via the standard tool
\texttt{pip}. Starting from version 8.0, installation via \texttt{pip}
is \Sage's default way of providing the \Pari interface.

\longdelivref{UI}{pari-python-lib2} has been delivered, further improving the \Pari packages
by adding new features, in particular to the Python interface to \Pari.
\emph{cypari2} has gained the ability produce high-resolution SVG plots.
It now also supports the dynamic array type from PARI/GP, \verb/t_LIST/.
The source code of cypari2 is automatically generated.
This automatic generation has been greatly improved
and can be re-used outside cypari2 for any Python package that wants to interface efficiently with PARI.
The cypari2 documentation is also greatly improved,
as a direct result of improvements to the Sphinx documentation system
in \localtaskref{UI}{sage-sphinx}.

\subparagraph{\longtaskref{UI}{oommf-tutorial-and-documentation}
  has been merged into
  \longlocaltaskref{dissem}{dissemination-of-oommf-nb-virtual-environment}
}
\label{UI@oommf-tutorial-and-documentation}

\subparagraph{\longtaskref{UI}{oommf-nb-ve}
  has been merged into
  \longlocaltaskref{dissem}{dissemination-of-oommf-nb-virtual-environment}
}
\label{UI@oommf-nb-ve}

%%% Local Variables:
%%% mode: latex
%%% TeX-master: "report"
%%% End:

%  LocalWords:  subsubsection Jupyter taskref notebook-collab ipython-kernels cfd-vis
%  LocalWords:  oommf-py-ipython-attributes oommf-nb-ve oommf-tutorial-and-documentation
%  LocalWords:  mathhub longmilestoneref emph visualization longdelivref UI-vre delivref
%  LocalWords:  jupyter-live-collab ipython-kernel-sage jupyter-collab texttt nbdime
%  LocalWords:  nbval textbf pytest Cython-generated ipython-advanced-interacts adstex
%  LocalWords:  adcomp optimizations thebelab ipyvolume pythreejs threejs ipyscales unray
%  LocalWords:  ipydatawidgets micromagnetic oommf-python-interface cysignals cypari2
%  LocalWords:  dissem dissemination-of-oommf-nb-virtual-environment

\subsubsection{WorkPackage 5: High Performance Mathematical Computing}
  \label{hpc}
%Explain, task per task, the work carried out in WP during the reporting period giving details of the work carried out by each beneficiary involved.


  %%%%%%%%%%%%%%%%%%%%%%%%%%%%%%%%%%%%%%%%%%%%%%%%%%%%%%%%%%%%%%%%%%%%%%%%%%%%%% 
  \paragraph{Overview}

  This work package is about better exploiting modern parallel
  computer architectures in computational mathematics software,
  notably when deployed within a Virtual Research Environment. It is
  addressed at the level of individual computational components
  (\Pari, \GAP, \Linbox, \MPIR, \Sage, \Singular, ...), and also at
  the level of interfacing and exposing core parallel features to
  higher level programming interfaces.

  Key results obtained over the reporting period are the following:
  %% Only list deliverables produced in the reporting period
  \begin{compactitem}
  \item \ednote{@ClementPernet: nothing to highlight about LinBox?}
  %% \item A closer integration of \Linbox in \Sage with improved reliability and
  %%   computing efficiency.
  \item A full-featured parallelisation engine, supporting POSIX threads and
    MPI, for \PariGP in production release of the software.
  \item Release of GAP-4.9 allowing compilation  in HPC-GAP compatibility mode.
  %    \item A new super-optimizer for vectorized assembly code and its
  %  exploitation to improve the performances of the MPIR code.
  \item A new symmetric matrix factorization algorithm over finite fields, and
    its high-performance implementation in the \texttt{fflas-ffpack} library.
  \item Major redesign of the polynomial arithmetic used in Singular,
    delivering state of the art efficiency.
  \end{compactitem}
  In addition, we investigated how to exploit parallelism when
  combining computational software; see
  \longlocaltaskref{component-architecture}{component-for-HPC}, and
  the following milestone.

%%%%%%%%%%%%%%%%%%%%%%%%%%%%%%%%%%%%%%%%%%%%%%%%%%%%%%%%%%%%%%%%%%%%%%%%%%%%%%
\paragraph{Milestones}

\ednote{@ClementPernet: this is a milestone for the previous reporting
period; do we want to keep it around? mention further work?}

\subparagraph{\longmilestoneref{hpc-prototype}}

\emph{“User story: Astrid wants to run compute intensive routines
    involving both dense linear algebra and combinatorics. She has
    access through a JupyterHub-based VRE to a high end multi-core
    machine which includes a vanilla \Sage installation. She
    automatically benefits from the HPC features of the underlying
    specialized libraries (\Linbox, ...). This is a proof of concept
    of the overall framework to integrate the HPC advances of
    specialized libraries into a general purpose VRE.
    %
    It will prepare the final integration of a broader set of such
    parallel features for the end of the project.”}

With Deliverable~\delivref{hpc}{LinBox-algo}, we developed, released and integrated in the
\Sage the LinBox library and its core dependencies: fflas-ffpack and givaro.
When installing the latest \Sage release on a multithreaded multicore server, it
only takes one configure option to let fflas-ffpack use a multi-threaded BLAS
and therefore expose its parallel speed-up to the end-user of Sage. This feature
is compliant with the use of a higher level of parallelism, through process
workstealing queues that \textit{Astrid} may be using in her combinatorics code, as those
exposed in \delivref{hpc}{sage-HPCcombi}. Now that this first proof of concept has been
successfully achieved, we are working in exposing the more advanced parallel
routines of fflas-ffpack into \Sage, following~\delivref{hpc}{LinBox-DSL}. It
should in particular make Gaussian elimination and related routines enjoy a
better scaling with respect to available CPU cores.

%%%%%%%%%%%%%%%%%%%%%%%%%%%%%%%%%%%%%%%%%%%%%%%%%%%%%%%%%%%%%%%%%%%%%%%%%%%%%%
\paragraph{Tasks}

\medskip
\subparagraph{\longtaskref{hpc}{hpc-pari}}
Deliverable~\longdelivref{hpc}{pari-hpc1} was merged with
\longdelivref{hpc}{pari-hpc2} in the revised workplan. The deliverable
\delivref{hpc}{pari-hpc2} is released on time.

The release 2.12 of the \PariGP suite features a MultiThread engine,
used transparently in all tools from the suite: the \Pari library, the
command line interface \texttt{gp} and the GP2C compiler. Written in 2015 and
2016, the engine supports sequential evaluation (no parallelism), POSIX
threads and MPI within the same code base. It is now being progressively
used wherever it makes sense in the code base and this is by nature work
in progress. In \Pari-2.12, the MT engine is a central component of
\begin{itemize}
\item fast (near linear time) Chinese remaindering;
\item fast linear algebra over $\mathbb{Q}$ and cyclotomic fields,
  a critical component of the new "Modular Forms" package;
\item polynomial resultants in
  $\mathbb{Z}[X] \times \mathbb{Z}[X,Y]$ (via fast Chinese remainders and
    evaluation / interpolation), a basic tool for algebraic number theory;
\item computation of classical modular polynomials for about 20 classical
invariants (j, Weber functions, small eta quotients\dots);
\item discrete logarithm over finite fields (prime fields and
$\mathbb{F}_{p^e}$ for word-sized prime $p$) ;
\item Adleman-Pomerance-Rumely-Cohen-Lenstra primality proof;
\item Fourier coefficients of $L$-functions (Hasse-Weil and Artin
  $L$-functions);
\item hi-resolution plot of mathematical functions using parallel evaluation.
\end{itemize}

The \texttt{master} branch on the public development server includes further
\begin{itemize}
  \item values of complex $L$-functions (via parallel computation of
    Meijer $G$-functions);
  \item a new thread-safe version of the Multiple Polynomial Quadratic Sieve
    (MPQS) integer factoring algorithm, ready to be parallelized;
\end{itemize}
The parallel-enabled components of the \PariGP suite have been advertised 
(including tutorial sessions) and tested by participants during PARI/GP
workshops in Grenoble (2016), Lyon (2017), Besançon (2018) and Bordeaux
(2019).

\medskip
\subparagraph{\longtaskref{hpc}{hpc-gap}}
\label{hpc@hpc-gap}

Deliverable~\longdelivref{hpc}{GAP-HPC-report} was completed at the end of this period,
reporting all of the developments in the \GAP system during the project which are relevant
to, or provide essential context for, this workpackage.

During reporting period 3 the main areas of effort were:
\begin{itemize}
\item Follow-up work to the integration of HPC-GAP into the main codebase reported in
  the previous period. This work improves the robustness of the system and dramatically reduced the
  differences between the two versions of the source code.
\item
  \ednote{@stevelinton, @alex-konovalov: briefly explain why meataxe is fundamental?
    Something like: this is the workhorse for higher level
    representation theoretical computations (character tables, etc).}
  Development of the \GAP interface to the meataxe64 high performance linear algebra library (to which
  we also contributed significant development effort). This system targets large calculations over
  small finite fields on multi-core shared memory computers.
  The interface makes almost all of the
  capabilities of meataxe64 callable from \GAP, something which is not only a quantum leap in
  performance for \GAP in this critical area, but also allows easy prototyping in \GAP of new
  algorithms for meataxe64.  This library can make use of multiple cores whether or not it is being
  called from \HPCGAP. A full set of benchmarks are included in D5.15, but as a highlight, two
  dense random $320\,000\times 320\,000$ matrices over $GF(2)$ can be multiplied in just over 1000
  seconds on a 64 core AMD ``bulldozer'' system.
  \item Release of ``libGAP'' a general-purpose C API for \GAP. This allows any program, including in particular HPC code, to call on the
    functionality of \GAP efficiently and without the need to run a separate \GAP process. This makes
    fine grained interaction possible.

    \ednote{@defeo, @embray: the following statement could be here or in WP3}
    To interface with \GAP, SageMath formerly used a bespoke
    implementation of ``libGAP``, requiring heavy patching of \GAP.
    Having the functionality available upstream reduced considerably
    the maintenance burden for SageMath developers and packagers
    alike.

  \item Development and release of a new linguistic reflection API in \GAP, allowing \GAP programs to
    access and modify the executable representation of their own functions at run-time. This will be
    the basis of future automatic parallelisation and optimisation tools.
  \item Very much improved profiling tools
  \item Release of the new package ``ferret'' which achieves world-leading performance in partition
    backtrack, a critical, and notoriusly challenging computational kernel
  \item Extensive developments in our testing and release infrastructure with the overall goal of
    ensuring that \GAP users have easy access to a reliable, up-to-date and mutually compatible set of
    versions of the large suite of packages redistributed with \GAP.
\end{itemize}

\medskip
\subparagraph{\longtaskref{hpc}{hpc-linbox}}
  \label{hpc@hpc-linbox}

During this reporting period, we delivered~\longdelivref{hpc}{LinBox-distributed}.

A first focus was made on distributed computing, with an MPI parallelization of a Chinese remainder based
algorithm. The first proof-of-concept implementation was then cleanly integrated in the mainstream code of the
library. Its performance shows a very nice scaling with the number of compute nodes on a 256 cores cluster.

\ednote{@ClementPernet: T5.3: proofread my changes; I was confused by the original text}

Although this approach is best suited for parallelization on a large
number of nodes, its total computational complexity ($O(n^?)$) becomes
a major concern on large instances.
%
The usual alternative approach based on $p$-adic lifting has a better total computational complexity ($O(n^?)$), but is intrinsically more
sequential, and therefore less suited for large scale parallelization.
A major contribution in this task is a new algorithm combining $p$-adic lifting and Chinese remaindering in order to
expose more parallelism without sacrifying the gain in complexity.
We also provide a full-featured  implementation of this new algorithm in \Linbox, which delivers high
sequential efficiency and nevertheless scales well up to 16 cores.
This compromise is a good fit for personal computers or the typical
lab-wide computational server researchers have access to.

Lastly, we introduced support for GPUs in the \texttt{fflasffpack} library and showed how matrix product over a finite
field benefit from these accelerators.

All these software improvements are closely integrated in the mainstream code of the \texttt{fflasffpack} and \Linbox libraries.

\medskip
\subparagraph{\longtaskref{hpc}{hpc-singular}}
  \label{hpc@hpc-singular}

  The only deliverable under consideration for this reporting period
  is~\longdelivref{hpc}{singular-polyarith}

Multivariate polynomials are represented in Singular using the sdmp format. While this data structure is generally amenable to parallelization, the implementation and some of the algorithms in Singular were not. Much work has been invested in updating the algorithms and data structures and making Singular polynomial arithmetic competitive with other systems. This work has been done in the Singular submodule Flint, whose code is available at \url{https://github.com/wbhart/flint2}.

We now support polynomial exponents of unlimited size with the three basic monomial orderings of lex, deglex, and degrevlex over the integers mod p and rationals. Continued engagement by colleagues in the HPC community including Bernard Parisse, Michael Monagan, Roman Pearce, and Micka\"el Gastineau has been invaluable.

Parallel and serial implementations of the operations of multiplication, division and GCD are complete and perform well in both the dense and sparse cases. The performance is more than competitive with all other systems we are aware of, both on a single core and on multiple cores.

Basic arithmetic in Singular now benefits directly from the implementation and researchers are already working on leveraging the new implementation in other areas, e.g. Gr\"{o}bner bases over rational functions, Gr\"{o}bner bases with a bottleneck on multivariate arithmetic and polynomial factorisation. Early indications are that all of these are going to experience a huge improvement for many real-world research applications.

Other systems such as the new Oscar computer algebra system already benefit directly from the new ODK implementation. 

Sage will automatically benefit directly at the next update of the Singular version in Sage.

\medskip
  \subparagraph{\longtaskref{hpc}{hpc-mpir}}
  \label{hpc@hpc-mpir}

  Not applicable for this period.
  
  \subparagraph{\longtaskref{hpc}{hpc-combi}}
  \label{hpc@hpc-combi}

  Not applicable for this period.


  \subparagraph{\longtaskref{hpc}{pythran}}
  \label{hpc@hpc-pythran}
  Not applicable for this period.

%%   The goal of this task is to make Pythran easily integratable in large-scale
%%   project, taking into account native dependencies, compilation time, memory
%%   footprint, speed and size of compiled binaries as well as multi-platform
%%   support. Integration with \software{cython} is a possible mean to achieve
%%   this goal.

%%   Two projects have been selected for this task: \software{scipy} and
%%   \software{scikit-image}. These projects are relevant for \software{pythran}
%%   because they have many small to medium kernels that can benefit from
%%   compilation. Even though \software{pythran} has not been selected as a scipy
%%   backend, the exchanges with the community have led to a great deal of
%%   improvements of which all \software{Pythran} users take advantage.  The
%%   \software{scikit-image} community is still examining the possibility of using
%%   \software{pythran} as an acceleration mean.

%%   Integration of  \software{pythran} as a \software{cython} backend for
%%   \software{numpy} has improved in various aspects: better error detection,
%%   more supported expression patterns and improved performance for the compiled
%%   expressions.

%% Deliverable~\longdelivref{hpc}{sage-HPCcombi} is shared with
%% Task~\longlocaltaskref{hpc}{hpc-combi}, the status of which we reported on above.

  \subparagraph{\longtaskref{hpc}{hpc-jupyter}}
  \label{hpc@hpc-jupyter}
  
%% It is common for academic High Performance Computing (HPC) clusters to make
%% use of schedulers based on Sun Grid Engine with Son of Grid Engine as one of
%% the most popular. It is used, for example, on the institutional HPC systems
%% in the Universities of Sheffield and Manchester in the United Kingdom. It is also used
%% on the regional N8 HPC facility, a system shared by the eight most research
%% intensive universities in the North of England.
  Not applicable for this period.

%%% Local Variables:
%%% mode: latex
%%% mode: visual-line
%%% TeX-master: "report"
%%% End:


%  LocalWords:  subsubsection hpc compactitem super-optimizer vectorized factorization
%  LocalWords:  texttt fflas-ffpack longmilestoneref emph JupyterHub-based specialized
%  LocalWords:  longtaskref longdelivref delivref finalized mathbb embarassingly taskref
%  LocalWords:  scienceproject refactorization organized LinBox-algo DumPerSul:fcrpmgbd16
%  LocalWords:  Pernet:cqm16,PerSto:tsegqm17 DumKalTho:lticmpdsm16,DumLucPer:cftearp17
%  LocalWords:  Cython Hongguang singular-polyarith sdmp parallelization deglex degrevlex
%  LocalWords:  Monagan Micka parallelized parallelize Imbach hpc-mpir sage-paral-tree
%  LocalWords:  Cilk libsemigroup optimized pythran integratable scipy scikit-image numpy
%  LocalWords:  sage-HPCcombi hpc-jupyter

\subsubsection{WorkPackage 6:  Data/Knowledge/Software-Bases}\label{dksbases}
%Explain, task per task, the work carried out in WP during the reporting period giving details of the work carried out by each beneficiary involved.

%%%%%%%%%%%%%%%%%%%%%%%%%%%%%%%%%%%%%%%%%%%%%%%%%%%%%%%%%%%%%%%%%%%%%%%%%%%%%%
\paragraph{Overview}

In a series of workshops (September 2015 in Paris, January 2016 in St. Andrews, June 2016 in Bremen, and July 2016 in Bia{\l}ystok, 2017 in Orsay, 2018 in Cernay, 2019 in Cernay), the participants working on \WPref{dksbases} met and discussed the topic of integrating the \pn systems into a mathematical VRE toolkit.
Additionally, Florian Rabe was employed at both FAU and UPSud throughout 2018 and 2019 to deepen the integration.

Key results of the first two reporting periods were
\begin{compactitem}[\bf R1.]
\item the observation that \emph{knowledge-aware interoperability of software and database-systems is the most critical objective} for \WPref{dksbases} in the \pn project.
\item the consensus that this can be achieved by \emph{aligning the mathematical knowledge underlying the various systems},
\item the existing integration of mathematical computation systems in the Sage and Jupyter systems must be complemented with a similar integration of mathematical databases.
\end{compactitem}
This requires explicitly representing the three aspects of math VREs -- Data (D), Knowledge (K), and Software (S) -- and basing computational services and inter-system communication on a joint \DKS-base.
These results are engrained in the ``Math-in-the-Middle'' (MitM) paradigm~\cite{DehKohKon:iop16}, which gives a representational basis for specification-based interoperability of mathematical software systems -- so that they can be integrated in a VRE toolkit.
In the MitM paradigm, the mathematical knowledge underlying the VREs (K) and the interface for each system (S) are represented as modular theory graphs in the OMDoc/MMT format.
For the data aspect (D) we have extended the concept of OMDoc/MMT theories to ``virtual theories'' that allow the practical management of possibly infinite theories, see~\cite{ODK-D6.5} for details.

Through the concerted effort of the WP6 participants, we have been able to implement this design and instantiate itwith generate theory graphs for the \GAP and \Sage systems and integrating the \LMFDB (see~\cite{ODK-D6.5}.
Based on this, we were able to generically integrate \GAP, \Sage, and \LMFDB via the standardised SCSCP protocol~\cite{HorRoz:ossp09}. This case study shows the feasibility of the design. 

\begin{wrapfigure}r{6cm}\vspace*{-1em}
\documentclass{standalone}
\usepackage{tikzinput}
\begin{document}
\providecommand\myscale{4.5}
\begin{tikzpicture}[scale=\myscale]
  \node (center) at (0,.15) {Organization};
  \node (left) at (.2,-.3) {Computation};
  \node (right) at (.4,0) {Tabulation};
  \node (back) at (-.5,0) {Inference};
  \node (up) at (0,.5) {Narration};

  \draw[very thick] (center) -- (left);
  \draw[very thick] (center) -- (right);
  \draw[very thick] (center) -- (back);
  \draw[very thick] (center) -- (up);
  \draw[dotted] (left) -- (right) -- (back) -- (left);
  \draw[dotted] (up) -- (left);
  \draw[dotted] (up) -- (right);
  \draw[dotted] (up) -- (back);
\end{tikzpicture}
\end{document}
%%% Local Variables: 
%%% mode: latex
%%% TeX-master: t
%%% End: 
\vspace*{.5em}
\caption{Five Aspects of Math VREs, a Tetrapod Structure}\label{fig:tetrapod}\vspace*{-1.5em}
\end{wrapfigure}
In the \textbf{third reporting period}, the focus was on the \textbf{representation and curation of mathematical data}, building on the earlier work. We have refined the original notion of \DKS-bases from the grant proposal into a tetrapodal structure which joins four primary aspects of ``doing Maths'' which have to be supported in a VRE toolkit via a modular organization aspect -- see the introduction  of \cite{ODK-D6.10} and \cite{CarFarKohRab:bmobb19} for a discussion.

We have taken up the general discussion of research data, the FAIR principles and have adapted them to the case of mathematical resarch data. The outcome of this was the concept that -- as mathematics deals with ideal and abstract objects -- it is possible to fully describe the objects in question symbolically and often losslessly represent them as database structures while at the same time binding them to the symbolic description. The codecs from the ``virtual theories'' approach developed in \WPref{dksbases} do just this: they link the database level of mathematical data sets with the MitM ontology -- and from there via the interface theory graphs interface it to the mathematical software systems in \pn.

We undertook three larger case studies to bring this about:
\begin{itemize}
\item Developing the system data.mathhub.info for managing mathematical data sets MitM-style and equipping it with a search UI; see the report on \taskref{dksbases}{data-LMFDB} below
\item Exporting a the Isabelle knowledge base (via a subcontract), and equipping it with a semantic search facility;  see the report on \taskref{dksbases}{isabelle} below 
\item Extending the formula search capabilities developed in the first reporting period to Jupyter notebooks; see the repoet on
 \taskref{dksbases}{mws} below.
\end{itemize}
This wraps up and integrates the work in \WPref{dksbases} combining aspects of Data (D), Knowledge (K), and Software (S). The joint system addreasses central aspects of all four FAIR concerns for research data semanticaly. 
For a joint and integrated final report on this, see ~\cite{ODK-D6.10}.


%%%%%%%%%%%%%%%%%%%%%%%%%%%%%%%%%%%%%%%%%%%%%%%%%%%%%%%%%%%%%%%%%%%%%%%%%%%%%% 
\paragraph{Milestones}

% month 36
\subparagraph{\longmilestoneref{dksbases-interop1}}
This milestone was addressed in the second reporting period.
\medskip
%\emph{“User story: thanks to a fully functional prototype integrating of at least the systems \GAP, \Sage, \Singular, and \LMFDB via the \SCSCP Protocol, end users shall be able to run calculations involving any combination of those systems from any of them.
%  This prototype will be the basis for integration work for additional systems and the user interface from WP4.”}
%\medskip
%
%Workpackage \textbf{WP6} is fully on track with this milestone.
%After first integration and DKS prototypes (the MitM VRE middleware  framework) became available in late fall 2017 (see~\cite{KohMuePfe:kbimss17,WieKohRab:vtuimkb17}) we were able to develop more sophisticated -- and mathematically more realistic/relevant -- use cases~\cite{CreLow:mdcmds18} and generalize those parts of the framework that had been overly specific to the first use cases.
%This involved non-trivial investments in all parts of the framework, as well as the system API theory generation systems and (in particular) the MitM ontology. 

% month 42
\subparagraph{\longmilestoneref{dksbases-interop2}}
\begin{oldpart}{copied from TR 2, needs update by FAU}
\emph{“The goal of this milestone is to take into account all the operational experiences with the first prototype and add more systems and integrate some of the UI components from WP4.
  The experiences with the preparation of this prototype will allow us to estimate the joining costs of adding a system to the OpenDreamKit VRE toolkit, which is an important measure of the flexibility of the Math-In-the-Middle approach.”}
The state of the MitM VRE middleware is sufficiently mature that most of the functionality can be configured by writing domain and system knowledge in form of OMDoc/MMT theories, but not extending the system (programming the VRE systems or the MMT mediator).
This means that additional systems can be added at the cost of generating system API theories, extending the  MitM ontology and supplying alignments.
We are targeting the knowledge bases OEIS, and FindStat (see \localtaskref{dksbases}{data-findstat}) as well as PARI/GP.
We plan to extend the worked use cases substantially.
To this end we already have statements of interest from external researchers, who want to use the flexible integration in the MitM framework and do not mind the communication overheads involved. 
First work on UI integration work has already begun; see  \longdelivref{UI}{jupyter-import}, which presents a Jupyter kernel for MMT and prototypical MitM-based integration of Jupyter widgets.
\end{oldpart}

%%%%%%%%%%%%%%%%%%%%%%%%%%%%%%%%%%%%%%%%%%%%%%%%%%%%%%%%%%%%%%%%%%%%%%%%%%%%%% 
\paragraph{Tasks}
\medskip

\subparagraph{\longtaskref{dksbases}{data-assessment}}
\label{dksbases@data-assessment}
This task was addressed in the first reporting period.
\medskip

\subparagraph{\longtaskref{dksbases}{data-triform}}
\label{dksbases@data-triform}
This task was addressed in the first reporting period.
%For this task we have specified and implemented the concept of virtual theories that can contain large -- theoretically even infinite -- numbers of declarations and objects (e.g. 3.5M declarations in the LMFDB data base for elliptic functions) in OMDoc/MMT.
%Virtual theories are characterized by the fact that they are too large to keep in main memory of the MMT System and have to be partially and lazily imported from an external data store.
%We have reported on the design in \longdelivref{dksbases}{design}, on a first implementation on the international conference (MACIS 2017)~\cite{WieKohRab:vtuimkb17}, and finally on an extended use-case in \LMFDB in \longdelivref{dksbases}{psfoundation}. 
\medskip

\subparagraph{\longtaskref{dksbases}{data-design}}
\label{dksbases@data-design}
This task was addressed in the first reporting period.
%This task was directly addressed in the \WPref{dksbases} workshops in the first year and has led to the design and implementation in \delivref{dksbases}{design}. A first implementation has been presented on the international conference (MACIS 2017)~\cite{WieKohRab:vtuimkb17}, and finally on an extended use-case in \LMFDB in \longdelivref{dksbases}{psfoundation}.
% \medskip

\subparagraph{\longtaskref{dksbases}{data-foundationCAS}}
\label{dksbases@data-foundationCAS}
This task was addressed in the first reporting period.
%In the course of the deliberations in the \WPref{dksbases} workshops we saw a shift from the development of computational foundations and verification towards API/Interface function specifications to enable semantic system interoperability via the Math-in-the-Middle (MitM) Ontology.
%Consequently, emphasis has changed to the generation of system API theories for \GAP, \Sage, \Singular, and \LMFDB, which act as OpenMath content dictionaries.
%The computational foundations exist but are rather more simple than originally anticipated.
%Much of the functionality has been offloaded to the SCSCP standard -- remote procedure call with OpenMath representations of the mathematical objects -- developed in the SCIENCE Project.
%As a direct consequence of the work in \pn the OpenMath Society has promoted the \SCSCP protocol into as an OpenMath Standard.
%
%Conversely, the \GAP and \Sage CDs are rather more elaborated than anticipated in the proposal, and thus form a viable basis for alignment with the MitM Ontology.
%
%The MitM integration paradigm is the result of our research and development on the computer algebra foundations in this task has been presented on the international conference MACIS 2017~\cite{KohMuePfe:kbimss17} and is described in deliverable \longdelivref{dksbases}{psfoundation}, which presents an advanced CAS integration use case. 
%The MitM ontology and the system API theories have been developed to the point, where the data model is fully developed and the contents cover the use cases corresponding to this task and \longlocaltaskref{dksbases}{data-design} are surveyed in \longdelivref{dksbases}{lfmverif}.
\medskip

\subparagraph{\longtaskref{dksbases}{research-categories}}
\label{dksbases@data-research-categories}

\begin{oldpart}{this is the text from TR 2, update by FAU needed}
The MitM architecture developed in \WPref{dksbases} has given important impulses to make the code infrastructure of \Sage and \GAP more declarative (knowledge-based).
In \Sage, the category infrastructure was validated (it seems to be the right level of abstraction to generate API theories) and extended; we explore ways to enrich it with additional semantic
trough the use of annotations, to maximize the chance of
them being accepted and adopted by the Sage community.

In \GAP, the facilities for ``constructors'' was reformed, extended by an infrastructure for documentation and static typing/type analysis, and the code base refactored for over 2000 constructors.
Similarly, the online documentation subsystem for \GAP has been regularized and synchronized with the constructor level.
Already at this early stage of the task the new ``knowledge-based perspective'' has revealed a plethora of errors and inefficiencies and has contributed to the code quality in both systems.
\end{oldpart}
\medskip

\subparagraph{\longtaskref{dksbases}{data-OEIS}}
\label{dksbases@data-OEIS}
This task was addressed in the first reporting period.
%For the OEIS case study we have parsed the OEIS data and converted it into OMDoc/MMT theories (ca. 260,000).
%The main problem solved here was to parse the formula section (generating functions, relations between sequences, \ldots): they are represented in a human-oriented ASCII syntax, which is highly irregular, ill-separated from surrounding text, and interpunctuation.
%Nonetheless we managed to recover ca. 90\% of the formulae and
%\begin{compactenum}[\em i\rm)]
%\item generate ca. 100,000 new relations between sequences and
%\item provide a package of ca. 50,000 generating functions to Sage (which can be used
%  e.g. in the FindStat database).
%\end{compactenum}
%We use this theory set to test the functionalities of ``virtual theory graphs'' (one step up from the ``virtual theories'' developed in \localtaskref{dksbases}{data-design}).
\medskip

\subparagraph{\longtaskref{dksbases}{data-findstat}}
\label{dksbases@data-findstat}
This task was addressed in the second reporting period.
%We have seen that the \LMFDB already shows all the complexities needed to develop full-coverage DKS functionality for the \pn VRE toolkit.
%On the other hand our survey shows that our DKS design (OMDoc/MMT virtual theories) is sufficient for covering the FindStat use case as well.
%Therefore we have delayed this taks to the last year of the \pn project, when the system API theories for \Sage and OEIS have matured. With the declarative design of the virtual theories, task \localtaskref{dksbases}{data-findstat} becomes a matter of writing down the schema theories system API theories for FindStat and defining the requisite codecs. We expect this to be a matter of one of two weeks of joint development of the FAU team together with UPSud. 
\medskip

\subparagraph{\longtaskref{dksbases}{data-LMFDB}}
\label{dksbases@data-LMFDB}

\begin{oldpart}{this is the text from TR 2, update by FAU needed}
Work on this task has started. Given the concept of virtual theories developed in \localtaskref{dksbases}{data-triform} the task is to build a database connector that converts the MongoDB tables in LMFDB into ``mathematical objects''.
We have identified the problems -- e.g. that objects are reduced to ad-hoc database records: for instance elliptic curves are represented as a quadruple of integers, where the last is represented as a string of digits as the range of MongoDB integers is too small.
We have developed an architecture of language-specific Codecs which mitigate these problems in a knowledge-centered way (Codecs are OMDoc/MMT objects) that interpret database records as OMDoc/MMT objects and can thus be used populate virtual theories.
The next step is to extend the existing MMT query language by a query compiler into the underlying data store system; concretely to MongoDB underlying LMFDB for \localtaskref{dksbases}{data-LMFDB}.
\end{oldpart}
\medskip

\subparagraph{\longtaskref{dksbases}{data-memo}}
\label{dksbases@data-memo}
We have developed persistent memoization modules for Sage and Gap that can use both local and remote data stores.
Both use the same format so they can share the same data stores.\ednote{to be finished by sites US,PS,UW}

We report on this task in detail in \delivref{dksbases}{persistent-memoization}.
\medskip

\subparagraph{\longtaskref{dksbases}{mws}}
\begin{oldpart}{this is the text from TR 2, update by FAU needed}
Work on the first work phase has proceeded as planned and has culminated in \longdelivref{dksbases}{mws}.
The second work phase on this task presupposes the Math-in-the-Middle ontology (as we call it now.)
Where we already have that, e.g. for the OEIS (see \localtaskref{dksbases}{data-OEIS}) we already have a running search engine.
The main problem here is to devise intuitive query interfaces and integrate them into the \pn VRE framework.
\end{oldpart}

We report on this task in detail in \delivref{dksbases}{nbad-search}.
\medskip

\subparagraph{\longtaskref{dksbases}{isabelle}}
\begin{newpart}{FR: adapted from the deliverable}
For many decades, the development of a universal database of all mathematical knowledge, as envisioned, e.g., in the QED manifesto \cite{qed}, has been a major driving force of computer mathematics.
Today a variety of such libraries are available.
These are most prominently developed in proof assistants such as Coq \cite{coq} or Isabelle \cite{isabelle} and are treasure troves of detailed mathematical knowledge.
Within \pn, we have developed interface standards, specifically OMDoc for symbolic and ULO for relational knowledge, that allow maintainers of formal libraries to make their content available to outside systems.

In this task (which has been added in the last amendment of the grant agreement), we have exported the large Isabelle knowledge bases as both OMDoc/MMT and ULO format
Concretely, we have built an exporter from the Isabelle Theorem prover library (Archive of Formal Proof) to both MMT and RDF data.
This exporter is now part of the latest releases of both Isabelle and MMT, and the exported data is available online.

We report on this task in detail in \delivref{dksbases}{nbad-search}.
\end{newpart}


%%% Local Variables:
%%% mode: latex 
%%% mode: visual-line
%%% fill-column: 5000
%%% TeX-master: "report"
%%% End:

%  LocalWords:  subsubsection dksbases ystok WPref compactitem emph DehKohKon:iop16 textbf taskref longdelivref lfmverif triformal formalized biformal HorRoz:ossp09 medskip longmilestoneref dksbases-interop1 dksbases-interop2 characterized WieKohRab:vtuimkb17 psfoundation delivref KohMuePfe:kbimss17 regularized synchronized ldots interpunctuation compactenum mws KohMuePfe:kbimss17,WieKohRab:vtuimkb17 CreLow:mdcmds18 jupyter-import Jupyter MitM-based Jupyter

\input{WPs/WP7}
\input{WPs/WP8}
\input{WPs/WP9}
\input{WPs/WP10}
\input{WPs/WP11}
\input{WPs/WP12}
\input{WPs/WP13}
\input{WPs/WP14}
\input{WPs/WP15}


\TODO{Milestones need to be discussed and then described here.}

\newpage

\TODO{Check this for any necessary changes.}


\subsection{Management Structure and Procedures}
\label{sect:mgt}

\eucommentary{Describe the organisational structure and the decision-making 
(including a list of milestones (table 3.2a)).\\
Explain why the organisational structure and decision-making mechanisms are
 appropriate to the complexity and scale of the project.\\
Describe, where relevant, how effective innovation management will be 
addressed in the management structure and work plan.\\
Describe any critical risks, relating to project implementation, that 
the stated project's objectives may not be achieved. Detail any risk 
mitigation measures. Please provide a table with critical risks 
identified and mitigating actions (table 3.2b).}

\draftpage
\subsection{Consortium as a Whole}
\eucommentary{\begin{itemize}
\item
Describe the consortium. How will it match the project's objectives? 
How do the members complement one another (and cover the value chain, 
where appropriate)? In what way does each of them contribute to the 
project? How will they be able to work effectively together?
\item
If applicable, describe the industrial/commercial involvement in the 
project to ensure exploitation of the results and explain why this is 
consistent with and will help to achieve the specific measures which 
are proposed for exploitation of the results of the project (see section 2.3).
\item
Other countries: If one or more of the participants requesting EU funding 
is based in a country that is not automatically eligible for such funding 
(entities from Member States of the EU, from Associated Countries and 
from one of the countries in the exhaustive list included in General 
Annex A of the work programme are automatically eligible for EU funding),
 explain why the participation of the entity in question is essential to carrying out the project
\end{itemize}
}

\draftpage

\subsection{Resources to be Committed}

\eucommentary{Please provide the following:
\begin{itemize}
\item
a table showing number of person/months required (table 3.4a)
\item
a table showing 'other direct costs' (table 3.4b) for participants where 
those costs exceed 15\% of the personnel costs (according to the budget 
table in section 3 of the administrative proposal forms)
\end{itemize}}

%\newpage

%\landscape

\fbox{\begin{minipage}{\textwidth}

\begin{center}\Large\bf
Summary of staff effort
\end{center}
\end{minipage}}

\eucommentary{Please indicate the number of person/months over the whole 
duration of the planned work, for each work package, for each participant. 
Identify the work-package leader for each WP by showing the relevant 
person-month figure in bold.}

\bigskip

\TODO{Update this once the list of parthers and the WPs are finalised.}

\newcommand{\wpleader}{\textbf}

\begin{center}
\begin{minipage}{\textwidth}
%\begin{center}
\begin{tabular}{| p{0.9cm} | p{1.5cm} | c | c | c | c | c | c | c | c | c | c |}  \hline
\textbf{Partic.} & \textbf{Partic.} 
& \multicolumn{9}{c|}{\textbf{Work package}} &
 \textbf{Total} \\
\textbf{no.} & \textbf{short} & WP1 & WP2 & WP3 & WP4& WP5 & WP6 & WP7 & WP8 & WP9 &
 \textbf{PMs} \\
 & \textbf{name} &
 &   &  &   &  &  &   &  & &
 \\
\hline

\textbf{1} & \shortparticipant{1} & 
%\wpleader{9} & 2 &  15 & \wpleader{24} &   & 3 &  &   & 53  & &
 & &   &  &   &  &  &   &  &
\\\hline

\textbf{2} & \shortparticipant{2} &
 & &   &  &   &  &  &   &  &
 \\\hline

\textbf{3} & \shortparticipant{3} &
 & &   &  &   &  &  &   &  &
 \\\hline

\textbf{4} & \shortparticipant{4} &
 & &   &  &   &  &  &   &  &
 \\\hline

\textbf{5} & \shortparticipant{5} &
 & &   &  &   &  &  &   &  &
 \\\hline

\textbf{6} & \shortparticipant{6} &
 & &   &  &   &  &  &   &  &
 \\\hline

\textbf{7} & \shortparticipant{7} &
 & &   &  &   &  &  &   &  &
 \\\hline

\multicolumn{2}{|c|}{\textbf{Total PM}} & 
 & &   &  &   &  &  &   &  &

\\\hline
\end{tabular}
%\end{center}
\end{minipage}
\end{center}

\fbox{\begin{minipage}{\textwidth}

\eucommentary{Please complete the table below for each participant if the sum of the costs for’ travel’, ‘equipment’,
and ‘goods and services’ exceeds 15% of the personnel costs for that participant (according to the
budget table in section 3 of the proposal administrative forms).}

\begin{center}\Large\bf
Other direct cost items
\end{center}
\end{minipage}}

\bigskip

\begin{tabular}{|r|l|p{9cm}|}
\hline
\textbf{} & \textbf{Cost (\euros)} & \textbf{Justification} \\\hline
\textbf{Travel} & & \\\hline
\textbf{Equipment} & & \\\hline
\textbf{Other goods and services} & & \\\hline
\textbf{Total} & \\\cline{1-2}
\end{tabular}


\subsubsection*{Management Level Description of Resources and Budget}

\TODO{This needs to be updated in line with the rest of the
project.}

The project will employ XX person-months of effort over YY years, 
comprising ...


% ---------------------------------------------------------------------------
%  Section 4: Members of the Consortium
% ---------------------------------------------------------------------------

\newpage

\eucommentary{This section is not covered by the page limit.\\ 
The information provided here will be used to judge the operational capacity.}

\section{Members of the Consortium}

\subsection{Participants}

\eucommentary{Please provide, for each participant, the following (if available):\\
\begin{itemize}
\item
a description of the legal entity and its main tasks, 
with an explanation of how its profile matches the tasks in the proposal;
\item
a curriculum vitae or description of the profile of the persons, 
including their gender, who will be primarily responsible for carrying 
out the proposed research and/or innovation activities;
\item
a list of up to 5 relevant publications, and/or products, services 
(including widely-used datasets or software), or other achievements 
relevant to the call content;
\item
a list of up to 5 relevant previous projects or activities, connected 
to the subject of this proposal;
\item
a description of any significant infrastructure and/or any major items 
of technical equipment, relevant to the proposed work;
\item
any other supporting documents specified in the work programme for this call.
\end{itemize}}

\cite{science-project}

\subsection{Third Parties Involved in the Project (including use of third party resources)}

\eucommentary{Please complete, for each participant, the table
(see page 27 of "VRETemplate.PDF"), 
or simply state "No third parties involved", if applicable.}

No third parties involved.


% ---------------------------------------------------------------------------
%  Section 5: Ethics and Security
% ---------------------------------------------------------------------------

\newpage

\section{Ethics and Security}

\eucommentary{This section is not covered by the page limit.}

\subsection{Ethics}

\eucommentary{
If you have entered any ethics issues in the ethical issue table in the administrative proposal forms, you must:\\
$\bullet$ submit an ethics self-assessment, which: \\
-- describes how the proposal meets the national legal and ethical requirements of the
country or countries where the tasks raising ethical issues are to be carried out;\\
-- explains in detail how you intend to address the issues in the ethical issues table, in
particular as regards: 
research objectives (e.g. study of vulnerable populations, dual use, etc.), 
research methodology (e.g. clinical trials, involvement of children and related
consent procedures, protection of any data collected, etc.), 
the potential impact of the research (e.g. dual use issues, environmental damage,
stigmatisation of particular social groups, political or financial retaliation,
benefit-sharing, malevolent use , etc.)\\
$\bullet$ provide the documents that you need under national law(if you already have them), e.g.:\\
-- an ethics committee opinion;\\
-- the document notifying activities raising ethical issues or authorising such activities\\
If these documents are not in English, you must also submit an English summary of them
(containing, if available, the conclusions of the committee or authority concerned).\\
If you plan to request these documents specifically for the project 
you are proposing, your request must contain an explicit reference to the project title}

\subsection{Security}

Please indicate if your proposal will involve:

\begin{itemize}
\item
activities or results raising security issues: NO
\item
'EU-classified information' as background or results: NO
\end{itemize}

\newpage

\label{bibliography}
\addcontentsline{toc}{section}{References}

\bibliographystyle{abbrv}
\bibliography{bibliography}

\end{document}
