\section*{\texorpdfstring{Deliverable description, as taken from GitHub
issue
\href{https://github.com/OpenDreamKit/OpenDreamKit/issues/83}{\#83} on
2016-09-07}{Deliverable description, as taken from GitHub issue \#83 on 2016-09-07}}\label{deliverable-description-as-taken-from-github-issues-83-on-2016-09-07}

\begin{itemize}
\tightlist
\item
  \textbf{WP4:}
  \href{https://github.com/OpenDreamKit/OpenDreamKit/tree/master/WP4}{User
  Interfaces}
\item
  \textbf{Lead Institution:} CNRS
\item
  \textbf{Due:} 2016-02-29 (month 6)
\item
  \textbf{Nature:} Other
\item
  \textbf{Task:} T4.12
  (\href{https://github.com/OpenDreamKit/OpenDreamKit/issues/80}{\#80})
\item
  \textbf{Proposal:}
  \href{https://github.com/OpenDreamKit/OpenDreamKit/raw/master/Proposal/proposal-www.pdf}{p.47}.
\end{itemize}

The SageMath project includes many different subsystems, mostly written
in C/C++. For each subsystem, Sage provides a low-level interface,
usually written in Cython, through which the higher level components
access the subsystem. The mathematical community would immensely benefit
if the low-level interfaces were maintained outside of the Sage project,
as separate Python packages. Indeed such decoupling would enable other
Python projects to build upon those externalized interfaces, thus
helping to improve them, and share maintenance effort.

Of all the subsystems, the case of Pari is of particular interest. Since
its inception, Sage has had a low-level Cython interface to Pari, which
has evolved over time in a highly coupled way. Around 2012, some Sage
developers forked this interface into a project they called
\href{https://bitbucket.org/t3m/cypari/}{CyPari}. One of the primary
goals of the CyPari fork was to make independent Python bindings to Pari
for use in a project called
\href{https://bitbucket.org/t3m/snappy}{Snappy}. Over time, the CyPari
fork has diverged from the Sage/Pari interface, and has gotten behind in
terms of functionality.

The goal of this deliverable is to reconcile the fork by externalizing
the Sage/Pari interface into an independent package, maintained by the
Sage community, which may ultimately replace CyPari inside Snappy. The
task happened to be more difficult than originally thought. The high
level of coupling between Sage internals and the Pari interface makes it
very delicate to pull the latter out of the SageMath codebase. The
process of making this possible has lead to a great amount of
refactoring inside the Sage project, which is summarized in
\href{http://trac.sagemath.org/ticket/20238}{Trac ticket 20238}.

Because of the high degree of coupling, and thanks to the availability
of Snappy, this deliverable constitutes a highly valuable case study for
future externalizations of low-level interfaces in SageMath. To bring
this deliverable to completion, we have decided to split it in several
steps:

\begin{itemize}
\tightlist
\item
  \(\checkmark\) Move SageMath's C signalling api to a separate
  Python/Cython package. The package is called
  \href{https://github.com/sagemath/cysignals}{cycsignals}, and is
  \href{http://trac.sagemath.org/ticket/20002}{integrated to SageMath
  7.1}.
\item
  \(\checkmark\) Decouple SageMath's Pari interface from the coercion
  model. This needs review at
  \href{http://trac.sagemath.org/ticket/21158}{Trac ticket 21158} and
  its dependencies.
\item
  {[} {]} Clean up the interface API, by removing unneeded object
  orientation and external dependencies. This work is in progress at
  \href{http://trac.sagemath.org/ticket/20241}{Trac ticket 20241}.
\item
  {[} {]} Move SageMath's Pari interface to a separate Python/Cython
  package, depending on cysignals. The package is planned to replace the
  PyPi package \href{https://pypi.python.org/pypi/cypari/}{CyPari}.
\end{itemize}

The first item is complete and delivered. Progress has been slowed down
by delays in the recruitment process in Université Paris Sud, and by the
aforementioned technical difficulties. The next two items are
approaching completion, we estimate the remaining effort to three more
person/weeks. The last item is very easy, once the previous ones are
completed.
