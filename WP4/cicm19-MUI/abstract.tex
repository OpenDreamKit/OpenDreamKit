There are two distinct traditions in interacting with knowledge in mathematics.
One is to use documents that present mathematical knowledge in the form of definitions, theorems, and proofs, and the other is to compute with mathematical objects.

Mathematical software systems try to support users in both kinds of interactions. Active document systems make documents interactive by hyperlinks, theming, user adaptivity. 
Computer algebra and simulation systems give users access to computational facilities on mathematical objects.

In this paper we show two major steps towards to the integration of these approaches.  
Firstly we present a Jupyter kernel for MMT, which provides the functionality of the MMT knowledge representation system in Jupyter, a uniform interface to computation facilities in the form of a Read-Eval-Print Loop.
Secondly, we show how to combine the advantages of Jupyter Notebooks and MathHub documents:
We also show how Jupyter widgets can be deeply integrated with the MMT knowledge management facilities to give semantics-aware interaction facilities.
In particular, we dynamically employ the highly interactive and often ephemeral Jupyter Notebooks as subdocuments of MathHub documents such as static HTML pages generated from scientific articles.

%%% Local Variables:
%%% mode: latex
%%% mode: visual-line
%%% fill-column: 5000
%%% TeX-master: "paper"
%%% End:

%  LocalWords:  Jupyter MitM-based textbf textbf textbf
