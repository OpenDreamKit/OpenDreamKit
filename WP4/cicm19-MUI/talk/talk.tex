\documentclass{beamer}
\usetheme{Madrid}

\usepackage{listings}

\AtBeginSection[]{
  \begin{frame}
  \vfill
  \centering
  \begin{beamercolorbox}[sep=8pt,center,shadow=true,rounded=true]{title}
    \usebeamerfont{title}\insertsectionhead\par%
  \end{beamercolorbox}
  \vfill
  \end{frame}
}

\makeatletter
\setbeamertemplate{footline}{%
  \leavevmode%
  \hbox{%
  \begin{beamercolorbox}[wd=.2\paperwidth,ht=2.25ex,dp=1ex,center]{author in head/foot}%
    \usebeamerfont{author in head/foot}\insertshortauthor\expandafter\beamer@ifempty\expandafter{\beamer@shortinstitute}{}{~~(\insertshortinstitute)}
  \end{beamercolorbox}%
  \begin{beamercolorbox}[wd=.4\paperwidth,ht=2.25ex,dp=1ex,center]{title in head/foot}%
    \usebeamerfont{title in head/foot}\insertshorttitle
  \end{beamercolorbox}%
  \begin{beamercolorbox}[wd=.4\paperwidth,ht=2.25ex,dp=1ex,right]{date in head/foot}%
    \usebeamerfont{date in head/foot}\insertshortdate{}\hspace*{2em}
    \insertframenumber{} / \inserttotalframenumber\hspace*{2ex} 
  \end{beamercolorbox}}%
  \vskip0pt%
}
\makeatother


%Packages
\usepackage[utf8]{inputenc}
\usepackage{hyperref}
\usepackage{amssymb}

%META-INFORMATION
\title[Semantic Documents and Dynamic Notebooks]{Integrating Semantic Mathematical Documents and Dynamic Notebooks}
\author[Tom Wiesing et al.]{Kai Amann, Michael Kohlhase, Florian Rabe, and \underline{Tom Wiesing}\\Computer Science, FAU Erlangen-N{\"u}rnberg}
\date[July 12 2019, CICM Prague]{July 12, 2019\\Conference on Intelligent Computer Mathematic\\Prague, Czech Republic}

\begin{document}
    %TITLEPAGE
    \frame{\titlepage}
    
    \section{Introduction}

    \begin{frame}{Documents}
        \begin{columns}
        \column{0.7\textwidth}
            \begin{itemize}
                \item \textbf{Documents} guide the user through a particular topic
                \begin{itemize}
                    \item Papers
                    \item Talks
                    \item Books
                    \item Websites
                \end{itemize}
                \item Useful for \textbf{presenting} knowledge
                \item Caveat: Documents are usually \textbf{static}
                \begin{itemize}
                    \item they do not allow interaction with this knowledge
                \end{itemize}
                \item Focus on \textbf{Mathematical Documents} in this talk
                \begin{itemize}
                    \item can be generalized to all of \textbf{STEM}
                \end{itemize}
            \end{itemize}
                \column{0.3\textwidth}
            \includegraphics[scale=0.25]{images/doc}
        \end{columns}
    \end{frame}

    \begin{frame}{What is a REPL?}
        \begin{columns}
        \column{0.5\textwidth}
            \begin{itemize}
                \item REPL = \textbf{R}ead \textbf{E}val \textbf{P}rint \textbf{L}oop
                \begin{enumerate}
                    \item\label{replbegin} Read a Command from the user
                    \item Evaluate the Command
                    \item Print the Result
                    \item Loop back to \ref{replbegin}
                \end{enumerate}
                \item commonly used for programming or command line interaction
                \begin{itemize}
                    \item e.g. Bash (Linux / Mac OS) or PowerShell (Windows)
                    \item e.g. Python (scientific programming)
                \end{itemize}
            \end{itemize}
        \column{0.5\textwidth}
            \includegraphics[scale=0.25]{images/repl}
        \end{columns}
    \end{frame}

    \begin{frame}{Notebooks + Jupyter}
        \begin{columns}
        \column{0.4\textwidth}
            \begin{itemize}
                \item Notebooks build on the REPL paradigm
                \item consist of a set of cells, each either have some code or documentation
                \item an OpenSource implementation \textbf{Jupyter} System
                \begin{itemize}
                    \item supports different programming languages using \textbf{kernels}
                \end{itemize}
                \item Interactive, but \textbf{requires programming}
            \end{itemize}
        \column{0.6\textwidth}
            \includegraphics[scale=0.2]{images/notebook}
        \end{columns}
    \end{frame}

    \begin{frame}{OMDOC and MMT and MathHub}
        \begin{itemize}
            \item \textbf{OMDoc} = format for \textbf{O}pen \textbf{M}athematical \textbf{Doc}uments
            \begin{itemize}
                \item format for encoding STEM documents and knowledge
                \item developed mainly by \textit{Michael Kohlhase}
                \item can handle formal, informal and flexiformal content
                \begin{itemize}
                    \item formal = e.g. formalized proof, well-typed program, formal library
                    \item informal = e.g. paper, textbook, presentation
                    \item flexiformal = anything in-between, e.g. informal document with well-annotated formulae in between
                \end{itemize}
            \end{itemize}
            \item \textbf{MMT} = Meta-Meta-Tool
            \begin{itemize}
                \item framework for knowledge representation, implemented in Scala
                \item developed mainly by \textit{Florian Rabe}
                \item avoids a specific representational paradigm and is language-independent
                \item makes use of OMDOC, can thus handle formal, informal and flexi-formal documents
            \end{itemize}
        \end{itemize}
    \end{frame}

    \begin{frame}{MathHub}
        \begin{columns}
        \column{0.4\textwidth}
            \begin{itemize}
                \item \textbf{MathHub} \url{https://mathhub.info/}
                \begin{itemize}
                    \item portal for active mathematical documents and an archive for flexiformal mathematics
                    \item uses MMT as a backend and OMDoc as a representational format
                \end{itemize}
            \end{itemize}
        \column{0.6\textwidth}
            \includegraphics[scale=0.2]{images/mathhub}
        \end{columns}
    \end{frame}

    \begin{frame}{Goals and Challenges}
        \begin{itemize}
            \item \textbf{Idea:} Combine interactive notebooks and static, semantically-annotated documents to get interactivity with little programming
            \begin{itemize}
                \item How can we combine the notebook and document paradigms?
                \item How can we support flexible interactions without forcing authors to pro-
gram?
            \end{itemize}

            \item Our solution
            \begin{itemize}
                \item Make use of \textbf{OMDOC} as it can represent flexi-formal documents
                \item Import them into MMT
                \item Build a \textbf{Jupyter Kernel} for an MMT REPL
                \item Expand this Kernel to support Widgets \textit{(more on what exactly those are later)}
                \item Make the result accessible directly from within documents
            \end{itemize}

            \item This talk: Key design choices + overview of the implementation + some examples
            \begin{itemize}
                \item \textit{Read the paper for more details}
            \end{itemize}
        \end{itemize}
    \end{frame}

    \begin{frame}{Overview}

        \begin{itemize}
            \item This talk: Key design choices + overview of the implementation + some examples
            \begin{itemize}
                \item \textit{Read the paper for more details}
            \end{itemize}
        \end{itemize}

        \begin{enumerate}
            \item Introduction
            \item A REPL for MMT
            \item Widgets for active documents
            \item Conclusion
        \end{enumerate}
    \end{frame}


    \section{A REPL for MMT}

    \begin{frame}{Architecture of Jupyter Kernels}
    \end{frame}

    \begin{frame}{MMT and Theory Graphs}
    \end{frame}

    \begin{frame}{MMT Kernel Architecture 1: Overview}
    \end{frame}

    \begin{frame}{MMT Kernel Architecture 2: Messages}
    \end{frame}

    \begin{frame}{An example: Writing MMT content using Jupyter}
    \end{frame}

    \section{Widgets for Active Documents}

    \begin{frame}{An Introduction to Widgets}
        \begin{itemize}
            \item widgets = \textbf{GUI component} that allows Jupyter kernels to provide graphical interfaces
            \begin{itemize}
                \item e.g. slider, input field, ouput field, ...
                \item go \textbf{beyond} the REPL paradigm
            \end{itemize}

            \item can be plugged together interactively
            \item consist of kernel-side and frontend code
            \begin{itemize}
                \item frontend contains HTML, CSS + JavaScript
                \item kernel holds the state of the widget
            \end{itemize}
            \item Kernels can implement custom widgets by plugging together existing ones
        \end{itemize}
    \end{frame}

    \begin{frame}{An example: The active computation widget}
        \includegraphics[scale=.5]{images/activecomp}
        \begin{itemize}
            \item the user can enter a \textbf{term} and a \textbf{set of variables}
            \item widget allows to change the variables and compute the result
        \end{itemize}
    \end{frame}

    \begin{frame}{MMT Kernel Architecture 3: Widgets}
        \centering
        \includegraphics[scale=.35]{images/widgets}
    \end{frame}

    \begin{frame}[fragile]{Widgets inside Static Documents}
        \begin{itemize}
            \item \textbf{Idea: } Make this usable directly from documents
            \begin{itemize}
                \item We need to know \textbf{the formula} that we want to interact with
                \begin{itemize}
                    \item Let the user click on it
                \end{itemize}
                \item We need to know the \textbf{context} and \textbf{variables}
                \begin{itemize}
                    \item Annotate them on the document
                \end{itemize}
            \end{itemize}
        \end{itemize}

        \begin{columns}
        \column{0.35\textwidth}
            \centering
            \includegraphics[scale=0.25]{images/doc}
        \column{0.65\textwidth}
            \input{html.tex}
        \end{columns}
    \end{frame}

    \begin{frame}[fragile]{Widgets inside Static Documents (2)}
        \includegraphics[scale=0.5]{images/acwidget}
    \end{frame}


    \section{Conclusion}

    \begin{frame}{Conclusion}
        \begin{itemize}
            \item We have presented an integration of Documents and Notebooks
            \begin{itemize}
                \item based on \textbf{OMDoc} and \textbf{MMT}
                \item consists of a \textbf{Jupyter Kernel} with widget support
                \item initial integration into \textbf{Active Documents}
            \end{itemize}

            \item We have only given an overview here, and shown a few examples
            \begin{itemize}
                \item Read the paper for more details!
            \end{itemize}

            \item Still a lot of work left!
            \begin{itemize}
                \item Deep MathHub/Jupyter Integration
                \item IDE Support for Documents with Active Computation
                \item REPL Cells/Documents as first-class citizens in MMT
                \item and more $\dots$
            \end{itemize}

            \item Questions, Comments, Concerns?

            \item Thank You For Listening!

        \end{itemize}
    \end{frame}

\end{document}