\section*{\texorpdfstring{Deliverable description, as taken from Github
issue
\href{https://github.com/OpenDreamKit/OpenDreamKit/issues/91}{\#91} on
2017-01-09}{Deliverable description, as taken from Github issue \#91 on 2017-01-09}}\label{deliverable-description-as-taken-from-github-issue-91-on-2017-01-09}

\begin{itemize}
\tightlist
\item
  \textbf{WP4:}
  \href{https://github.com/OpenDreamKit/OpenDreamKit/tree/master/WP4}{User
  Interfaces}
\item
  \textbf{Lead Institution:} Jacobs University Bremen
\item
  \textbf{Due:} 2016-02-29 (month 6)
\item
  \textbf{Nature:} Report
\item
  \textbf{Task:} T4.6
  (\href{https://github.com/OpenDreamKit/OpenDreamKit/issues/74}{\#74})
\item
  \textbf{Proposal:}
  \href{https://github.com/OpenDreamKit/OpenDreamKit/raw/master/Proposal/proposal-www.pdf}{p.~47}
\item
  \textbf{\href{https://github.com/OpenDreamKit/OpenDreamKit/raw/master/WP4/D4.2/report-final.pdf}{Final
  report}}
\end{itemize}

One of the most prominent features of a virtual research environment
(VRE) is a unified user interface. The OpenDreamKit approach is to
create a mathematical VRE by integrating various pre-existing
mathematical software systems. There are two approaches that can serve
as a basis for the OpenDreamKit UI: \emph{computational notebooks} and
\emph{active documents}. The former allow mathematical text around the
computation cells of a real-eval-print loop of a mathematical software
system and the latter make semantically annotated documents semantic.

We report on two systems in the OpenDreamKit project: Jupyter -- a
notebook server for various kernels, and MathHub.info -- a platform for
active mathematical documents. We identify commonalities and differences
and develop a vision for integrating their functionalitities.

Related projects:

\begin{itemize}
\tightlist
\item
  MathBookXML: \url{http://mathbook.pugetsound.edu/}
\item
  Jupyter notebook exporter for Sphinx:\\
  \url{https://github.com/sphinx-doc/sphinx/pull/2117}
\item
  Jupyter javascript plugin for static sites:\\
  \url{https://github.com/oreillymedia/thebe}\\
  See also: \url{https://www.oreilly.com/ideas/jupyter-at-oreilly}
\item
  ReST to IPython Notebook converter through pandoc and markdown:\\
  \url{https://github.com/nthiery/rst-to-ipynb/}
\end{itemize}
