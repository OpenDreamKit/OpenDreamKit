\section{Examples of In-Situ Computation}\label{sec:examples}

In the following we will look at some examples to get a feeling for the applications
\subsection{Unit Conversions}\label{sec:units}
\ednote{Ulrich has some text on this}

\subsection{Equations}\label{sec:equations}

A common example of in-situ computation is the exploration of mathematical models that are
given as equations. In the simplest case, this can be equations like Einstein's
mass-energy equivalence (\ref{f:emc}) -- which we will use as a running example -- and in
other cases, this can be complex models like van Roosbroeck's models for drift and
diffusion of eletrons and holes in semiconductor devices~\cite{Kopruckietal}\footnote{We
  are currently studying this model, formalizing the inherent knowledge and augmenting
  (parts of) \cite{Kopruckietal} into an active document, see~\cite{KohKopMueTab:RCS} for
    first results. The methods reported on here will be employed in this case study, which
    itself is beyond the scope of this deliverable report}, which comprises partial
  differential equations, boundary conditions, and physical constants --

\begin{equation}\label{f:emc}
  E=mc^2
\end{equation}

where $E$ is the energy, $m$ is the mass, and $c$ is the speed of light. There we might be
interested to see what the energy equivalent of 1 gram of matter might be. So we would
like to just right click on the $m$, instantiate it to $1g$ and simplify the
expression. Conversely, we might want to know how many grams of matter it would take to
drive from Erlangen to Paris. So we would instantiate $E$ with $122 kWh$ and solve
$122kWh=mc^2$ for $m$. Actually since we end up with $10^-{???}$ grams we would directly
convert the quantity expression to \ednote{continue}.

\subsection{Computation with Proofs}
\begin{itemize}
  \item calling automated theorem provers on a goal in a document
  \item extending the level of explanation by doing that on a subgoal or deepening the
    level of explanation. E.g. from ``obviously'' to a full proof.  
  \end{itemize}

\subsection{Playing with global/local values}
What would be paper look like if the speed of light were $88 mph$. 

\subsection{Updating Values to current or historical values}
spreadsheets (a well-understood form of active documents) can already do that. 
\begin{itemize}
\item global warming papers with newer models or data
\item Wolfram alpha: ``does it snow in hell?''
\end{itemize}




%%% Local Variables:
%%% mode: latex
%%% TeX-master: "report"
%%% End:
