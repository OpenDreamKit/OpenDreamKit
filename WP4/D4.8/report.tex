\documentclass{deliverablereport}

\usepackage[style=alphabetic,backend=bibtex]{biblatex}
\addbibresource{../../lib/kbibs/kwarcpubs.bib}
\addbibresource{../../lib/kbibs/extpubs.bib}
\addbibresource{../../lib/kbibs/kwarccrossrefs.bib}
\addbibresource{../../lib/kbibs/extcrossrefs.bib}
\addbibresource{../../lib/deliverables.bib}
%\addbibresource{../../lib/publications.bib}
\addbibresource{rest.bib}
% temporary fix due to http://tex.stackexchange.com/questions/311426/bibliography-error-use-of-blxbblverbaddi-doesnt-match-its-definition-ve
\makeatletter\def\blx@maxline{77}\makeatother

\deliverable{UI}{jupyter-test}
\deliverydate{02/27/2017}
\duedate{02/27/2017 (Month 18)}
\def\pn{OpenDreamKit}
\author{Martin Sandve Aln\ae{}s \& Hans Fangohr \& Vidar Fauske \& Thomas Kluyver \& Benjamin Ragan-Kelley \& MORE?}

\begin{document}
\maketitle
%There is a large ecosystem of mathematical software systems and knowledge bases.
Individually, these are optimized for particular domains and functionalities, and together they cover many needs of practical and theoretical mathematics.
However, each system specializes on one particular area, and it remains very difficult to solve problems that need to involve multiple systems.
Some integrations exist, but they are ad-hoc and have scalability and maintainability issues.
In particular, there is not yet an interoperability layer that combines the various systems into a virtual research environment (VRE) for mathematics.
  
The OpenDreamKit project aims at building a toolkit for such VREs.
It suggests using a central system-agnostic formalization of mathematics (Math-in-the-Middle, MitM) as a mathematical pivot point for semantic-preserving translations in the needed interoperability layer.
In this \papertype, we report on a series of case studies that instantiates the MitM paradigm with the systems \GAP, \Sage, \LMFDB, and \Singular to perform distributed computation in group, ring, and number theory.
 
Our work involves massive practical efforts, including a novel formalization of computational group theory, improvements to the involved software systems, an extension of the underlying knowledge management system to cope with large theories, and a novel mediating system that sits at the center of a star-shaped integration layout between mathematical software systems and knowledge bases.

Together with deliverable report\textbf{D6.8}, this report describes the implementation and initial evaluation of the MitM integration and interoperability paradigm initially envisioned in deliverables \textbf{D6.2} and \textbf{D6.3}.
The MitM paradigm constitutes the core development goal of \textbf{WP6} and the curated content described in this report enables running non-trivial integration case studies.
In the future we hope to further consolidate content, increase coverage of alignment, and greatly extend the reach of the integration both interms of OpenDreamKit systems covered as well as knowledge available in the MitM Core ontology.

%%% Local Variables:
%%% mode: visual-line
%%% fill-column: 5000
%%% mode: latex 
%%% TeX-master: "report"
%%% End:

%  LocalWords:  optimized formalization papertype textbf textbf textbf

\strut\githubissuedescription
\newpage\tableofcontents\newpage

\newcommand{\nbval}{\texttt{nbval} }

\section{Background on validation} % (fold)

Jupyter notebook documents have become an important part of the development and communication of computational ideas.
While these documents contain code and the output produced by running that code,
they do not guarantee that the code remains valid in changing circumstances or that it continues to function after updates to underlying packages.
These are common problems in all software, but existing testing frameworks do not support notebooks.
Because notebooks contain both input and output, there are additional opportunities and challenges for testing notebooks,
because outputs can be compared to previous runs.
Further, Jupyter notebooks aim to be a tool for enabling reproducible computation and communication,
and the ability to validate and verify the contents of notebooks is critical to that goal.
For these reasons, it is important that notebooks can be tested efficiently,
so that authors and readers alike can verify that the notebooks contents remain valid.


\section{Validating notebooks with \nbval} % (fold)

\begin{figure}[ht]
  \centering
  \includegraphics[width=.7\textwidth]{img/nbval-terminal}
  \caption{\nbval output, showing that output changed.}\label{fig:nbval}
\end{figure}

\nbval (\url{https://github.com/computationalmodelling/nbval}) is a new tool for testing notebooks, built as a plugin for the pytest testing framework (\url{http://pytest.org}).
By leveraging existing tools,
\nbval fits well into the software development tools such as continuous integration services and testing environments.

When \nbval encounters a notebook to test, it identifies the language of the notebook from the notebook's metadata and starts a process to run the code found in notebook, called the Kernel.
\nbval communicates with this Kernel via the Jupyter protocol, as in the notebook environment.
Each cell in a notebook constitutes a test, and is executed in order,
as if the notebook had been executed in the interactive notebook environment.

Unlike traditional source code files,
notebooks contain both code to execute and the output.
Validating notebooks can take the output into account or not.
At its most basic level, called `lax mode', \nbval executes a cell, only checking for errors.
This verifies that execution of a notebook completes without error,
but makes no effort to guarantee that the result is the same across executions.

\nbval's default mode is to record the output produced by executing the notebook,
and compare it with the output stored in the notebook.
At its strictest, any visible change in the output will result in a failed test.
Many times, output can contain transient values that are not informative,
such as memory addresses or dates.
To deal with this, \nbval provides an extensible mechanism for normalizing output,
so that these transient values may be excluded from the comparison.
Developers 


\section{nbval and reproducibility} % (fold)

\nbval facilitates integrating notebooks into a reproducible scientific workflow.
Tests are integral to maintaining and verifying software,
which is critical for validating and reproducing scientific computation.
A publication can include a notebook that produces its results or figures.
By using \nbval, the validation of this notebook and output can be automated,
to make it more convenient, and thus more likely,
for scientific publications to follow reproducible practices.


\section{nbval and nbdime} % (fold)

Testing notebooks with \nbval involves comparing the notebook as saved,
and the notebook recently run.
This is comparing two notebooks,
which can build on earlier OpenDreamKit work.
\nbval can use nbdime, produced in \delivref{UI}{jupyter-collab},
to display the difference between the before and after notebooks,
for more effective comparison and identification of changes.

\begin{figure}[ht]
  \centering
  \includegraphics[width=.7\textwidth]{img/nbval-nbdime}
  \caption{An \nbval output rendered with nbdime}\label{fig:nbval-nbdime}
\end{figure}

\section{Future work} % (fold)

\nbval has been integrated successfully into some repositories of notebook-based educational materials by OpenDreamKit participants,
and is being integrated into the existing notebook-based documentation of projects in the wider Jupyter community.
We will work to encourage adoption of \nbval for verifying documentation,
and would like to see nbval used to enable verification of scientific publications
now that it has proved its effectiveness in educational materials.

\printbibliography
\end{document}

%%% Local Variables:
%%% mode: latex
%%% TeX-master: t
%%% End:

%  LocalWords:  githubissuedescription newpage tableofcontents newpage printbibliography
