\section{Introduction}\label{sec:intro}

\ednote{MK: more intro blabla about narration and computation -- can also use this for the
  one of the tetrapod edges.}

We have proposed ``Active Structured Documents'' as a natural interface for interacting
with mathematical knowledge for the working mathematician and thus as a UI component for
the \pn Virtual Math Research Environment Toolkit~\cite{ODK-D4.2}.

In a nutshell, active documents are documents which make aspects of the meaning of their
contents explicit enough that it becomes machine-actionable in a document player that
delivers services -- in our case for computation -- that can be triggered 

In~\cite{KohDavGin:psewads11} we
present a syste

 \ednote{MK: continue, cite
  ADP\cite{KohDavGin:psewads11}, put the ADP picture here.}

In the long run, we propose to integrate active structured documents with the Jupyter
notebooks, and as a first step into this direction we explore the requirements of
integrating \emph{in-situ} (i.e. in-document) \emph{computation} -- a forte and the indeed
the reason-d'etre of notebooks -- into conventional, narrative-structured mathematical
documents; Section~\ref{sec:examples} presents, analyzes, and classifies examples for
in-situ computation.  We also explore how the active documents technology has to be
extended to accomodate this functionality as a semantic service -- see
Section~\ref{sec:infarch} for details.\ednote{MK: also an implementation section? We have
  to put the screenshots somewhere} Finally Section~\ref{sec:concl} concludes the report
and gives diretions of future research and development.

%%% Local Variables:
%%% mode: latex
%%% TeX-master: "report"
%%% End:
