\section{Distributed, Collaborative, Versioned Editing}\label{sec:editing}

For the \pn project, we have developed and are hosting on \sys
\begin{compactenum}
\item the ``Math-in-the-Middle ontology''~\cite{MitM:on}, which hosts a flexiformal
  development of the mathematical knowledge for system interoperability in \pn
  (see~\cite{DehKohKon:iop16}, \delivref{dksbases}{design}),
\item the ``ODK System ontologies''~\cite{ODKsysonto:on}, a collection of \omdoc/\mmt
  theories and codecs that describe the mathematical content of (parts of) the
  LMFDB~\cite{lmfdb:on}, OEIS~\cite{oeis} and other data sources, bridging the internal
  (i.e. system/database oriented) and external (mathematical) views of the content.
\end{compactenum}
These two \sys libraries are in an early state currently and curating them is a major task
in \pn's \longWPref{dksbases}. Therefore, the editing workflows we
report on here are crucial to the \pn project.

Given the distributed and collaborative nature of the \pn project, the editing facilities
need to be as well. As experience in software engineering shows -- and flexiformal active
documents are like software in many respects -- this is only possible using (concepts of)
revision control systems. In \sys we took the major design decision to use the distributed
revision control system GIT~\cite{GIT:on} as the basis for all general storage,
authentification, distribution, and collaboration functionalities and build
domain-specific functionality on top of that. We took the minor design decision to use the
GitLab~\cite{GitLab:on} system as the repository management system -- rather than e.g.
GitHub~\cite{GitHub:on}, since it is open source and allows us to run our own server and
configure it more by patching the code.

We organize the content into \textbf{libraries} by area or intended application --
e.g. the two libraries discussed at the top of this section and further divide them up
into \textbf{math archives}~\cite{HorIacJuc:cscpnrr11}, which standardize the file system
layout of all dimensions of mathematical representations: source, content, presentation,
narration, etc. At the GitLab level, libraries are modeled as ``groups'' and individual
math archives as repositories. As GIT -- and thus repository managers like GitLab -- only
allow authentification at the repository level, math archives are mostly used for
authentification and access control in \sys.

The advantage of the GIT-based setup is that we can combine two methods for accessing the
contents of \sys:
\begin{compactenum}[\em i\rm)]
\item an online, web-based editing/interaction workflow for the casual user, in the spirit
  of the Planetary system and
\item an offline editing/authoring workflow based on a GIT working copy.
\end{compactenum}
We will describe both separately in sections~\ref{sec:lmh} and \ref{sec:web}, after
clarifying the setting in \sys a bit more.

\subsection{Building a Math Knowledge Base by Editing Surface Language
  Documents}\label{sec:surface}

The unifying representation format of the \sys system is flexiformal \omdoc/\mmt, which is
structured as a theory graph. As \omdoc/\mmt is optimized for machine processing, actual
content is authored and edited as documents in a \textbf{surface format}: a human-oriented
syntax that can be compiled into \omdoc/\mmt. \sys currently supports five surface formats 
\begin{enumerate}
\item HTML5 and {\TeX/\LaTeX} (as a minimally flexiformal document formats)
\item \sTeX (semantic {\TeX/\LaTeX}), an extension of {\LaTeX} that allows to annotate
  {\LaTeX} documents with semantic properties and relations.
\item \mmt surface syntax.
\item TWELF (Edinburgh Logical Framework in TWELF syntax)
\end{enumerate}
Additionally, the native formats of the theorem prover libraries imported into MathHub are
handled by special importers on a system-by-system basis. Only \sTeX and \mmt syntax are
relevant for the \pn project, so we will ignore the others in this report. In all cases,
the conversion to \omdoc/\mmt is a multi-step, multi-document, dependency-driven process,
which is handled by the \mmt build system~\cite{mmt:buildsys:on} (see the upper right hand
corner in Figure~\ref{fig:arch}). The \sys system uses a separate build system process
that pulls changes from the \sys GitLab server, re-builds any affected files, and pushes
the results to the \sys GitLab server again. From that the \sys system can pull the new
state of the libraries and update the local \mmt API.

\subsection{Local \sys Editing}\label{sec:lmh}

The local editing workflow is important for power authors who want to edit more than one
file simultaneously or have customized modes for the surface languages
in their preferred text
editors. A user can fork or pull the relevant repositories from the \sys GitLab, edit them
and submit them back to \sys either via a pull request to the repository masters or a
direct commit/push. As the content is usually highly networked and distributed across
multiple math archives (and thus GIT repositories), we have developed a command line tool
\lmh (\underline{l}ocal \underline{M}ath\underline{H}ub; see~\cite{lmh:on}
and~\cite{lmh-docker:on} for a docker container that bundles up all system requirements
for \lmh) that manages working copies across repository borders taking into account e.g.
cross-archive dependencies.  \lmh also supports running the build system locally and
previewing HTML5 renderings of the generated \omdoc/\mmt.

The concrete editing support for flexiformal, active mathematical documents crucially
depends of the depth of formalization. Completely informal mathematical documents are
traditionally written in {\LaTeX} and the traditional authoring tools are well-suited for
this task. Any cross-repository effects can be handled by \lmh, so we will discuss more
semantic document formats next.

\subsection{Local Editing Support for Lightly Formalized Mathematical Content}

\sTeX~\cite{Kohlhase:ulsmf08,sTeX:github:on} is an extension of the {\TeX/\LaTeX} that
allows the annotation of semantic properties into the document source. These annotations
are largely invisible in the PDF output, but can be transferred into \omdoc/\mmt when
formatted with the \sTeX plugin to {\LaTeX}ML~\cite{GinStaKoh:latexmldaemon11}. We call
the process of adding such source annotations \textbf{semantic preloading}.

Both formatting pipelines highlight different features of the lifecycle of flexiformal
documents. Usually, the content is initially developed with casual preloading -- whatever
is convenient while authoring -- when first writing the original material. This is then
formatted to PDF, which is checked and iterated for mathematical and visual
correctness. Even though we have a long history of supporting surface language editing in
editor-based IDEs~\cite{JucKoh:sidesc10,,JucEth12:redsys}, the most important practical
aspects of knowledge curation in \sys are well-handled by the \lmh tool, which provides
archive curation functionalities such as basic link checking, pre-translation, and build
system support. An advantage of this approach is that it is largely editor-independent and
thus does not constrain the user in their choice of tools.

\begin{figure}[ht]\centering
  \includegraphics[width=\textwidth]{errorview}
  \caption{The \sys Error Viewer}\label{fig:errorview}
\end{figure}

In a second phase, the authors fully preload the documents semantically. In this phase,
the content is built into \omdoc/\mmt, loaded into the \mmt API and checked for semantic
link-consistency. In our experience the most important ``semantic editing service'' is to
track the errors encountered during these semantic checks. While it would be helpful to
have editor/IDE integration for the ensuing semantic debugging process, the main
advantage\footnote{This effect may however be partially caused by the fact that we
  currently still also have errors in the transformation pipeline, which necessitates full
  rebuilding of the \sys libraries interleaved with source debugging.}  can already be
reaped by a special, error aggregating viewer.  Figure~\ref{fig:errorview} shows a
fragment, where we see source errors (the first two) and error messages probably caused by
compiler problems. In both cases, the user can click on a particular occurrence and
enter a web-based editing cycle which gives access to the logs and results of the
respective transformation steps. In particular, source errors can directly be fixed using the web
editing facilities (see Section~\ref{sec:web}).

% Nicolas: I want this feature in github pages :-)

\subsection{Local Editing Support for Formal Mathematics}\label{sec:local-formal}

Editing formal mathematical content poses a different set of challenges. As the content is
formal (i.e expressed in a formal language with well-defined semantics), its syntax and
semantics are fully machine-checkable, and this defines the standards of mathematical
quality. Consequently, an editing process that integrates syntax/semantics-checking --
which takes the form of type-checking in \mmt -- is needed for editing formal mathematics.

Figure~\ref{fig:jedit2} shows a JEdit-based IDE for \omdoc/\mmt content in \mmt surface
syntax that is tightly coupled with the \mmt API for semantic services;
see~\cite{Rabe:LII14} for details. In a nutshell, we can see the structural view of the
files in the panel on the left and the source file in the central pane. Both are
cross-linked for better structural and source navigation. In this example we also have a
type checking error in line 51 of the file \texttt{algebra.mmt}. It is indicated by the
red underline and by hovering the mouse over it gives the local error message, and the
error console at the bottom give additional details.

\begin{figure}[ht]\centering
  \includegraphics[width=\textwidth]{jedit2}
  \caption{Local Editing of \mmt Surface Syntax in JEdit}\label{fig:jedit2}
\end{figure}

In our experience, a dedicated, tightly integrated IDE is indispensable for the
development of formal mathematical content. This is analogous to the development of
software -- another form of formal content -- but for formal mathematical documents
type-checking -- which includes proof checking via the Curry-Howard isomorphism -- puts
even more semantic constraints on the edited material and thus an even heavier strain on
the author/maintainer.

\todo{specify what part of this section, if anything, has been
  achieved as part of this deliverable}

\subsection{Web-based \sys Editing}\label{sec:web}

\begin{figure}[ht]\centering
  \fbox{\includegraphics[width=.52\textwidth]{bijective}}\quad
  \fbox{\includegraphics[width=.42\textwidth]{bijective-source}}
  \caption{Presentation and Source Views of A Glossary Entry}\label{fig:bijective}
\end{figure}

In the web-based workflows, we have the same requirements for integrating editing and
semantic correctness management as in the local workflows. In the current state of
development of \sys, we aim for more granularity of integration: we integrate at the
document level, which is commensurate with the fact that the quest for reusability in
active documents leads to small documents overall. Figure~\ref{fig:bijective} shows the
presentation of (the english language binding of) a glossary module for bijectivity. The
breadcrumbs at the top localize it in the \texttt{sets} of archive in the \textsf{SMGloM}
library. The content presentation shows its dependencies and a human-oriented presentation
of the definition of bijectivity. There are five tabs with different forms of the same
content, the source view (expanded on the right of Figure~\ref{fig:bijective}), the
generated \omdoc, and an edit tab. The latter is just a link to the corresponding
GitLab edit page; see Figure~\ref{fig:bijective-edit}.

\begin{figure}[ht]\centering
  \includegraphics[width=\textwidth]{bijective-edit}
  \caption{Editing a Glossary Entry in GitLab}\label{fig:bijective-edit}
\end{figure}

This delegation is commensurate with the division of labor in the \sys system; GitLab
suports the full range of revision control interactions. In this case, users with
sufficient access rights can directly commit to the repository (and thus change the math
archive directly) users with less rights can directly initiate a pull request and submit
their changes to the archive maintainers in this way. 

But this delegation is not without technical challenges. An important concern is that the
user identities of the Drupal and GitLab instances have to be synchronized. Fortunately,
users can log into both systems with their GitHub account, which solves the problem at
least for the \pn project, where GitHub is universally employed and thus all partners
have GitHub accounts. Another problem is that GitLab only offers editing facilities for
files it considers text files, and it considers \mmt syntax files as binary, since they
contain lower ASCII symbols for component, declaration, and module separators.
Unfortunately, the media types are not configurable in GitLab, so we have to patch the
code. Building on an open-source system like GitLab and not the closed-source GitHub
allows to do that, but this still poses administration hassles with the fast-moving
release cycle of GitLab.

%%% Local Variables:
%%% mode: latex
%%% TeX-master: "report"
%%% End:

%  LocalWords:  sec:editing MitM:on DehKohKon:iop16,ODK-D6.2 ODKsysonto:on omdoc textbf
%  LocalWords:  GIT:on GitLab:on GitHub:on HorIacJuc:cscpnrr11 standardize ldots sec:lmh
%  LocalWords:  sec:web sec:surface optimized fig:arch lmh ocal lmh:on lmh-docker:on oeis
%  LocalWords:  Formalized Kohlhase:ulsmf08 GinStaKoh:latexmldaemon11 JucKoh:sidesc10
%  LocalWords:  JucEth12:redsys centering includegraphics fig:errorview fig:jedit2 jedit2
%  LocalWords:  JEdit-based JEdit lmfdb:on sec:local-formal fbox fig:bijective textsf
%  LocalWords:  synchronized
