\begin{participant}[type=leadPI, gender=male, PM = 8]{\bf Hans Petter Langtangen}
 is director of Center for Biomedical Computing at Simula Research Laboratory, a Norwegian Center of Excellence doing inter-disciplinary research in the intersection of mathematics, physics, computer science, geoscience and medicine. Langtangen is on 80\% leave from a position as professor at the Department of Informatics, University of Oslo.

Langtangen received his PhD from the Department of Mathematics, University of Oslo, in 1989, and then worked at SINTEF before being hired as assistant professor at the University of Oslo in 1991. After being promoted to full professor of mechanics at the Department of Mathematics in 1998, he moved in 1999 to a professorship in computer science. In the period 1999-2002 he also held an adjunct professor position at the Department of Scientific Computing at Uppsala University in Sweden. The Simula Research Laboratory was formed in 2001, and Langtangen has since then worked with research and management at this laboratory. The scientific computing activity at Simula has been awarded the highest grade, Excellent, by five panels of top-ranked international scientists in the period 2001-2012.

Langtangen's research is inter-disciplinary and involves continuum mechanical modeling, applied mathematics, stochastic uncertainty quantification, and scientific computing, with applications to biomedicine and geoscience in particular. He has also been occupied with developing and distributing scientific software to make the research results more widely accessible and help accelerate research elsewhere. For over three decades he has been very active with teaching and supervision.

The scientific production consists of 4 authored books, 3 edited books, about 60 papers in international journals, about 60 peer-reviewed book chapters and conference papers, and over 130 scientific presentations. The publications cover fluid flow, elasticity, wave propagation, heat transfer, finite element methods, uncertainty quantification, and implementation techniques for scientific software. Langtangen is on the editorial board of 7 journals and serves as Editor-in-Chief of the leading SIAM Journal on Scientific Computing. He is also a member of the Norwegian Academy of Science and Letters.

\end{participant}
