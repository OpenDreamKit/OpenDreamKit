\eucommentary{Milestones means control points in the project that help to chart progress. Milestones may
correspond to the completion of a key deliverable, allowing the next phase of the work to begin.
They may also be needed at intermediary points so that, if problems have arisen, corrective
measures can be taken. A milestone may be a critical decision point in the project where, for
example, the consortium must decide which of several technologies to adopt for further
development.}

The work in the \TheProject project is structured by four milestones,
which could be briefly characterised as:
starting up and building prototypes; 
moving from prototypes to fully functional implementations; 
further engagement with the community and producing research outputs; 
evaluation and final releases. They coincide with the project meetings 
held at the end of each year of the project (four other meetings will 
be held in the middle of each year). 
Given the nature of the project, with a
large number of essentially independent tasks, there is no need for
milestones attached to specific collections of tasks or
deliverables.
Given that the meetings are the main
face-to-face interaction points in the project, it's suitable to
schedule the milestones for these events, where they can be discussed
in detail, tracking the progress in each work package through status
reports on the tasks and deliverables and take corrective measures,
where necessary, and critical decisions regarding further plans.
We envisage that this setup will give the project the vital coherence
in spite of the broad interdisciplinary mix of various backgrounds of the
participants.

% \newcommand{\WPall}{\WPref{management}, \WPref{dissem}, \WPref{component-architecture}, \WPref{UI}, \WPref{hpc}, \WPref{dksbases}, \WPref{social-aspects}}

% \newcommand{\WPnoUI}{\WPref{management}, \WPref{dissem}, \WPref{component-architecture}, \WPref{hpc}, \WPref{dksbases}, \WPref{social-aspects}}

% \begin{center}
%   \begin{tabular}{|m{.05\textwidth}|m{.30\textwidth}|m{.15\textwidth}|m{.05\textwidth}|m{.22\textwidth}|}
%     \hline
%     Mile-stone nr. & Milestone name & Related work packages & Est. date & Means of verification \\\hline
%     M1 & Requirements study, design and prototype implementations. Start of
%          community building.
%        & \WPall 
%        & 12 
%        & 2nd Project meeting report. Completion of corresponding deliverables. \\\hline
%     M2 & First fully functional interface implementations.
%          Enhanced versions of \TheProject components.
%          Training early adopters.
%        & \WPall 
%        & 24 
%        & 4th Project meeting report. Completion of corresponding deliverables. \\\hline
%     M3 & Evaluating \TheProject software. Working with the community 
%          and building portfolio of experiments produced with \TheProject.
%        & \WPall 
%        & 36 
%        & 6th Project meeting report. Completion of corresponding deliverables. \\\hline
%     M4 & Project evaluation and final versions of all \TheProject components.
%        & \WPnoUI 
%        & 48 
%        & 8th Project meeting report. Completion of corresponding deliverables. \\\hline
%   \end{tabular}
% \end{center}

\begin{milestones}
  \milestone[id=startup,month=12,
  verif={Completed all corresponding deliverables and reported the progress in the 2nd Project meeting report.}]
  {Startup}
  {By Milestone 1 we will have carried out the requirements study, design and prototype implementations and started community building activities.}

  \milestone[id=proto1,month=24,
  verif={Completed all corresponding deliverables and reported the progress in the 4th Project meeting report.}]
  {Prototypes}
  {By Milestone 2 we will have constructed first fully functional interface implementations and released enhanced versions of \TheProject components, and train early adopters of \TheProject.}

  \milestone[id=community,month=36,
  verif={Completed all corresponding deliverables and reported the progress in the 6th Project meeting report.}]
  {Community/ Experiments}
  {By Milestone 3 we will have gathered and evaluated feedback on \TheProject software and established the portfolio of experiments produced with \TheProject through further engaging with the community.}

  \milestone[id=eval,month=48,
  verif={Completed all corresponding deliverables and reported the progress in the 8th Project meeting report.}]
  {Evaluation}
  {By Milestone 4 we will have released final versions of all \TheProject components and completed the project evaluation.}
\end{milestones}

%%% Local Variables:
%%% mode: latex
%%% TeX-master: "proposal"
%%% End:

%  LocalWords:  verif ldots
