\begin{event}{Sage Days 74: Differential geometry and topology}{SD74}{Observatoire de Paris, Meudon, France, 30 May - 2 June 2016}{PS, UB}{26}{https://wiki.sagemath.org/days74}

\textbf{Main goals.} This workshop was dedicated to the implementation of some
topology and differential geometry in SageMath, partly in connection with the
SageManifolds project \url{http://sagemanifolds.obspm.fr/}. 3D visualisation
in the Jupyter notebook was also discussed.

\textbf{ODK implication.} ODK, via its Orsay and Bordeaux nodes, supported the travel and living expenses of 7 speakers:
\begin{itemize}
\item Marck Bell (U. Illinois, Urbana-Champaign)
\item Marck Culler (U. Illinois, Chicago)
\item Nathan Dunfield (U. Illinois, Urbana-Champaign)
\item Patrick Hooper (City College of New York)
\item Vincent Delecroix (U. Bordeaux)
\item Jeremy L. Martin (U. Kansas, Lawrence)
\item John Palmieri (U. Washington, Seattle)
\end{itemize}

\textbf{Event summary.} Morning sessions were devoted to talks on various
topics relevant to the workshop theme, some of them involving codes that are
not part of SageMath (SnapPy, Flipper, Gyoto).
Afternoon sessions were devoted to working groups and coding sprints.

\textbf{Demographic.}
26 persons took part in these Sage Days: 5 females and 21 males, originating
from the following countries: France (11), USA (8), Poland (3), Germany (2), Russia (1) and UK (1).


\textbf{Results and impact.}
41 SageMath tickets have been written or reviewed during the workshop; the list
of them is available at \url{https://trac.sagemath.org/query?keywords=~sd74&or&keywords=~days74}
Progresses on the K3D-jupyter visualisation are reported at
\url{https://wiki.sagemath.org/K3D-tools}.

\end{event}
