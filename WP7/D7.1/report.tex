\documentclass{deliverablereport}

\usepackage[style=alphabetic,backend=bibtex]{biblatex}
\addbibresource{../../lib/kbibs/kwarcpubs.bib}
\addbibresource{../../lib/kbibs/extpubs.bib}
\addbibresource{../../lib/kbibs/kwarccrossrefs.bib}
\addbibresource{../../lib/kbibs/extcrossrefs.bib}
\addbibresource{../../lib/deliverables.bib}
%\addbibresource{../../lib/publications.bib}
\addbibresource{rest.bib}
% temporary fix due to http://tex.stackexchange.com/questions/311426/bibliography-error-use-of-blxbblverbaddi-doesnt-match-its-definition-ve
\makeatletter\def\blx@maxline{77}\makeatother

\deliverable{social-aspects}{social-datareport}
\deliverydate{02/27/2017}
\duedate{02/27/2017 (Month 18)}
\def\pn{OpenDreamKit}
\author{ }

\begin{document}
\maketitle
%There is a large ecosystem of mathematical software systems and knowledge bases.
Individually, these are optimized for particular domains and functionalities, and together they cover many needs of practical and theoretical mathematics.
However, each system specializes on one particular area, and it remains very difficult to solve problems that need to involve multiple systems.
Some integrations exist, but they are ad-hoc and have scalability and maintainability issues.
In particular, there is not yet an interoperability layer that combines the various systems into a virtual research environment (VRE) for mathematics.
  
The OpenDreamKit project aims at building a toolkit for such VREs.
It suggests using a central system-agnostic formalization of mathematics (Math-in-the-Middle, MitM) as a mathematical pivot point for semantic-preserving translations in the needed interoperability layer.
In this \papertype, we report on a series of case studies that instantiates the MitM paradigm with the systems \GAP, \Sage, \LMFDB, and \Singular to perform distributed computation in group, ring, and number theory.
 
Our work involves massive practical efforts, including a novel formalization of computational group theory, improvements to the involved software systems, an extension of the underlying knowledge management system to cope with large theories, and a novel mediating system that sits at the center of a star-shaped integration layout between mathematical software systems and knowledge bases.

Together with deliverable report\textbf{D6.8}, this report describes the implementation and initial evaluation of the MitM integration and interoperability paradigm initially envisioned in deliverables \textbf{D6.2} and \textbf{D6.3}.
The MitM paradigm constitutes the core development goal of \textbf{WP6} and the curated content described in this report enables running non-trivial integration case studies.
In the future we hope to further consolidate content, increase coverage of alignment, and greatly extend the reach of the integration both interms of OpenDreamKit systems covered as well as knowledge available in the MitM Core ontology.

%%% Local Variables:
%%% mode: visual-line
%%% fill-column: 5000
%%% mode: latex 
%%% TeX-master: "report"
%%% End:

%  LocalWords:  optimized formalization papertype textbf textbf textbf

\strut\githubissuedescription
\newpage\tableofcontents\newpage

Large-scale and to a lesser extent medium-scale open-source software 
is as a rule a product of a collaborative effort spanning many years of
development, improvements, bug fixes, ports to new platforms,
and partial or even full rewrites. A number of interesting related questions
arise in this context.
\begin{enumerate}
\item What are available data, measurement parameters and tools?
\item Can one assess the usefulness of the project by estimating
how ``alive'' it is, i.e. how much it is changing over time?
\item Can one reliably range the contributors
by the effort put into the project?
\item Can one produce recommendations on the team size and composition
to ensure project's well-being?
\item Does the ``openness'' of the project matter?
\item Reliability, reproducibility, etc.
\end{enumerate}

\section{Data, parameters and tools}

The main sources of information about the history of a project are versions of
its source code and logs of various relevant communications, discussions, and
test results.  Before the wide acceptance of {\em revision control systems} 
(RCS) \cite{OSullivan:MakingSenseOfRCS} such as  CVS \cite{CVSWeb} and
Git \cite{ChaStr:pg14} the only readily available source code data came from
regular (often infrequent) public releases. Then it has become
more and more widespread, although not universal (cf. e.g. GAP \cite{gap},
which only in the past few years made its RCS 
public---and has not released earlier RCS data)
to keep the RCS {\em trees} holding {\em commits}---code changes
accompanied by comments---available online with read access for the public.

Communications on the project take basically three (not totally disjoint)
forms: mailing lists/bulletin
boards and tracker/code reviewing systems,
such as Trac \cite{wp7:trac}, Redmine, Github \cite{wp7:github},
Bitbucket, Gitlab, etc, and documentation
systems/wikis. The latter is open by nature, whereas for the first two
the prevailing trend, at least in the domain related to the
ODK themes, for these is the ever increased
openness of the development process.

The current prevailing form of the analysis of the source code is
based on analysing authorship, frequency, and other parameters of commits
in the RCS. Most of the tools are in one or another way related to 
Github and its APIs to access RCS trees and collect statistics,
see e.g. \cite{wp7:afronshapeoss}.
Communications are analysed using various text mining and analysis tools,
such as FOSS Heartbeat \cite{wp7:fossheartbeat};
these are not dissimilar to tools used to analyse {\em social networks}
such as Facebook.


\section{Alive or dead?}

\section{Ranging contributors}

\section{Team composition}

\section{Openness, licensing, etc}

\section{Reliability and reproducibility}

\printbibliography
\end{document}

%%% Local Variables:
%%% mode: latex
%%% TeX-master: t
%%% End:

%  LocalWords:  githubissuedescription newpage tableofcontents newpage printbibliography
