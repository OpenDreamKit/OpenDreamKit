\addtocounter{wpno}{1}
\begin{Workpackage}{\thewpno}
\wplabel{wp:x}
\WPTitle{\wpname{\thewpno}}
\WPStart{Month 1}
\WPParticipant{PS}{1}
\WPParticipant{LL}{1} % Pythran
\WPParticipant{SA}{1} % GAP
\WPParticipant{UK}{1} % Singular
\WPParticipant{UB}{1} % Pari
\WPParticipant{UG}{1} % Pari

\begin{WPObjectives}
  The objective of this work package is to improve performance, in
  particular via massive parallelism features at the level of the
  components of \TheProject, and share expertise and best practices on
  this topic. This includes notably native support for:
  \begin{itemize}
  \item Fine grained High Performance Computing on many-cores
    architectures.
  \item Coarse grained or embarrassingly parallel computing on grids
    or on the cloud.
  \item Compilation of high level Python code to optimized C/C++.
  \end{itemize}
\end{WPObjectives}

\begin{WPDescription}
  HPC expertise was already gained by the community at the occasion of
  the 2009-2013 EPSRC project where HPC features were developed for
  the \GAP component. One of the main developer was then rehired
  within the \Singular team to bring expertise and develop similar
  features there.  In this work package, we will build on this
  momentum to implement HPC support in the components
  Tasks~\ref{task:hpc_pari,task:hpc_linbox,task:hpc_singular}

  Many of the computational components of \TheProject use a high level
  interpreted language for their library. This is notably the case of
  \Sage. Performance is achieved by compiling critical sections using
  the \Cython \Python-to-C compiler. In this work package, we will
  boost performance by further developing and applying such
  compilation tools. A key asset will be the recruitment of two of the
  lead developers of the \Pythran \Python-to-C compiler.
\end{WPDescription}

\begin{task}{Pari}
  \label{task:hpc_pari}
  \TOWRITE{KB}{Task around HPC/parallelism in Pari}
\end{task}

\begin{task}{Linbox}
  \label{task:hpc_linbox}
  \TOWRITE{JGD/CP}{Task around HPC/parallelism in Linbox}
\end{task}

\begin{task}{Singular}
  \label{task:hpc_singular}
  \TOWRITE{WD}{Task around HPC/parallelism in Singular}
\end{task}


\begin{task}{Pythran-Cython convergence}
  \label{task:pythran_cython}

  \Pythran is a \Python to C++ compiler for a subset of the \Python
  language. It is meant to efficiently compile scientific programs,
  and takes advantage of multi-cores and SIMD instruction units.
  Thanks to type inference, it requires little annotations.

  \Cython is a \Python to C compiler that was originally developed for
  \Sage and is now a thriving project of its own. It can handle
  essentially any \Python code, and in particular classes, but relies
  heavily on annotations for producing optimized code.

  Therefore, \Pythran and \Cython are similar in spirit but have
  complementary feature sets. In this task, we will investigate the
  opportunity and feasibility of a convergence between \Cython and
  \Pythran into a single tool: depending on the code at hand, one
  strategy or the other would be automatically selected.

  This work will be achieved by a close collaboration between the
  \Pythran developers hired for \TheProject and \Cython developers
  involved in the \Sage project.

  \TODO{Importance and outreach to the numerical community}

  \TODO{deliverable}
\end{task}

\begin{task}{Pythran for Sage}
  \label{task:pythran_sage}
  Currently, \Sage uses \Cython for compiling the critical sections of
  its libraries. In this task, we will explore opportunities to
  benefit from Pythran compilation within the Sage library. A specific
  challenge is that this library uses quite heavily object-oriented
  programming.

  This task will strongly benefit from Task~\ref{task:pythran_cython},
  while providing in return a real life large-scale use case for it.
\end{task}




\begin{WPDeliverables}
  \WPdeliverable{del:ipython_kernels_basic}{12}{Basic Jupyter
    interface for GAP, Pari, Sage, Singular}
  \WPdeliverable{del:ipython_kernels}{12}{Full featured Jupyter
    interface for GAP, Pari, Singular}
  \WPdeliverable{del:ipython_kernels_sage}{12}{Sage notebook / IPython
    notebook convergence}
\end{WPDeliverables}
\end{Workpackage}

%%% Local Variables: 
%%% mode: latex
%%% TeX-master: "../proposal.tex"
%%% End: 
