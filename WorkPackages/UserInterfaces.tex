\begin{workpackage}[id=UI,wphases=24-48,
  title=User Interfaces,
  PSRM=1,
  JURM=12, % active documents
  LLRM=1, % IPython
  SARM=1, % GAP
  UKRM=1, % Singular
  UBRM=1] % Pari
  
\begin{wpobjectives}
  The objective of this work package is to provide a modern, robust,
  and flexible user interface for computation, supporting real-time
  sharing, integration with collaborative problem-solving,
  multilingual documents, paper writing and publication, links to
  databases, etc.
\end{wpobjectives}

\begin{wpdescription}
  \TODO{What is a notebook interface}

  \TOWRITE{IPython}{improve this draft presentation of IPython}

  IPython is a leading notebook interface in the world of interactive
  computations, and use massively by biologists, physicists, \TODO{and
    outside academia!}. Originally tailored for Python, it has been
  language agnostic, and can communicate through a standardized
  interface to various computation kernels \TODO{cite a few}. It can
  transparently run kernels locally or remotely (e.g. on the cloud),
  and has built in support for parallel computing and HPC.
\end{wpdescription}

\begin{tasklist}
\begin{task}[title=Uniform notebook interface for all interactive components]
  \label{task.ipython_kernels}
  In this task, we will implement Jupyter interfaces for the
  interactive computation components of \TheProject, including GAP,
  Pari, Sage, and Singular. A first release
  D~\ref{del:ipython_kernels_basic} will focus on basic functionality,
  and a second release D~\ref{del:ipython_kernels} will cover advanced
  features like 3D graphics or transparent documentation browsing (as
  live worksheets whenever relevant).

  % Note from William: my student Andrew Ohana just mostly did
  % something like that for IPython, but then stopped.  Anyway, it's
  % very do-able based on a summer project from another student and a
  % bunch of work I did with THREE.js for SMC.

  Sage itself will require a specific treatment as it already has a
  notebook interface. Its development started about at the same time
  as the IPython notebook, with similar target features but a
  different agenda: the Sage notebook had to be available very quickly
  to solve pressing needs of the Sage community; instead the IPython
  notebook was to take its time and build robust foundations from the
  ground up. The two projects have exchanged a lot, and the IPython
  notebook, which benefits from a much larger user base and thus
  developer pool, has mostly caught up with the Sage notebook in terms
  of functionality. It's thus time for the Sage community to outsource
  this key but non disciplinary component and phase out the Sage
  notebook in favor of the IPython notebook.

  % In charge: IPython dev + dev in Orsay + community?
  The Sage and IPython convergence D~\ref{del:ipython_kernel_sage} will
  require:
  \begin{itemize}
  \item Robust migration path and tools for Sage worksheets,
  \item Support for math, 2D, and interactive 3D output.,
    % \item Bundling of the IPython notebook and its dependencies within
    %   the Sage distribution. DONE
  \item Import (and export?) of ReST documents, with full support for
    Sage's specific roles (math, ...),
  \item Support for remote Sage kernel, typically on the cloud, or
    running with a different Python version (Sage as a library),
  \item A migration path for interactive widgets implemented with
    Sage's \texttt{@interact} functionality.
  \end{itemize}

  Joint meetings and visits between the developers of Jupyter and of
  the computing components will be a key asset for this task.

  \TODO{convergence with TeXmacs?}
\end{task}

\begin{task}[title=Notebook interface usability]
  In this task, we will further improve Jupyter as a uniform user
  interface for interactive computation and database query.

  \TOWRITE{IPython}{Split off the work into an appropriate list of deliverables}

  \begin{itemize}
  \item Collaborative and shared worksheets,
  \item Multilingual notebooks?
  \item Improved 2D/3D graphics: maybe architecture for integrating
    VPython, vispy, ...? 3D visualization of algebraic surfaces (in
    collaboration with Singular)
  \item Native folding support for sections and the like, with mouse
    and keyboard commands.
  \item Version control
  \item Reproducibility: \ref{del:ipython_test} Support for tested
    notebooks. The writer specifies the expected outputs, e.g. in text
    format, and can check at any point that the full execution of the
    notebook yields exactly the expected output, as can be done with
    e.g. ReST files in Sage: \lstinline{sage -t notebook.rst}
  \item \ref{del:ipython_docking} (Month ???) Heavyweight (e.g. QT
    based and not web based) user interface with docking support in
    the style of \texttt{Spyder}
    \url{https://code.google.com/p/spyderlib/}. Can possibly be
    implemented by extending the QT IPython console, or by letting
    \texttt{Spyder} use the IPython protocol (\TODO{If that's not yet
      the case}).
  \end{itemize}
\end{task}

\begin{task}[title=Dynamic documentation and exploration system]
  % \item \ref{del:ipython_dynamic_doc} (Month ???)

  Introspection has become a critical tool in interactive computation,
  allowing user to explore on the fly the properties and capabilities
  of the objects under manipulation. This becomes particularly acute
  in systems like Sage where large parts of the class hierarchy is
  built dynamically, and static documentation builders like Sphinx
  cannot anymore render all the available information.

  In this task, we will investigate how to further enhance the user
  experience. This will include:

  \begin{itemize}
  \item On the fly generation of Javadoc style documentation, through
    introspection, allowing e.g. the exploration of the class
    hierarchy, available methods, etc.
  \item \TOWRITE{Logilab}{Inclusion of database queries and views}
  \item \ref{del:ipython_advanced_interacts} (Month 36) Exploratory
    support for semantic-aware interactive widgets providing views on
    objects represented and or in databases

    Preliminary steps are demonstrated in the \texttt{Larch
      Environment} project (see demo vidéo on
    \url{http://www.larchenvironment.com/}) and sage-explorer.

    Ultimate goal: automatically generated LMFDB-style interfaces.
    Mention Knowls, as dynamic context-free items of knowledge
  \end{itemize}




  Whenever possible, those features will be implemented generically
  for any computation kernel by extending the Jupyter protocol with
  introspection and documentation queries.

  % In charge: IPython dev + dev in Orsay + NT?

  \item .
  \item     % In charge: IPython dev

\end{task}

\begin{task}[title=Structured documents]
  % \item \ref{del:ipython_structured_documents} (Month ???) 
  Support for writing interactive structured documents, and in
  particular papers, books, experimentation log books and reports,
  presentations, course notes, etc, with the following features:
  \begin{itemize}
  \item Static printed/PDF/HTML version and interactive version.\\
    Achieved by either importing or exporting document files in some
    standard format (LaTeX, ReST, Markdown, ...).
  \item Tests (see above).
  \item Collaborative edition.
  \item Version control.
  \end{itemize}
\end{task}
\end{tasklist}

\TODO{include here everything about this topic in Needs.rst}

\TODO{Wherever relevant, create tickets with details, and refer to
  them here.}

\begin{wpdelivs}
  \begin{wpdeliv}[due=12,id=del:ipython_kernels_basic,dissem=??,nature=??]
      {Basic Jupyter interface for GAP, Pari, Sage, Singular}
\end{wpdeliv}
  \begin{wpdeliv}[due=12,id=del:ipython_kernels,dissem=??,nature=??]
      {Full featured Jupyter interface for GAP, Pari, Singular}
\end{wpdeliv}
  \begin{wpdeliv}[due=12,id=del:ipython_kernel_sage,dissem=??,nature=??]
      {Sage notebook / IPython notebook convergence}
\end{wpdeliv}

  % Shared Jupyter sessions embedded in voice-over-IP or
  % teleconference calls or reciprocally.
  %
  % NOTE: This task is probably outdated by appear.in which makes
  % video-conferencing in the browser trivial
  %
  % \ref{del:ipython_collaborative}
  % Eugen Dedu:
  % I think such a module can be thought of as a screen-capturing
  % module, i.e. allow Ekiga to capture the screen of a Sage user (this
  % is currently not possible).  This is not a difficult task to do.
  % Julien Puydt: ekiga can do that since something like 2008 with my
  % experimental gstreamer plugin, and I shall be able to present
  % interesting sample code to the ekiga-devel mailing-list in something
  % like two-three weeks (after I'm done with my students), which will
  % hopefully be part of the next version.
  % 
  % But as Nicolas noted in his answer, some kind of interative session
  % where people can share a sage session would be better.
  % 
  % I think the feature decomposes in the following pieces:
  % - IPython should have a way to share sessions between several
  % participants using an open and standard protocol ;
  % - ekiga should implement it.
  % 
  % In my opinion ekiga, because of its dependency on ptlib and opal
  % libraries and the use of complex protocols like SIP and H323, needs
  % highly technical people.  Students cannot help much, but engineers
  % are appropriate.

\end{wpdelivs}
\end{workpackage}

\begin{verbatim}

About the availability of people to hire, I have a full-time,
experienced developer whose contract runs out in fall 2015, he would be
ideal for the project. I also have a doctoral student who needs
employment after the MathSearch project (until 10/2015) runs out. So I
do have people who would directly be available.

Michael

===================8<---------------------------------

Task 4.10. Structured Documents (12 PM total, 3 PM per deliverable)
   -> This existing task we could just take over based on our MathHub.info
         system, which would need to be adapted to the task.
Deliverables:
   D1: Active Documents based on sTeX
   D2: Distributed, Collaborative, Versioned Editing of Active Documents
in MathHub.info
   D3: Notebook Import into MathHub.info (interactive display)
   D4: in-place computation in active documents (context/computation).
Comments:
  MathHub.info is a portal for reading and interacting with "active
documents"
  (i.e. documents that have an additional semantic layer that supports
semantic services like
   - definition lookup, type-inference, unit conversion, ...)
  Notebooks are essentially "programs with documentation", whereas
active documents are
  documents with a semantic knowledge layer. Regular publications are an
important
  boundary case: Active Documents look like papers, but are
web-standards compatible
  and interactive.
  sTeX is a semantic variant of LaTeX that we can transform into OMDoc/MMT,
  which is the native knowledge representation format for active documents
  and machine-actionable knowledge about math and symbolic programs.

===================8<---------------------------------

Task K-4.11 Math Search Engine (10 PM total; 2 each for D1/2, 3 each for
D3/4)
   D1: Full-text search (formulae + Keywords) over LaTeX-based documents
         (e.g. arXiv subset)
   D2: Full-text search (F+K) over Notebooks (in the format determined
in task 4.7)
   D3: Formula search in CAS programs and Software Modules
   D4: Search from Notebooks/Active Documents (for local context to
inform search)
Comments:
   We already have a search engine, therefore we only need to build
harvesters for D1/2;
   D3/4 are more speculative.

I am going to get some sushi now, I have 2 or 3 tasks for WP6 in mind,
but not written up.

\end{verbatim}

%%% Local Variables: 
%%% mode: latex
%%% TeX-master: "../proposal.tex"
%%% End: 
