%\subsubsection{Jupyter}
\label{sec:jupyter}

\begin{wrapfigure}{r}{0.50\textwidth}
%\includegraphics[scale=0.23]{Pictures/jupyterdemo1.png}
\includegraphics[scale=0.4]{Pictures/jupyterdemo-largefont.pdf}
\caption{\label{fig:jupyterdemo} Self-contained \Jupyter Notebook demonstrating the concepts of cells that contain different types of material and can be executed (or updated) in arbitrary or sequential order.}
\end{wrapfigure}

Project \Jupyter \cite{Jupyter} is a set of open-source software projects for
interactive and exploratory computing emerging from \IPython \cite{IPython}. These
software projects help make scientific computing and data science
reproducible and multi-language (Python, Julia, R, Haskell, Bash, R,
\ldots). The main component offered by \Jupyter is the \Jupyter
notebook, a web-based interactive computing platform that allows users
to create data- and code-driven narratives that combine live
(re-executable) code, equations, narrative text, interactive
dashboards and other rich media.

Figure~\ref{fig:jupyterdemo} shows a Python-based sample
session. Within the Python session, all libraries available in Python
can be imported and combined flexibly, a number of interfaces between
different languages exist. Many more examples are available, for
example \cite{IPython-demo-hyperbolic-conservation-laws} and
within \cite{IPython-sload-foundation-report-2013}.

The \Jupyter notebook is being used widely in academia
(e.g. University of California, Berkeley, Stanford,
MIT, Harvard, Cambridge, Oxford, Imperial College, Southampton,
Hamburg, Paderborn, Vienna, Paris, Katowice, and Oslo) and government
(NASA JPL, LBL, KBase, White House Hackathon) as well as
industry (e.g. Google, IBM, Facebook, Oracle, Otto Group, Microsoft,
Bloomberg, JP Morgan, WhatsApp, O’Reilly, Quantopian, Logilab,
GraphLab, Enthought, Continuum, Authorea, BuzzFeed)  and
journalism (e.g. 538 and New York Times). \\
%
Because the architecture and building blocks of \Jupyter are open,
they are used to build numerous other commercial and non-profit
products and services. The \Jupyter Notebook has between 500,000 and
1.5 million individual users worldwide.

These notebook documents provide a \emph{complete} and
\emph{executable} record of a computation that can be shared with
others in a way that has not been possible before. This has led, among
other things, to a huge boost in reproducible, interactive
teaching/education documents in recent years. A paradigm that Fernando Perez, creator of the project, has referred to as ``literate computing''.\footnote{\url{http://blog.fperez.org/2013/04/literate-computing-and-computational.html}}

We will build on this technology by extending \Jupyter with new
functionality, unifying other computational tools to be usable as
components in this framework, and merging the \Sage and \Jupyter
development.  \TOWRITE{NT}{Is the 'merge' too strong a claim? Please
  correct / remove.}



