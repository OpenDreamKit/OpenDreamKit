\begin{sitedescription}{UK}
\subsubsection{University of Kaiserslautern}

% See proposal.tex, Members of the Consortium for a complete description of what should go there
The University of Kaiserslautern (UK) is a medium sized university founded in 1970. It currently 
consists of 12 departments, ranging from mathematics and business studies and economics, 
computer sciences and electrical and computer technology over mechanical and process engineering, 
biology, chemistry and physics to architecture, regional and environmental planning, and social sciences. 
The university has 13,725 students, 3636 of whom are remote study students. The scientific location 
of Kaiserslautern is also distinguished by the presence of multiple external research institutes of 
considerable reputation, including two facilities of the Fraunhofer network and the German Research 
Institution for Artificial Intelligence. All these institutions maintain close links and even share staff 
with the corresponding departments of UK, which is chairing the Science Alliance Kaiserslautern, 
a network of these research institutions. The university conducts a number of international 
collaborations and successfully participated in projects funded under several EU Framework 
Programs and has gathered comprehensive experience both as coordinator and partner in research 
networks and projects. Besides projects with national funding, the UK is 
also very active in the field of international research. In this context, the funding instruments 
available in the EU Framework Programmes play an important role. In total, the UK is partner 
to a total of 11 projects (as of January 2015) conducted under the 7th FP and Horizon 2020. 
Nine further individual projects funded by ERC (2) or Marie-Curie measures (7) are being co-ordinated 
by researchers at UK. By this involvement to date, UK has procured more than 13 million Euros under the 7th FP.

\medskip In the context of this proposal, the
\emph{Algebra, Geometry, and Computer Algebra Group}  of the Department
of Mathematics at UK is widely known for its long tradition in 
computational algebraic geometry and algebra, with particular emphasis on the 
development of the computer algebra system \Singular and its satellites and  
subsystems such as {\sc{Factory}}, {\sc{PolyBori}}, and {\sc{Plural}}.
Kaiserslautern's main tasks in this project are to add very fine-grained 
parallelism to some key components of {\sc{Singular}} and to work
on the maintainance and improvement of \MPIR.

\subsubsection*{Curriculum vitae}

\paragraph{Principal investigator Prof. Dr. Wolfram Decker}

\begin{description}
  \item[Personal Data]\ 

\begin{tabular}{ll}
Gender:& male \\
Nationality:  & German  \\
Address:        & Department of Mathematics\\
                & Technical University of Kaiserslautern\\
                & P.O. Box 3049\\
                & Germany \\
Phone:          & +49 631 205 2253 \\
Email:          & decker@mathematik.uni-kl.de \\
URL:            & www.mathematik.uni-kl.de/agag/mitglieder/professoren/\\ & prof-dr-wolfram-decker/\\
Status:         & Professor\\
\end{tabular}

\item[Scientific Qualification]\ 

\begin{tabular}{ll}
PhD 1984& University of Kaiserslautern \\
Habilitation 1989& University of Kaiserslautern \\
\end{tabular}

\item[Academic Career]\ 

\begin{tabular}{ll}
 1984 -- 1990& Assistant Professor, University of Kaiserslautern\cr
 1990 -- 2009& C3 Professor, University of the Saarland\cr
 2009 -- present& W3 Professor, Technical University (TU) Kaiserslautern\cr
\end{tabular}

\item[Synergistic Activities]\ 
\begin{description}
\item[PhD students]\ 

  Hirotachi Abo, Holger Cr\"ni, Hiep Dang, Hanieh Keneshlou, Dereje
  Kifle, Michael Messollen, Ngoc Anh Pham, Sorin Popescu, Andreas
  Steenpa\ss, Isabel Stenger, I. Made Sulandra, Shrawan Tiwari

\item[University Service]\ 

\begin{tabular}{ll}
 1994--1996 & Dean, Dep. of Math., University of the Saarland\\
  2014--present & Dean, Dep. of Math., TU Kaiserslautern\\
\phantom{A}
 \end{tabular}

\item[Scientific Service]\ 

\begin{tabular}{ll}
 1996--1999 & Coordinator of EuroProj (a European algebraic geometry network)\\
 2000--2004 & Chair of the programme management committee of EAGER\\ 
                      &  (a European algebraic geometry network)\\
 2010--present & Coordinator of the DFG Priority Programme SPP1489\\
                      & `Algorithmic and experimental methods\\
                      & in algebra, geometry, and number theory'\\
\end{tabular}

\item[Conferences (co)organized]\ 

\begin{tabular}{ll}
 1997-- 2004 & About 30 conferences, summer schools, and workshops\\ 
                           & in the framework of EuroProj and EAGER.\\
 1992--present & More than 20 coding sprints, conferences, summer schools,\\
                          & and workshops outside EuroProj and EAGER,\\
                          & including 3 conferences at Dagstuhl and 1 at Banff.\\
 2000                 & Chair of the Minisymposium on computer algebra, third ECM.\\
\end{tabular}
\end{description}

\item[Selected Grants]\ 

\begin{tabular}{ll}
1986-1987 & NATO-Grant of the DAAD (visit UC Berkeley)\\
1987--1994& In: DFG Priority Programme `Complex manifolds'\\
\phantom{1992--}1993 & Grant: Japanese Society for the Promotion of Science (visit Kyoto)\\
1992--1997& In: DFG Priority Programme `Algorithmic number theory and algebra'\\
2002--2006& In: DFG Priority Programme `Global methods in complex geometry'\\
2010--present & In: DFG Priority Programme SPP1489 (two grants)\\
1997--2004 & Seven grants for EU Highlevel Scientific Conferences
\end{tabular}

\item[Selected Publications]\ 
\medskip\noindent
\begin{enumerate}[1.]

\item On the uniqueness of the Horrocks-Mumford-bundle (with F.-O. Schreyer).
\emph{Math. Ann.} {\bf{273}} (1986), 415--443.  (MR0824431)

\item Stable rank 2 vector bundles with Chern-classes $c_1=-1$, $c_2=4$. 
\emph{Math. Ann.}  {\bf{275}} (1986), 481–-500. (MR0858291)

\item Construction of Surfaces in $\mathbb{P}_4$ (with L. Ein, F.-O. Schreyer). 
\emph{J. Algebraic Geometry} {\bf{2}} (1993), 185--237. (MR0858291)

\item Computational algebraic geometry today. In: \emph{Applications of algebraic geometry 
to coding theory, physics and computation (Eilat, 2001)}, 65--119, 
NATO Sci. Ser. II Math. Phys. Chem., 36, Kluwer Acad. Publ., Dordrecht, 2001. (MR1866896)

\item 
Computing in Algebraic Geometry. A Quick Start using SINGULAR (with C. Lossen).
\emph{Algorithms and Computation in Mathematics, 16}.  Springer, Berlin, 2006. xvi+327 pp. (MR2220403)

\item Parallel algorithms for normalization (with J.B\"ohm, S. Laplagne, G. Pfister, A. Steenpa\ss, S. Steidel).
\emph{J. Symbolic Comput.} {\bf{51}} (2013), 99--114.  (MR3005784)

\item A first course in computational algebraic geometry (with  G. Pfister). 
\emph{African Institute of Mathematics (AIMS) Library Series}. Cambridge 
University Press, Cambridge, 2013. viii+118 pp. (MR3052757)

\item
Local analysis of Grauert-Remmert-type normalization algorithms (with J.B\"ohm, M. Schulze). 
\emph{Internat. J. Algebra Comput.} {\bf{24}} (2014), 69--94. (MR3189667)
\end{enumerate}

\item[Selected Mathematical Software]\ 

\begin{tabular}{ll}
 1997--present & Coauthor of {\sc{Singular}} libraries for adjoint ideals, absolute factorization, \\
                          & integral bases, invariant theory, parametrization of rational curves, \\
                          & primary decomposition, normalization, and sheaf cohomology\\

 2009--present& Head of the {\sc{Singular}} developers group\\
\end{tabular}
\end{description}

%%% Local Variables: 
%%% mode: latex
%%% TeX-master: "../proposal.tex"
%%% End: 

\begin{participant}[type=PI,gender=male]{Dr. William Hart}

William Hart is a postdoctoral researcher at the University of
Kaiserslautern. He is the lead developer of the Flint and MPIR projects
as well as the main author and maintainer of the BSDNT bignum library, the
Nemo and ANTIC libraries and a contributor to various other software packages.

Before coming to Kaiserslautern, held a prestigious five year Career
Acceleration Fellowship ``Algorithms in Algebraic Number Theory'' at Warwick
University in the UK. He has been involved in a number of high performance
computing records, including computation of congruent numbers (subject to the
BSD conjecture) up to a trillion ($10^{12}$).

William is the main author of the FFT code for multiplication of large integers
and polynomials in both MPIR and Flint, which are used extensively by the Sage,
Singular and Macaulay 2 computer algebra systems.

The main focus of William's research has been in algorithms for fast arithmetic, fast integer and polynomial factorisation and to algebraic number theory,
including computation of modular equations and class invariants.
\end{participant}
%%% Local Variables:
%%% mode: latex
%%% TeX-master: "../proposal"
%%% End:


\begin{participant}[type=res,PM=48]{NN}
\end{participant}
\begin{participant}[type=res,PM=36]{NN}
  We will hire two full time experienced software developers to work
  under the leadership of Wolfram Decker on adding very fine-grained 
  parallelism to some key components of {\sc{Singular}}. The fellows
  will have past experience of parallism in software development.
  We further require good communication and team working
  skills.
\end{participant}

\begin{participant}[type=res,PM=12]{NN}
  We will hire a full time highly specialized software developer and
  assembly expert, to work under the leadership of William Hart on the
  performance task \taskref{hpc}{hpc-mpir} for \MPIR.
\end{participant}

\subsubsection*{Publications, products, achievements}

\begin{enumerate}
\item {\sc{Singular}} computer Algebra system.
\item Wolfram Decker is coordinator of the DFG Priority Project SPP1489 \emph{Algorithmic and Experimental Methods in Algebra, Geometry, and
Number Theory'}.
\item {\sc{Flint and MPIR}} C libraries for number theory and bignum arithmetic.
\item William Hart held an EPSRC Career Acceleration Fellowship EP/G004870/1 
from 2008-2013, \emph{Algorithms in Algebraic Number Theory}
\end{enumerate}

\subsubsection*{Previous projects or activities}

\begin{enumerate}
\item Member of the DFG Priority Project \emph{Algorithmic Number Theory and Algebra}.
\end{enumerate}

\subsubsection*{Significant infrastructure}

Excellent computing infrastructure (high end servers), access to 
different types of compute clusters through the IT-Center of the 
TU Kaiserslautern.
\end{sitedescription}



\begin{draft}
\vspace{1cm}\TOWRITE{WD}{Complete check list below -- delete completed items if you wish}

\begin{verbatim}
- [ ] checked that sum of person months put into finance request is
  the same as sum of person months associated with the Work Packages
  (in proposal.tex, as defined as part of the \begin{workpackage}"
  command.
  
  Take into account person months associated with work package 1, time
  of all staff to be hired and work on the project (including
  investigators). Figure 5 helps with a quick check of the sums over
  different work packages.

- [ ] completed site specific resource summary in resources.tex,
  including table of non-staff costs. This is compulsory (EU
  regulations) if the non-staff cost exceed 15% of the total cost, and
  is likely to be the case for most of the partners. We ask everybody
  to do it, to be consistent and show transparently how we have
  planned our total budget.

- [x] Have all our tasks a designated lead institution? Check in the
  Work Packages that all the tasks you are involved in have a
  dedicated lead party. If the lead party is "USO", then use:
  \begin{task}[lead=USO]

- [ ] Have all our deliverables a designated lead institution [using
  the 'lead=' key]?

- [x] In the "Members of the consortium section", have we addressed "a
  description of the legal entity and its main tasks, with an
  explanation of how its profile matches the tasks in the
  proposal"? See Entry for Paris Sud and Southampton as examples.

- [x] In the Members of the consortium section, have we given
  descriptions of all the people we intend to hire (even if we don't
  know who that is yet). 
  
- [ ] Do all our tasks include us in the list of sites involved?
\end{verbatim}
\end{draft}

%KEY-MORE-TODOS



%%% Local Variables:
%%% mode: latex
%%% TeX-master: "../proposal"
%%% End:

%  LocalWords:  sitedescription subsubsection sc emph bignum
