\begin{sitedescription}{UB}
% PIC: 
% see: http://ec.europa.eu/research/participants/portal/desktop/en/orga

% See ../proposal.tex, section Members of the Consortium for a
% complete description of what should go there

Bordeaux is an important center of studies and research in France
with approximately 50,000 students, 2,000 PhD students and 5,000
researchers. The University of Bordeaux was founded in the XVth
century and nowadays the city regroups two universities, dozens of
schools as well as \TOWRITE{VD}{NUMBER} research laboratories with partner institution such
as CNRS, Inserm, INRA and INRIA.

The Institut Math\'emathiques de Bordeaux (IMB) is a leading
institution in Number Theory. It is the home of the software
\PariGP and the Journal de Th\'eorie des Nombres de Bordeaux.

The city of Bordeaux also hosts two important young laboraties
for computer science: Laboratoire Bordelais d'Informatique (LaBRI)
and INRIA-Bordeaux.

\medskip
In the context of this proposal, Bordeaux has a long standing experience in
Algorithmic Number Theory and two significant hardware infrastructures
(Plafrim and Avakas). The CNRS Aquitaine main task in this project is the
extension of \PariGP in relation with the other software components and
the .

\subsubsection*{Curriculum vitae}
Note that for purely administrative reasons, Loic Gouarin will be rattached to
CNRS Aquitaine but is naturally based at Paris sud.

% Curriculum of the personnel at this institution

\begin{participant}[type=leadPI,PM=12,salary=2500,gender=male]{Vincent Delecroix}
  \TOWRITE{VD}{Finalize CV}
  CNRS researcher at the Laboratoire Bordelais de recherche en informatique, Vincent
  Delecroix is a junior researcher in Number theory, Combinatorics and Dynamical systems.

  X publications.  He has several international collaboration (England, Mexico,
  United-States).

  \begin{itemize}
\item  Bobo
\item  cours Sage Bordeaux
\end{itemize}

Delecroix is a major developer of Sage in various components:
\begin{itemize}
\item  integers
\item  combinatorics
\item dynamical systems
\end{itemize}
\end{participant}
%%% Local Variables:
%%% mode: latex
%%% TeX-master: "../proposal"
%%% End:


\begin{participant}[type=PI,PM=12,salary=9800,gender=male]{Karim Belabas}
is a Professor of Mathematics in Bordeaux (France) since 2005. He is
a senior researcher in Number Theory, with particular interest in
computational and effective aspects. Karim has published about 25 articles
in international journals, including papers in Duke and Compositio (one of
which co-authored with Manjul Bhargava, 2014 Fields Medalist), and
edited the book ``Explicit methods in number theory''.

Karim was head of the Pure Math teaching department in Bordeaux from 2010 to
2014 and is vice-head of the Institut de Math\'ematiques de Bordeaux since
2015. He has held a grant from the French ANR worth \euro$200$k (ALGOL
project, 2007--2011) and was part of a \euro{2.5}m Marie-Curie Research
Training Network (GTEM project 2006--2010); he was responsible for three
deliverables and supervision of an early stage researcher during her PhD in
the Work Package ``Constructive Galois Theory''. He has (co-)organised 8
international conferences, including a special Trimester at IHP in 2004 and
an influential recurrent workshop on ``Explicit methods'' in Oberwolfach
(every two years since 2007) and 5 \PariGP Ateliers. He has (co-)supervised
11 PhD students and about 15 masters students.

Karim is a leading computational number theorist in France. He
is the project leader for the \PariGP free computer algebra system since
1995, which has had a major impact on hundreds of publications. He is one of
the system's main developers (about 60000 lines of code written, most of the
documentation, and 1300 bug-tracking tickets authored).
\end{participant}
%%% Local Variables:
%%% mode: latex
%%% TeX-master: "../proposal"
%%% End:

\paragraph{Bill Allombert}

% months=6
% salary=YYY

CNRS Ing\'enieur de Recherche. One of the main pari developer.


\begin{participant}[type=R,PM=6,salary=5700,gender=male]{Adrien Boussicault}
  Maître de Conférences at the LaBRI (Laboratoire Bordealais de Recherche en 
  informatique), Adrien Boussicault is a young researcher in Algebraic and 
  Enumerative Combinatorics. He has 8 papers in international journals.
  
  He is a young contributor in Sage. He wrote 3 tickets to implement 
  combinatorial objects. He co-organized Sage-Combinat Days 57.
\end{participant}
%%% Local Variables:
%%% mode: latex
%%% TeX-master: "../proposal"
%%% End:


\begin{participant}[type=R,PM=0,gender=male]{Lo\"ic Gouarin}
  Research Engineer since 2005 at CNRS and more specifically since
  2010 at the Laboratoire de Mathématique d'Orsay, Université
  Paris-Sud, Loïc Gouarin develops scientific computing software in
  different fields like Lattice-Boltzmann methods, Stokes solvers for
  fluid particles interaction, ...

  He is also director of the ``GdR Calcul'' and co-director of the
  ``Réseau Calcul''. These two entities form the ``Groupe Calcul'' of
  the CNRS whose role is to animate the scientific and high
  performance computing community in France, in particular by
  organising conferences, meetings, and seminars. In this context, he
  organises himself 3 to 4 training and development workshops per
  year, and promotes the use of \Python for teaching and research in
  France.

  Organisationally, due to purely administrative constraints within
  CNRS, Loïc Gouarin will be attached to \site{UB}.
\end{participant}

%%% Local Variables:
%%% mode: latex
%%% TeX-master: "../proposal"
%%% End:

\begin{participant}[type=R, PM=48]{NN}
We will hire a computational support research engineer to work at Bordeaux
 under the leadership of Professor Karim Belabas and Doctor Vincent
Delecroix on the tasks of \WPref{hpc}, \WPref{component-architecture} and \WPref{UI}.
He or she will work on some or all of the following tasks
\begin{itemize}
\item parallelization of some algorithms in \PariGP,
\item the creation of a Python library for \PariGP and its usage within \Sage,
\item and the graphics capabilities inside Sage.
\item \TOWRITE{VD}{OTHERS??}
\end{itemize}
\end{participant}


\subsubsection*{Publications, products, achievements}

Some recent publications :
\begin{enumerate}
\item 
Karim Belabas, Eduardo Friedman,
\textit{Computing the residue of the Dedekind zeta function}.
Math. Comp. 84 (2015), no. 291, 357–369. 

\item
The PARI Group; PARI/GP version 2.7.0, Bordeaux, 2014,
http://pari.math.u-bordeaux.fr/.

\item
Karim Belabas et al.
\textit{Explicit methods in number theory. Rational points and Diophantine equations},
179 pages, Panoramas et Synthèses 36, 179p., 2012.

\item
Bill Allombert, Yuri Bilu and Amalia Pizarro-Madariaga,
\textit{CM-Points on Straight Lines}, to appear in "Analytic Number Theory" (dedicated do H. Maier),
Springer.

\item
Vincent Delecroix,
\textit{Cardinality of Rauzy classes}
Ann. Inst. Fourier, 63 no 5 (2013), p. 1651-1715.

\item
Jean-Christophe Aval, Adrien Boussicault, Mathilde Bouvel, Matteo Silimbani
\textit{Combinatorics of non-ambiguous trees},
Advances in Applied Mathematics 56 (2014), p. 78-108.
\end{enumerate}

\subsubsection*{Previous projects or activities}

Current grants:
\begin{enumerate}
\item
 ANR PEACE (2012-2015)
    Goal: The discrete logarithm problem on algebraic curves is one of the rare
    contact points between deep theoretical questions in arithmetic geometry and
    every day applications. On the one side it involves a better understanding,
    from an effective point of view, of moduli space of curves, of abelian
    varieties, the maps that link these spaces and the objects they classify.
    On the other side, new and efficient algorithms to compute the discrete
    logarithm problem would have dramatic consequences on the security and
    efficiency of already deployed cryptographic devices. 

\item
ERC starting grant ANTICS (2011-2016) 
    Goal: "Rebuild algorithmic number theory on the firm grounds of theoretical
    computer science".
    Challenges: complexity (how fast can an algorithm be?), reliability
    (how correct should an algorithm be?), parallelisation.
\end{enumerate}

\subsubsection*{Significant infrastructure}
\begin{enumerate}
\item The Plafrim is a regional federation hosted at INRIA Bordeaux (in partnership with the LaBRI and IMB). It has an important cluster devoted to experimental code (1188 cores).
\item The M\'esocenter de Calcul Intensif Aquitain (MCIA) is localized
in Bordeaux. It hosts the Avakas cluster (3328 cores,  38 TFlops) and the
M3PEC cluster (432 cores).
\end{enumerate}

\end{sitedescription}


\begin{draft}
\vspace{1cm}\TOWRITE{VD}{Complete check list below -- delete completed items if you wish}

\begin{verbatim}
- [ ] checked that sum of person months put into finance request is
  the same as sum of person months associated with the Work Packages
  (in proposal.tex, as defined as part of the \begin{workpackage}"
  command.
  
  Take into account person months associated with work package 1, time
  of all staff to be hired and work on the project (including
  investigators). Figure 5 helps with a quick check of the sums over
  different work packages.

- [X] completed site specific resource summary in resources.tex,
  including table of non-staff costs. This is compulsory (EU
  regulations) if the non-staff cost exceed 15% of the total cost, and
  is likely to be the case for most of the partners. We ask everybody
  to do it, to be consistent and show transparently how we have
  planned our total budget.

- [X] Have all our tasks a designated lead institution? Check in the
  Work Packages that all the tasks you are involved in have a
  dedicated lead party. If the lead party is "USO", then use:
  \begin{task}[lead=USO]

- [X] Have all our deliverables a designated lead institution [using
  the 'lead=' key]?

- [ ] In the "Members of the consortium section", have we addressed "a
  description of the legal entity and its main tasks, with an
  explanation of how its profile matches the tasks in the
  proposal"? See Entry for Paris Sud and Southampton as examples.

- [ ] In the Members of the consortium section, have we given
  descriptions of all the people we intend to hire (even if we don't
  know who that is yet). 
  
- [ ] Do all our tasks include us in the list of sites involved?
\end{verbatim}
\end{draft}

%KEY-MORE-TODOS


%%% Local Variables:
%%% mode: latex
%%% TeX-master: "../proposal"
%%% End:

%  LocalWords:  sitedescription th eorie des nombres Plafrim mesocentre Avakas developped
%  LocalWords:  subsubsection Belabas Synthèses Allombert Bilu Pizarro-Madariaga Maier
%  LocalWords:  parallelisation TOWRITE
