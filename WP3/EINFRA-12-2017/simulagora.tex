\begin{longtable}{|p{7cm}|p{7cm}|}
\hline
 & SERVICE OVERVIEW\\
\\\hline
Thematic Service Name&Simulagora\\
\\\hline
Service description&
%
Simulagora is the Web platform supporting the
https://www.simulagora.com web site, which enable users to launch
scientific computations to on-demand deployed cloud computation
resources. The proposal aims at installing Simulagora on EIG cloud to
make it available to its users.
\\
\\\hline
Service provider&Logilab\\
\\\hline
Service catalogue&\\
%
Not Applicable.

\\\hline
Value&
%
Simulagora features improve users' experience dramatically compared to
classical HPC batch managers:
\begin {itemize}
\item no software installation required (a recent browser is enough)
\item users can choose between several managed OS versions
\item users have administrator privileges on the computing machine
\item users can connect to the computing
\item computations are reproducible as everything is stored, from the
  OS itself to the input data and programs used to perform them
\item users can collaborate, share data, code and studies
\item users can use a programmable API to automate the launch of
  several computations (e.g. a parametric study or a design of
  experiment)
\end {itemize}
%
As a result, users reduce the maintenance tasks of their software and
hardware, decrease the time they need to get their computations done
and analyse them, even more so if they need to collaborate with other
people.
\\
\\\hline
Current TLR&
%
Simulagora has been in production for 2 years and used by French
companies (from SMEs like Fluidyn and Phimeca to big companies like
EDF and SNCF) for very different use cases, proving its scalability
and reliability.

\\
\\\hline
Access policy&
%
The access policy would preferably be Policy-based, typically granting
access to any academic in Europe (to be negociated).

\\
\\\hline
Terms of use&TODO\\
\\\hline
%
The service is intendend to all scientific people who need to perform
computations on HPC-like infrastructures and usually use a batch
manager for this purpose. The typical workloads of Simulagora are
Computation Fluid Dynamics (OpenFOAM) or Finite Element problems in
thermics, mechanics or electromagnetics (using Code\_ASTER, FENICS,
...), using Open Source solvers.

\\
\\\hline
Service business model&
%
The initial integration of Simulagora in the EGI cloud platforms can
be estimated to 3 man.months. Its maintenance costs during the project
can vary from 1 to 3 man.month per year depending on the specific
software the users would eventually ask for.

\\
\\\hline
\end{longtable}


\begin{tabular}{|p{7cm}|p{7cm}|}
&SERVICE INTEGRATION WITH GENERIC E-INFRASTRUCURES
\\\hline
Integration activity and concerned service components&TODO
\\\hline
Overall necessary effort (Person-Months) and timeline&TODO
\\\hline
List of requested service components&TODO
\\\hline
\end{tabular}

\begin{tabular}{|l|l|l|}
&SERVICE ARCHITECTURE&
\\\hline
Name&Description, standards, resource capacity&Provider (if appointed)
\\\hline
mybinder&todo&todo
\\\hline
JupyterHub&todo&todo
\\\hline
\end{tabular}

\begin{tabular}{|p{7cm}|l|}
  &Infrastructure integration
  \\\hline
  Description of infrastructure integration activities relevant to the proposed thematic service (to be planned in the project)&TODO
  \\\hline
\end{tabular}

\begin{tabular}{|p{7cm}|l|}
  &Training
  \\\hline
  Description of training activities relevant to the proposed service (to be planned in the project)&TODO
  \\\hline
\end{tabular}
