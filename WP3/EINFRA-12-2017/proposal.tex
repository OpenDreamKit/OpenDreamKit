\section{Context}

There is a fast growing open source ecosystem around the Jupyter [1]
technology for interactive data science and scientific computing; the
flagship is the Jupyter notebook: web-based narrative documents mixing
live computations, equations rich texts and medias, interactive
visualization, etc.

Although rapidly evolving, this is a mature ecosystem, based on open
and acclaimed technologies (e.g. docker/kubernetes, ...), and already
used by hundreds of thousand of people, as much in academia as in the
industry (e.g. the Bloomberg company is actively contributing to
Jupyter to support its internal needs).

\section{Aim}

At this stage, a low hanging fruit with high return value would be to
leverage the access to this technology to academics in Europe
(possibly beyond) by *deploying, hosting, and maintaining* Jupyter
based web services like JupyterHub, tmpnb, or mybinder. All of them
are TRL-8 level, with instances already deployed. Some may require
customization or some specific development for integration in the
EGI/EUDAT/... infrastructure (authentication and data sharing,
container orchestration, ...).

In fact, the need for leveraging the access to such services is so
pressing that we would be very interested in actually starting a
collaboration as soon as possible, without waiting for the application
to be written / accepted.

Some more details are available in our Use Case description [2] we had
presented at the "Design your E-Infrastructure workshop" in Krakow.

\section{OpenDreamKit's involvement}

The OpenDreamKit EInfra-9 project aims at fostering a flexible toolbox
from which researchers in computational mathematics can easily build
and deploy VRE's taylored to their specific needs. Jupyter (together
with SageMathCloud) is its backbone for all the collaborative
workspaces, user interfaces, and web-services aspects. OpenDreamKit is
particularly involved in the collaborative, reproducibility, and ease
of deployment features. We effectively have on board many of the
academic sites in Europe involved in the Jupyter project. Since
OpenDreamKit focuses on the software aspects, our actions would
complement each other well with a EInfra-12 (A) proposal.

In this context, I see OpenDreamKit as a mere umbrella over a
collection of great people. If it would be useful for the EGI/EUDAT
proposal to use the OpenDreamKit "brand", that's great. If it's best
for the would be consortium to instead build a specific "Jupyter
group" mixing people from within OpenDreamKit and outside, that's very
fine too.  OpenDreamKit's spirit is strongly rooted on a «by users,
for users» approach, and our main interest for participating is that
we critically need the outcome.

\section{Proposed services}

\subsection{tmpnb}

...

\subsection{mybinder}

Description: builds a container with ...

Rationale: mybinder enables trivial to use deployment and sharing of
jupyter notebooks; fosters dissemination and reproducible research,
... its success proves that it has isolated just the right service for
a critical need.

Issue: too successful and lacking computational resources

Goal: power up mybinder with additional computational resources to
leverage its use. Option 1: run an alternative instance of mybinder.
Better option: add a button "Run on EGI cloud" on the official
mybinder instance, enabling anyone with appropriate credentials to use
the EGI cloud to run the container.

\subsection{jupyterhub}

Run an instance of JupyterHub enabling anyone with appropriate
credentials to run jupyter notebooks on the EGI cloud.
