\documentclass{deliverablereport}

\deliverable{component-architecture}{multiplatform-buildbot}
\deliverydate{31/08/2018}
\duedate{31/08/2018 (M36)}
\author{Erik Bray, et. al.}

\usepackage{graphicx}
\usepackage{subcaption}

\begin{document}
\maketitle

\hypertarget{introduction}{%
\section{Introduction}\label{introduction}}

\hypertarget{changes-to-deliverable}{%
\section{Changes in scope of the deliverable}\label{changes-to-deliverable}}

% * Explosion in the last 5+ years in free CI services and integration into
%   issue trackers (GitHub especially); fewer infrastructure requirements
%   for most projects in this regard.
%
% * Different projects have different needs in this area; most can get by
%   with advisory CI provided by free services.  For some, especially larger
%   projects such as Sage and GAP, it is still useful to have additional
%   CI for checking integrations between multiple changes across a larger
%   matrix of platforms
%
% * Some projects simply have fewer cross-platform continuous integration
%   needs; especially true of projects that are mostly numerical in nature
%   such as linbox
%
% * Other projects can largely achieve multi-platform testing by making
%   their project available in conda-forge, and reusing the conda-forge
%   CI architecture [Julian may wish to say something about this, being
%   more knowledgeable about it.]
%
% * Note issues with obtaining necessary hardware and software licenses for
%   CI/build on non-free platforms: Windows, and ESPECIALLY OSX.
%
% Note: maybe Sage (specifically due to sage-the-distribution) is kind of
% special among all other projects.


\hypertarget{project-reports}{%
\section{Continuous integration achievements of OpenDreamKit projects}\label{project-reports}}

\subsection{SageMath}

% Julian should remark on his efforts here: In particular what problems
% motivated his efforts, and how he thinks this will improve things in the
% future.  Maybe make note here about producing Docker images as build
% artifacts (TODO: confirm if GAP is doing this too), and further how
% mybinder.org turns out to be a useful development tool in conjunction
% with Dockerized builds.  Not sure how much to say about that here, or
% leave it to the conclusion.

\subsection{GAP}

% Via Alex [can he write a couple paragraphs about this?]:
% I will be ready to add a section with the overview of the current state of
% continuous integration in GAP. In addition to the private Jenkins CI instance
% that we use for wrapping and testing release candidates, checking for GAP
% package updates and testing new versions of GAP packages, we now use Travis
% CI to run a number of tests for the main GAP repositories at
% https://travis-ci.org/gap-system/. These include not only CI test for the
% main GAP development repository, but also package integration tests that use
% a set of Docker containers build in various settings
% (https://hub.docker.com/r/gapsystem/). We have a standard CI setup for GAP
% packages, which package authors may adopt and customise. GAP packages using
% Travis CI in the gap-packages VO can be seen at
% https://travis-ci.org/gap-packages/ (some more at
% https://gap-packages.github.io/). Also, GAP and packages use CodeCov to
% measure code coverage: see https://codecov.io/gh/gap-system/gap for GAP and
% https://codecov.io/gh/gap-packages/ for GAP packages.
% 
% @embray
% > Much of the context to this deliverable is about multi-platform testing and
% support. What aspect of our work addresses this? The Travis-CI builds for GAP
% appear to be entirely, or at least almost entirely for Linux (@alex-konovalov
% can clarify). I don't see any OSX builds, and Travis does not support
% Windows.
% Good point about Windows - forgot to mention that we also use AppVeyor:
% https://ci.appveyor.com/project/gap-system/gap

% About Linux vs macOS: package integration tests (all but one badges at
% https://github.com/gap-system/gap-distribution) use Docker container, hence
% use Linux. The core system tests used both Linux and macOS builds in the
% past, but then were disabled because of performance problems:

% https://github.com/gap-system/gap/blob/c8104cd056833f60c9f40efcb81b37b9cec76198/.travis.yml#L66

% We have all three systems - Linux, Windows and macOS - available as Jenkins
% nodes in St Andrews, so we find this to be satisfactory for the moment.


\subsection{Singular}


% Maybe ??

\subsection{PARI}

% Maybe ??

\hypertarget{best-practices}{%
\section{Lessons learned and best practices}\label{best-practices}}
\end{document}

%%% Local Variables:
%%% mode: latex
%%% TeX-master: t
%%% End:
