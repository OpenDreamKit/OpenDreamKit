\section*{\texorpdfstring{Deliverable description, as taken from Github
issue
\href{https://github.com/OpenDreamKit/OpenDreamKit/issues/62}{\#62} on
2017-02-24}{Deliverable description, as taken from Github issue \#62 on 2017-02-24}}\label{deliverable-description-as-taken-from-github-issue-62-on-2017-02-24}
\begin{itemize}
\tightlist
\item
  \textbf{WP3:}
  \href{https://github.com/OpenDreamKit/OpenDreamKit/tree/master/WP3}{Component
  Architecture}
\item
  \textbf{Lead Institution:} University of St Andrews
\item
  \textbf{Due:} 2016-08-31 (month 12)
\item
  \textbf{Nature:} Other
\item
  \textbf{Task:} T3.2 (\#51)
\item
  \textbf{Proposal:}
  \href{https://github.com/OpenDreamKit/OpenDreamKit/raw/master/Proposal/proposal-www.pdf}{p.43}
\item
  \textbf{Final report:} due 2017-02-27
\end{itemize}

\textbf{SCSCP} stands for the \textbf{Symbolic Computation Software
Composability Protocol} - the remote procedure call framework by which
different software components (primarily mathematical software systems)
may offer computational services to a variety of possible clients using
the \href{http://www.openmath.org/}{OpenMath} encoding both for the data
and protocol instructions (see the
\href{http://www.symbolic-computing.org/scscp}{\textbf{SCSCP
specification}} for further details).

SCSCP has been developed in the EU FP6 project 026133
\href{http://www.symbolic-computing.org/}{SCIEnce - Symbolic Computation
Infrastructure for Europe}. In the duration of the project (2006-2011)
and subsequent years, several native CAS implementations of SCSCP client
and server, and also APIs for Java, C and C++ had appeared (see the
complete list \href{http://www.symbolic-computing.org/}{here}). However,
there were no Python OpenMath SCSCP implementations (except a prototype
quality client supporting only lists of integers) and that hindered
further extension of the SCSCP framework.

In this deliverable, we have extended support for SCSCP to other
relevant systems involved in the project. This builds foundation to D3.9
``Semantic-aware Sage interface to GAP'' (\#68) and other activities
outlined in our paper
\href{https://dx.doi.org/10.1007/978-3-319-42547-4_9}{``Interoperability
in the OpenDreamKit Project: The Math-in-the-Middle Approach''}
(Intelligent Computer Mathematics. CICM 2016. Lecture Notes in Computer
Science, vol 9791. Springer). More specifically, we have achieved the
ability to communicate using SCSCP protocol to the following
systems/languages: 

- {[}x{]} Python (both versions 2 and 3): via pure
pip-installable packages 

- {[}x{]} openmath 

- PyPI: https://pypi.python.org/pypi/openmath 

- GitHub: https://github.com/OpenMath/py-openmath 

- {[}x{]} scscp 

- PyPI: https://pypi.python.org/pypi/scscp 

- GitHub: https://github.com/OpenMath/py-scscp 

- {[}x{]} SageMath: via Python
packages listed above 

- {[}x{]} LMFDB: via Python packages listed above

- {[}x{]} PARI: 

- {[}x{]} support via D4.1 ``Python/Cython bindings for
PARI and its integration in Sage'' (\#83) 

- {[} {]} later via D4.10
``Second version of the PARI Python/Cython bindings'' (\#84) 

- {[}x{]}
GAP: via (updated versions of) GAP packages: 

- {[}x{]} OpenMath 

- Website: https://gap-packages.github.io/openmath/ 

- GitHub: https://github.com/gap-packages/openmath 

- {[}x{]} SCSCP 

- Website: https://gap-packages.github.io/scscp/ 

- GitHub: https://github.com/gap-packages/scscp 

- {[}x{]} Singular: via GAP and/or SageMath 

- {[} {]} MathHub: to be written by @tkw1536

\begin{center}\rule{0.5\linewidth}{\linethickness}\end{center}

Further notes: - Relevant tickets in Sage:
https://trac.sagemath.org/ticket/19970 and
http://trac.sagemath.org/ticket/19971 - In view of
https://wbhart.blogspot.co.uk/2016/11/new-computer-algebra-system-oscar\_20.html
it's desirable to implement OpenMath and SCSCP in Julia and later use
Singular through it.
