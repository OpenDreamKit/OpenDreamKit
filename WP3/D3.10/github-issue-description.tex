\begin{itemize}
\tightlist
\item
  \textbf{WP3:}
  \href{https://github.com/OpenDreamKit/OpenDreamKit/tree/master/WP3}{Component
  Architecture}
\item
  \textbf{Lead Institution:} Université Joseph Fourier
\item
  \textbf{Due:} 2019-08-31 (month 48)
\item
  \textbf{Nature:} Other
\item
  \textbf{Task:} T3.3 (\#52)
\item
  \textbf{Proposal:}
  \href{https://github.com/OpenDreamKit/OpenDreamKit/raw/master/Proposal/proposal-www.pdf}{p.~42}
\item
  \textbf{\href{https://github.com/OpenDreamKit/OpenDreamKit/raw/master/WP3/D3.10/report-final.pdf}{Final
  report}}
  (\href{https://github.com/OpenDreamKit/OpenDreamKit/raw/master/WP3/D3.10/}{sources})
\end{itemize}

This deliverable addresses the following objectives of OpenDreamKit:

\textbf{Objective 1:} To develop and standardise an architecture
allowing combination of mathematical, data and software components with
off-the-shelf computing infrastructure to produce specialised VRE for
different communities.

\textbf{Objective 3:} To bring together research communities (e.g.~users
of Jupyter , Sage , Singular , and GAP) to symbiotically exploit
overlaps in tool creation building efforts, avoid duplication of effort
in different disciplines, and share best practice.

\textbf{Objective 4:} Update a range of existing open source
mathematical software systems for seamless deployment and efficient
execution within the VRE architecture of objective 1.

\textbf{Objective 5:} Ensure that our ecosystem of interoperable open
source components is sustainable by promoting collaborative software
development and outsourcing development to larger communities whenever
suitable.

We contribute to the achievements of these objectives through the
creation of source and binary packages for major Linux distributions,
for all OpenDreamKit components.

Sage has a long history of integrating and distributing large
mathematical libraries/software as a whole, with relatively little
attention given to defining and exposing interfaces. Component
re-usability used not to be a main focus for the Sage community, at the
same time the non-standard and relatively underused package system
discouraged writing and maintaining autonomous libraries. These factors
contributed to make the Sage distribution what is usually described as a
``monolith'' (Sage library code alone, not counting included libraries,
makes up for 1.5M lines of code and documentation), hard to distribute,
to maintain, to port, and to develop with. On the other hand, GAP has
been distributing community-developed ``GAP packages'' for a long time,
but faced fragmentation issues, at the code and at the community level.
The rudimentary package system added more technical difficulties to
GAP's development model.

We achieve the stated goal of packaging for major Linux distributions
through several actions:

\begin{itemize}
\tightlist
\item
  Workshops dedicated to packaging,
\item
  Limiting \emph{patched} dependencies in OpenDreamKit software,
\item
  Updating dependencies of OpenDreamKit software,
\item
  Modularization of OpenDreamKit software,
\item
  Providing alternate workflows for user-contributed code, thanks to
  system-specific packaging tools and repositories.
\end{itemize}
