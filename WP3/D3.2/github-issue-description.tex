\section*{\texorpdfstring{Deliverable description, as taken from Github
issue
\href{https://github.com/OpenDreamKit/OpenDreamKit/issues/61}{\#61} on
2017-02-25}{Deliverable description, as taken from Github issue \#61 on 2017-02-25}}\label{deliverable-description-as-taken-from-github-issue-61-on-2017-02-25}

\begin{itemize}
\tightlist
\item
  \textbf{WP3:}
  \href{https://github.com/OpenDreamKit/OpenDreamKit/tree/master/WP3}{Component
  Architecture}
\item
  \textbf{Lead Institution:} Université Paris-Sud
\item
  \textbf{Due:} 2016-02-29 (month 18; originally: month 6)
\item
  \textbf{Nature:} Report
\item
  \textbf{Task:} T3.6
  (\href{https://github.com/OpenDreamKit/OpenDreamKit/issues/55}{\#55})
\end{itemize}

Part of OpenDreamKit's mission is to work on component architecture and
also user interfaces for better collaboration. In this regard, the open
source on-line platform
\href{http://cloud.sagemath.com/}{SageMathCloud}, even thought it is not
directly developed by ODK, is among our primal interests. It has been
described as an emergent technology in
\href{https://github.com/OpenDreamKit/OpenDreamKit/issues/43}{D2.3} and
it is used as a support for a teaching tutorial using OpenDreamKit
technologies in
\href{https://github.com/OpenDreamKit/OpenDreamKit/issues/44}{D2.4}.

Some of the most important technologies of OpenDreamKit, such as Jupyter
and SageMath, are distributed on-line through the SageMathCloud
platform. It makes it a good mean to distribute some of the newly
developed ODK features, for example new Jupyter kernels. It is very
probable that many users will benefit from some of the ODK new
developments through SageMathCloud. Besides, the inner technologies of
SageMathCloud are of special interest to ODK developers: they show
advanced uses of cutting-edge web technologies and explore new leads
that could inspire the work we do in ODK.

For all these reasons, is has been the wish of OpenDreamKit to get
involved into the development of the SageMathCloud platform. This
deliverable marks the first step in this direction. We start exploring
the main layers of SageMathCloud backend code and give a general
overview of its functioning. The material we have produced can directly
help the platform attract more developers. One of the expected follow-up
is an easy install for a local version of SageMathCloud especially
designed for development which could be part of up-coming
\href{https://github.com/OpenDreamKit/OpenDreamKit/issues/63}{D3.5}. The
long term goal however is to understand the extent of a full install of
a SageMathCloud instance on a server: how hard is it? How much does it
cost to maintain? Is it a reasonable solution for an institution do so?
