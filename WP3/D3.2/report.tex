\documentclass{deliverablereport}

\usepackage[style=alphabetic,backend=bibtex]{biblatex}
\addbibresource{../../lib/kbibs/kwarcpubs.bib}
\addbibresource{../../lib/kbibs/extpubs.bib}
\addbibresource{../../lib/kbibs/kwarccrossrefs.bib}
\addbibresource{../../lib/kbibs/extcrossrefs.bib}
\addbibresource{../../lib/deliverables.bib}
%\addbibresource{../../lib/publications.bib}
\addbibresource{rest.bib}
% temporary fix due to http://tex.stackexchange.com/questions/311426/bibliography-error-use-of-blxbblverbaddi-doesnt-match-its-definition-ve
\makeatletter\def\blx@maxline{77}\makeatother

\deliverable{component-architecture}{smc-documentation}
\deliverydate{02/27/2017}
\duedate{02/27/2017 (Month 18)}
\def\pn{OpenDreamKit}
\author{ }

\begin{document}
\maketitle
%There is a large ecosystem of mathematical software systems and knowledge bases.
Individually, these are optimized for particular domains and functionalities, and together they cover many needs of practical and theoretical mathematics.
However, each system specializes on one particular area, and it remains very difficult to solve problems that need to involve multiple systems.
Some integrations exist, but they are ad-hoc and have scalability and maintainability issues.
In particular, there is not yet an interoperability layer that combines the various systems into a virtual research environment (VRE) for mathematics.
  
The OpenDreamKit project aims at building a toolkit for such VREs.
It suggests using a central system-agnostic formalization of mathematics (Math-in-the-Middle, MitM) as a mathematical pivot point for semantic-preserving translations in the needed interoperability layer.
In this \papertype, we report on a series of case studies that instantiates the MitM paradigm with the systems \GAP, \Sage, \LMFDB, and \Singular to perform distributed computation in group, ring, and number theory.
 
Our work involves massive practical efforts, including a novel formalization of computational group theory, improvements to the involved software systems, an extension of the underlying knowledge management system to cope with large theories, and a novel mediating system that sits at the center of a star-shaped integration layout between mathematical software systems and knowledge bases.

Together with deliverable report\textbf{D6.8}, this report describes the implementation and initial evaluation of the MitM integration and interoperability paradigm initially envisioned in deliverables \textbf{D6.2} and \textbf{D6.3}.
The MitM paradigm constitutes the core development goal of \textbf{WP6} and the curated content described in this report enables running non-trivial integration case studies.
In the future we hope to further consolidate content, increase coverage of alignment, and greatly extend the reach of the integration both interms of OpenDreamKit systems covered as well as knowledge available in the MitM Core ontology.

%%% Local Variables:
%%% mode: visual-line
%%% fill-column: 5000
%%% mode: latex 
%%% TeX-master: "report"
%%% End:

%  LocalWords:  optimized formalization papertype textbf textbf textbf

\strut\githubissuedescription
\newpage\tableofcontents\newpage

\section{Introduction}

%% quote from section of the proposal relevant to this work
%% and expand slightly on it.  What is the goal in examining SMC and how does
%% it fit into OpenDreamKit?

\SMC (SMC) provides an important, working example of a Virtual Research
Environment (VRE)
%% Ref: ODK proposal section 1.3.2 %%.
%% Ref: ODK proposal section 1.3.9, pg. 17 %%
with capabilities suitable for reasearchers, software developers, and teachers.
Launched in 2013, \SMC presently hosts over 100,000 projects and 10,000 weekly
active users. This fast adoption by a wide variety of users demonstrates the
relevance and the long term impact this kind of collaborative environments can
have.  Although it is just one example of a VRE, it serves as an important
prototype for OpenDreamKit in a couple key ways: 

Because there is no one VRE that suits all needs, a framework for VREs must be
highly flexible, allowing researchers to choose from the tools that best suit
the task and combine them in arbitrary, yet interoperable ways.  \SMC has
already taken steps in this direction by providing a unified interface to
software systems such as \Sage, \GAP, and other OpenDreamKit components, as
well as a full \Linux shell with common software development tools, \LATEX
document editing and display, \Jupyter notebooks, computing resources, and
other capabilities that one would need for a VRE with unlimited possibilities.
While there is still work to be done on further integrating the components of
\SMC, this is work that OpenDreamKit can feed back into SMC while
simultaneously using SMC as an example VRE.

Another major effort in building a VRE, beyond innovations on the environment
itself and development of effective workflows within a VRE, is the underlying
software and computing infrastructure needed to make VREs accessible and widely
available.  \SMC has given us a real working example, with open source code, of
how to build a cloud-based VRE, accessible from a web browser from anywhere in
the world, with a consistent user interface around its components.  There are
many technical details we can learn from it, such as what software technologies
its framework is built on, and how its underlying computing resources are
organized and administered.

The purpose of this deliverable is to better understand the technical details
of how \SMC works, so that the OpenDreamKit members can learn from it, and also
eventually contribute innovations from OpenDreamKit project back into SMC.

\section{History of \SMC's development}

%% Provide a brief history of SMC's development, and why it is difficult to
%% document, and difficult for newcomers to contribute to.
William Stein, the creator of \Sage and \SMC, starting working on SMC in 2012
as a way to make \Sage--notoriously difficult to install--more accessible to
users via a cloud service.  Around the same time, it became apparent that in
order to survive and to compete with commercial mathematical software systems,
it would be extremely helpful to have a self-sustained income stream based
around \Sage.  Stein realized that, with enough value-added on top of \Sage
itself, users might be willing to pay for access (and in particular computing
and network resources) for this service.  So \SMC became more than just \Sage,
but  rather a cloud-based research and teaching environment built in large part
around but not solely focused on \Sage.

For most of its development history, Stein has been \SMC's sole developer,
working in his spare time whole teaching at the University of Washington at
Seattle.  It was not until September 2013 when its second most extensive
contributor, Harald Schilly, made his first commit to the SMC source code
repository.  Since then, especially once the code was made open source, a
little more than two dozen people have made developments, but the Stein has
still done more than ten times as much work (roughly, in terms of number of
respoitory commits) than anyone else on the project, and as such is the only
person who truly understands its full design.

Because Stein has been practically \SMC's sole developer, and because of the
breakneck pace at which it was built, very little of SMC's internal
design--both overall and the lower-level details, is documented in an
accessible manner.  In order to understand SMC's design one currently has to
read the code, and get inside Stein's head a little bit to understand how he
might have been thinking.  Further complicatiing matters is that SMC has gone
through multiple partial rewrites (most notably a rewrite of the server code,
originally in \Python, to \JavaScript).  Because these rewrites have been only
partial (due to Stein's limited time and developer resources), this has
resulted in what amounts to layers of digital sediment that must be sifted
through carefully in order understand how and why some design decisions came
about.

Because of the fast pace of development, documentation on how to help with
development of \SMC itself--something of interest if OpenDreamKit is to
contribute back to SMC--has also fallen behind.  Even the talk
\href{https://youtu.be/GOuy07Kift4}{How to contribute to SageMathCloud} given
by William Stein in late 2015 and mentioned in the report for D2.2
%% Ref: D2.2 report, p. 3%%
is no longer \emph{as} immediately useful for getting started on development as
it was a year before this report.

None of this should be read as criticism of Stein or \SMC--the reasons for the
relative opacity of SMC's design are understandable.  It is just helpful to
understand why it is particularly challenging to understand and document this
already enormously complex piece of software.  An additional challenge of this
task is to write documentation and development procedures that will be
maintainable through future development without always becoming immediately
obsolete.

\begin{figure}
\includegraphics[width=\textwidth]{images/smc-contributions-2017-02-17.png}
\caption{top \SMC source contributors from the project's beginning in 2012 through February 2017}
\end{figure}

\section{Current state of \SMC's source code and documentation}

%% Summarize the current state of the source code, the current state of the
%% architecture documentation, and of the development documentation and
%% resources.  This is really "where the work started from".

\section{Overview of \SMC's internal design}

%% Provide a brief summary of my documentation of SMC's internals, with link
%% to the full documentation.  Explain how this should help others understand
%% how SMC works, even given its relative flux.  Discuss future work needed
%% to understand the source code and its modularization.

\section{Overview of \SMC's software development process}
%% Summarize work done improving development processes and explain ongoing
%% and future work needed in this area.

\section{Conclusion and future work}
%% Conclude with summary of where we started, where we are now, and what work
%% is still needed.  Perhaps raise open questions about whether or not D3.4, as
%% currently documented in the proposal, is worth pursuing, and why/why not.


\printbibliography
\end{document}

%%% Local Variables:
%%% mode: latex
%%% TeX-master: t
%%% End:

%  LocalWords:  githubissuedescription newpage tableofcontents newpage printbibliography
