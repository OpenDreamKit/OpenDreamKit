\documentclass{deliverablereport}

\usepackage[style=alphabetic,backend=bibtex]{biblatex}
\addbibresource{../../lib/kbibs/kwarcpubs.bib}
\addbibresource{../../lib/kbibs/extpubs.bib}
\addbibresource{../../lib/kbibs/kwarccrossrefs.bib}
\addbibresource{../../lib/kbibs/extcrossrefs.bib}
\addbibresource{../../lib/deliverables.bib}
%\addbibresource{../../lib/publications.bib}
\addbibresource{rest.bib}
% temporary fix due to http://tex.stackexchange.com/questions/311426/bibliography-error-use-of-blxbblverbaddi-doesnt-match-its-definition-ve
\makeatletter\def\blx@maxline{77}\makeatother

\deliverable{component-architecture}{smc-documentation}
\deliverydate{02/27/2017}
\duedate{02/27/2017 (Month 18)}
\def\pn{OpenDreamKit}
\author{ }

\begin{document}
\maketitle
%There is a large ecosystem of mathematical software systems and knowledge bases.
Individually, these are optimized for particular domains and functionalities, and together they cover many needs of practical and theoretical mathematics.
However, each system specializes on one particular area, and it remains very difficult to solve problems that need to involve multiple systems.
Some integrations exist, but they are ad-hoc and have scalability and maintainability issues.
In particular, there is not yet an interoperability layer that combines the various systems into a virtual research environment (VRE) for mathematics.
  
The OpenDreamKit project aims at building a toolkit for such VREs.
It suggests using a central system-agnostic formalization of mathematics (Math-in-the-Middle, MitM) as a mathematical pivot point for semantic-preserving translations in the needed interoperability layer.
In this \papertype, we report on a series of case studies that instantiates the MitM paradigm with the systems \GAP, \Sage, \LMFDB, and \Singular to perform distributed computation in group, ring, and number theory.
 
Our work involves massive practical efforts, including a novel formalization of computational group theory, improvements to the involved software systems, an extension of the underlying knowledge management system to cope with large theories, and a novel mediating system that sits at the center of a star-shaped integration layout between mathematical software systems and knowledge bases.

Together with deliverable report\textbf{D6.8}, this report describes the implementation and initial evaluation of the MitM integration and interoperability paradigm initially envisioned in deliverables \textbf{D6.2} and \textbf{D6.3}.
The MitM paradigm constitutes the core development goal of \textbf{WP6} and the curated content described in this report enables running non-trivial integration case studies.
In the future we hope to further consolidate content, increase coverage of alignment, and greatly extend the reach of the integration both interms of OpenDreamKit systems covered as well as knowledge available in the MitM Core ontology.

%%% Local Variables:
%%% mode: visual-line
%%% fill-column: 5000
%%% mode: latex 
%%% TeX-master: "report"
%%% End:

%  LocalWords:  optimized formalization papertype textbf textbf textbf

\strut\githubissuedescription
\newpage\tableofcontents\newpage

\section{Introduction}

%% quote from section of the proposal relevant to this work
%% and expand slightly on it.  What is the goal in examining SMC and how does
%% it fit into OpenDreamKit?

\SMC (SMC) provides an important, working example of a Virtual Research
Environment (VRE)
%% TODO: Reference ODK proposal section 1.3.2 %%.
%% TODO: Reference ODK proposal section 1.3.9, pg. 17 %%
with capabilities suitable for reasearchers, software developers, and teachers.
Launched in 2013, \SMC presently hosts over 100,000 projects and 10,000 weekly
active users. This fast adoption by a wide variety of users demonstrates the
relevance and the long term impact this kind of collaborative environments can
have.  Although it is just one example of a VRE, it serves as an important
prototype for OpenDreamKit in a couple key ways: 

Because there is no one VRE that suits all needs, a framework for VREs must be
highly flexible, allowing researchers to choose from the tools that best suit
the task and combine them in arbitrary, yet interoperable ways.  \SMC has
already taken steps in this direction by providing a unified interface to
software systems such as \Sage, \GAP, and other OpenDreamKit components, as
well as a full \Linux shell with common software development tools, \LATEX
document editing and display, \Jupyter notebooks, computing resources, and
other capabilities that one would need for a VRE with unlimited possibilities.
While there is still work to be done on further integrating the components of
\SMC, this is work that OpenDreamKit can feed back into SMC while
simultaneously using SMC as an example VRE.

Another major effort in building a VRE, beyond innovations on the environment
itself and development of effective workflows within a VRE, is the underlying
software and computing infrastructure needed to make VREs accessible and widely
available.  \SMC has given us a real working example, with open source code, of
how to build a cloud-based VRE, accessible from a web browser from anywhere in
the world, with a consistent user interface around its components.  There are
many technical details we can learn from it, such as what software technologies
its framework is built on, and how its underlying computing resources are
organized and administered.

The purpose of this deliverable is to better understand the technical details
of how \SMC works, so that the OpenDreamKit members can learn from it, and also
eventually contribute innovations from OpenDreamKit project back into SMC.

\section{History of \SMC's development}

%% Provide a brief history of SMC's development, and why it is difficult to
%% document, and difficult for newcomers to contribute to.
William Stein, the creator of \Sage and \SMC, starting working on SMC in 2012
as a way to make \Sage--notoriously difficult to install--more accessible to
users via a cloud service.  Around the same time, it became apparent that in
order to survive and to compete with commercial mathematical software systems,
it would be extremely helpful to have a self-sustained income stream based
around \Sage.  Stein realized that, with enough value-added on top of \Sage
itself, users might be willing to pay for access (and in particular computing
and network resources) for this service.  So \SMC became more than just \Sage,
but  rather a cloud-based research and teaching environment built in large part
around but not solely focused on \Sage.

For most of its development history, Stein has been \SMC's sole developer,
working in his spare time whole teaching at the University of Washington at
Seattle.  It was not until September 2013 when its second most extensive
contributor, Harald Schilly, made his first commit to the SMC source code
repository.  Since then, especially once the code was made open source, a
little more than two dozen people have made developments, but the Stein has
still done more than ten times as much work (roughly, in terms of number of
respoitory commits) than anyone else on the project, and as such is the only
person who truly understands its full design.

Because it has had effectively one developer, and because of the breakneck pace
at which it was built, very little of \SMC's internal design--both overall and
the lower-level details, is documented in an accessible manner.  In order to
understand SMC's design one currently has to read the code, and get inside
Stein's head a little bit to understand how he might have been thinking.
Further complicatiing matters is that SMC has gone through multiple partial
rewrites (most notably a rewrite of the server code, originally in \Python, to
\JavaScript).  Because these rewrites have been only partial (due to Stein's
limited time and developer resources), this has resulted in what amounts to
layers of digital sediment that must be sifted through carefully in order
understand how and why some design decisions came about.

Because of the fast pace of development, documentation on how to help with
development of \SMC itself--something of interest if OpenDreamKit is to
contribute back to SMC--has also fallen behind.  Even the talk
\href{https://youtu.be/GOuy07Kift4}{How to contribute to SageMathCloud} given
by William Stein in late 2015 and mentioned in the report for D2.2
%% TODO: Reference D2.2 report, p. 3%%
is no longer \emph{as} useful for getting started on full stack development
of SMC as it was a year before this report (though it still contains some
helpful information).

None of this should be read as criticism of Stein or \SMC--the reasons for the
relative opacity of SMC's design are understandable.  It is just helpful to
understand why it is particularly challenging to understand and document this
already enormously complex piece of software.  An additional challenge of this
task is to write documentation and development procedures that will be
maintainable through future development without always becoming immediately
obsolete.

\begin{figure}
\includegraphics[width=\textwidth]{images/smc-contributions-2017-02-17.png}
\caption{top \SMC source contributors from the project's beginning in 2012 through February 2017}
\end{figure}

\section{Current state of \SMC's source code and documentation}

%% Summarize the current state of the source code, the current state of the
%% architecture documentation, and of the development documentation and
%% resources.  This is really "where the work started from".
\SMC's source code currently resides in a single \software{git} repository
hosted on GitHub.  This repository contains all code written uniquely for
SMC--web-based frontend and all backend servers--as well as scripts and tools
for development of SMC.

It also has a collection of many small scripts and utilities used for managing
the live SMC site and the many servers that comprise it (hubs, compute nodes,
etc.)  Many of these are just single command-line commands for what might be
common tasks.  Though there is no documentation for most of these tools or when
and how they should be used.  Most of them are not directly relevant to
doing development of SMC.

Most of the main SMC source code is split across two \Python packages, five
\JavaScript, and one directory containing additional web resources such as
fonts and images, as well as "vendored" \JavaScript--third-party source code
that may require special handling to integrate into the SMC web app.

There is also a configuration script for Webpack,
%% TODO: Reference link to webpack here %%
a tool used to bundle \JavaScript sources, HTML, images, and other resources
into a working website.  This file serves as something of a map to how all of
SMC's frontend code fits together.  Because there is no equivalent for the
backend, it is a little more difficult at first to understand how the
server-side code works.  That said, the general layout is clear enough that an
experienced developer, armed with high-level documentation of SMC's
infrastructure, can gain insight into what features might be implemented where.
More such high-level documentation is needed, however.

Beyond a few brief "README" files in the main source packages, and scattered
in-line documentation in some of the source files, there is very little
documentation of what each source file is or what internal or external APIs
might exist.  So while the high-level structure is not difficult to understand,
it is somewhat more difficult to understand the flow of data and program
control.

\section{Documenting \SMC's internal design}

%% Provide a brief summary of my documentation of SMC's internals, with link
%% to the full documentation.  Explain how this should help others understand
%% how SMC works, even given its relative flux.  Discuss future work needed
%% to understand the source code and its modularization.
In order to help newcomers better undertand \SMC's internal design, we have
begun documenting some aspects of it in better detail.  The first version of
document is included in the annex of this report. It has also been accepted
into SMC's source tree for easier discovery by anyone interested in how SMC
works.

This document covers two topics in particular: It goes into greater detail than
any previous documentation on the core \JavaScript libraries and tools that SMC
is built on, including a brief introduction to those tools, and some discussion
of how and why they are used by SMC.  This list is still not exhaustive, but it
covers most of the key technologies one needs to have understanding of in order
to understand SMC's code base.

Our documentation also provides a brief tour through SMC's code base--
specifically the client-side code--explaining how a page is rendered by SMC's
client code step by step.  This provides specific examples in SMC's code base
of how its dependencies are used, and how the source code is organized
internally.  Some of the exact details may change in the future, but the basic
design (at least on the client side) is likely to remain stable for some time.
This is due in part to the mostly finished work of rewriting SMC's web client
on top of React--a \JavaScript UI framework created and heavily supported by
Facebook.  This React-based design is not likely to change in the forseeable
future.

The walkthrough we have written on SMC's code base is for a very simple
example.  For future work, it would be instructive to include a more complex
example, such as a walkthrough of how SMC's advanced real-time documentation
collaboration features work.  A detailed walkthrough of the server-side
code-base is still needed as well, but Stein and Schilly are in the process of
significantly reorganizing much of that code. Thus it would be advisable to
wait until more of that work is complete before writing further detailed
documentation.


\section{Overview of \SMC's software development process}
%% Summarize work done improving development processes and explain ongoing
%% and future work needed in this area.
A large, complex web application like \SMC is non-trival to do development
on compared to more self-contained software projects.  A full SMC deployment
consists of multiple server processes that must work in concert, and other
moving parts such as Webpack builds for assembling the web client.  SMC's
source code documents a few different ways to work on SMC:

\begin{itemize}
    \item The simplest is to run a development SMC server directly on one's
        personal computer, reloading the server as necessary while making
        changes to the source code.  SMC in fact has a "development" mode
        wherein all of SMC's server components run on one's local machine,
        which also acts as the sole compute node.  Any projects on the
        development server are created and run on the local machine, under the
        developer's login account.  This is certainly the simplest way to work,
        but the code currently has some (known) bugs such that a SMC server run
        in this way is not actually fully functioning.  Thus, while this is
        feasible for some development tasks, it is not currently possible to
        develop the full SMC stack in this way.  We recommend trying to fix
        that if at all possible.  Another downside to this method is that the
        developer must install all of SMC's requirements manually, and this is
        not well documented at the moment.  As much of SMC is UNIX-oriented
        this also creates a barrier for would-be developers working on Windows.

    \item Another compelling way to develop on SMC is referred to as
        "SMC-in-SMC".  It is possible, on an existing SMC server (such as the
        main one hosted at {\tt cloud.sagemath.com}), download the SMC source
        code into a project, and run a full SMC server from within the SMC
        project.  This is somewhat similar to the previous method, but takes
        advantage of the fact that all of SMC's runtime dependencies are
        already available inside an SMC project.  It is easier to get
        SMC-in-SMC up and running than most other methods, and it works
        impressively well considering its recursive nature.  While one is
        restricted to working within an SMC project this is not \emph{much} of
        a downside considering the flexibility afforded SMC users).  It does,
        however, require having a non-free SMC account (the default free
        accounts for SMC do not allow internet access from within projects),
        though Stein is generous in giving less restricted free accounts to
        users who wish to help with SMC development.

    \item The easiest way to get a fully functioning SMC server up and running
        on one's personal computer is with the official Docker container for
        SMC.
        %% TODO: Reference D3.1 report here, on Docker containers
        The Docker container image is well-maintained to work with the latest
        SMC source code, and provides a turn-key solution for a full-stack
        single-server SMC deployment.  There is a slight downside, however,
        that its current design is not very amenable to development.  For
        example, the SMC source code is contained in the container itself,
        which means any and all development, including editing the source code,
        must be done within the Docker container.  This may require a would-be
        developer to manually reproduced their preferred software development
        environment (tools and settings) and edit all source code directly
        inside the Docker container.  A better approach is to design a Docker
        container for development that allows the developer to keep the SMC
        source code on their local machine.  The SMC Docker container would
        then mount the source code as an external "volume" (this is a feature
        of containers similar to a shared folder between the container and its
        host).  The developer can than do all development work on their local
        machine, but \emph{execute} the source code inside the Docker
        container, freeing them of the need to worry about setting up SMC's
        dependencies.  We have begun work on a development-friendly Docker
        container for SMC, but more work is required to make this fully viable.
        When completed this will likely be the easiest way for anyone to get
        quickly up and running with SMC development on their personal computer.

\end{itemize}


\section{Conclusion and future work}
%% Conclude with summary of where we started, where we are now, and what work
%% is still needed.  Perhaps raise open questions about whether or not D3.4, as
%% currently documented in the proposal, is worth pursuing, and why/why not.
Prior to work on this deliverable there was scant documentation on \SMC, its
overall design, its dependencies, or its internals.  There was some
documentation on how to do development on it, but not enough, and sometimes
out of date.  With the documentation we have added it should require less
effort for a motivated developer to gain the necessary background knowledge
(such as knowledge of the core technologies SMC is built on) to development on
SMC.  Our documentation should also help lead to a faster understanding of
the general flow of SMC's design, though it has not been feasible to document
the implementations of all of its features.  Some of its more advanced
features--especially key features such as its real-time collaborative document
editing--require further documentation.  More documentation is needed on the
design of SMC's server components, though that work should wait until their
implementation has stabilized more.

We have also gained experience within the OpenDreamKit team on setting up a
\SMC server and the process of doing development on SMC.  Gaining this shared
knowledge is a prerequisite for future work on determining what aspects of
SMC--at its core a framework for cloud-based hosting of complex mathematical
software, and thus a powerful tool for building VREs--can be integrated into
OpenDreamKit.  However, further work is needed on improving the development
tools for SMC--in particular we recommend improving the existing Docker
container for SMC for development on one's personal computer.


\printbibliography
\end{document}

%%% Local Variables:
%%% mode: latex
%%% TeX-master: t
%%% End:

%  LocalWords:  githubissuedescription newpage tableofcontents newpage printbibliography
